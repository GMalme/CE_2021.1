\subsection{Minhas impressões iniciais sobre a ciência, por Gustavo Rodrigues dos Santos}

Desde o primórdio da humanidade a ciência sempre foi uma área intrínseca ao ser humano, sempre tentamos explicar o porquê das coisas que acontecem ao nosso redor, os denominados fenômenos ( \gls{fenomeno}). Dessa forma, podemos analisar que a resposta para a pergunta \textit{O que é ciência?} vem se modificando juntamente à humanidade. Inicialmente, a ciência podia se apresentar na forma de um mito o qual retratava a cosmogonia de um povo. Porém ao longo da evolução da humanidade, a ciência tende a acompanhá-la, passando desde os filósofos gregos até a modernidade, quando ocorre um marco para a história da ciência, a Revolução Científica. Desse modo, a ciência sai de um estado nascente rumo a um estado final, sendo este o estado o qual a produção do conhecimento esteja totalmente e precisamente correta acerca dos fenômenos estudados.

Atualmente, nos encontramos no meio dessa transição, nosso conhecimento científico é repleto de erros, e estamos numa fase onde a verdade não é absoluta, como relatado por \citet{fernandes_consideracoes_2021}. Assim, como qualquer outro setor da sociedade, a ciência vivencia regularmente crises não só ligadas à \textit{``incapacidade dos paradigmas científicos vigentes para explicar a realidade''} \citep{fernandes_consideracoes_2021}. Um grande exemplo disso são os movimentos anti-ciência vistos na atual conjuntura onde é notável a crise da elitização da ciência. Esse fenômeno em nada se relaciona à não competência das atuais teorias em explicar os eventos, mas sim com o descolamento da população da comunidade acadêmica. Os cientistas são seres humanos, e portanto, naturalmente, possuem defeitos e consequentemente vieses e conceitos que muitas vezes refletem a própria sociedade a qual estão inseridos. Portanto, se almejarmos chegar a ciência em seu estado final, logo devemos nos opor à movimentos reacionários para assim dar continuidade ao processo espontâneo da ciência. Nós como sociedade devemos, primeiramente, buscar formas de superar nossos vieses e consequentemente alterar as estruturas em nosso meio que perpetuam esses vieses e quem sabe assim conseguiremos progredir de uma maneira mais fluida e assertiva para o objetivo final ou sua proximidade.
