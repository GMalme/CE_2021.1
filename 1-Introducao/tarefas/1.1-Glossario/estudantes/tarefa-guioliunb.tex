	\newglossaryentry{Reprodutibilidade}{
name = {Reprodutibilidade},
description  = {
A reprodutibilidade de uma experiência científica é uma das condições que permitem incluir no processo de progresso do conhecimento científico as observações realizadas durante a experiência. Essa condição origina-se no princípio de que não se pode tirar conclusões senão de um evento bem descrito, que aconteceu várias vezes, provocado por pessoas distintas. Essa condição permite se livrar de efeitos aleatórios que podem afetar os resultados, de erros de julgamento ou de manipulações por parte dos cientistas.
Também, significa que os dados da investigação e o código são disponibilizados de forma a que outros possam atingir os mesmos resultados que são declarados em resultados científicos . Relacionado está com o conceito de replicabilidade, o ato de repetir uma metodologia científica para atingir conclusões semelhantes. 
Fonte:	\cite{noauthor_reproducibility_nodate}
}
}
	