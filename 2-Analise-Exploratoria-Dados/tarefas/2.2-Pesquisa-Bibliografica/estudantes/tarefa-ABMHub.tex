\chapter{Análise Bibliográfica sobre Processamento de Linguagem Natural, por Lucas de Almeida Bandeira Macedo}

\section{Planejamento do estudo}

Com a vinda de assistentes virtuais, como a Alexa (Amazon), Cortana (Microsoft) ou Siri (Apple), as pessoas costumam se perguntar cada vez mais: "como que esse programa está entendendo o que eu falo?".

Mas não só de assistentes virtuais vive o Processamento de Linguagem Natural (também conhecido como NLP - Natural Language Processing), afinal, qualquer texto ou fala pode ser interpretado por uma máquina e devidamente classificado. Por exemplo, uma aplicação famosa é o "classificador de sentimentos", em que um modelo treinado consegue classificar textos entre sentimentos "positivos" ou "negativos". Com a ascensão do Twitter, uma rede social baseada em pequenos textos de não mais que 280 caracteres, NLP se torna cada vez mais interessante.

Assim, as perguntas que traçam o norte para este estudo são:

\begin{itemize}
    \item Quais os principais conceitos ligados com Processamento de Linguagem Natural?
    \item Como se dá o progresso das pesquisas em NLP ao longo dos anos? As redes sociais influenciaram esse crescimento?
    \item Qual o estado da estrutura social da comunidade de NLP?
\end{itemize}

\subsection{Uso do Bibliometrix e Biblioshiny}

Será usada a ferramenta Bibliometrix, com sua função Biblioshiny, para gerar gráficos e grafos iterativos e personalizáveis, para auxiliar na interpretação da realidade científica do tópico.

\section{Coleta de dados}

A coleta de dados foi feita utilizando o site Web of Science (WoS), no dia 03/02/2022, através do portal periódico da capes.

A pesquisa foi realizada utilizando as edições "Science Citation Index Expanded" e "Conference Proceedings Citation Index – Science", ambas coleções são voltadas para, principalmente, as ciências exatas.

A \textit{string} (ou \textit{query}) de busca inicialmente utilizada foi a seguinte:

\lstinputlisting[numbers=left,basicstyle=\normalsize\ttfamily,caption={Query de busca sobre Procesasmento de Linguagem Natural.},label=queryNLP03022022]
{experiments/ABMHub/PesquisaBibliometrica/NLP/pesquisa_velha.txt}

\subsection{Explicação para os termos de busca usados}

A proposta é apenas pesquisar sobre Processamento de Linguagem Natural, sem muito rigor na aplicação em que essa arquitetura de rede neural é aplicada. Portanto, inicialmente a pesquisa foi apenas "natural language processing".

Porém, uma rápida olhada pelos artigos retornados evidenciou uma grande quantidade de artigos sobre linguísticas, e áreas que não são da computação. Como o objetivo aqui adquirir modelos de Deep Learning, a pesquisa foi ajustada para filtrar apenas por NLP ligadas diretamente a computação e inteligência artificial, evidenciado pelas cláusulas "neural network", "(machine or deep) and learning" e "artificial intelligence". Essa nova pesquisa trouxe melhores resultados, todos evidenciando redes neurais e variadas técnicas de machine learning. O total de registros retornado pela query foi 

\subsubsection{Refinamento da Coleta de Dados}

 Em seguida, em uma análise mais fina, utilizando a \textbf{Rede de Co-ocorrências de Palavras-chave}, podemos evidenciar outras palavras chaves que estavam aparecendo entre os registros da pesquisa, que não deveriam estar aparecendo. É possível observar na imagem \ref{fig:ABMHub:NLPgraph1}, palavras como "câncer" ou "diagnóstico" que estão relacionadas a visão computacional mais que NLP, aparecendo com pesos não-desprezíveis.
 
 \begin{figure}
    \centering
    \includegraphics[angle=0,width=1\textwidth]{experiments/ABMHub/PesquisaBibliometrica/NLP/network.png}
    \caption{Grafo de relação de keywords}
    \label{fig:ABMHub:NLPgraph1}
\end{figure}

Assim, é necessário uma nova iteração da pesquisa, para evitar que registros de visão computacional corrompam a pesquisa de NLP. É delicado fazer isso, pois existem muitas menções a Visão Computacional nos registros de LP, já que ambos são ligados a Deep Learning, então retirar a keyword "Visão Computacional" provavelmente removeria muitos registros que não gostaríamos de remover da pesquisa. Assim, a melhor solução encontrada foi remover palavras que não têm intersecção entre os dois assuntos. Por exemplo, "medical", "cancer" e "diagnosis".

Assim, chegamos na mais recente query:

\lstinputlisting[numbers=left,basicstyle=\normalsize\ttfamily,caption={Query de busca sobre Procesasmento de Linguagem Natural.},label=queryNLP03022022]
{experiments/ABMHub/PesquisaBibliometrica/NLP/pesquisa_nova.txt}

