\chapter{Análise Bibliográfica sobre Sistemas de Recomendação de Notícias, por Tatiana Franco Pereira\label{chap:bibliometria:Tatianafp}}

\section{Planejamento do estudo}

Após a transposição do conteúdo dos jornais impressos para o formato digital, sites e aplicativos de notícias passaram a ser fonte de um conteúdo gigantesco de informações, que é distribuído ao longo de inúmeras páginas. Com isso, a navegação do usuário durante o seu consumo de informação se tornou mais complexa, ao mesmo tempo em que sua navegação pelo conteúdo apresentado tem o potencial de se tornar mais personalizado. 

Sistemas de recomendação de notícias tem por objetivo manter o usuário interessado no conteúdo que lhe é ofertado no site ou aplicativo de notícias, com usuário mais engajados e uma maior navegação em suas páginas, a fonte de informação ganha maior relevância, além de arrecadar mais dinheiro por meio de campanhas e marketing para outras empresas. 

Tendo isso em vista, este trabalho teve por objetivo responder as seguintes perguntas:
\begin{itemize}
    \item Qual a base de conhecimentos científicos produzida em torno do tema sistemas de recomendação de notícias? 
    \item Quais os principais termos e conceitos ligados à frente de pesquisa no tema sistemas de recomendação de notícias? 
    \item Qual a estrutura social da comunidade, se é que existe, que pesquisa sobre o tema sistemas de recomendação de notícias?
    \item Quanto o Brasil estaá inserido na comunidade de pesquisadores do tema sistemas de recomendação de notícias? 
\end{itemize}

\subsection{Uso do Bibliometrix e Biblioshiny}

Para este estudo, serão utilizadas as ferramentas Bibliometrix e Biblioshiny, executadas por meio do RStudio.

\subsection{Limitações} 

O exercício foi feito em uma semana, envolvendo aproximadamente 7 horas de trabalho, utilizando a base de dados Web of Science (WoS).

\section{Coleta de dados}

A coleta de dados feita usando o WoS no dia 07 de fevereiro de 2022, acessado por meio do Portal de Periódicos da CAPES.

Foram feitas buscas nas coleções \textbf{Science Citation Index Expanded (SCI-EXPANDED), Social Sciences Citation Index (SSCI), Conference Proceedings Citation Index – Science (CPCI-S), Emerging Sources Citation Index (ESCI)}, que contém registros relativos a vários campos do conhecimento. Os artigos nessas duas coleções são indexados desde 1945. 

\subsection{Query de Busca}

Foi usada a \textit{query} de busca ilustrada na listagem:

\lstinputlisting[numbers=left,basicstyle=\normalsize\ttfamily,caption={Query de busca sobre Sistemas de Recomendação de Notícias.}]
{experiments/Tatianafp/PesquisaBibliometrica/query}

\subsubsection{Explicação para os termos de busca usados}

A busca consistiu de duas cláusulas unidas por uma conjunção \textit{and}. A primeira foi aplicada à busca por tópico, ou seja, o termo de busca poderia aparecer no Título, no Abstract, na Author Keywords, ou nas Keywords Plus da referência. Já a segunda foi aplicada à busca por categorias. 

O termo \texttt{news recommendation systems} foi usados na primeira cláusula da \query\  para recuperar artigos que tenham em seu título, palavras-chave e resumo, termos relacionados a sistemas de recomendação de notícias. Foi usado um único termo devido à forte adesão ao termo por parte dos pesquisadores desta área.

A busca foi limitada às categorias \texttt{ Computer Science Information Systems}, \texttt{ Computer Science Artificial Intelligence} e \texttt{Computer Science Interdisciplinary Applications} por meio da segunda cláusula, visando concentrar os resultados da pesquisa à trabalhos relacionados às áreas de Sistemas de Informação, Inteligência Artificial e Aplicações Interdisciplinares. 

\subsection{Registros recuperados}

Os 452 registros obtidos como resultado da busca encontram-se em \url{https://www.webofscience.com/wos/woscc/summary/a921d816-49be-44eb-94ec-d0430ce03243-22c82fa6/relevance/1}. 

Foram utilizadas as opções \textit{Exportar registros para arquivo Bibtex} e \textit{Gravar Conteúdo: Registro completo e Referências citadas} no WoS, para que as citações também fosse usadas em análises da citações.

\section{Análise dos Dados}

\subsection{Filtragem de registros}

Antes da análise, foi aplicado um filtro sobre o \dataset\ inicial, que era formado por 452 registros,contendo pŕevias de artigos, artigos de conferência, entre outros. Foram removidos os registros de prévias de artigos, restando 446 registros após a filtragem. Este conjunto de artigos será chamado de NewsRec@Tatianafp. 

\subsection{Análise bibliográfica descritiva do \textit{dataset} }

Será apresentado a seguir, uma análise bibliométrica descritiva do \dataset\   NewsRec@Tatianafp, a qual visa fazer um descrição inicial do conjunto de registros. Tal análise foi gerada pela função \texttt{biblioAnalysis}.

As informações mais gerais sobre o \dataset\ NewsRec@Tatianafp são as seguintes:

\begin{description}
    \item [\textit{Timespan}]  Os artigos que atenderam aos critérios de busca e filtragem foram publicados a partir de 1994, até 2021. Ou seja, não foram encontrados registros entre 1945 e 1993.
    \item [\textit{Sources (Journals, Books, etc)}] São 311 fontes de informação que publicaram os documentos recuperados no \dataset\  NewsRec@Tatianafp. Ou seja, em média, cada \textit{scientific journal} publicou $446/311=1,43$ artigos. 
    \item [\textit{Average years from publication}] A média do tempo de publicação dos artigos no \dataset\  NewsRec@Tatianafp é de 6,37 anos.
    \item [\textit{Average citations per documents}] Cada artigo no \dataset\   NewsRec@Tatianafp  foi citado, em média 13,11 vezes.
    \item [\textit{Average citations per year per doc}] Cada artigo no \dataset\   NewsRec@Tatianafp  foi citado, em média 1,704 vezes por ano.
    \item [\textit{References}] O \dataset\  NewsRec@Tatianafp contém 10463 referências citadas.
    \item [\textit{Keywords Plus (ID)}] 193 distintas palavras-chave do tipo Keywords Plus (ID) foram encontradas no \dataset\ NewsRec@Tatianafp. 
    \item [\textit{Author's Keywords (DE)}] 1.180 distintas palavras-chave indicadas pelos autores foram encontradas no \dataset\ .
    \item [\textit{Authors}] 1.222 distintos nomes de autores foram encontrados no \dataset\ .
    \item [\textit{Author Appearances}] Os 1.528 distintos (nomes de) autores foram encontrados 23.470 vezes, como autores de artigos.
    \item [\textit{Authors of single-authored documents}] Dentre os 1.528 distintos (nomes de) autores encontrados, 24 deles editaram artigos individualmente, isso é, sem co-autores.
    \item [\textit{Authors of multi-authored documents}] Dentre os 1.528 distintos (nomes de) autores encontrados, 1.198 deles editaram artigos com um ou mais co-autores"
    \item [\textit{Single-authored documents}] Dentre os 446 documentos presentes no \dataset\   NewsRec@Tatianafp, 26 foram escritos por um único autor, e os 420 restantes foram elaborados em co-autoria.
    \item [\textit{Documents per Author}] Dentre os 1.528 distintos (nomes de) autores, cada um publicou em média 0,365 artigos.
    \item [\textit{Authors per Document}] Cada um dos 446 documentos presentes no \dataset\   NewsRec@Tatianafp foi autorado com 3,35 autores em média ($19.410 / 5.787 = 3,35$).
    \item [\textit{Co-Authors per Documents}] As 1.528 aparições de (nomes de) autores (``Author Appearances''), sem distribuem, em média 2,74 vezes para os 446 documentos do \dataset\  NewsRec@Tatianafp.
    \item [\textit{Collaboration Index}] Os 1.198 (nomes de) autores que editaram artigos com um ou mais co-autores, colaboraram em media 3,43 vezes para editar os 446 artigos elaborados em co-autoria, gerando, assim, um índice de colaboração 2,85. 
\end{description}

\subsection{Análise da Evolução da Produção Científica}

Com o Bibliometrix, podemos analisar como a produção científica de artigos sobre o tema Sistemas de Recomendação de Notícias evoluiu ao longo dos anos, segundo o \dataset\  NewsRec@Tatianafp. Ao realizar essa análise, encontramos a imagem \ref{fig:evol_anual_NewsRec_Tatianafp}. 

\begin{figure}
    \centering
    \includegraphics[width=1\textwidth]{experiments/Tatianafp/PesquisaBibliometrica/images/AnnualScientificProduction.png}
    \caption{Evolução da produção científica no \dataset\  NewsRec@Tatianafp.}
    \label{fig:evol_anual_NewsRec_Tatianafp}
\end{figure}

O \textit{Annual Growth Rate} encontrado no \dataset\  NewsRec@Tatianafp é de 5.12\%, consideravelmente maior que a taxa de crescimento anual da publicação científica mundial, de cerca de 3.3\%.

Ao observarmos a diferença entre a quantidade de publicações entre os anos de 2014 e 2018, os dois picos de crescimento, nota-se que o número quase que duplica. Isso indica que provavelmente ocorrera uma descoberta importante no ano de 2018, o que incentivou a pesquisa sobre o tema de sistemas de recomendação de notícias. No entanto, houve uma queda na produção científica nos anos posteriores ao ano de 2018, o motivo para tal ainda necessita de maiores investigações.  

\subsection{Análise da Colaboração entre Países}

Ao explorar o \dataset\  NewsRec@Tatianafp com o auxílio da ferramenta Bibliometrix, também é possível analisar como ocorre a colaboração entre os diversos países na produção científica relacionada ao tema de sistemas de recomendação de notícias. A imagem \ref{fig:country_collab_NewsRec_Tatianafp} retrata o gráfico que fora obtido. 

\begin{figure}
    \centering
    \includegraphics[width=1\textwidth]{experiments/Tatianafp/PesquisaBibliometrica/images/CountryCollaborationMap.png}
    \caption{Colaboração entre países segundo o \dataset\  NewsRec@Tatianafp.}
    \label{fig:country_collab_NewsRec_Tatianafp}
\end{figure}

Ao analisar a imagem \ref{fig:country_collab_NewsRec_Tatianafp} nota-se uma alta colaboração entre os Estados Unidos e China, acompanhados por colaborações com países da Europa e mais alguns países da Ásia como Índia, Japão e Coréia do Sul. Também se tem um volume considerável de contribuições com a Austrális e com o Canadá. Apesar de estar destacado no mapa, o Brasil não tem suas colaborações representadas pelas linhas que ligam os países. Isso ocorre devido ao seu número pequeno de colaborações, se resumindo a apenas três, conforme apresentado na tabela \ref{tab:Brasil_collab_Tatianafp}. 

\begin{center}
\begin{table}[h]
\begin{tabular}{||c c c||} 
 \hline
 País & País & Frequência \\ 
 %[0.5ex] 
 \hline\hline
 Brasil & Austria & 1 \\ 
 \hline
 Brasil & Portugal & 1 \\
 \hline
 Brasil & Rússia & 1  \\ \hline
% \textbf{}
\end{tabular}
 \caption{Colaborações entre o Brasil e demais países no \dataset\ NewsRec@Tatianafp.}
\label{tab:Brasil_collab_Tatianafp}
\end{table}
\end{center}

\subsection{Análise da Fontes mais relevantes }

Além das análises apresentadas anteriormente, também foi analisado as fontes mais relevantes entre aquelas nas quais foram publicados os registros. Observa-se que as fontes textit{Experts Systems with Applications} e textit{IEE Acesss} empataram na primeira posição com o total de 15 documentos, conforme apresentado na imagem \ref{tab:Brasil_collab_Tatianafp}. Fora isso, também é possível perceber que os registros estão bem distribuídos ao longo das demais fontes de publicação. 

\begin{figure}
    \centering
    \includegraphics[width=1\textwidth]{experiments/Tatianafp/PesquisaBibliometrica/images/MostRelevantSources.png}
    \caption{Fontes mais relevantes segundo o \dataset\ NewsRec@Tatianafp.}
    \label{fig:relevant_sources_NewsRec_Tatianafp}
\end{figure}

\subsection{Análise da WordCloud }

Utilizamos a WordCloud gerada pelo Bibliometrix a fim de analisar os termos mais relevantes presentes nos documentos. Como apresentado na figura , pode-se notar um aumento de termos como textit{privacy}, textit{trust} e textit{diversity}. Mostrando uma recente preocupação da comunidade de produção científica  sobre o tema Sistemas de Recomendação de Notícias com os impactos que podem ser gerados por seus softwares. 

\begin{figure}
    \centering
    \includegraphics[width=1\textwidth]{experiments/Tatianafp/PesquisaBibliometrica/images/WordCloudDocuments.png}
    \caption{Fontes mais relevantes segundo o \dataset\ NewsRec@Tatianafp.}
    \label{fig:wordloud_NewsRec_Tatianafp}
\end{figure}

\section{Resultados e interpretação}

Os picos apresentados no gráfico de evolução da produção científica precisam ser melhor compreendidos por meio de uma busca mais aprofundada. Ademais, esta pesquisa bibliográfica indicou um crescente preocupação da comunidade de produção científica  sobre o tema Sistemas de Recomendação de Notícias com os impactos que podem ser gerados por seus softwares, uma grande colaboração mundial e uma presença mesmo que tímida de pesquisas brasileiras na comunidade científica deste tema. 