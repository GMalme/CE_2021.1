\subsection{Minhas impressões iniciais sobre a ciência, por João Pedro Felix}

A noção do que de fato é ciência por vezes pode ser equivocada e a realidade é que muito do que se considera ciência é na realidade uma aplicação da mesma.
Como visto \hyperlink{1-Introducao/aulas/Ciencia-e-sua-Avaliacao.pdf.2}{acima}, a ciência tem por objetivo, através de um estudo metodicamente organizado, descrever, explicar e prever fenômenos \gls{fenomeno} a partir de dados \gls{Dado} obtidos através do estudo desses próprios fenômenos.

Isso por si só desconstrói a ideia que boa parte das pessoas tem do que é ciência, ao classificar áreas de estudo como a matemática não como ciência, mas sim como uma ferramenta utilizada em prol da ciência. Isso não retira uma parte sequer da importância da matemática e muito menos muda a necessidade de se investir no desenvolvimento da área, mas nos convida a refletir também sobre outras áreas que talvez funcionem muito mais como um meio pelo qual se faz ciência do que como ciência propriamente dita.

A ciência da computação por exemplo é uma dessas áreas. Boa parte das pessoas que trabalham na área não atua descrevendo, explicando e prevendo fenômenos. Até porque, podemos dizer que no fim boa parte da computação se resume a aplicação da matemática. Trata-se de uma área totalmente diferente da biologia por exemplo, não é algo que se pode tirar proveito simplesmente observando e coletando dados. A computação depende dos seres humanos para existir, bem ao contrário da vida, no caso da biologia.

A ciência da computação tem como principal objetivo intervir na realidade. Tornar situações adversas em outras mais favoráveis. Ela ainda faz uso da observação, mas de problemas externos a própria computação, ela analisa o cotidiano das pessoas, identifica aquilo que não funciona bem e procura maneiras de fazer funcionar melhor. Ou seja, a maioria dos profissionais da área de ciência da computação não atua diretamente na produção de conhecimento cientifico, mas sim em sua aplicação na vida das pessoas.

Mas será que isso não seria o suficiente para classificá-la como ciência ao invés de uma ferramenta como a matemática? A observação se faz presente, mas ao invés de acontecer sobre um vírus, uma molécula, ela acontece sobre a sociedade, assim como nas ciências humanas. Ela ainda busca descrever fenômenos, mas não a partir de palavras e sim da poderosa matemática, de modo a explicá-los não com um simples objetivo da reflexão, mas sim com uma visão atuante na intenção de intervir nesse meio para tornar as condições mais favoráveis e é justamente a intenção de intervir que traz a tona a previsão. Não há como intervir em uma realidade sem conhecê-la. É necessário ter em mente o que acontece no meio ao qual se está inserido se o objetivo for mudá-lo, e é fato que isso é fundamental no desenvolvimento de qualquer software que será utilizado por seres que vivem em uma realidade que muda constantemente.

De fato, existem contribuições dentro da área de ciência da computação que visam o desenvolvimento interno da própria área, que visam evoluir a própria computação. Assim como é fato que a ciência da computação pode muito bem ser usada como uma ferramenta em outras áreas do conhecimento para atingir objetivos próprios dentro dessas áreas. Contudo, o fato de poder ser utilizada como ferramenta não muda o fato de que existem situações que o papel do profissional é exatamente o mesmo de um cientista de qualquer outra área: descrever, explicar e prever, a maior diferença no fato de que a isso é adicionado o "intervir".

A diferença então estaria na parte de que tudo isso deveria ser feito a partir de um estudo metodicamente organizado? De fato, a maneira como se constrói software hoje em dia se difere muito da maneira como outras áreas realizam a produção cientifica. Abordagens de desenvolvimento comuns no mercado atual, como a abordagem de desenvolvimento ágil, se baseia em metas, em marcos que devem ser obtidos a cada período de tempo, sem especificar exatamente como. Se por um lado isso garante certa organização a equipe de desenvolvimento, não há como dizer que ela surge de forma metódica.

Mas e quanto as abordagens usadas em sistemas criticos? O desenvolvimento nesses casos costuma ser baseado em rigorosa documentação em que um fenômeno (problema) é analisado, descrito de maneira matemática, de forma que possa ser explicado sem ambiguidades, e assim é construído um esquema capaz de prever o comportamento do mesmo de maneira que seja possível intervir no mesmo. Feito isso, cada detalhe de como "intervir" é planejado metodicamente de modo que a primeira linha de código só seja digitada no momento que o plano estiver claro e revisado diversas vezes.

O fato de buscar a intervenção tornaria a área algo diferente de ciência? Ou o sentido de descrever, explicar e prever não deveria ser aplicado sobre uma sociedade caso o objetivo seja intervir? Fato é que se a produção cientifica não servir para alterar a realidade de nada ela servirá. Felizmente, como preferimos ver áreas A ou B pouco importa na produção do conhecimento cientifico.