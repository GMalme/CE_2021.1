\newglossaryentry{Paradigma}{
	name={Paradigma},
	description={Paradigma (do latim tardio paradigma, do grego parádeigma, derivado de paradeiknúō -mostrar, apresentar, confrontar-) é um conceito das ciências e da epistemologia (a teoria do conhecimento) que define um exemplo típico ou modelo de algo. É a representação de um padrão a ser seguido. É um pressuposto filosófico, matriz, ou seja, uma teoria, um conhecimento que origina o estudo de um campo científico; uma realização científica com métodos e valores que são concebidos como modelo; uma referência inicial como base de modelo para estudos e pesquisas.. Fonte: \citet{noauthor_paradigma_nodate} (Tradução livre)
	}
}