\subsection{Minhas impressões iniciais sobre a ciência, por Marcus Vinícius Oliveira de Abrantes}
Francis Bacon, um dos precursores do método científico, inaugura em \citep{bacon_novrum_1620} uma interpretação da natureza baseada em experimentação e investigação. Através dos ídolos, expõe as deviações do conhecimento decorrentes das armadilhas mentais e limitações humanas. No ídolo da tribo, mostra como a percepção humana por meio dos sentidos difere da natureza da realidade. No ídolo da caverna, trata dos vieses concebidos internamente por cada indivíduo, frutos de sua experiência e construção social, que afetam a formação de ideias . No ídolo do foro demonstra que na própria fala existem dificuldades  em perpassar o conhecimento de forma concisa e clara. Por fim no ídolo do teatro, vai de encontro ao aprisionamento frente às formas tradicionais de se pensar, como fábulas concebidas como explicação.
É nesse contexto que surge o método científico, um \gls{Paradigma}  que busca esquivar-se das ilusões e erros da mente para se obter um modelo o mais próximo o possível da realidade. É descrito muitas vezes como uma forma de desafiar as próprias conclusões, questionando todas as possibilidades para encontrar frestas que solicitem um modelo novo. Vai de encontro não a uma resposta final, mas a uma suficientemente condizente com a realidade, até que se prove o contrário.