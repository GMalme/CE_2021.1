\section{Respostas de Guilherme Oliveira Loiola}

Iniciando como motivador da discussão a seguinte pergunta:


A atividade profissional de um cientista da computação utiliza todos os componentes de um experimento científico?


Introduzindo o raciocínio para responder o questionamento proposto, fica claro que é necessário definições dos itens essenciais da ciência e da atuação profissional focada em \gls{Computacao}. 


A \gls{Ciencia} tem como fundamento o estudo racional através de métodos científicos que buscam mapear padrões sobre as experiências avaliadas. Tudo está voltado para a produção de conhecimento pela construção de \gls{ModeloCientifico} que facilite o entendimento sendo condizente com os resultados observados.

A construção do conhecimento científico tem relação com descobertas empíricas podendo fazer previsões futuras ou não. Dessa forma, ampliando o conhecimento sobre a natureza, contudo se difere de uma ciência aplicada pelo interesse no resultado produzido. Como explicação, a ciência pura tem o único intuito sua progressão mesmo que não haja uma demanda instantânea do conhecimento produzido. Diferente de uma atuação profissional que visa exclusivamente a solução de demandas atuais com demonstração lógica sobre o resultado.

\subsection{A computação é uma ciência empírica e uma atividade científica?}

As construções de soluções computacionais também são embasadas em metodologias preocupadas com o resultado empírico, ou seja, a corretude para atingir um resultado prático. Suas bases de conhecimento são: ciência, engenharia e matemática. Com o funcionamento fundamentado em matemática e lógica a computação em si não ocorre por estudo empírico, embora os resultados tenham alta qualidade são construídos por deduções de axiomas matemáticos e não observações empíricas. Então, o processamento de informação em si não configura ciência.

Todavia, a computação funciona para necessidades humanas e que são associadas com acontecimentos empíricos. Dessa forma, parte das atuações se enquadram como ciência aplicada sendo parte das atuações contribuições de arte, ciência empírica comparada ou embasada teoricamente.

\subsection{A computação é uma ciência experimental?}

Uma ciência experimental utiliza da experimentação para validar usas hipóteses ou teorias. As hipóteses são verificadas pelo \gls{MetodoExperimental} cuja as variáveis do experimento sejam controladas, assim como o fenômeno causado para estudo.

A solução de problemas na computação é feita com o consumo de dados de entrada para o processamento através de algoritmos, então ocorre a produção da saída como resultado. Também, é evidente o controle sistemático das variáveis definidas no processo, assim como toda construção lógica para solução da temática. Com isso é razoável a similaridade e compatibilidade da computação como uma ciência experimental

\subsection{Similaridade da computação com os componentes do Experimento Científico}.

Não só a congruência destes aspectos, mas também é importante a análise de existência de cada componente do experimento científico nos atuações profissionais em computação. Os Componentes Científico são descritos nos itens a seguir e serão posteriormente chamados de \texttt{CC}:

\begin{itemize}
    \item 1º - Uma Classe de Fenômenos de Interesse;
    \item 2º - Hipóteses e Modelos;
    \item 3º - Causa e Efeito;
    \item 4º - Procedimento Repetíveis;
    \item 5º - Demonstração, Manipulação e Controle de Fatores;
    \item 6º - Coleta de Dados;
    \item 7º - Análise Lógica;
\end{itemize}

Utilizando as implementações de software, hardware e firmware como um conjunto de atividades de computação. Para início de resposta temos Sommerville definindo que qualquer produção de software deve passar por 4 fases essenciais: 

\begin{itemize}
    \item Especificação do software: Fase de idealização do projeto de software. Tipicamente, a definição dos fenômenos de interesse (1º \texttt{CC}).
    Também, é notável a busca por modelos existentes que possam ajudar na solução, juntamente, com as hipóteses de solução que serão propostas pelos responsáveis do projeto (2º \texttt{CC}).
    \item Projeto e implementação de software:
    
    A fase de implementação transita por diversos componentes. Com a produção de um sistema informatizado está agregado a execução de algoritmos para seu funcionamento. Atualmente, são utilizados um gama numerosa de métodos escritos em códigos, porém se concentram em determinísticos, probabilísticos ou uma combinação de ambos. Esses códigos são característicos por gerarem saídas após o consumo de entradas do mundo real, logo estabelecendo uma causa e efeito (3° \texttt{CC}). 
    
    Nessa fase também é característico a produção de artefatos que muitas vezes não são científicos. Documentações, abstrações de processo e escolhas arbitrárias não são associadas ao método científico. Mesmo não sendo puramente científico esses artefatos, muitas vezes, são construídos para facilitar o entendimento da complexa relação de múltiplos programas trabalhando em conjunto.
    
    Abstraindo a parte que facilita a construção humana no ambiente computacional pode-se afirmar que o resultado esperado da construção é o total domínio da solução proposta com manipulação/controle das variáveis trabalhadas (5º \texttt{CC}). 
    
    \item Validação de software;
    
    Uma etapa imprescindível é a verificação dos requisitos sobre o trabalho construído. Expondo a implementação contra testes rigorosos é comprovado por experimentação a validade e \gls{ReprodutibilidadeCientifica} (4° \texttt{CC}).
    
    Vale ressaltar a análise de comportamento sobre diversas perspectivas que garantem bom funcionamento do software ou não, segundo os postulados teóricos que guiaram o projeto. Para realizar essa análise, sem dúvidas, é necessário a coleta de dados produzido pelo sistema e a conferência de validade do funcionamento. (6° \texttt{CC}) 
    
    \item Evolução de software;
    
    Concluindo, para questões de progressão no projeto de software é necessário o conhecimento do funcionamento atual, também do objetivo desejado. A evolução deve ter alguma métrica de base, caso contrário não é possível distinguir evolução ou regressão do software. Portanto, é necessário uma análise lógica para discernir, criticar e evoluir o produto de software (7° \texttt{CC}).
    

\subsection{Conclusão}
glsss
\end{itemize}


