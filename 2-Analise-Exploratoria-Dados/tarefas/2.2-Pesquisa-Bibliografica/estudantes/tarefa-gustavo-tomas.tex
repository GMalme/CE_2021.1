%% Keywords usadas: (graphic processing unit or GPU) and (lighting or light or shadow*)

\chapter{Análise Bibliográfica sobre , por Gustavo Tomás}

\section{Planejamento do estudo}

O objetivo do trabalho é analisar o impacto das GPUs (Graphic Processing Units) no processamento e simulação da luz. Para isso, foram utilizadas as ferramentas Bibliometrix e Biblioshiny.

\subsection{O que já existe de pesquisa bibliométrica sobre esse tema?}

\subsection{Limitações} O exercício relatado foi feito em cerca de uma semana, entre os dias 02 e 10 de fevereiro de 2022 e a base de dados utilizada foi Web Of Science (WoS).

\section{Coleta de dados}

A coleta de dados foi feita usando o WoS no dia 03/02/2022, por meio do Portal de Periódicos da CAPES. Foram feitas buscas nas coleções Science Citation Index Expanded (SCI-EXPANDED) e Social Sciences Citation Index (SSCI), mas com o foco em registros relativos a área de ciências naturais e exatas. A busca utilizada foi a seguinte:

\begin{verbatim}
(graphic processing unit or GPU) and (lighting or light or shadow*)
\end{verbatim}

Essa busca consiste em dois termos, sendo que o primeiro é composto pela GPU (por extenso ou pela sigla) e o segundo pelas palavras luz ou iluminação ou sombra(s). Dessa forma, foram encontrados 1311 registros, sendo que nesse trabalho foram utilizados os primeiros 1000 registros, disponíveis em \ref{}.

\section{Análise dos dados}

\subsection{Filtragem de registros}
Antes da análise, foram aplicados filtros aos registros, de forma que apenas registros do tipo \textit{article}, de qualquer ano e com qualquer número de citações, fossem analisados. O resultado consiste em 850/1000 registros originais.

\subsection{Análise descritiva do dataset}

As informações mais gerais sobre o \textit{dataset} MASSA@jhcf são as seguintes:
\begin{description}
    \item [\textit{Timespan}] Os artigos que atenderam aos critérios de busca e filtragem foram publicados a partir de 1990, até 2021. Ou seja, não foram encontrados registros entre 1945 e 1989.
    \item [\textit{Sources (Journals, Books, etc)}] São 2.319 fontes de informação que publicaram os documentos recuperados no dataset MASSA@jhcf. Ou seja, em média, cada \textit{scientific journal} publicou $5.787/2.319=2,5$ artigos. \footnote{Note que a média, enquanto medida de tendência central, pode não ser a que melhor reflete a tendência a quantidade de artigos publicados por revista.}
    \item [\textit{Average years from publication}] A média do tempo de publicação dos artigos no dataset MASSA@jhcf é de 7,36 anos.
    \item [\textit{Average citations per documents}] Cada artigo no dataset MASSA@jhcf foi citado, em média 20,7 vezes\footnote{Note que a média, enquanto medida de tendência central, pode não ser a que melhor reflete a tendência de  citações a artigos.}.
    \item [\textit{Average citations per year per doc}] Após publicado, cada um dos 5.787 artigos do dataset MASSA@jhcf  foi citado 2,262 vezes por ano, em média.
    \item [\textit{References}] O dataset MASSA@jhcf contém 201.464 referências citadas (tags CR).
    \item [\textit{Keywords Plus (ID)}] 13.735 distintas palavras-chave do tipo Keywords Plus (ID)\footnote{\textit{KeyWords Plus} são ``termos de índice gerados automaticamente a partir dos títulos de artigos citados. Os termos do KeyWords Plus devem aparecer mais de uma vez na bibliografia e são ordenados de frases com várias palavras a termos únicos. O KeyWords Plus aumenta o número de resultados tradicional de palavras-chave ou títulos.'' Fonte: \url{https://images.webofknowledge.com/WOKRS410B4/help/pt_BR/WOS/hp_full_record.html}} foram encontradas no dataset MASSA@jhcf. 
    \item [\textit{Author's Keywords (DE)}] 15.704 distintas palavras-chave indicadas pelos autores foram encontradas no \textit{dataset}.
    \item [\textit{Authors}] 19.410 distintos nomes de autores foram encontrados no dataset\footnote{Um mesmo autor pode ter uma ou mais diferentes grafias no dataset, e serão reconhecidos dois ou mais autores diferentes, embora de fato sejam apenas um. Isso significa que a quantidade de \textbf{nomes de autores} equivale à quantidade de \textbf{autores}. Adicionalmente, é possível que distintos autores sejam reconhecidos com o mesmo nome, isso é, que sejam homônimos. Ou seja, o dataset em geral conterá erros de contagem na quantidade de autores reais.}.
    \item [\textit{Author Appearances}] Os 19.410 distintos (nomes de) autores foram encontrados 23.470 vezes, como autores de artigos.
    \item [\textit{Authors of single-authored documents}] Dentre os 19.410 distintos (nomes de) autores encontrados, 375 deles editaram artigos individualmente, isso é, sem co-autores.
    \item [\textit{Authors of multi-authored documents}] Dentre os 19.410 distintos (nomes de) autores encontrados, 19.035 deles editaram artigos com um ou mais co-autores"
    \item [\textit{Single-authored documents}] Dentre os 5.787 documentos presentes no dataset MASSA, 409 foram escritos por um único autor, e os 5.378 restantes foram elaborados em co-autoria.
    \item [\textit{Documents per Author}] Dentre os 19.410 distintos (nomes de) autores, cada um publicou em média 0,298 artigos.
    \item [\textit{Authors per Document}] Cada um dos 5.787 documentos presentes no dataset MASSA foi autorado com 3,35 autores em média ($19.410 / 5.787 = 3,35$).
    \item [\textit{Co-Authors per Documents}] As 23.470 aparições de (nomes de) autores (``Author Appearances''), sem distribuem, em média 4,06 vezes para os 5.787 documentos do dataset MASSA@jhcf.
    \item [\textit{Collaboration Index}] Os 19.035 (nomes de) autores que editaram artigos com um ou mais co-autores, colaboraram em media 3,54 vezes para editar os 5.378 artigos elaborados em co-autoria, gerando, assim, um índice de colaboração 3,54. 
\end{description}

\subsection{Evolução da Produção Científica}

\subsection{Interpretação do Crescimento}

\subsection{Evolução das Citações}

\subsection{Interpretação das Citações}

\subsection{\textit{Three-Field Plots (Sankey diagram)}}

\subsection{Interpretação da figura}

\subsection{Análises Bibliométricas: Fontes de Informação}

\subsection{Análises Bibliométricas: Autores}

\subsection{Análises Bibliométricas: Documentos}

