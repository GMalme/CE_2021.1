\section{Respostas de Stefano Luppi Spósito}

\subsection{A computação é uma atividade científica?}
Podemos definir a computação como a busca de soluções a partir de entradas, de forma a se obter saídas que são processadas por algoritmos. A computação não visa a produção de conhecimentos científicos, uma vez que a computação nada mais é do que uma ferramenta que pode ou não ser utilizada na produção destes conhecimentos, uma vez que ela pode reunir dados ou ajudar na confirmação de hipóteses.
Portanto, podemos afirmar que a computação não pode ser considerada como uma atividade científica.
\subsection{A computação é uma ciência experimental?}

\subsection{A computação é uma ciência empírica?}