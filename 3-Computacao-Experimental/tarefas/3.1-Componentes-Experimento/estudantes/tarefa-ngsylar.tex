\section{Respostas de Gabriel Rocha Fontenele\label{tarefa-ngsylar-componentes-experimento}}

\subsection{A computação como atividade científica}

A \gls{Computacao} é uma atividade que se aproxima da atividade científica no que concerne a uma série de passos para resolver um problema. Os pontos comuns entre a computação e a atividade científica vão desde:

\begin{itemize}
\item \textbf{Ideação de um modelo}, como por exemplo, buscar meios, teorias, técnicas ou ferramentas para dar solução a um dado problema ou utilidade a algum tipo de operação - ou seja, computacionar um problema da realidade;
\item \textbf{planejamento} e elaboração de um sistema baseado no modelo, como por exemplo a fase de planejamento e codificação de um software ou construção de um hardware;
\item \textbf{experimentação}, representando a fase de testes da solução ou protótipo produzido, seja por parte dos desenvolvedores ou de usuários selecionados para a realização de testes - podendo culminar numa série de ajustes e modificações do protótipo durante este período;
\item e \textbf{análise dos dados}, que representa o período de uma possível validação do protótipo e posterior lançamento do produto final computacional.
\end{itemize}

\subsection{Os componentes da experimentação}

A computação basea-se na inserção de dados de entrada e obtenção de dados de saída que, após interpretados ou analisados, fornecem resultados e informação relevantes à solução do problema proposto. Dessa forma, a computação pode ser definida como uma atividade empírica, ou seja, basea-se na coleta de dados, assim como o \gls{EmpirismoCientifico}. Sendo assim, é fácil observar que a atividade de um \gls{Cientista} da computação está fortemente atrelada a \gls{Experimentacao}.

A experimentação por sua vez consiste na realização de experimentos que podem ser divididos em partes ou componentes distintos, mas complementares entre si, sendo eles:

\begin{itemize}
\item a escolha de uma classe de \gls{fenomeno}s de interesse;
\item a formulação de hipóteses e modelos;
\item observação de causa e efeito;
\item criação ou reprodução de procedimento repetíveis;
\item manipulação e controle de fatores;
\item coleta de dados;
\item e análise lógica dos resultados.
\end{itemize}

Ora, o cientista é o indivíduo responsável pela produção da \gls{Ciencia}, que está intrinsicamente baseada na produção de conhecimento através da experimentação. Sendo assim, o cientista da computação se utiliza de todos os componenetes da experimentação para produzir um novo conhecimento científico na área da computação.

\subsection{Como o cientista da computação usa os componentes da experimentação}

O cientista da computação pode construir a base de um estudo sobre experimentos ao utilizar, por exemplo, técnicas de \gls{Simulacao} de um contexto no qual se localiza os fenomenos do seu interesse. Assim, o cientista formula hipóteses que dão explicações para o comportamento dos fenômenos observados e elabora um modelo sobre o qual o experimento será realizado, simulando uma abstração para um contexto real.

O observação de um fenômeno real ao longo de um período de tempo determinado fornece dados sobre a estabilidade e variação ocorridos neste mesmo período. Tal comportamento se dá pela relação entre causa e efeito dos fatores envolvidos na observação do fenômeno. Determinado um fator, pode-se observar se ação dele causa um efeito em quais demais fatores, que se responderem aos estímulos do fator inicial podem produzir um efeito comportamental que representam novas ações de causa para possíveis efeito consequentes em outros fatores, possibilitando também que essas relações se tornem cíclicas ao longo do tempo. Entretanto, em um sistema real, provavelmente poucos ou nenhum fator pode ser controlado pelo cientista, além de que muitos deles podem estar ligados a outros tipos de fenômenos que não são do interesse do cientista e não o ajudarão a validar as hipóteses inicialmente criadas. Assim, o cientista usa um modelo formulado que delimita o contexto simluado a fatores que considera relevantes para a realização de experimentos.

Manipulando os fatores escolhidos dentro do ambiente simulado, o cientista dispõe de controle para que o comportamento dos agentes não seja discrepante ou fuja da proposta do que se quer observar, o que faz com que os procedimentos realizados possam ser repetidos por outros indivíduos interessados na produção cinetífica ou experimentação. Por fim, o cientista coleta e analisa os dados resultantes de seu experimento, validando ou refutanto as hipótese iniciais ou formulando novas hipóteses para o comportamento dos fenômenos observados.
