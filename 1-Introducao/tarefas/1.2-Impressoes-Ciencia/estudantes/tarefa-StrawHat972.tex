\subsection{Minhas impressões iniciais sobre a ciência, por João Víctor Siqueira de Araujo}

Como podemos definir o que é \gls{Ciencia}? Essa pergunta a priori parece bastante simples, mas na realidade trata-se de uma questão muito mais complexa. É difícil definir um conceito para Ciência, o que se deve à inexistência de um consenso sobre uma definição exata, o que temos na verdade são noções que variam ao longo do tempo e do espaço \citep{schwartzman_ciencia_1984}. Mesmo com essa dificuldade de uma definição exata, podemos enxergar a Ciência como um \gls{fenomeno} de caráter humano e principalmente social, em virtude que a Ciência é feita por pessoas, os \gls{Cientista}s, e essas pessoas interagem socialmente através da \gls{ComunidadeCientifica}, a partir da qual se instaurou uma \gls{Cultura} que preza pelo desenvolvimento do conhecimento a partir de observações empíricas e de análise racional. Dessa forma, para se entender melhor o que é Ciência, é imprescindível a compreensão que a mesma trata-se de um fato social e, por isso, não podemos desatrelar seus aspectos sociais ao próprio conceito de Ciência, visto que esta é intrinsecamente uma atividade social.