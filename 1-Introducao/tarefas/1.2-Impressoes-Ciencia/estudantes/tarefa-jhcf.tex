\subsection{Minhas impressões iniciais sobre a ciência, por Jorge Henrique Cabral Fernandes}

A ciência da computação \citep{baldwin_three-fold_1994}, uma ciência específica dentre as várias existentes, apresenta-se muito relacionada com a filosofia da ciência \citep{floridi_blackwell_2004}, ou como uma meta-ciência (A filosofia é um campo do pensamento humano que é base para a ciência). Um exemplo que comprova essa versatilidade - ou até mesmo universalidade - da Ciência da Computação é o emprego da computação, sendo como exemplo mais concreto o uso de simulação, como método de pesquisa empírica, isso é pesquisa baseada em coleta e análise de dados \citep{marcolino_a_2014,tedre_experiments_2014}. A versatilidade do computador, decorrente do software, inclusive de simulação, gera uma plataforma empírica - o computador é prodigioso na coleta, cálculo e geração de dado - para o desenho e execução de quase todo tipo de experimento científico em quaisquer dos outros campos do conhecimento. Situações mais específicas podem ser vistas no campo da \gls{EA}.
