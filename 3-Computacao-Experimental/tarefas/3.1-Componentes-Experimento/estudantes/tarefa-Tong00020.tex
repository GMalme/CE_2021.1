\section{Respostas de Tong Zhou}

\subsection{A computação é uma atividade científica?}

Primeiramente devemos definir o que é uma atividade científica. Ela é um método de investigação em que, ao identificar um problema, são feitas observações e experimentos com dados relevantes para construir ou testar hipóteses o resolvam.

Agora definimos computação, que é o estudo, a busca e a experimentação de processos algorítmicos para desenvolvimento de hardware e software que solucionem um problema. 

   
Por fim, chegamos a conclusão de que a definição de computação se enquadra na definição de atividade científica. 

\subsection{A computação é uma ciência experimental?}

A computação é considerada experimental quando usada como uma \gls{dc}, que envolve a criação, experimentação e análise de artefatos computacionais. Nele são feitos observações de dados e  experimentos para a implementação fenômenos computacionais.


\subsection{A computação é uma ciência empírica?}

A computação é considerada empírica quando estuda os acontecimentos por meio da observação e da experimentação, um exemplo é quando tenta modelar a \gls{cg} com experimentos computacionais. 


Uma área da computação que é considerada ciência empírica é a Inteligência artificial
