\newglossaryentry{RIS}{
name={RIS},
description={
RIS é um formato de arquivo criado pela Research Information Systems. Tem suporte em vários serviços, como bibliotecas digitais (Ex: IEEE Xplore) e aplicativos de catálogo (Ex: Zotero). 
A estrutura de arquivos com essa extensão tem o seguinte formato:
\begin{lstlisting}
    ID1 - Informação1
    ID2 - Informação2
    ...
\end{lstlisting}
    Em que esses "IDs" são identificadores únicos, que denotam o contexto da informação
Fonte: \cite{wikipedia_ris_2017}.
}
}

\newglossaryentry{ModeloCientifico}{
name={Modelo Científico},
description={
    São ferramentas utilizadas para tornar uma característica específica do mundo mais fácil de entender ou lidar, fazendo referências a conhecimentos existentes e mais aceitos.
Fonte: \cite{wikipedia_scientific_2022}.
}
}