\subsection{Minhas impressões iniciais sobre a ciência, por Gabriel Rocha Fontenele}

A \gls{Ciencia} ampara-se fundamentalmente na filosofia como base para servir o propósito de descrever, explicar ou prever \gls{fenomeno}s do universo \citep{floridi_blackwell_2004}. Tal filosofia abrange a sub-área da Filosofia da Ciência, que estuda aspectos mais subjetivos da ciência ao tentar responder perguntas como "o que é o cientista?", "como os fenômenos devem ser observados?" ou "como deve se estruturar um método para produzir conhecimento científico?".

Tais perguntas concebem conceitos e definições que a ciência se apropria para estudar os fenômenos de forma objetiva -, com perguntas sobre como se manifestam, quais o fatores desencadeadores, o que provocam e explicações sobre os resultados obtidos. Para isso, a ciência faz uso do \gls{EmpirismoCientifico}: os dados coletados produzirão informações relevantes e possíveis soluções para um dado problema inserido no contexto do fenômeno.