\section{Tarefa: Pesquisa Bibliométrica}

A tarefa vale 20 pontos, e consiste em produzir individualmente uma análise bibliométrica inicial, abordando um tema de um dos submodelos de produção da ciência escolhidos. Escolha um dos temas que te interessa. 

A análise bibliométrica, o produto da tarefa, deve estar apresentada num capítulo da parte \ref{part:analises:bibliometricas}, no relatório da turma no Overleaf, no diretório ``2-Analise-Exploratoria-Dados/tarefas/2.2-Pesquisa-Bibliografica/estudantes/'', em um arquivo de nome "tarefa-\githubusername.tex. Ver o exemplo no arquivo ``tarefa-jhcf.tex''

O dataset de análise bibliométrica deve conter, minimamente, 200 registros.

A análise precisa ser descrita em cinco etapas, deve seguir as todas orientações feitas em \ref{metodo:analise:bibliografica}, e o detalhamento proposto por \citet{aria_bibliometrix_2017}:
\begin{enumerate}
    \item Study design (planejamento do estudo);

    \item  Data collection (coleta de dados);

    \item Data analysis (análise dos dados);

    \item Data visualization (visualização dos dados representados de forma gráfica, em vários formatos, vários tipos de diagrama);

    \item  Interpretation (interpretação, traçar conclusões, reflexões, sugestões de aprofundamento).
\end{enumerate}

A análise deve ser entregue na forma de texto e figuras montadas no arquivo "2-Analise-Exploratoria-Dados/tarefas/2.1-Pesquisa-Bibliografica/estudantes/tarefa-<githubusername>.tex".

É necessário fazer o input do arquivo acima em "2-Analise-Exploratoria-Dados/tarefas/2.1-Pesquisa-Bibliografica/estudantes/main.tex".

A entrega da tarefa é concluída quando houver um commit no github, originado do Overleaf, feita diretamente pelo usuário estudante, deixando um comentário: 
"Tarefa 2.1 Concluída por " + nome completo do estudante + " no commit de número:" + número do commit no repositório git.

Na edição do \LaTeX~ deve-se atentar aos seguintes aspectos:
\begin{enumerate}
    \item Todos os dados, inclusive as imagens, usadas na produção da análise, deve estar inseridos no diretório de experimentos individuais do estudante, no experimento de nome ``Analise-Bibliometrica"
    \item Todas as figuras e gráficos inseridos na análise devem ser individualmente rotulados com label, sem conflitar com os labels já existentes, devem ter um título (caption) descritivo do que apresenta a figura e o nome do dataset usado, e também a figura/gráfico deve ser explicitamente descritas e citadas, usando referencia (ref);
    \item As figuras deve ser automaticamente dimensionadas, e eventualmente rotacionadas,  para caber na largura e (ou) altura do texto. 
\end{enumerate}