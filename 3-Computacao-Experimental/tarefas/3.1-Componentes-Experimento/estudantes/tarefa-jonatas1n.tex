\section{Respostas de Jônatas Gomes Barbosa da Silva}

\subsection{O que é a computação?}
A \gls{computacao} pode ser definida como a busca de solução para um problema a partir de entradas (inputs), de forma a obter resultados (outputs) depois de processada a informação através de um algoritmo \citet{wikipedia_computacao_2022}. É um subcampo da ciência da computação e da matemática, desenvolvida durante milhares de anos.

\subsection{A computação é uma atividade científica?}
De acordo com o Wikipedia, \gls{ciencia} se refere ao sistema de adquirir conhecimento baseado no método científico, bem como ao corpo organizado de conhecimento conseguindo através de tais pesquisas. \cite{wikipedia_ciencia_2022}
    

\subsection{Os componentes de um experimento Científico}
Um \gls{experimento} é um procedimento para apoiar ou refutar uma hipótese. Esse procedimento deve seguir determinadas etapas para aferir sua eficiência. Os componentes de um experimento científico são:
\begin{itemize}
    \item Uma Classe de Fenômenos de Interesse
    \item Hipóteses e Modelos
    \item Causa e Efeito
    \item Procedimento Repetíveis
    \item Demonstração, Manipulação e Controle de Fatores
    \item Coleta de Dados
    \item Análise Lógica
\end{itemize}
