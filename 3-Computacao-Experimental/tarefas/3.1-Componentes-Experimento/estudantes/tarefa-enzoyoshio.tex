\section{Respostas de Enzo Yoshio Niho}

\subsection{A computação é uma atividade científica? Justifique. }

Sim, com as várias fontes bibliográficas vistas acerca da \gls{Ciencia} e da computação, pode-se afirmar que é uma atividade científica. Como visto em \citep{fernandes_consideracoes_2021}, artigo ao qual nos basearemos sobre o que é ciência, diz que para haver ciência devemos ter alguns pré-requisitos como \gls{ComunidadeCientifica}, entre outros princípios e fenômenos. É importante também que seja adotada uma \gls{Metodologia} durante a ciência para que ela não se perca no meio do caminho durante o processo de produção de ciência, logo podemos perceber essa correlação com a computação também pois precisamos de definir certas coisas antes de começar um projeto em computação, há também outras semelhanças com a ciência, logo é uma atividade científica.

\subsection{A computação é uma ciência experimental? Justifique. }

Sim, a computação é muito fundamentada na experimentação, no processo de \gls{Causalidade} e efeito, logo a computação é muito experimental, podendo nos basear bastante em muitas experiências e verificar como será o resultado pelo efeito dessa causa. Em computação podemos fazer muitos experimentos e simulações para conseguir prever vários resultados em várias outras áreas ou a sobre a própria área da computação, logo a experimentação é muito importante na área da computação, o processo de automatização provido pelos computadores facilita o uso de experimentos e testes que não seriam muito viáveis caso tivessem que ser feitos a mão por atividade humana. Então a computação é excelente para a ciência experimental, para poder experimentar e simular várias operações ou cenários dado um contexto.

\subsection{A computação é uma ciência empírica? Justifique. }

A computação é um ciência empírica, pois assim como a ciência experimental, a ciência empírica é muito material, tem que ser possível visualizar como se manifesta essa ciência. Em computação podemos fazer isso e logo gerar uma saída disponível, dependendo do que você fornece como hipótese, ele irá gerar uma saída, logo é uma ciência empírica. A necessidade de provar a \gls{Teoria}, usando as hipóteses e as saídas de um computador, faz com que a computação seja uma ciência empírica.