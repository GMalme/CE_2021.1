\chapter{Análise Bibliográfica sobre Análise de Dados integradas em Nuvem para auxílio de Fenômenos Sociais, por Guilherme Oliveira Loiola\label{chap:bibliometria:guioliunb}}

\section{Planejamento do estudo\label{MASSA:coleta}}
O planejamento o  desenho do estudo deve descrever as motivações, questões de interesse, escopo, limitações e objetivos do trabalho.

O planejamento do estudo deve motivar o tema escolhido e o interesse do autor.

No caso do meu trabalho, as perguntas que o nortearam foram:
\begin{itemize}
    \item Qual a base de conhecimentos científicos produzida com análises de dados com auxílio da computação em nuvem voltada à compreensão de fenômenos sociais? 
    \item Como a capacidade de processamento da nuvem está proporcionando análises sociais mais precisas e eficientes? 
    \item Quais são as principais aplicações de interesse da comunidade científica diante da evolução tecnológica dos últimos anos?
    \item Quais os principais métodos de análise e ferramentas de aplicação nos problemas atuais?
\end{itemize}

\subsection{O que já existe de pesquisa bibliométrica sobre esse tema?}

\cite{ma_method_2020} realizou uma pesquisa que visava combinar a capacidade de processamento da computação em nuvem com os algoritmos de inteligência artificial para auxiliar na identificação de padrões e criação de metodologias de tratamento para dificuldades enfrentadas na psiquiatria.

Assim como \cite{archenaa_interactive_2016} que promoveu um estudo enfrentando desafios da saúde utilizando computação em nuvem (utilizando a tecnologia \textit{Apache Spark}).


\subsection{Uso do Bibliometrix e Biblioshiny}
Serão usadas a ferramenta e o \textit{workflow} proposto pelos autores do pacote Bibliometrix.

\subsection{Limitações} O exercício relatado foi feito em uma semana, envolvendo de um autor. As bases de dados foram buscadas na plataforma WoS,


\section{Coleta de dados\label{MASSA:coleta}}

A coleta de dados feita usando o WoS no dia 08 de janeiro de 2022, acessado por meio do Portal de Periódicos da CAPES.

Foram feitas buscas na coleções \textbf{Web of Science - Coleção Principal (Clarivate Analytics) } que possui uma vasta bibliografia sobre a área de tecnologia. Através do refinamento nessa grande base foi obtido a base de dados composta por duas \querys\ de pesquisa. A primeira coleta é mais simples com as palavras de maior relevância. Já a segunda, inclui alguns termos mais abrangentes e ponderando mais a questão de influência na sociedade. 

Ao todo foi agrupado um total de 2906 registros para a análise bibliométrica. Sendo 1000 registros originados da primeira \query\ e a segunda query contando com 1906 registros.

Além disso, foi realizado um refinamento das \query\ originais para eliminar algumas categorias de registros que fugiam do interesse do estudo. Temos os filtros: de linguagem para inglês (linguagem amplamente utilizada), restrição para o período de 2014 até 2022, documentos do tipo artigo/artigo de conferência e áreas de pesquisa ligadas à computação, engenharia e telecomunicações.

\subsection{Query de Busca}

Foram utilizadas duas \querys\ de busca que estão ilustradas a seguir: %\ref{fig:}.

\lstinputlisting[numbers=left,basicstyle=\normalsize\ttfamily,caption={\query\  de busca sobre análises de dados que utilizam computação em nuvem para impactar a sociedade positivamente},label=query20210803-2]
{experiments/guioliunb/AnaliseBibliometrica/SocialBigDataAnalysis/WoS-dataset1/query1.txt}

\lstinputlisting[numbers=left,basicstyle=\normalsize\ttfamily,caption={\query\  de busca sobre análises de dados que utilizam computação em nuvem para impactar a sociedade positivamente},label=query1002]
{experiments/guioliunb/AnaliseBibliometrica/SocialBigDataAnalysis/WoS-dataset2/query2.txt}

\subsubsection{Explicação para os termos de busca usados\label{MASSA:query}}

As \querys\ utilizadas foram criadas com intuitos semelhantes. A primeira mais restrita pelas palavras-chaves ligadas com tecnologia e separando a expressão de \texttt{big data} em \texttt{big} e \texttt{data} podendo ser duas possibilidades de ocorrência nos artigos. E a segunda \query\ reforça o intuito que a pesquisa tenha aplicação social e possibilitando também que o assunto  \texttt{database} ingresse ao conjunto.

A busca consistiu basicamente de quatro principais assuntos (\textit{Big Data}, \textit{Cloud} ,\textit{Analytics}, \textit{Social}) aplicados à busca por tópico (O termo de busca pode aparecer no Título, no Abstract, na Author Keywords, ou nas Keywords Plus da referência).

Com a intersecção das áreas de competência sendo filtradas pela pesquisa de registros, juntamente, na união dos conjuntos de registros de ambas \querys/ obtivemos artigos que focavam bem no assunto proposto.

\subsection{Registros recuperados}

Os 2.906 registros obtidos como resultado da busca encontram-se no projeto Overleaf da ferramenta utilizada na tarefa Computação Experimental - Turma A em \url{https://www.overleaf.com/project/618e9b4b0db7234d6d9fbfc0} . 

Foram utilizadas as opções \textit{Exportar registros para arquivo de texto sem formatação} e \textit{export full record / Gravar Conteúdo: Seleção personalizada, com todos os 29 campos disponíveis, inclusive referências citadas} no WoS, para que as citações também fosse usadas em análises da citações (estrutura intelectual do conhecimento). Os 2906 registros foram recuperados em dois blocos com mil registros da primeira consulta (1000) e quatro blocos da segunda consulta (2906), assim totalizando os 2906 registro  separados de até 500 registros por vez (1-500, 501-1000, 1001-1500, 1501-2000, ...).

A listagem \ref{record20210803-2} apresenta as 66 linhas de um registro no formato RIS, referentes a um artigo recuperado da Web of Science. Cada um dos campos de um registro é marcado por um código de dois caracteres, nas colunas 1 e 2 de cada linha. Se a coluna está em branco repete-se o mesmo campo da linha anterior.

Alguns campos específicos serão comentados a seguir:
\begin{description}
    \item [PT - Publication Type] indica o tipo da publicação, no caso específico um artigo de \textit{journal} (J);
    \item [AU - Author] Nome de um autor;
    \item [AF - Author Full Name] Nome completo de um autor;
    \item [TI - Title] Título da publicação;
    \item [SO - Source] Nome da revista;
    \item [DE - Descriptor] Palavras-chave;
    \item [AB - Abstract] Resumo;
    \item [CR - Cited Referente] Cada uma das referências citadas no artigo;
    \item [TC - Times Cited] Quantidade de vezes que esse artigo foi globalmente citado;
    \item [PY - Publication Year] Ano de publicação;
    \item [VL - Volume, IS - Issue] Volume e número onde o artigo foi publicado, na revista;
    \item [BP - Begin page, EP - End page] Páginas inicial e final do artigo dentro do volume e número da revista;
    \item [DI - Digital Object Identifier] Identificador único do artigo no sistema \url{http://doi.org};
    \item [DA - Date of Acquisition] Data em que o registro foi obtido da WoS;
    \item [ER - End of Record] Fim do registro.
\end{description}

\lstinputlisting[language={},numbers=left,basicstyle=\tiny\ttfamily,caption={Exemplo de um registro recuperado no formato RIS, sobre o tema Cloud-based Big Data Analysis},label=record20210803-2]
{experiments/guioliunb/AnaliseBibliometrica/SocialBigDataAnalysis/record.txt}

\section{Análise dos dados}

\subsection{Filtragem de registros}
Antes da análise, é possível aplicar filtros sobre os registros obtidos. Anteriormente, a primeira buscava contava com 1.343 registros e a segunda com 3.360 registros. Em ambas foi aplicado a mesma combinação de registros.

Foi aplicado um filtro ao \dataset\ inicial, com 4.703 registros, que continham pŕevias de artigos, artigos de conferência, capítulos de livro etc. Foram mantidos apenas os registros de artigos publicados em revistas científicas\footnote{A suposição é que que o conhecimento de maior qualidade sobre o tema está nas publicações em revistas.}. Após a aplicação desse filtro, 2.906 registros foram mantidos no \dataset, que será doravante chamado SocialBigDataAnalysis/Artigos, ou SBDAA@guioliunb.

\subsection{Análise descritiva do \dataset\   SBDAA@guioliunb}

A análise bibliométrica descritiva faz uma descrição inicial do \dataset\  . Para explicação detalhada de como são calculadas as diversas taxas geradas pelo Bibliometrix veja a documentação do \textit{package} a partir da página \url{https://cran.r-project.org/web/packages/bibliometrix/index.html}. A análise bibliométrica descritiva é gerada pela função \texttt{biblioAnalysis}.

As informações mais gerais sobre o \dataset\   SBDAA@guioliunb são as seguintes:
\begin{description}
    \item [\textit{Timespan}] Os artigos que atenderam aos critérios de busca e filtragem foram publicados a partir de 2014, até 2022. Registros anteriores não formavam um conjunto de dados expressivo.
    \item [\textit{Sources (Journals, Books, etc)}] São 871 fontes de informação que publicaram os documentos recuperados no \dataset\   SBDAA@guioliunb. Ou seja, em média, cada \textit{scientific article} publicou $2.906/871=2,5$ artigos. \footnote{Note que a média, enquanto medida de tendência central, pode não ser a que melhor reflete a tendência a quantidade de artigos publicados por revista.}
    \item [\textit{Average years from publication}] A média do tempo de publicação dos artigos no \dataset\   SBDAA@guioliunb é de 4.4 anos.
    \item [\textit{Average citations per documents}] Cada artigo no \dataset\   SBDAA@guioliunb foi citado, em média 8,896 vezes\footnote{Note que a média, enquanto medida de tendência central, pode não ser a que melhor reflete a tendência de  citações a artigos.}.
    \item [\textit{Average citations per year per doc}] Após publicado, cada um dos 2.906 artigos do \dataset\   SBDAA@guioliunb  foi citado 1,575 vezes por ano, em média.
    \item [\textit{References}] O \dataset\   SBDAA@guioliunb contém 47.165 referências citadas (tags CR).
    \item [\textit{Keywords Plus (ID)}] 1.186 distintas palavras-chave do tipo Keywords Plus (ID)\footnote{\textit{KeyWords Plus} são ``termos de índice gerados automaticamente a partir dos títulos de artigos citados. Os termos do KeyWords Plus devem aparecer mais de uma vez na bibliografia e são ordenados de frases com várias palavras a termos únicos. O KeyWords Plus aumenta o número de resultados tradicional de palavras-chave ou títulos.'' Fonte: \url{https://images.webofknowledge.com/WOKRS410B4/help/pt_BR/WOS/hp_full_record.html}} foram encontradas no \dataset\   SBDAA@guioliunb. 
    \item [\textit{Author's Keywords (DE)}] 5.403 distintas palavras-chave indicadas pelos autores foram encontradas no \dataset\  .
    \item [\textit{Authors}] 6.106 distintos nomes de autores foram encontrados no \dataset\  \footnote{Um mesmo autor pode ter uma ou mais diferentes grafias no \dataset\  , e serão reconhecidos dois ou mais autores diferentes, embora de fato sejam apenas um. Isso significa que a quantidade de \textbf{nomes de autores} equivale à quantidade de \textbf{autores}. Adicionalmente, é possível que distintos autores sejam reconhecidos com o mesmo nome, isso é, que sejam homônimos. Ou seja, o \dataset\   em geral conterá erros de contagem na quantidade de autores reais.}.
    \item [\textit{Author Appearances}] Os 11.668 distintos (nomes de) autores foram encontrados 23.470 vezes, como autores de artigos.
    \item [\textit{Authors of single-authored documents}] Dentre os 6.106 distintos (nomes de) autores encontrados, 109 deles editaram artigos individualmente, isso é, sem co-autores.
    \item [\textit{Authors of multi-authored documents}] Dentre os 6.106 (nomes de) autores encontrados, 5.997 deles editaram artigos com um ou mais co-autores"
    \item [\textit{Single-authored documents}] Dentre os 2.906 documentos presentes no \dataset\   SBDAA@guioliunb, 175 foram escritos por um único autor, e os 2.731 restantes foram elaborados em co-autoria.
    \item [\textit{Documents per Author}] Dentre os 6.106 distintos (nomes de) autores, cada um publicou em média 0,476 artigos.
    \item [\textit{Authors per Document}] Cada um dos 2.906 documentos presentes no \dataset\  SBDAA foi autorado com 2,1 autores em média .
    \item [\textit{Co-Authors per Documents}] As 11.668 aparições de (nomes de) autores (``Author Appearances''), sem distribuem, em média 4,02 vezes para os 2.906 documentos do \dataset\  SBDAA.
    \item [\textit{Collaboration Index}] Os 6.106 (nomes de) autores que editaram artigos com um ou mais co-autores, colaboraram em media 2,2 vezes para editar os 2.906 artigos elaborados em co-autoria, gerando, assim, um índice de colaboração 2,2. 
\end{description}

\subsection{Evolução da Produção Científica}

\begin{figure}
    \centering
    \includegraphics[width=1\textwidth]{experiments/guioliunb/AnaliseBibliometrica/SocialBigDataAnalysis/anual-scientific-production.JPG}
    \caption{Evolução da produção científica no \dataset\   SBDAA@guioliunb.}
    \label{fig:evol:anual:SBDAA@guioliunb}
\end{figure}

A figura \ref{fig:evol:anual:SBDAA@guioliunb} apresenta um declínio na produção mundial no tema de interesse, segundo o \dataset\  SBDAA@guioliunb. O ápice de produção foi alcançado no ano de 2018. Fica evidente que o ano de 2022 ainda não pode contribuir para a análise anual, pois ainda está em progresso.

\begin{figure}
    \centering
    \includegraphics[width=1\textwidth]{experiments/guioliunb/AnaliseBibliometrica/SocialBigDataAnalysis/AVG citation per year.png}
    \caption{Evolução da produção científica no \dataset\   SBDAA@guioliunb.}
    \label{fig:evol:anual:cit:SBDAA@guioliunb}
\end{figure}




Já figura \ref{fig:evol:anual:cit:SBDAA@guioliunb} mostra que o número de citações é na verdade crescente. Então segundo o podemos questionar se o tema realmente está em declínio ou em dispersando entre outras bases e autores. Um fato comum após algum tempo da iniciação da pesquisa sobre alguma área é a ampliação do conhecimento, logo, a distribuição também. Com a expressiva quantidade dos dados gerados nos desde 2018 é relevante considerar que os trabalhos tenham crescido, porém diluídos em outros temas e profissionais.

\subsection{Interpretação do Crescimento} 

A taxa de crescimento anual teve o resultado negativo no, porém os últimos dois anos apresentou grande concentração de investimentos nas áreas de \textit{big data} e \textit{cloud computing}. Dado isso, podemos questionar se a área está em declínio ou outro motivo pode ter ocasionado isso na base adquirida.

\subsection{Evolução das Citações}

Somado a isso, temos a Figura \ref{fig:evol:anual:cit:SBDAA@guioliunb} que explicita que o uso do \dataset\  SBDAA@guioliunb em citações e estudos posterior é crescente. Logo, a produção de conteúdo e a evolução da área demonstra evolução e dispersão. Podendo assim responder o questionamento sobre o declínio da área.

Outro fato, é o acúmulo de citações por ano facilitando o gráfico continuar em uma crescente.



\subsection{Interpretação das Citações}
Com o crescimento das citações fica evidente o repetição do uso nos artigos e o ganho de relevância do \dataset\ diante o tempo. Dessa forma, os novos registros bibliográficos herdam conceitos produzidos por esse bibliografia. Assim, demonstrando crescimento do interesse pela comunidade científica e aceitação de outros profissionais. Ou seja, uma tendência positiva do material produzido.
O crescimento das citações junto ao declínio de produções similares pode significar também a evolução da área para novos desafios. Assim o material registrado não é mais produzido mas utilizado para novas produções de artigo.

\subsection{\textit{Three-Field Plots (Sankey diagram)} \label{SBDAA:Sankey}}

\begin{figure}
    \centering
    \includegraphics[angle=0,width=1\textwidth]{experiments/guioliunb/AnaliseBibliometrica/SocialBigDataAnalysis/3FSE-referencesXauthorsXkeywords.png}
    \caption{Plotagem ``Três Campos'' (Sankey plot) do \dataset\ SBDAA@guioliunb: Referências, Autores, e Palavras-Chave.}
    \label{fig:SBDAA@guioliunb:ThreeFieldPlot}
\end{figure}

As \textit{Three-Field Plots (Sankey diagram)} (plotagens do tipo ``Três Campos'') apresentam afinidades entre três conjuntos de atributos agregados que ocorrem no \dataset. Uma plotagem do tipo Sankey busca mostrar os principais fluxos entre diferentes conjuntos de itens. \footnote{Para uma introdução ver \url{https://en.wikipedia.org/wiki/Sankey_diagram}. Para obter detalhes sobre a forma de geração e utilização desse gráfico, inclusive de forma interativa, veja o vídeo em \url{https://www.youtube.com/watch?v=jBb1iha6-sg}.} 



A figura \ref{fig:SBDAA@guioliunb:ThreeFieldPlot} apresenta a plotagem do tipo ``Três Campos'' do \dataset\   SBDAA@guioliunb, vinculando, ao centro, os 20 Autores mais proeminentes (AU), à esquerda, as 20 Citações mais frequentes (CR - Cited Records), e à direita, as 20 Palavras-Chave mais frequentes empregadas pelos autores.

Já a figura \ref{fig:SBDAA@guioliunb:ThreeFieldPlot2} apresenta a plotagem do tipo ``Três Campos'' do \dataset\   SBDAA@guioliunb, vinculando, ao centro, os 20 Autores mais proeminentes (AU), à esquerda, os países de origem mais recorrentes , e à direita, os títulos com maior intersecção entre os artigos.



\begin{figure}
    \centering
    \includegraphics[angle=0,width=1\textwidth]{experiments/guioliunb/AnaliseBibliometrica/SocialBigDataAnalysis/3FSE-countriesXauthorsXtitles.png}
    \caption{Plotagem ``Três Campos'' (Sankey plot) do \dataset\   SBDAA@guioliunb: 20 Países, Autores e Títulos.}
    \label{fig:SBDAA@guioliunb:ThreeFieldPlot2}
\end{figure}

\subsection{Interpretação da figura \ref{fig:SBDAA@guioliunb:ThreeFieldPlot}}



A figura \ref{fig:SBDAA@guioliunb:ThreeFieldPlot} explicita uma rede de produção de conhecimento bem conectada entre os 20 autores, ou seja, são cientistas bem ativos no tema e trabalham em contribuição para um objetivo comum.

\textit{Big Data} e \textit{Cloud Computing} são as palavras-chaves com maior relevância fazendo jus à necessidade de integração entre as duas áreas na formação da base estudada.
Também, pode-se observar o surgimento de áreas relacionadas aos dois aspectos principais como: edge computing, data mining, data analytics e outros. Esses assuntos de fato são utilizados com as duas principais \textit{keywords}, então é interessante observar suas possíveis ramificações.
Além disso, os principais autores possuem um grande coeficiente de citações feitas e recebidas. Podendo assim ser um indicador de qualidade de seus trabalhos científicos.

\subsection{Interpretação da figura \ref{fig:SBDAA@guioliunb:ThreeFieldPlot2}}

Na segunda figura \ref{fig:SBDAA@guioliunb:ThreeFieldPlot2} é razoável  observar a grande contribuição da China no \dataset\ e consequentemente entender a grande participação dos chineses nos autores mais relevantes como mostrado na \ref{fig:SBDAA@guioliunb:ThreeFieldPlot}. Acrescentado com a contribuição chinesa fica também evidente uma porcentagem razoável de EUA e Austrália.

O termo mais utilizado nos títulos é \textit{data} concordando com as palavras-chaves anteriormente analisadas. As outras categorias tem um significado acoplado com o termo \textit{data}, pois ao observar os artigos formam: modelos, métodos, \textit{frameworks},\textit{networks} e sistemas especializados no uso de \textit{data}.


\subsubsection{Autores mais relevantes\label{MASSA:Sankey:AutoresRelevantes}}

\begin{figure}
    \centering
    \includegraphics[angle=0,width=1\textwidth]{experiments/guioliunb/AnaliseBibliometrica/SocialBigDataAnalysis/most relevant authors.png}
    \caption{Autores mais relevantes do \dataset\   SBDAA@guioliunb.}
    \label{fig:SBDAA@guioliunb:relevantauthors}
\end{figure}

A figura \ref{fig:SBDAA@guioliunb:relevantauthors} confirma uma grande participação chinesa nessa área de pesquisa. Porém, deve ser levado em conta a possibilidade de nomes homônimos ou participações do mesmo contribuidor com assinatura de sobrenome diferente nos trabalhos.


\subsection{Medidas bibliométricas}

As medidas bibliométricas propriamente ditas, relativas ao \dataset\ SBDAA@guioliunb, serão exploradas nesta subseção, e são organizadas em três conjuntos:
\begin{description}
    \item [Relativas às Fontes de Informação] Uma vez que foram consideradas apenas as publicações em revistas, todas as fontes de informação mensuradas serão revistas científicas, ou \textit{journals}. As principais medidas são de impacto das fontes, mensuradas com base no número de citações que os artigos publicados nas revistas obtiveram de outras publicações, possivelmente feitas em outras fontes de informação, como outras revistas, seções de livros, artigos de conferência etc. As citações são registradas pelas organizações que fazem indexação de artigos, como a Web of Science e SCOPUS;
    \item [Relativas aos Autores] Sempre que um artigo publicado por um ou mais autores e também indexado por uma organização (Web of Science,  SCOPUS etc), é citado em um outro artigo também indexado por essa mesma organização, então é feita a anotação de uma citação ao mesmo, e o impacto potencial desse autor sobre a ciência é atestado pelo valor mais alto da citação do conjunto de seus artigos indexados. Várias métricas (índice H, G, M etc) podem ser derivadas dessa medida (quantidade de citações), e são exploradas tanto em relação aos autores como em relação às revistas onde esses artigos foram publicados;
    \item [Relativas aos Documentos] Cada citação adicional a  um documento (artigo de revista, de conferência, livro, ou  capítulo de livro) é um indicador do impacto do documento em si, que evidencia a sua importância. Além das citações, a ocorrência de palavras dentro dos documentos, inclusive ordenada pelo tempo, também produz indicadores numéricos (métricas) relevantes para analisar a importância do documento em relação a outros. 
\end{description}

Essas medidas serão apresentadas a seguir.

\subsubsection{Bibliometrias aplicadas aos documentos (Artigos científicos) no \dataset}

\paragraph{Citações globais aos artigos no \dataset}

Cada registro recuperado possui nos seus metadados informações, podendo constar a quantidade de vezes que uma citação ao mesmo foi registrada no índice do WoS (\textit{TC - Times Cited}).
A figura \ref{fig:SBDAA@guioliunb:relevantdocuments} evidência as publicações mais referenciadas. Os 4 primeiros itens tem uma dominância considerável sobre os outros, contudo é interessante observar que são dois trabalhos com referências acopladas, pois o número e título do primeiro se assemelha muito ao segundo, assim como o terceiro se assemelha ao quarto. Como confirmação ambos código DOI são igual par a par.

Os outros registros se comportam como registros individuais. Porém, vale ressaltar que uma relevante parte deles foi publicado na organização IEEE. A ordem de relevância das fontes será analisada em seção posterior.




\begin{figure}
    \centering
    \includegraphics[angle=0,width=1\textwidth]{experiments/guioliunb/AnaliseBibliometrica/SocialBigDataAnalysis/MOST GLOBAL CITED.png}
    \caption{Autores mais relevantes do \dataset\ SBDAA@guioliunb no contexto global .}
    \label{fig:SBDAA@guioliunb:relevantdocuments}
\end{figure}

Após a visitação do resumo do texto de vários dos documentos citados, percebe-se que não refletem bem o foco do \dataset, o que se justifica pelo fato de que esses documentos são os de maior citação global, e não necessariamente os que tem maior citação local ao \dataset. Dessa forma, procede-se à próxima análise.

\paragraph{Citações locais aos artigos no \dataset}

Podemos também visualizar as citações feitas apenas dentro do \dataset\. Com isso o mapeamento da relevância da citações pode ser observada internamente, ou seja, com foco nas situações segundo as cláusulas definidas na consulta.  


\begin{figure}
    \centering
    \includegraphics[angle=0,width=1\textwidth]{experiments/guioliunb/AnaliseBibliometrica/SocialBigDataAnalysis/MOST LOCAL CITED.png}
    \caption{Autores mais relevantes do \dataset\ no contexto local   SBDAA@guioliunb.}
    \label{fig:SBDAA@guioliunb:relevantdocuments}
\end{figure}


Após a apresentação das citações it

Após essa breve análise dos 25 artigos mais citados do \dataset\ ganhou-se uma confiança de que os dados são de boa qualidade quanto à representatividade do uso de simulações multiagentes para modelagem de fenômenos sociais. Os principais fenômenos sociais investigados nesses artigos são relacionados com as ideias de:
\begin{enumerate}
    \item Formação de consenso em coletivos;
    \item Influência da liderança na formação de consensos;
    \item Gestão coletiva no uso da terra, em interfaces ecológico-sociais;
    \item Modelagem do comportamento de \textit{traders} no mercado financeiro, especialmente formação de preços;
    \item Comportamento das pessoas em situações de desastre, especialmente incêndio;
    \item Difusão de inovações em comunidades, especialmente agrícolas;
    \item Migração de populações; 
    \item Idade em que as pessoas casam; e
    \item Modelagem de marketing (uma espécie de fenômeno de difusão de inovações em rede).
\end{enumerate}

\paragraph{Referências a outros documentos (artigos, capítulos de livros etc) citados pelos artigos no \dataset}

Cada registro do \dataset\ SBDAA@guioliunb contém o conjunto das referências citadas, no campo CR. A tabelas \ref{tab:MASSA2:LocalCitationsReferences} apresentada de forma sumária quais foram as 25 citações mais frequentes, sugerindo os principais documentos que contém o conhecimento mais basilar usado pelos pesquisadores nesse campo do conhecimento da simulação multiagente de fenômenos sociais. Note que os registros de muitos desses documentos poderão não estar presentes no \dataset\ SBDAA@guioliunb.

\begin{table}[htp]
    \centering
\footnotesize
\csvreader[tabular = |r|l|l|r|,
separator=semicolon
%,filter not strcmp={\csvcolii}{},
, table head = \hline\hline \# & Artigo Citado & DOI & Tot.Cit.\\ \hline\hline,
table foot = \hline\hline
]{experiments/jhcf/PesqBibliogr/SimulacaoMultiagente/WoS-20220203/Metricas/Documentos/MASSA2-Most-Local-Cited-References.csv}{Cited References=\paper, DOI=\doi,Citations=\totcit}{ \thecsvrow & {\tiny\paper} & {\tiny \doi} & \totcit}

    \caption{25 referências (artigos, capítulos de livros etc) mais citadas localmente no \dataset\ SBDAA@guioliunb.}
    \label{tab:MASSA2:LocalCitationsReferences}
\end{table}

Alguns desses registros serão brevemente comentados a seguir, acrescentando-se que os dez mais citados já foram descritos na plotagem Sankey em \ref{MASSA:Sankey:AutoresRelevantes}.

\begin{enumerate}
\item \textit{\textbf{Collective dynamics of ‘small-world networks'}} \citep{watts_collective_1998}, já descrito na plotagem Sankey em \ref{MASSA:Sankey:AutoresRelevantes} tem 25136 citações globais; 
\item \textit{\textbf{Consensus problems in networks of agents with switching topology and time-delays}} \citep{olfati-saber_consensus_2004}, já descrito na plotagem Sankey em \ref{MASSA:Sankey:AutoresRelevantes}, possui 8465 citações globais;

\item \textit{\textbf{Emergence of Scaling in Random Networks}}\citep{barabasi_emergence_1999}, já descrito na plotagem Sankey em \ref{MASSA:Sankey:AutoresRelevantes};

\item \textit{\textbf{Statistical physics of social dynamics}}\citep{castellano_statistical_2009}, já descrito na plotagem Sankey em \ref{MASSA:Sankey:AutoresRelevantes}; 
\item \textit{\textbf{Novel Type of Phase Transition in a System of Self-Driven Particles}} \citep{vicsek_novel_1995}, já descrito na plotagem Sankey em \ref{MASSA:Sankey:AutoresRelevantes}; 

\item \textit{\textbf{Flocks, herds and schools: A distributed behavioral model}} \citep{reynolds_flocks_1987}, já descrito na plotagem Sankey em \ref{MASSA:Sankey:AutoresRelevantes};

\item \textit{\textbf{Agent-based modeling: Methods and techniques for simulating human systems}} \citep{bonabeau_agent-based_2002}, já descrito na plotagem Sankey em \ref{MASSA:Sankey:AutoresRelevantes};

\item \textit{\textbf{Opinion dynamics and bounded confidence: models, analysis and simulation}} \citep{hegselmann_opinion_2002}, já descrito na plotagem Sankey em \ref{MASSA:Sankey:AutoresRelevantes}; 

\item \textit{\textbf{The ODD protocol: A review and first update}} \citep{grimm_odd_2010}, já descrito na plotagem Sankey em \ref{MASSA:Sankey:AutoresRelevantes};

\item \textit{\textbf{Consensus seeking in multiagent systems under dynamically changing interaction topologies}} \citep{ren_consensus_2005} explora novas propriedades da formação de consenso em redes de múltiplos agentes com topologia variável;


\item \textit{\textbf{Coordination of groups of mobile autonomous agents using nearest neighbor rules}} \citep{jadbabaie_coordination_2003}
apresenta modelos teóricos que embasam o já citado trabalho de simulação feito por \citet{vicsek_novel_1995}.

\item \textit{\textbf{Mixing beliefs among interacting agents}} \citep{deffuant_mixing_2000} descreve um modelo de dinâmica de opinião em grupos de agentes;

\item \textit{\textbf{NetLogo (and NetLogo User Manual)}}  \citep{wilensky_netlogo_1999} é uma página web que hospeda a descrição da linguagem de simulação multiagente chamada NetLogo;


\item \textit{\textbf{A standard protocol for describing individual-based and agent-based models}} \citep{grimm_standard_2006} introduz o protocolo ODD, para descrição de simulações multia-agentes;

\item \textit{\textbf{Growing Artificial Societies: Social Science from the Bottom Up}} \citep{epstein_growing_1996} é um livro que descreve o uso de um simulador de sistemas multi-agentes chamado Sugarscape, explorando propriedades de sistemas sociais complexos, como acumulação de riqueza e formação de cultura;

\item \textit{\textbf{Flocking for multi-agent dynamic systems: algorithms and theory}} \citep{olfati-saber_flocking_2006} apresenta um arcabouço para o projeto e análise de algoritmos de simulação voo/nado de pássaros/peixes;

\item \textit{\textbf{Simulating dynamical features of escape panic}} \citep{helbing_simulating_2000} apresentam estudos de simulação multiagente do comportamento de multidões em situação de pânico;

\item \textit{\textbf{Distributed Event-Triggered Control for Multi-Agent Systems}} \citep{dimarogonas_distributed_2012} investigam propriedades de simulações de coordenação de ações em sistemas multiagentes, baseadas no processamento distribuído de eventos;

\item \textit{\textbf{Consensus and Cooperation in Networked Multi-Agent Systems}} \citep{olfati-saber_consensus_2007}  apresenta um arcabouço teórico para análise de algoritmos de formação de consenso em sistemas multiagente;

\item \textit{\textbf{Opinion evolution in closed community}} \citep{sznajd-weron_opinion_2000} emprega simulações de um modelo magnético de física do estado sólido (Ising spin model) para analisar propriedades da formação de opiniões em comunidades sintéticas ``democráticas'' ou  ``ditatoriais'' (sistemas multiagentes);

\item \textit{\textbf{Reinforcement Learning: An Introduction}} \citep{sutton_reinforcement_2014} é um livro que apresenta as principais ideias e algoritmos da aprendizagem por reforço;

\item \textit{\textbf{Birds of a Feather: Homophily in Social Networks}} \citep{mcpherson_birds_2001} apresenta, sobre uma perspectiva sociológica, o conceito-chave da homofilia, usado em análise de redes sociais;


\item \textit{\textbf{Social force model for pedestrian dynamics}} \citep{helbing_social_1995} emprega modelos analíticos físicos e simulações computacionais, para descrever fenômenos típicos da movimentação de pedestres em espaços urbanos;

\item \textit{\textbf{Statistical mechanics of complex networks}} \citep{albert_statistical_2002}  apresenta uma ampla compilação de propriedades físicas, matemáticas, estatísticas e algorítmicas de grafos que representam redes complexas de relações entre entidades que ocorrem na natureza, em sistemas físicos, biológicos, humanos e tecnológicos;

\item \textit{\textbf{The Dissemination of Culture: A Model with Local Convergence and Global Polarization
}} \citep{axelrod_dissemination_1997} utiliza modelos de simulação baseados em agentes para esclarecer de que modo surgem heterogeneidades e homogeneidades culturais em sociedades artificiais geograficamente dispersas, que interagem por meio de múltiplos traços culturais.
\end{enumerate}

\paragraph{Espectroscopia das referências}

A técnica de espectroscopia das referências bibliográficas (``reference publication year spectroscopy'' (RPYS)) de um \dataset\cite{marx_detecting_2014} possibilita identificar as raízes históricas  de um campo de conhecimento. 

A figura \ref{fig:MASSA2-ReferenceSpectroscopy} apresenta, distribuída ao longo do tempo, a quantidade de referências citadas no \dataset\, para cada ano, bem como os desvios dessa quantidade em relação à média (em vermelho). A mais antiga das referências usadas é do ano de 1705, e não se detectam evidentes picos isolados de referências em anos específicos, que indicariam surgimento de publicações mais importantes que as sucederam. Há, entretanto, uma oscilação dos desvios na quantidade de citações, em períodos de aproximadamente seis anos, especialmente entre 1970 e 2013 (linha vermelha).

\begin{figure}
    \centering
    \includegraphics[width=1\textwidth]{experiments/jhcf/PesqBibliogr/SimulacaoMultiagente/WoS-20220203/Metricas/Documentos/MASSA2-ReferenceSpectroscopy.png}
    \caption{Espectroscopia (RPYS) completa das referências do \dataset\ SBDAA@guioliunb.}
    \label{fig:MASSA2-ReferenceSpectroscopy}
\end{figure}

Se observamos a espectroscopia do mesmo \dataset\ apenas entre os anos de 1901 a 1970, como na figura \ref{fig:MASSA2-ReferenceSpectroscopy:1901:1970}, pode-se perceber um ponto de inflexão na curva, a partir do ano de 1946, que foi o período após o término da Segunda Guerra Mundial, quando o conhecimento científico que havia sido produzido sob sigilo no período da guerra começou a se disseminar pelo mundo.

\begin{figure}
    \centering
    \includegraphics[width=1\textwidth]{experiments/jhcf/PesqBibliogr/SimulacaoMultiagente/WoS-20220203/Metricas/Documentos/MASSA2-ReferenceSpectroscopy-1901-1970.png}
    \caption{Espectroscopia (RPYS) das referências do \dataset\ SBDAA@guioliunb, entre 1901 e 1970.}
    \label{fig:MASSA2-ReferenceSpectroscopy:1901:1970}
\end{figure}
    
Se observamos a espectroscopia do mesmo \dataset\ apenas entre os anos de 1971 a 2019, como na figura \ref{fig:MASSA2-ReferenceSpectroscopy:1971:2019}, pode-se perceber um ponto de declínio no volume de citações a partir do ano de 2010, sugerindo que esse campo de conhecimento atingiu sua maturidade cerca de 10 anos atrás.

\begin{figure}
    \centering
    \includegraphics[width=1\textwidth]{experiments/jhcf/PesqBibliogr/SimulacaoMultiagente/WoS-20220203/Metricas/Documentos/MASSA2-ReferenceSpectroscopy-1971-2019.png}
    \caption{Espectroscopia (RPYS) das referências do \dataset\ SBDAA@guioliunb, entre 1971 e 2019.}
    \label{fig:MASSA2-ReferenceSpectroscopy:1971:2019}
\end{figure}

\paragraph{Uso de palavras dentro dos artigos no \dataset}

As últimas das métricas aplicadas a documentos, disponíveis para aplicação no Bibliometrix é baseada na ocorrência de termos no texto dos documentos. A mais comum delas é baseada na simples contagem de frequência das palavras, como ilustra a tabela \ref{tab:MASSA2:Word:Occurrences}, com os 40 termos mais frequentes em uso.

\begin{table}[htp]
    \centering
\footnotesize
\csvreader[tabular = {|l|r|l|},
separator=semicolon,
filter={\value{csvrow}<40}
%,filter not strcmp={\csvcolii}{},
, table head = \hline\hline \# & Palavra (termo) & Frequência \\ \hline\hline,
table foot = \hline\hline
]{experiments/jhcf/PesqBibliogr/SimulacaoMultiagente/WoS-20220203/Metricas/Documentos/MASSA2-Most_Frequent_Words.csv}{Words=\palavra, Occurrences=\freq}{ \thecsvrow & \palavra & \freq}

    \caption{40 palavras (termos) mais frequentes no \dataset\ SBDAA@guioliunb.}
    \label{tab:MASSA2:Word:Occurrences}
\end{table}

Outras formas de apresentação alternativas são apresentadas nas duas figuras a seguir, que ilustram de forma diferente a mesma informação, como em:
\begin{description}
    \item [Word Cloud] Uma nuvem de palavras, na figura \ref{fig:MASSA2-WordCloud-100words}, com evidencias para as 100 palavras mais frequentes;
    \item [Tree Map] Um mapa em árvore, na figura \ref{fig:MASSA2-TreeMap}, com evidências para as 50 palavras mais frequentes;
\end{description}

\begin{figure}
    \centering
    \includegraphics[width=1\textwidth]{experiments/jhcf/PesqBibliogr/SimulacaoMultiagente/WoS-20220203/Metricas/Documentos/MASSA2-WordCloud-100words.png}
    \caption{Nuvem dos 100 termos mais frequentes do \dataset\ SBDAA@guioliunb.}
    \label{fig:MASSA2-WordCloud-100words}
\end{figure}

\begin{figure}
    \centering
    \includegraphics[width=1\textwidth]{experiments/jhcf/PesqBibliogr/SimulacaoMultiagente/WoS-20220203/Metricas/Documentos/MASSA2-TreeMap.png}
    \caption{\textit{Tree Map} dos 50 termos mais frequentes do \dataset\ SBDAA@guioliunb.}
    \label{fig:MASSA2-TreeMap}
\end{figure}

Por fim, o Bibliometrix permite apresentar o uso dos termos ordenado temporalmente, como nas duas figuras a seguir:
\begin{description}
    \item [Word Growth / Word Dynamics] que mostra o crescimento de uso das palavras mais frequentes, como na figura \ref{fig:MASSA2-WordDynamics};
    \item [Trending topics] que mostra as tendências para uso de determinadas palavras em determinadas faixas de tempo, como em \ref{fig:MASSA2-TrendTopics}. Para obtenção do gráfico foram determinados os seguintes valores para os parâmetros: frequência mínima de ocorrência para que um termo seja considerado = 15, quantidade máxima de tópicos por ano = 7.
\end{description}

\begin{figure}
    \centering
    \includegraphics[width=1\textwidth]{experiments/jhcf/PesqBibliogr/SimulacaoMultiagente/WoS-20220203/Metricas/Documentos/MASSA2-WordDynamics.png}
    \caption{Dinâmica de uso ao longo do tempo, dos 20 termos mais frequentes do \dataset\ SBDAA@guioliunb.}
    \label{fig:MASSA2-WordDynamics}
\end{figure}

\begin{figure}
    \centering
    \includegraphics[angle=90,width=1\textwidth,height=0.93\textheight]{experiments/jhcf/PesqBibliogr/SimulacaoMultiagente/WoS-20220203/Metricas/Documentos/MASSA2-TrendTopics-WF=15:WPY=7.png}
    \caption{\textit{Trending Topics} do \dataset\ SBDAA@guioliunb, WF = 15, WPY=7.}
    \label{fig:MASSA2-TrendTopics}
\end{figure}

\subsection{Métricas para Autores}

Um autor escreve um documento científico, que eventualmente é publicado. Ao escrever esse documento cita outros, e eventualmente o trabalho do autor também é citado. Se é citado, é porque é relevante, e daí infere-se que teve impacto. Com base na medica bruta da citação, várias métricas podem ser criadas. Algumas delas são aplicadas a seguir. 

\subsubsection{Autores mais produtivos no \dataset}

A tabela \ref{tab:MASSA2:Author:Production} apresenta a lista ordenada e decrescente dos autores com maior número de artigos no \dataset. A coluna mais à direita divide esse primeiro valor pelo número de autores nos artigos.

\begin{table}[htp]
    \centering
\footnotesize
\csvreader[tabular = {|r|l|r|r|},
filter={\value{csvrow}<40}
%,filter not strcmp={\csvcolii}{},
, table head = \hline\hline \# & Autor & Qtd. de artigos & Qtd. Proporcional \\ \hline\hline,
table foot = \hline\hline
]{experiments/jhcf/PesqBibliogr/SimulacaoMultiagente/WoS-20220203/Metricas/Authors/MASSA2-Most-Productive-Authors.csv}
{Authors=\autor,Articles=\qtdart,Fractionalized=\artfrac}{ \thecsvrow & \autor & \qtdart & \artfrac}
    \caption{20 autores com mais artigos no \dataset\ SBDAA@guioliunb.}
    \label{tab:MASSA2:Author:Production}
\end{table}

\subsubsection{Autores mais relevantes localmente citados}

A tabela \ref{tab:MASSA2:Author:Production:Local} apresenta a lista ordenada e decrescente dos autores com maior número de artigos citados por outros artigos no \dataset. Esses são, possivelmente, os autores mais impactantes para o estado da arte no \dataset\ SBDAA@guioliunb.

\begin{table}[htp]
    \centering
\footnotesize
\csvreader[tabular = {|r|l|r|},
filter={\value{csvrow}<40},
%,filter not strcmp={\csvcolii}{},
table head = \hline\hline \# & Autor & Qtd artigos\\ \hline\hline,
table foot = \hline\hline]
{experiments/jhcf/PesqBibliogr/SimulacaoMultiagente/WoS-20220203/Metricas/Authors/MASSA2-Most-Local-Cited-Authors.csv}
{Author=\autor,LocalCitations=\qtdcit}
{ \thecsvrow & \autor & \qtdcit}
\caption{20 autores com mais artigos citados por outros artigos no \dataset\ SBDAA@guioliunb.}
    \label{tab:MASSA2:Author:Production:Local}
\end{table}

Uma busca na aba Filter, do Bibliometrix, permite explorar os artigos mais citados desses autores, alguns tem seus títulos listados a seguir:
\begin{description}
    \item [SCHWARZ N] 
    \begin{itemize}
        \item DESCRIBING HUMAN DECISIONS IN AGENT-BASED MODELS - ODD PLUS D, AN EXTENSION OF THE ODD PROTOCOL;
        \item AGENT-BASED MODELING OF THE DIFFUSION OF ENVIRONMENTAL INNOVATIONS - AN EMPIRICAL APPROACH;
    \end{itemize}
    \item [SU HS, WANG XF] 
    \begin{itemize}
        \item FLOCKING OF MULTI-AGENTS WITH A VIRTUAL LEADER;
        \item AGENT-BASED MODELING OF THE DIFFUSION OF ENVIRONMENTAL INNOVATIONS - AN EMPIRICAL APPROACH;
    \end{itemize}
    \item [LIAO XF] 
    \begin{itemize}
        \item EVENT-TRIGGERING SAMPLING BASED LEADER-FOLLOWING CONSENSUS IN SECOND-ORDER MULTI-AGENT SYSTEMS;
    \end{itemize}    
\end{description}

\subsubsection{Variação da produtividade dos autores ao longo do tempo}

Os cientistas são seres humanos, que também tem um ciclo de vida de iniciante, muito produtivo, menos produtivo, e não produtivo.
Diagramas como o da figura \ref{fig:MASSA2-TopAuthorsProductionOverTime} apresentam esse ciclo de vida, em relação ao \dataset\ SBDAA@guioliunb.

\begin{figure}
    \centering
    \includegraphics[angle=90,width=1\textwidth,height=0.93\textheight]{experiments/jhcf/PesqBibliogr/SimulacaoMultiagente/WoS-20220203/Metricas/Authors/MASSA2-TopAuthorsProductionOverTime.png}
    \caption{Variação da produção dos autores de maior impacto, do \dataset\ SBDAA@guioliunb.}
    \label{fig:MASSA2-TopAuthorsProductionOverTime}
\end{figure}

\subsubsection{Lei de Lotka}

A Lei de Lotka (ver \url{https://en.wikipedia.org/wiki/Lotka\%27s_law}) estabelece uma distribuição de frequência aproximadamente inversamente quadrática ou cúbica, para o número de artigos publicados pelos autores de qualquer área do conhecimento. Isso é, se 1000 pessoas publicam ao longo de sua contribuição para o campo de conhecimento apenas um documento, então
entre $1000/x^{2}$ a $1000/x^{3}$ publicam $x$ documentos. Ou seja, entre
$1000/2^{2}$ a $1000/2^{3}$ pessoas publicam dois documentos, $1000/3^{2}$ a $1000/3^{3}$ pessoas publicam três documentos etc.

Se os dados empíricos do \dataset\ são alinhados à essas curvas, então supor dizer que o \dataset\ é bem formado. Será que isso ocorre com a tabela \ref{}, criada para o \dataset\ SBDAA@guioliunb?

\begin{table}[htp]
    \centering
\footnotesize
\csvreader[
separator=semicolon,
tabular = {|r|l|r|r|r|},
filter={\value{csvrow}<40},
%,filter not strcmp={\csvcolii}{},
table head = \hline\hline \# & Qtd Autores & Qtd artigos & Lotka 2 & Lotka 3\\ \hline\hline,
table foot = \hline\hline]
{experiments/jhcf/PesqBibliogr/SimulacaoMultiagente/WoS-20220203/Metricas/Authors/MASSA2-Lotka_Law.csv}
{}
{ \thecsvrow & \csvcolii & \csvcoli & \csvcoliii & \csvcoliv}
\caption{Comparação do \dataset\ SBDAA@guioliunb com a formulação geral da Lei de Lotka.}
    \label{tab:MASSA2:Author:Lotka}
\end{table}

\subsubsection{Medidas de Impacto dos Autores}

A partir da medida básica de citações (TC) podem ser criados vários índices, sendo os mais conhecidos os índices H (Ver \url{https://en.wikipedia.org/wiki/H-index}), G (Ver \url{https://en.wikipedia.org/wiki/G-index})e M (ver \url{https://en.wikipedia.org/wiki/Author-level_metrics#m-index}).

As tabelas \ref{tab:MASSA2:Author:Impacto:H}, \ref{tab:MASSA2:Author:Impacto:G}, \ref{tab:MASSA2:Author:Impacto:M} e \ref{tab:MASSA2:Author:Impacto:Qtd:Publicacoes} mostram os autores mais proeminentes do \dataset\ ordenados com base em um desses índices ou com o simples volume total de publicação, em comparação aos demais índices.

Além dos índices e da quantidade total de citações (coluna TC), apresenta-se o volume de publicações (NP) e o ano de primeira publicação do autor (PY Start).

Observe que os índices H e G tendem a valorizar os autores mais estabilizados, enquanto que o índice M mostra os autores mais recentes, que tem menos anos de publicação.

Observe, adicionalmente, que a base para cálculo desses índices é o número total de citações, que tem alcance global, enquanto que o número de artigos publicados é o valor local.

\begin{table}[htp]
    \centering
\footnotesize
\csvreader[
separator=semicolon,
tabular = {|r|l|r|r|r|r|r|r|},
filter={\value{csvrow}<40},
%,filter not strcmp={\csvcolii}{},
table head = \hline\hline \# & Autor & Índice H & Índice G & Índice M & TC & NP & PY Start\\ \hline\hline,
table foot = \hline\hline]
{experiments/jhcf/PesqBibliogr/SimulacaoMultiagente/WoS-20220203/Metricas/Authors/MASSA2-H-Index-Author-Impact.csv}
{}
{ \thecsvrow & \csvcoli & \csvcolii & \csvcoliii & \csvcoliv & \csvcolv & \csvcolvi & \csvcolvii}
\caption{10 autores de maior impacto no \dataset\ SBDAA@guioliunb, conforme o índice H.}
    \label{tab:MASSA2:Author:Impacto:H}
\end{table}

\begin{table}[htp]
    \centering
\footnotesize
\csvreader[
separator=semicolon,
tabular = {|r|l|r|r|r|r|r|r|},
filter={\value{csvrow}<40},
%,filter not strcmp={\csvcolii}{},
table head = \hline\hline \# & Autor & Índice H & Índice G & Índice M & TC & NP & PY Start\\ \hline\hline,
table foot = \hline\hline]
{experiments/jhcf/PesqBibliogr/SimulacaoMultiagente/WoS-20220203/Metricas/Authors/MASSA2-G-Index-Author-Impact.csv}
{}
{ \thecsvrow & \csvcoli & \csvcolii & \csvcoliii & \csvcoliv & \csvcolv & \csvcolvi & \csvcolvii}
\caption{10 autores de maior impacto no \dataset\ SBDAA@guioliunb, conforme o índice G.}
    \label{tab:MASSA2:Author:Impacto:G}
\end{table}

\begin{table}[htp]
    \centering
\footnotesize
\csvreader[
separator=semicolon,
tabular = {|r|l|r|r|r|r|r|r|},
filter={\value{csvrow}<40},
%,filter not strcmp={\csvcolii}{},
table head = \hline\hline \# & Autor & Índice H & Índice G & Índice M & TC & NP & PY Start\\ \hline\hline,
table foot = \hline\hline]
{experiments/jhcf/PesqBibliogr/SimulacaoMultiagente/WoS-20220203/Metricas/Authors/MASSA2-M-Index-Author-Impact.csv}
{}
{ \thecsvrow & \csvcoli & \csvcolii & \csvcoliii & \csvcoliv & \csvcolv & \csvcolvi & \csvcolvii}
\caption{25 autores de maior impacto no \dataset\ SBDAA@guioliunb, conforme o índice M.}
    \label{tab:MASSA2:Author:Impacto:M}
\end{table}

\begin{table}[htp]
    \centering
\footnotesize
\csvreader[
separator=semicolon,
tabular = {|r|l|r|r|r|r|r|r|},
filter={\value{csvrow}<40},
%,filter not strcmp={\csvcolii}{},
table head = \hline\hline \# & Autor & Índice H & Índice G & Índice M & TC & NP & PY Start\\ \hline\hline,
table foot = \hline\hline]
{experiments/jhcf/PesqBibliogr/SimulacaoMultiagente/WoS-20220203/Metricas/Authors/MASSA2-Total-Citations-Author-Impact.csv}
{}
{ \thecsvrow & \csvcoli & \csvcolii & \csvcoliii & \csvcoliv & \csvcolv & \csvcolvi & \csvcolvii}
\caption{25 autores de maior impacto no \dataset\ SBDAA@guioliunb, conforme a quantidade de vezes que seus artigos foram globalmente citados.}
    \label{tab:MASSA2:Author:Impacto:Qtd:Publicacoes}
\end{table}

\subsubsection{Filiações dos autores às instituições de P\&D}

Os autores de documentos científicos são filiados como estudantes ou empregados em universidades e centros de pesquisa, e quando os publicam colocam o nome de suas filiações, criando a possibilidade de produção de \textit{rankings} para essas instituições. A tabela \ref{tab:MASSA2-Most-Relevant-Affiliations} apresenta as 40 instituições com maior produtividade, conforme o volume de artigos publicados presentes no \dataset. Ou seja, é nas instituições listadas na tabela onde possivelmente está mais avançado o conhecimento nesse tema. 

\begin{table}[htp]
    \centering
\footnotesize
\csvreader[
separator=comma,
tabular = {|r|l|r|},
filter={\value{csvrow}<40},
%,filter not strcmp={\csvcolii}{},
table head = \hline\hline \# & Instituição de filiação & Qtd de artigos publicados\\ \hline\hline,
table foot = \hline\hline]
{experiments/jhcf/PesqBibliogr/SimulacaoMultiagente/WoS-20220203/Metricas/Authors/MASSA2-Most-Relevant-Affiliations.csv}
{}
{ \thecsvrow & \csvcoli & \csvcolii }
\caption{40 Instituições mais produtivas no tema do \dataset\ SBDAA@guioliunb, conforme a quantidade de artigos publicados por pessoas a elas filiadas.}
    \label{tab:MASSA2-Most-Relevant-Affiliations}
\end{table}

\subsubsection{Filiações dos autores aos Países}

As instituições às quais são filiados os autores são sediadas em países, o que permite estimar a produtividade e (ou) impacto dos países em um tema de conhecimento específico, como ilustra a tabela \ref{MASSA2-Most-Cited-Countries}, que mostra a quantidade de citações obtidas pelos artigos das instituições sediadas nesses países.

\begin{table}[htp]
    \centering
\footnotesize
\csvreader[
separator=semicolon,
tabular = {|r|l|r|r|},
filter={\value{csvrow}<40},
%,filter not strcmp={\csvcolii}{},
table head = \hline\hline \# & País de Filiação & Qtd. Citações & Média de Citação por Artigo\\ \hline\hline,
table foot = \hline\hline]
{experiments/jhcf/PesqBibliogr/SimulacaoMultiagente/WoS-20220203/Metricas/Authors/MASSA2-Most-Cited-Countries.csv}
{}
{ \thecsvrow & \csvcoli & \csvcolii & \csvcoliii}
\caption{40 países com maior impacto de citações no tema do \dataset\ SBDAA@guioliunb.}
    \label{tab:MASSA2-Most-Cited-Countries}
\end{table}

\paragraph{Países dos autores principais dos artigos}

Cada artigo escrito por mais de um autor tem um autor responsável por receber as correspondências relativas ao artigo. Esse é o  \textit{corresponding author}. A figura \ref{fig:MASSA2-Corresponding-Authors-Country} apresenta, em ordem do maior para o menor volume de correspondentes por artigo, o volume publicado por cada país. SCP são as publicações cujos autores são todos do mesmo país. MCP envolve autores de mais de um país. Note que países como Japão, Índia e Irã, no \dataset\ em pauta, não parecem colaborar muito, de forma internacional.

\begin{figure}
    \centering
    \includegraphics[angle=90,width=1\textwidth,height=0.93\textheight]{experiments/jhcf/PesqBibliogr/SimulacaoMultiagente/WoS-20220203/Metricas/Authors/MASSA2-Corresponding-Authors-Country.png}
    \caption{Variação da produção dos autores de maior impacto, do \dataset\ SBDAA@guioliunb.}
    \label{fig:MASSA2-Corresponding-Authors-Country}
\end{figure}

\subsection{Métricas para Fontes de Informação}

As métricas para fontes de informação permitem avaliar a qualidade das revistas, conferências, em relação ao impacto de publicações em determinado tema. Valem a maioria das medidas já vistas para autores.

\subsubsection{Fontes mais relevantes, conforme número de artigos publicados sobre o tema}

\begin{figure}
    \centering
    \includegraphics[width=1\textwidth]{experiments/jhcf/PesqBibliogr/SimulacaoMultiagente/WoS-20220203/Metricas/Sources/MASSA2-Most-Relevant-Sources.png}
    \caption{Revistas mais relevantes no  \dataset\ SBDAA@guioliunb.}
    \label{fig:MASSA2-Most-Relevant-Sources}
\end{figure}

\subsubsection{Fontes mais relevantes, conforme o número de citações globais}

\begin{figure}
    \centering
    \includegraphics[width=1\textwidth]{experiments/jhcf/PesqBibliogr/SimulacaoMultiagente/WoS-20220203/Metricas/Sources/MASSA2-Total-Citation-Source-Local-Impact.png.png}
    \caption{Revistas mais relevantes no  \dataset\ SBDAA@guioliunb, conforme a soma de citações aos artigos publicados.}
    \label{fig:MASSA2-Total-Citation-Source-Local-Impact.png}
\end{figure}

\subsubsection{Fontes mais relevantes, conforme o número de citações na lista de referências locais}

\begin{figure}
    \centering
    \includegraphics[width=1\textwidth]{experiments/jhcf/PesqBibliogr/SimulacaoMultiagente/WoS-20220203/Metricas/Sources/MASSA2-Most-Local-Cited-Sources(from-Reference-Lists).png}
    \caption{Revistas mais relevantes no  \dataset\ SBDAA@guioliunb, conforme a soma de citações aos artigos no \dataset.}
    \label{fig:MASSA2-Most-Local-Cited-Sources(from-Reference-Lists).png}
\end{figure}

\subsubsection{Lei de Bradford}

\begin{figure}
    \centering
    \includegraphics[width=1\textwidth]{experiments/jhcf/PesqBibliogr/SimulacaoMultiagente/WoS-20220203/Metricas/Sources/MASSA2-Bradfords-Law.png}
    \caption{Revistas mais relevantes no  \dataset\ SBDAA@guioliunb, conforme a Lei de Bradford.}
    \label{fig:MASSA2-Bradfords-Law.png}
\end{figure}

\subsubsection{Medidas de Impacto das fontes}

\paragraph{Índice H}

\begin{figure}
    \centering
    \includegraphics[width=1\textwidth]{experiments/jhcf/PesqBibliogr/SimulacaoMultiagente/WoS-20220203/Metricas/Sources/MASSA2-H-Index-Source-Local-Impact.png}
    \caption{Revistas de maior impacto no  \dataset\ SBDAA@guioliunb,  conforme o índice H.}
    \label{fig:MASSA2-H-Index-Source-Local-Impact.png}
\end{figure}


\paragraph{Índice G}

\begin{figure}
    \centering
    \includegraphics[width=1\textwidth]{experiments/jhcf/PesqBibliogr/SimulacaoMultiagente/WoS-20220203/Metricas/Sources/MASSA2-G-Index-Source-Local-Impact.png}
    \caption{Revistas de maior impacto no  \dataset\ SBDAA@guioliunb,  conforme o índice G.}
    \label{fig:MASSA2-G-Index-Source-Local-Impact.png}
\end{figure}

\paragraph{Índice M}

\begin{figure}
    \centering
    \includegraphics[width=1\textwidth]{experiments/jhcf/PesqBibliogr/SimulacaoMultiagente/WoS-20220203/Metricas/Sources/MASSA2-M-Index-Source-Local-Impact.png}
    \caption{Revistas de maior impacto no  \dataset\ SBDAA@guioliunb,  conforme o índice M.}
    \label{fig:MASSA2-M-Index-Source-Local-Impact.png}
\end{figure}

\subsubsection{Dinâmica de publicação nas fontes}

\begin{figure}
    \centering
    \includegraphics[width=1\textwidth]{experiments/jhcf/PesqBibliogr/SimulacaoMultiagente/WoS-20220203/Metricas/Sources/MASSA2-Source-Dynamics.png}
    \caption{Revistas com maior volume de publicações no tema no  \dataset\ SBDAA@guioliunb, ao longo do tempo.}
    \label{fig:MASSA2-Source-Dynamics.png}
\end{figure}

\subsection{Mapas de Acoplamento}

\begin{figure}
    \centering
    \includegraphics[width=1\textwidth]{experiments/jhcf/PesqBibliogr/SimulacaoMultiagente/WoS-20220203/Clustering/MASSA2-CouplingMap-4Labels-Per-Cluster.png}
    \caption{Palavras-chave mais evidentes que surgem nos clusters formados pelo Acoplamento bibliográfico entre documentos, no  \dataset\ SBDAA@guioliunb.}
    \label{fig:MASSA2-CouplingMap-4Labels-Per-Cluster}
\end{figure}


\subsection{Estrutura Conceitual do Conhecimento}

O Conhecimento científico é um fenômeno complexo que emerge a partir da agregação memética de termos e palavras, que representam conceitos e ideias, que se organizam em tópicos, temas, e que evoluem ao longo do tempo (ver \url{https://en.wikipedia.org/wiki/Memetics}).

A estrutura conceitual do conhecimento pode ser produzida pela análise de relacionamento estabelecidos entre esses termos. O bibliometrix apresenta um conjunto de técnicas para evidenciar essa estrutura conceitual, e que se organizam em dois grupos:
\begin{description}
    \item [Métricas em rede] que usam grafos para representar relacionamentos entre termos, evidenciando, por meio de métricas de análise de redes sociais, como o conhecimento conceitualmente se organiza.
    \item [Análise Fatorial] Que emprega métricas de redução da dimensionalidade, para explorar, usualmente em mapas bidimensionais, como os termos e palavras se relacionam. 
\end{description}

\subsubsection{Métricas aplicadas a grafos (redes)}

\paragraph{Redes de Coocorrências}

As redes de coocorrências apresentam importantes padrões que se formam nas publicações, e podem revelar a estrutura conceitual de uma área do conhecimento.

No Biblioshiny quatro tipos de redes coocorrências podem ser geradas:
\begin{itemize}
    \item Baseadas na coocorrência de termos, revelando quais são os termos mais comumente citados 
\end{itemize}

\begin{figure}
    \centering
    \includegraphics[width=1\textwidth]{experiments/jhcf/PesqBibliogr/SimulacaoMultiagente/WoS-20220203/Estrutura/Conceitual/MASSA2-Co-occurrence-Network-50nodes-louvainclustering.png.png}
    \caption{50 palavras-chave mais evidentes, clusterizadas pela coocorrência em documentos, no  \dataset\ SBDAA@guioliunb.}
    \label{fig:MASSA2-Co-occurrence-Network-50nodes-louvainclustering.png}
\end{figure}

\begin{figure}
    \centering
    \includegraphics[width=1\textwidth]{experiments/jhcf/PesqBibliogr/SimulacaoMultiagente/WoS-20220203/Estrutura/Conceitual/MASSA2-Cluster1-Co-occurrence-Network-50nodes-louvainclustering.png.png}
    \caption{Detalhamento do cluster 1, na rede das 50 palavras-chave mais evidentes, clusterizadas pela coocorrência em documentos, no  \dataset\ SBDAA@guioliunb.}
    \label{fig:MASSA2-Cluster1-Co-occurrence-Network-50nodes-louvainclustering.png}
\end{figure}

\begin{figure}
    \centering
    \includegraphics[width=1\textwidth]{experiments/jhcf/PesqBibliogr/SimulacaoMultiagente/WoS-20220203/Estrutura/Conceitual/MASSA2-Cluster2-Co-occurrence-Network-50nodes-louvainclustering.png.png.png}
    \caption{Detalhamento do cluster 2, na rede das 50 palavras-chave mais evidentes, clusterizadas pela coocorrência em documentos, no  \dataset\ SBDAA@guioliunb.}
    \label{fig:MASSA2-Cluster2-Co-occurrence-Network-50nodes-louvainclustering.png.png}
\end{figure}

\begin{figure}
    \centering
    \includegraphics[width=1\textwidth]{experiments/jhcf/PesqBibliogr/SimulacaoMultiagente/WoS-20220203/Estrutura/Conceitual/MASSA2-Cluster3-Co-occurrence-Network-50nodes-louvainclustering.png.png.png}
    \caption{Detalhamento do cluster 3, na rede das 50 palavras-chave mais evidentes, clusterizadas pela coocorrência em documentos, no  \dataset\ SBDAA@guioliunb.}
    \label{fig:MASSA2-Cluster3-Co-occurrence-Network-50nodes-louvainclustering.png.png}
\end{figure}

\paragraph{Mapas Temáticos}

\begin{figure}
    \centering
    \includegraphics[width=1\textwidth]{experiments/jhcf/PesqBibliogr/SimulacaoMultiagente/WoS-20220203/Estrutura/Conceitual/MASSA2-ThematicMap.png}
    \caption{Mapa temático do  \dataset\ SBDAA@guioliunb.}
    \label{fig:MASSA2-ThematicMap}
\end{figure}

\paragraph{Evolução Temática}

\begin{figure}
    \centering
    \includegraphics[width=1\textwidth]{experiments/jhcf/PesqBibliogr/SimulacaoMultiagente/WoS-20220203/Estrutura/Conceitual/MASSA2-Thematic-Evolution.png}
    \caption{Evolução temática do  \dataset\ SBDAA@guioliunb.}
    \label{fig:MASSA2-ThematicMap}
\end{figure}

\subsubsection{Métricas de redução da dimensionalidade (Análise Fatorial)}

\begin{figure}
    \centering
    \includegraphics[width=1\textwidth]{experiments/jhcf/PesqBibliogr/SimulacaoMultiagente/WoS-20220203/Estrutura/Conceitual/MASSA2-FactorialAnalysis-MCA-FactorialMap.png}
    \caption{Dimensões de variabilidade mais relevantes, nas palavras-chave do  \dataset\ SBDAA@guioliunb.}
    \label{fig:MASSA2-FactorialAnalysis-MCA-FactorialMap}
\end{figure}

\begin{figure}
    \centering
    \includegraphics[width=1\textwidth]{experiments/jhcf/PesqBibliogr/SimulacaoMultiagente/WoS-20220203/Estrutura/Conceitual/MASSA2-FactorialAnalysis-MCA-Dendrogram.png}
    \caption{Dendograma das dimensões de variabilidade mais relevantes, nas palavras-chave do  \dataset\ SBDAA@guioliunb.}
    \label{fig:MASSA2-FactorialAnalysis-MCA-Dendrogram}
\end{figure}

\begin{figure}
    \centering
    \includegraphics[width=1\textwidth]{experiments/jhcf/PesqBibliogr/SimulacaoMultiagente/WoS-20220203/Estrutura/Conceitual/MASSA2-FactorialAnalysis-MCA-MostContribDocuments.png}
    \caption{Documentos que mais contribuíram para determinar das dimensões de variabilidade mais relevantes, nas palavras-chave do  \dataset\ SBDAA@guioliunb.}
    \label{fig:MASSA2-FactorialAnalysis-MCA-MostContribDocuments}
\end{figure}

\subsection{Estrutura Intelectual  do Conhecimento}

Conhecimento científico é produzido por processos intelectuais onde autores de trabalho escolhem deliberadamente referenciar trabalhos de outros, por meio de documentos publicados, que são encaminhados para publicações em fontes de informação de sua escolha, e que evoluem ao longo do tempo.

O Bibliometrix permite exploração da estrutura intelectual do conhecimento, usando basicamente duas abordagens:
\begin{itemize}
    \item Redes de Co-Citação, abordagem bastante comum;
    \item Historiografia, abordagem pouco usual.
\end{itemize}

\subsubsection{Redes de Co-Citação}

\begin{figure}
    \centering
    \includegraphics[width=1\textwidth]{experiments/jhcf/PesqBibliogr/SimulacaoMultiagente/WoS-20220203/Estrutura/Intelectual/MASSA2-CoCitation-Network-50-Papers.png}
    \caption{Rede de cocitação entre as 50 referências mais presentes no  \dataset\ SBDAA@guioliunb.}
    \label{fig:MASSA2-CoCitation-Network}
\end{figure}

\subsubsection{Historiografia}

\begin{figure}
    \centering
    \includegraphics[width=1\textwidth]{experiments/jhcf/PesqBibliogr/SimulacaoMultiagente/WoS-20220203/Estrutura/Intelectual/MASSA2-HistoricalDirectCitationNetwork-100docs.png}
    \caption{Mapa histórico das citações diretas entre os documentos mais evidentes no  \dataset\ SBDAA@guioliunb.}
    \label{fig:MASSA2-HistoricalDirectCitationNetwork-100docs}
\end{figure}

\subsection{Estrutura Social  do Conhecimento}

Conhecimento científico é produzido socialmente, por meio de autores trabalhando em conjunto, e uma estrutura de filiações a organizações permanentes ou periódicas, que realizam ou promovem pesquisas, nelas incluídos os centros de pesquisa, universidades, departamentos, institutos, faculdades, eventos, revistas, conferências, e que evoluem ao longo do tempo. A análise da estrutura social do conhecimento evidencia esses relacionamentos, que iniciam no plano pessoal, e evoluem para outros escopos.

\subsubsection{Rede de Colaboração}

\begin{figure}
    \centering
    \includegraphics[width=1\textwidth]{experiments/jhcf/PesqBibliogr/SimulacaoMultiagente/WoS-20220203/Estrutura/Social/MASSA2-Collaboration-Network-150instit.png}
    \caption{Rede de colaboração entre as 150 instituições mais evidentes, no  \dataset\ SBDAA@guioliunb.}
    \label{fig:MASSA2-Collaboration-Network-150instit}
\end{figure}

\begin{figure}
    \centering
    \includegraphics[width=1\textwidth]{experiments/jhcf/PesqBibliogr/SimulacaoMultiagente/WoS-20220203/Estrutura/Social/MASSA2-Collaboration-Network-150authors.png}
    \caption{Rede de colaboração entre os 150 autores mais evidentes, no  \dataset\ SBDAA@guioliunb.}
    \label{fig:MASSA2-Collaboration-Network-150authors}
\end{figure}

\begin{figure}
    \centering
    \includegraphics[width=1\textwidth]{experiments/jhcf/PesqBibliogr/SimulacaoMultiagente/WoS-20220203/Estrutura/Social/MASSA2-Collaboration-Network-50country.png}
    \caption{Rede de colaboração entre os 150 países, no  \dataset\ SBDAA@guioliunb.}
    \label{fig:MASSA2-Collaboration-Network-50country}
\end{figure}

%\subsubsection{Mapa da Colaboração Mundial}

\section{Análises\label{MASSA2:Analises}}

A pesquisa mundial no tema da simulação multiagente de fenômenos sociais é pontuada por dois grandes temas, ilustrados de forma bastante clara no mapa de estrutura conceitual da figura \ref{fig:MASSA2-FactorialAnalysis-MCA-FactorialMap}, na página \pageref{fig:MASSA2-FactorialAnalysis-MCA-FactorialMap}:
\begin{itemize}
    \item 
(Em azul no mapa) um grupo de pesquisadores que estuda questões ligadas a sistemas de controle, em busca de compreender fenômenos de sincronização, consenso, rastreabilidade, coordenação, rastreamento de agentes artificiais, voando ou navegando em bando e se comunicando em redes com topologia variável. Parecem estar em busca de uma inteligência artificial distribuída e robotizada;
\item (em vermelho no mapa) um grupo mais difuso de pesquisadores, que busca compreender a emergência de fenômenos de natureza bem mais variada, inclusive os que ocorrem em grupos sociais humanos. Esses fenômenos pode ser de cooperação, evolução, participação em jogos, crescimento, difusão de inovações, escolhas, tomadas de decisão, percepção, gerenciamento e uso da terra (no sentido geográfico), formação de preços de ativos financeiros etc. Parecem estar em busca das aplicações das leis gerais dos sistemas complexos adaptativos, bem como na investigação de novos fenômenos que expliquem a dinâmica das sociedades humanas e biológicas.
\end{itemize}

\section{Conclusões}

Este trabalho está incompleto, mas apresenta o arcabouço geral de informações que possibilitam responder às  questões formuladas no início da pesquisa, em \ref{SBDAA@guioliunb:questoes}, para a qual serão apresentadas breves respostas preliminares:

\subsection{Base de conhecimentos}

Qual a base de conhecimentos científicos produzida em torno do tema simulação multiagente voltada à compreensão de fenômenos sociais, com ênfase em métodos experimentais?
 
Resposta: Ver, em \ref{MASSA2:Analises}, que ela se estrutura em dois grupos principais: um voltado para a inteligência artificial distribuída, baseada em agentes artificiais, como drones, possivelmente com aplicações no campo militar.
O outro grupo é mais abrangente, voltado para a compreensão de fenômenos na sociedade humana, e fundamentado
nos fenômenos emergentes.
Tomando-se também por base os mapas \ref{fig:MASSA2-Co-occurrence-Network-50nodes-louvainclustering.png} e seus detalhamentos em \ref{fig:MASSA2-Cluster1-Co-occurrence-Network-50nodes-louvainclustering.png}, \ref{fig:MASSA2-Cluster2-Co-occurrence-Network-50nodes-louvainclustering.png.png}, percebe-se que esse ultimo grupo pode ser subdidivido em dois: um de ordem mais teórica, representado pelo cluster em \ref{fig:MASSA2-Cluster3-Co-occurrence-Network-50nodes-louvainclustering.png.png}, fundamentado na evolução, possivelmente compreendida como um fenômeno emergente, e um mais aplicado, pesadamente lastreado em simulação, construção de modelos para várias áreas de aplicação.

Nota-se, com base na análise da espectroscopia mais recente das referências bibliográficas do \dataset, sumarizada em \ref{fig:MASSA2-ReferenceSpectroscopy:1971:2019} entre os anos de 1971 e 2019, que a área parece atingir sua maturidade por volta do ano de 2011.

\subsection{Fenômenos sociais}
   
Como a simulação multiagente tem sido usada para compreender fenômenos sociais, com ênfase em métodos experimentais? 

Resposta: Ver \ref{MASSA2:Analises}.

\subsection{Termos e conceitos centrais}

Quais os principais termos e conceitos ligados à frente de pesquisa no tema simulação multiagente de fenômenos sociais, com ênfase em métodos experimentais? 

Resposta: Ver e explorar os mapas das figuras \ref{fig:MASSA2-Co-occurrence-Network-50nodes-louvainclustering.png}, \ref{fig:MASSA2-ThematicMap}, entre outros.

\subsection{Estrutura Social da Comunidade}

Qual a estrutura social da comunidade, se é que existe, que pesquisa sobre o tema simulação multiagente de fenômenos sociais, com ênfase em métodos experimentais?

Resposta: Ver e analisar os mapas das figuras \ref{fig:MASSA2-Collaboration-Network-150authors}, \ref{fig:MASSA2-Collaboration-Network-150instit} e \ref{fig:MASSA2-Collaboration-Network-50country}.

A rede de cocitação 