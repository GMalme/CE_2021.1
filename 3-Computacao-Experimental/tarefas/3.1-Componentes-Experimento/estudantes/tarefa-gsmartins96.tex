\section{Respostas de Gabriel dos Santos Martins}

\subsection{A computação é uma atividade científica?}
Descobrir novas tecnologias, métodos, encontrar e resolver problemas, descobrir novas coisas em geral são uma das essências da atividade cientifica e podemos ver isso constantemente na computação que a cada ano evolui e vem sendo melhorada com o passar do tempo, como prova disso hoje temos computadores muito mais rápidos que o ENIAC de 1946, mostrando a evolução e descobertas que levaram a esse grande avanço computacional e tecnológico. É fato que a computação é uma atividade científica, não em todas as áreas mas grande parte envolve pesquisas científicas e estudos cientificas. A \gls{Ciencia} em si utiliza de computadores a da computação pra avançar, hoje muito processamento de dados e testes utiliza a computação para isso, e para melhorar tal processamento e armazenamento dos dados, é preciso usar a ciência para evoluir neste aspecto. 
Se analisarmos desde o inicio da computação até hoje, foi realizado diversas atividades cientificas que contribuíram com o avanço da computação, logo podemos definir a computação como uma atividade científica.

\subsection{A computação é uma ciência experimental?}
Como visto na resposta da pergunta a cima a computação de fato é uma atividade científica, e atividades cientificas envolve muitos experimentos e testes para alcançar algum resultado que levam anos para chegar em algo que de fato possa ser utilizado e comprovado que realmente funciona. Como definimos a computação como uma atividade cientifica é fato que neste caso também podemos defini-la como uma ciência experimental, visto que é preciso utilizar e medir resultados para ser ter algum avanço cientifico neste caso. 

\subsection{A computação é uma ciência empírica? Justifique}
Primeiro, precisamos definir o que é uma ciência empírica. Na ciência, muitas pesquisas são realizadas inicialmente através da observação e da experiência, que é o que podemos definir de ciência empírica. A \gls{Ciencias} empírica se encarregam de estudar os feitos auxiliando-se na observação e na experimentação. 
Dado esse pequeno conceito para entendermos um pouco sobre ciência empírica e com base nas resposta vistas acima, a computação como uma atividade cientifica e uma ciência experimental, utiliza-se de observações e experiências no processo de evolução da computação, dessa forma podemos definir a computação também como uma ciência empírica. 
