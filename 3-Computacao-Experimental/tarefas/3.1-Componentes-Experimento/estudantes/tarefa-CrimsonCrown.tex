\section{Respostas de Arthur da Silveira Couto}

Para responder à pergunta, primeiro precisamos definir os componentes de um experimento científico. Como definido na seção \ref{fundamentos:ce}, os componentes são:
\begin{itemize}
    \item Uma classe de fenômenos de interesse.
    \item Hipóteses e modelos.
    \item Causa e Efeito.
    \item Procedimentos repetíveis.
    \item Demonstração, manipulação e controle de fatores.
    \item Coleta de dados.
    \item Análise lógica.
\end{itemize}

Todo cientista da computação tem seus tópicos de interesse. Os fenômenos computacionais são muito estudados, mas além disso, cientistas da computação podem utilizar técnicas computacionais para estudar fenômenos mais associados com outras áreas. A ciência da computação é extremamente versátil e abrangente, porém pode se dizer que o primeiro componente de um experimento está sendo suprido, já que sempre há um fenômeno de interesse para cada cientista.

Hipóteses são sempre utilizadas no planejamento de um experimento ou projeto na ciência da computação. Um programa pode ser entendido como um modelo matemático e computacional de um determinado fenômeno, atrelando a computação à modelagem científica. Esse componente está bem presente na atividade de um cientista da computação.

Como programas estabelecem uma relação entre entrada e saída, relações de causa e efeito se tornam bem presentes na ciência da computação. A lógica e matemática que transforma a entrada na saída por meio de funções pode ser entendida como um modelo computacional das relações de causa e efeito hipotetizadas sobre o fenômeno sendo modelado. Dessa forma, a simulação resultante iria expressar essas relações por meio das mudanças que ocorrem com alterações na parametrização no resultado.

Uma simulação feita por computador também pode ser entendida como um procedimento repetível. Se um determinado programa de simulação é rodado com os mesmos parâmetros, o resultado deverá ser o mesmo. Além disso, qualquer cientista com acesso a um computador compatível, ao código ou executável da simulação, e aos parâmetros usados, poderá repetir a mesma simulação feita anteriormente por outros cientistas.

Por meio de parametrização, simulações podem ter seus fatores controlados e manipulados facilmente. Fenômenos computacionais estudados diretamente também passam por esse processo, por meio do uso de laboratórios com condições controladas, computadores com componentes específicos, e versionamento controlado do software utilizado em experimentos e testes. Também são usados regularmente conjuntos específicos de entradas para testar programas em situações controladas.

Os resultados dessas simulações ou testes são registrados em forma de dados. O trabalho de um cientista da computação em geral envolve aglomerar a saída de um programa para várias entradas específicas, e tirar conclusões a partir delas. Nesse tipo de atividade, os registros de saída são transformados em um banco de dados para ser analisado pelo cientista. Quando um teste é executado, os resultados do teste são usados como dados, e quando uma simulação é executada, o estado final é usado. Entendendo o programa ou simulação como um modelo operacional do fenômeno sendo estudado, e os dados coletados de saída como um resultado das relações de causa e efeito sendo aplicadas aos parâmetros definidos pelo cientista, essa atividade se torna claramente empírica em natureza.

O último componente, análise lógica, está presente também no trabalho de um cientista da computação. As conclusões tiradas de um teste nunca são apenas o aglomerado de todos os dados obtidos. Sempre queremos concluir algo a partir desses dados, como a eficiência de um algoritmo com relação a uma métrica especifica, ou se uma simulação se assemelha ao fenômeno observado para validar o modelo da sua construção. Isso pode ser feito comparando resultados criados com parâmetros diferentes, ou comparando os resultados de uma simulação com observações do fenômeno quando aplicável.

Dessa forma, todos os componentes parecem estar presentes na ciência da computação. Um cientista se interessa por um assunto, e resolve estuda-lo por meio de um experimento. Hipóteses são criadas, e modelos em forma de programas ou simulações são construídos para representar as ideias do cientista. Esse programa ou simulação é construído com um código que representa nas suas funções as relações de causa e efeito entre os elementos desse modelo. Após a sua construção, o programa ou simulação podem ser rodados inúmeras vezes com parâmetros diferentes, e podem ser rodados por qualquer outro cientista com acesso a infra-estrutura necessária. O programa ou simulação é rodado com parâmetros específicos definidos pelo projeto do experimento, de modo a obter dados relevantes para provar ou refutar as hipóteses do cientista. Os resultados são coletados como dados, e são depois analisados para obter as conclusões desejadas.