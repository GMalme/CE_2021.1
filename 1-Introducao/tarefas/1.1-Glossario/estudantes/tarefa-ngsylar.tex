\newglossaryentry{Empirismo Cientifico}{
    name = {Empirismo Científico},
    description = {
    Empirismo científico é um princípio da produção científica baseado na filosofia positivista, cuja obtenção de conhecimento se dá pela observação ou análise da coleta de dados por meio da experimentação de um dado fenômeno. Tal princípio foi utilizado, por exemplo, no teste de Turing, onde um observador deveria decidir a natureza de seus interlocutores, se homem ou máquina, analisando as respostas através de um critério comportamental e não metafórica.
    %
    Fonte: \cite{floridi_blackwell_2004}.
    }
}
