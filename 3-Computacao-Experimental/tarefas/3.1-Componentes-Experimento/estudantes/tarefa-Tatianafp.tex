\section{Respostas de Tatiana Franco Pereira\label{tarefa-Tatianafp-componentes-eperimento}}

\subsection{A computação é uma atividade científica? }

Sabe-se que \gls{Ciencia} pode ser definida como o estudo metodicamente organizado dos fenômenos que ocorrem no universo, com o intuito de descrever, explicar e prever o comportamento e a estrutura dos mesmos. Por sua vez, a \gls{Computacao} pode ser definida como sendo "a busca de solução para um problema a partir de entradas, de forma a obter resultados depois de processada a informação através de um \gls{Algoritmo}". 

Com base nessas duas definições, verifica-se que a Ciência da Computação não é necessariamente uma atividade científica, pois seu foco nem sempre é estudar a verdade sobre um determinado fenômeno. No entanto, isso não impede que ambas utilizem de procedimentos similares a fim de alcançar seus objetivos. Com a intenção de verificar se ocorre a utilização de todos os componentes de um experimento científico em uma atividade profissional de um cientista da computação, utilizaremos da reflexão de mais duas perguntas basilares: se a computação é uma ciência experimental e se a computação é uma ciência empírica.  

\subsection{A computação é uma ciência experimental? }

Na reflexão de Roger Bacon sobre a ciência experimental, afirmar-se que é preciso provar tudo pela \gls{Experiencia}. Segundo ele, argumentos não são o suficiente para se comprovar algo, necessita-se da experimentação para se averiguar as conclusões de uma hipótese \cite{bacon_opus_1268}. Ao observar como isso se aplica na rotina profissional de um cientista da computação, tem-se a importância da fase de testes de um software por exemplo, ou então o processo de se colocar algo em produção gradativamente a fim de se evitar maiores problemas. Mesmo ao se ter uma conhecimento profundo sobre a lógica de funcionamento de um algoritmo de software, é apenas ao colocá-lo em prática que tem-se uma noção realista de possíveis contratempos que possam vir a ocorrer, como por exemplo um entrada inesperada, ou falhas relacionadas a causas externas. Com base nisso, creio ser seguro afirmar que a computação é uma ciência experimental.

\subsection{A computação é uma ciência empírica? }

Por sua vez, o \gls{EmpirismoCientifico} é baseado no princípio de que a obtenção de \gls{Conhecimento} se dá pela observação ou análise da coleta de dados obtidos no processo de experimentação de determinado fenômeno. Ao se realizar os testes necessários para confirmar se o algoritmo está funcionando corretamente, observa-se os dados de saída que foram gerados durante a execução experimental do software. Com isso, pode-se afirmar que a computação possui as características necessárias para ser considerada uma ciência empírica.