\documentclass[12pt]{book}

\input{packages}
\input{packages-estudantes}

% define cores personalizadas para o texto de cada autor
\usepackage
%[final]
{changes}
%\usepackage{changes}
%\url{https://ctan.org/pkg/changes}

\definechangesauthor[name={Jorge Henrique Cabral Fernandes}, color=orange]{jhcf} % git-user: jhcf

\definechangesauthor[name={Alexsander Correa de Oliveira}, color=black]{KvotheKS} % git-user: KvotheKS OK

\definechangesauthor[name={Allann Gois Hoffmann}, color=orange]{AllannH} % git-user: AllannH OK

\definechangesauthor[name={André Larrosa Chimpliganond}, color=orange]{andrelarrosacrypt} % git-user: andrelarrosacrypt OK

\definechangesauthor[name={André Cássio Barros de Souza}, color=green]{andreloff} % git-user: andreloff OK

\definechangesauthor[name={Bruno Sanguinetti Regadas de Barros}, color=blue]{Jaxiii} % git-user: Jaxiii OK

\definechangesauthor[name={Enzo Nunes Leal Sampaio}, color=orange]{enzodevs2000} % git-user: enzodevs2000 OK

\definechangesauthor[name={Felipe Gomes Paradas}, color=pink]{fparadas} % git-user: fparadas OK

\definechangesauthor[name={Lucas de Almeida Bandeira Macedo}, color=teal]{ABMHub} % git-user: ABMHub OK

\definechangesauthor[name={Fernanda Macedo de Sousa}, color=magenta]{fernandams} % git-user: fernandams Ok

\definechangesauthor[name={Gabriel dos Santos Martins}, color=green]{gsmartins96} %  git-user: gsmartins96 OK

\definechangesauthor[name={Gabriel Faustino Lima da Rocha}, color=gray]{Faustino27} %  git-user: Faustino27 OK

\definechangesauthor[name={Gabriel Martins de Almeida}, color=purple]{GMalme} %  git-user: GMalme OK

\definechangesauthor[name={Gabriel Rocha Fontenele}, color=pink]{ngsylar} % git-user: ngsylar OK

\definechangesauthor[name={Ítalo Eduardo Dias Frota}, color=pink]{titofrota} % git-user: titofrota OK

\definechangesauthor[name={João Antonio Desidério de Moraes}, color=teal]{joaoadm94} % git-user: joaoadm94 OK

\definechangesauthor[name={Ualiton Ventura da Silva}, color=orange]{uventura} % git-user: uventura OK

\definechangesauthor[name={Pedro de Torres Maschio}, color=orange]{pedro-maschio} % git-user: pedro-maschio OK

\definechangesauthor[name={Tong Zhou}, color=orange]{Tong00020} % git-user: Tong00020 Ok

\definechangesauthor[name={Gustavo Rodrigues dos Santos}, color=pink]{gutorsantos} % git-user: gutorsantos OK

\definechangesauthor[name={Gustavo Tomás de Paula}, color=green]{gustavo-tomas} % git-user: gustavo-tomas OK

\definechangesauthor[name={Gustavo Macedo de Carvalho}, color=purple]{GustavoMacCar} % git-user: GustavoMacCar OK

\definechangesauthor[name={Arthur da Silveira Couto}, color=purple]{CrimsonCrown} % git-user: CrimsonCrown OK?

\definechangesauthor[name={Vitor de Oliveira Araujo Araruna}, color=orange]{vitorararuna} % git-user: vitorararuna OK

\definechangesauthor[name={Rafael dos Santos Silva}, color=red]{rafaelsilva21} % git-user: rafaelsilva21 OK

\definechangesauthor[name={Marcus Vinicius Oliveira de Abrantes}, color=red]{MarcusABR} % git-user: MarcusABR OK

\definechangesauthor[name={Mateus de Paula Rodrigues}, color=cyan]{MoustacheGolem} % git-user: MoustacheGolem OK

\definechangesauthor[name={Leonardo Alves Riether}, color=blue]{LeoRiether} % git-user: LeoRiether OK

\definechangesauthor[name={Tatiana Franco Pereira}, color=cyan]{Tatianafp} % git-user: Tatianafp OK

\definechangesauthor[name={Vinícius Caixeta de Souza}, color=orange]{vinis-caixe} % git-user: vinis-caixe OK

\definechangesauthor[name={Conrado Nunes Barbosa Neto}, color=blue]{Conras21} % git-user: Conras21 OK

\definechangesauthor[name={Stefano Luppi Sposito}, color=pink]{KawaiiStheno} % git-user: KawaiiStheno OK

\definechangesauthor[name={João Pedro Felix de Almeida}, color=teal]{DYosplay} % git-user: DYosplay OK

\definechangesauthor[name={João Víctor Siqueira de Araujo}, color=red]{StrawHat972} % git-user: StrawHat972 OK

\definechangesauthor[name={Raylan da Silva Sales}, color=pink]{Rayxan} % git-user: Rayxan OK

\definechangesauthor[name={Guilherme Oliveira Loiola}, color=blue]{guioliunb} % git-user: guioliunb OK

\definechangesauthor[name={Paulo Alvim Alvarenga}, color=purple]{alvimpaulo} % git-user: alvimpaulo OK

\definechangesauthor[name={Léo Akira Abe Barros}, color=red]{leoakir} % git-user: leoakir OK

\definechangesauthor[name={Enzo Yoshio Niho}, color=purple]{enzoyoshio} % git-user: enzoyoshio OK

\definechangesauthor[name={Daniel Rodrigues Cardoso}, color=blue]{DanielrCardoso} % git-user: DanielrCardoso OK

\definechangesauthor[name={Fernando Ferreira Cordeiro}, color=blue]{FernandoCordeiro} % git-user: FernandoCordeiro OK

\definechangesauthor[name={Jônatas Gomes Barbosa da Silva}, color=cyan]{jonatas1n} % git-user: jonatas1n OK

\definechangesauthor[name={Lucas Gabriel de Oliveira Gurgel Fernandes}, color=black]{lggurgel} % git-user: lggurgel OK

\definechangesauthor[name={Bruno Esteves Dalla Costa Filho}, color=red]{brunoedcf} % git-user: brunoedcf OK

\definechangesauthor[name={Paulo Mauricio Costa Lopes}, color=red]{RequiemDosVivos} % git-user: RequiemDosVivos OK

\definechangesauthor[name ={Caio Bernardon Nascif Kaawi Massucato}, color=blue]{CaioMassucato} % git-user: CaioMassucato OK 

\makenoidxglossaries
\loadglsentries{1-Introducao/tarefas/1.1-Glossario/estudantes/main}

\setcounter{tocdepth}{5}
\setcounter{secnumdepth}{5}
\captionsetup[table]{name=Quadro}
\renewcommand{\lstlistingname}{Listagem de Código}

\newcommand{\dataset}{\textit{dataset}}
\newcommand{\query}{\textit{query}}
\newcommand{\githubusername}{\textless githubusername\textgreater}

\begin{document}
\pagenumbering{gobble}% Remove page numbers (and reset to 1)
\clearpage
\thispagestyle{empty}

\begin{titlepage}
\begin{center}
 {\huge\bfseries \includegraphics[width=4cm]{unb-logo.jpg}\\
	CIC0203 - Computação Experimental - TA - 2021.2\\\url{https://www.overleaf.com/project/618e9b4b0db7234d6d9fbfc0}\\
	Notas e Registros\\}
 % ----------------------------------------------------------------
 \vspace{1.5cm}
{\large	% author names and affiliations
	Jorge Henrique Cabral Fernandes}\\
	André Larrosa Chimpliganond\\
	Alexsander Correa de Oliveira\\
	Allann Gois Hoffmann\\
	André Cássio Barros de Souza\\
	Arthur da Silveira Couto\\
	Bruno Esteves Dalla Costa Filho\\
	Bruno Sanguinetti Regadas de Barros\\
	Caio Bernardon Nascif Kaawi Massucato\\
	Conrado Nunes Barbosa Neto\\
	Daniel Rodrigues Cardoso\\
	Enzo Nunes Leal Sampaio\\
	Enzo Yoshio Niho\\
	Fernanda Macedo de Sousa \\
	Fernando Ferreira Cordeiro \\
	Felipe Gomes Paradas\\
	Gabriel dos Santos Martins\\
	Gabriel Faustino Lima da Rocha\\
	Gabriel Martins de Almeida\\
	Gabriel Rocha Fontenele\\
	Gustavo Macedo de Carvalho\\
	Gustavo Rodrigues dos Santos \\
	Gustavo Tomás de Paula\\
	Ítalo Eduardo Dias Frota\\
	João Pedro Felix de Almeida\\
	João Víctor Siqueira de Araujo\\
	Jônatas Gomes Barbosa da Silva\\
	Léo Akira Abe Barros\\
	Leonardo Alves Riether\\
	Lucas de Almeida Bandeira Macedo\\
	Lucas Gabriel de Oliveira Gurgel Fernandes \\
	Mateus de Paula Rodrigues\\
	Marcus Vinicius Oliveira de Abrantes\\
	Paulo Mauricio Costa Lopes\\
	Pedro de Torres Maschio\\
	Rafael dos Santos Silva\\
	Rafael Henrique Nogalha de Lima\\
	Raylan da Silva Sales\\
	Stefano Luppi Spósito\\
	Tatiana Franco Pereira\\
	Tong Zhou\\
	Ualiton Ventura da Silva\\
	Vinícius Caixeta de Souza\\
	Guilherme Oliveira Loiola\\
	Artur Filgueiras Scheiba Zorron\\
	Vitor de Oliveira Araujo Araruna \\
	\vspace{1.5cm}
	{\large Brasília, \DTMnow}
\end{center}
\end{titlepage}
	%\date{10 de março de 2021}
	% make the title area
%	\maketitle
    \listoftodos
	\printnoidxglossary
	\tableofcontents
	\listoffigures
	\listoftables
	\clearpage
\pagenumbering{arabic}

\pagestyle{fancy}

	\chapter*{Resumo}

	Este documento contém notas de aula e registros em geral, produzidos de forma textual, pelo professor e turma da disciplina CIC0203 - Computação Experimental - TA - 2021.2.
	
	A disciplina adotará a abordagem de estudos estatísticos de simulações computacionais, como base para a construção de experimentos.

\chapter{Orientações Iniciais}

Leia atentamente as orientações a seguir, tendo em vista que lhe auxiliarão no melhor desempenho neste curso-disciplina. 

\section{Importância deste documento}

Este documento contém o registro de todas as evidências de aprendizagem suas, e dos demais estudantes que fazem parte do curso Computação Experimental.
Quando for editar este documento tome cuidado para sempre deixá-lo em plena condição compilável, e sem erros. Busque também resolver os \textit{warnings} causados pelas suas edições.

Todos os pacotes de trabalho que você depositar deverão ser acessíveis, de forma direta ou indireta, a partir do arquivo ``main.tex'', no diretório raiz.

\section{Uso sincronizado de Repositório no Github e Overleaf}

Todo o código deste documento está na plataforma Overleaf.com, mantido sincronizado como o repositório CE-20212 na plataforma Github.com, disponível na url \url{https://github.com/jhcf/Comput-Experim-20212}, que é o repositório \textbf{origin}.


O acesso de gravação no Repositório ser-lhe-á concedido pelo professor do curso ao final da primeira semana de aula.
Dessa forma, para poder realizar a disciplina todos os estudantes devem ter uma conta pessoal em github.com, bem como uma conta pessoal em overleaf.com. Inicie registrando o seu nome completo e \textbf{github username} no arquivo autores.tex, onde você também deve escolher uma cor para uso no registro de alterações, conforme disponível no pacote ``changes''.

\section{Ingresso no grupo Zotero ``RESIC'', para  compartilhar referências bibliográficas com a turma}

Você deve cadastrar-se como usuário do Zotero, registrar-se no grupo Zotero de compartilhamento de referências bibliográficas RESIC.
A inscrição no grupo é feita no endereço abaixo:
\url{https://www.zotero.org/groups/2465026/resic}

\begin{quote}
Opcionalmente, você deve instalar a versão \textit{standalone} do Zotero em uma máquina de seu uso, a fim de ter acesso facilitado à base de referências bibliográficas a ser usada durante a execução da disciplina. Caso não instale, poderá acessar a base diretamente pelo navegador.
\end{quote}

Após ingressar no grupo, visite a referência ao livro ``Complex Adaptive Systems: An introduction to computational models of social life'', que está dentro da coleção ``Complexity -> Complex Adaptive Systems (CAS)''.

Busque acessar o livro completo ou o seu sumário, como disponível em \url{https://www.google.com.br/books/edition/Complex_Adaptive_Systems/XQUHZC8wcdMC}. Se possível adquira o livro em formato digital ou impresso. Ele será usado como referência para discussões avançadas sobre simulação, mais adiante na disciplina.

Veja o que contém o sumário do livro, parte do prefácio e introdução, e use o Zotero para adicionar uma pequena nota com sua impressão inicial sobre o livro, vinculada a essa referência. A Nota deve conter o seu nome completo e qualquer observação que você ache relevante, sobre o livro referenciado.

Depois de registrar uma nota vinculada à referência do livro citado, use o Zotero para exportar essa referência no formato / estilo ABNT [use a opção ``Create Bibliography'' no Zotero online, ou ``criar bibliografia a partir do item selecionado''  no Zotero standalone], e deposite o texto da referência formatada no texto online da tarefa T1 (ver Moodle/Aprender da disciplina).

\section{Depósito dos pacotes de trabalho referentes à execução das tarefas}

A execução de toda e qualquer tarefa pontuável neste curso é feita por meio do depósito de um pacote de trabalho no repositório, contendo textos, programas de computador e outros dados coletados e (ou) analisados. Todos os textos e códigos no repositório devem estar em formato não compactado. Apenas os arquivos de dados muito grandes devem ser compactados em formato zip. Não usar rar, gz etc. 

Todo pacote de trabalho a ser avaliado precisa estar integralmente armazenado no Repositório CE-20212 / origin (e consequentemente sincronizável com o Overleaf).

\subsection{Onde depositar os pacotes de trabalho?}

Toda e qualquer inserção de texto, programa de computador, dados, enfim, qualquer documento, feito por estudante, deve ocorrer em um dos seguintes pontos:
\begin{enumerate}
    \item Dentro de um subdiretório com o código da tarefa, no diretório de ``experiments'' do estudante, onde o nome do diretório de \texttt{experiments} de um estudante é o seu github username (veja, por exemplo, o professor, que tem como github username: jhcf);
    \item Dentro dos diretórios ``tarefas'', nos diretórios dos temas de estudo;
    \item No arquivo ``autores.tex'';
    \item No arquivo ``packages-estudantes.tex'', onde eventualmente podem ser inseridos novos pacotes para apoiar o uso de algum recurso específico; e
    \item Na substituição do arquivo RESIC.bib por outro mais recente, obtido pela exportação completa da biblioteca RESIC que se encontra na plataforma Zotero, na url \url{https://www.zotero.org/groups/2465026/resic}.
\end{enumerate}

Não serão avaliados os pacotes de trabalhos entregues em local distinto do especificado, ou não acessíveis por meio do output em PDF, resultante da compilação de ``main.tex''.

\section{Entrega das tarefas}

Toda pontuação concedida ao/à estudante será feita mediante:
\begin{enumerate}
    \item Sincronização plena entre o documento no Overleaf.com e o repositório no Github.com;
    \item O registro de execução da correspondente tarefa no ambiente Moodle do curso, feita pelo estudante até o limite de prazo informado;
\end{enumerate}

Se a tarefa é feita em grupo, cada um dos membros do grupo deve obrigatoriamente fazer o registro da execução da tarefa, com as mesmas informações. Os membros do grupo devem evitar qualquer duplicação de código e textos no Overleaf. Ou seja, depositar os pacotes de trabalho referentes a cada tarefa em apenas um dos diretórios de experimentos de um dos membros do grupo. Até o final do curso, todos os membros de um grupo devem aparecer como produtores de pelo menos um pacote de trabalho feito pelo grupo.

\section{Escrevendo o texto}

Cuidados ao escrever texto:
\begin{description}
\item [Reconhecimento de autorias] Faça citações a textos e ideias que não são de sua autoria, usando referências registradas no grupo Zotero RESIC. Sempre informe os nomes completos dos autores dos relatórios, logo após o título do relatório;
\item [Grafia] A boa ortografia e gramática são essenciais à valorização de um trabalho. Descuido com essa questão revela, de forma geral, descuido e (ou) desinteresse pelo próprio trabalho, influenciando a pontuação obtida;
\item [Referências a códigos e dados] Toda e qualquer de escrita de relatório deverá fazer referência explícita ao diretório no Repositório, onde se encontram os códigos e dados usados para produção do relatório. No ambiente Moodle não será aceita a entrega de arquivos compactados contendo os resultados de realização das tarefas;
\end{description}

\section{Organização do ambiente experimental}

A organização do ambiente experimental de cada estudante é essencial para a realização adequada das tarefas.
Alguns dos problemas típicos que refletem descuido com a organização do trabalho de laboratório, são:

\begin{enumerate}
\item Problemas com merge durante operação no repositório git;
\item Uso inadequado de caracteres com acentuação nos nomes dos arquivos;
\item Não informar os ponteiros adequados, seja por meio das tags \LaTeX~ input e (ou) includepdf, de modo que a tarefa fica inacessível e invisível na tabela de conteúdos gerada a partir da compilação do documento main.tex.
\end{enumerate}

\subsection{Problemas com merge durante operações de push no Git}

Cada estudante é inteiramente responsável por fazer os merges manuais para entrega de sua tarefa no Repositório, caso o branch correspondente ao seu pacote de trabalho não consiga ser feito de forma automática. 

Estudantes que deixarem merges em conflito, prejudicando o processo de envio dos trabalhos dos demais, poderão perder pontos pela não entrega do trabalho no prazo, bem como pelo conflito causado para os demais estudantes do curso-disciplina. 

Busque informações sobre como resolver merges no livro Pro Git \cite{chacon_pro_2014}, ou em  urls como \url{https://www.zotero.org/groups/2465026/resic/collections/C7BG9S2W/items/HCY5X8PT/note/9M2BFISL/collection}

\subsection{Uso inadequado de nomes para arquivos}

Devido ao fato de que estaremos trabalhando em um ambiente laboratorial compartilhado, com uso de muitas linguagens combinadas, como \LaTeX, Python e R, entre outras, é fundamental adotar um padrão de nomeação para arquivos e diretórios. Atualmente esse padrão é composto pelas seguintes regras:
\begin{enumerate}
    \item Não use caracteres de acentuação em nomes de arquivos;
    \item Não use espaço em branco em nomes de arquivos;
    \item Não use underline em nomes de arquivos. Usar hífen em vez de Underline;
    \item Para nomes de arquivos longos use a \url{https://pt.wikipedia.org/wiki/CamelCase}, e eventualmente misture com o uso de hifens, para melhor legibilidade.
\end{enumerate}

\subsection{Não colocar os ponteiros adequados para que a tarefa seja visível}

É necessário que todo trabalho a ser avaliado esteja visível a partir da tabela de conteúdos gerada pela compilação do documento main.tex, sejam em um capítulo, seção ou subseção em \LaTeX.



\part{Introdução\label{part:intro}}

    \chapter{Aula/Texto de Apoio: O que é a Ciência?}

    
Existem várias definições sobre o que é a ciência. Algumas questões sobre essa definição são apresentadas por \citep{fernandes_consideracoes_2021}, cujo texto está reproduzido em ~\ref{sobre:ciencia}. 




	\includepdf[pages=-]{1-Introducao/aulas/Ciencia-e-sua-Avaliacao} %o link permite que isso possa ser citado - DYosplay


    \chapter{Tarefas: Glossário e impressões Iniciais Sobre a Ciência}
    
    \section{Tarefa: Criar um item no glossário deste documento \label{tarefa:glossario}}

Registre, no glossário do relatório da disciplina, uma definição de um termo relacionado com o texto de considerações preliminares sobre a ciência, de autoria do professor, em para ``1-Introducao/aulas/Consideracoes-Preliminares-Sobre-a-Ciencia-e-sua-Avaliacao.pdf''.

O diretório onde o arquivo deve ser criado é: 
1-Introducao/tarefas/1.1-Glossario/estudantes

O nome do arquivo á ser criado por você deve ter a forma:
tarefa-\githubusername.tex, e deve seguir o modelo da tarefa exemplo do professor, em
1-Introducao/tarefas/1.1-Glossario/estudantes/tarefa-jhcf.tex

O texto do item de glossário que você vai criar deve ser escrito em língua portuguesa, e conter uma definição para o termo e um exemplo de caso concreto.

A definição deve referenciar pelo menos um item bibliográfico presente no grupo Zotero.

A sua resposta a essa atividade vale até 3\% da pontuação total da disciplina.
Veja como exemplo a resposta do professor, na primeira seção.


 
    \section{Tarefa 3: Registrar suas impressões iniciais sobre a ciência, citando pelo menos  um item no glossário}

Cada estudante, a partir do que já sabe e leu, deve criar no diretório a seguir uma seção com o seu nome, a fim de apresentar um ou dois parágrafos de sua autoria, apresentando as suas \textbf{Impressões iniciais sobre o que é a Ciência}: 

O diretório onde o arquivo deve ser criado é: 
1-Introducao/tarefas/1.2-Impressoes-Ciencia/estudantes

O nome do arquivo á ser criado deve ter a forma:
tarefa-\githubusername.tex, e deve seguir o modelo da tarefa exemplo do professor, em
1-Introducao/tarefas/1.2-Impressoes-Ciencia/estudantes/tarefa-jhcf.tex

No registro de suas impressões iniciais no texto da tarefa, você precisa:
\begin{itemize}
    \item Usar um ou mais dos itens do glossário, incluído o que você criou na tarefa \ref{tarefa:glossario};
    \item Usar uma ou mais citações a referências bibliográficas usando a tag \verb|\citet{}| ou \verb|\citep{}|. Não use a tag \verb|\url{}|. Note que todas as referências citadas devem estar registradas no arquivo RESIC.bib, gerado a partir da bibliografia no grupo RESIC em \url{https://www.zotero.org/groups/2465026/resic}, ao qual você deve ter acesso, como feito em tarefa anterior.
\end{itemize}

A sua resposta a essa atividade vale até 3\% da pontuação total da disciplina.
Veja como exemplo a resposta do professor, na primeira seção.

\chapter{Análise Bibliográfica sobre Simulação Multiagente e Fenômenos Sociais, por Jorge Fernandes\label{chap:bibliometria:jhcf}}

\section{Planejamento do estudo}
O planejamento o  desenho do estudo deve descrever as motivações, questões de interesse, escopo, limitações e objetivos do trabalho.

O planejamento do estudo deve motivar o tema escolhido e o interesse do autor.

No caso do meu trabalho, as perguntas que o nortearam foram:
\begin{itemize}
    \item Qual a base de conhecimentos científicos produzida em torno do tema simulação multiagente voltada à compreensão de fenômenos sociais, com ênfase em métodos experimentais? 
    \item Como a simulação multiagente tem sido usada para compreender fenômenos sociais, com ênfase em métodos experimentais? 
    \item Quais os principais termos e conceitos ligados à frente de pesquisa no tema simulação multiagente de fenômenos sociais, com ênfase em métodos experimentais? 
    \item Qual a estrutura social da comunidade, se é que existe, que pesquisa sobre o tema simulação multiagente de fenômenos sociais, com ênfase em métodos experimentais?
\end{itemize}

\subsection{O que já existe de pesquisa bibliométrica sobre esse tema?}

\cite{gore_classifying_2016} fizeram uma pesquisa que visava aprofundar a questão da simulação multiagente em relação à computação experimental.

A pesquisa é base para um posterior aprofundamento no campo da Cientometria, como fez \cite{chavalarias_whats_2017}.

\subsection{Uso do Bibliometrix e Biblioshiny}
Serão usadas a ferramenta e o \textit{workflow} proposto pelos autores do pacote Bibliometrix, conforme indica a figura ~\ref{fig:bibliometrix:workflow}.

\subsection{Limitações} O exercício relatado foi feito em uma semana, envolvendo entre 5 a 10 horas de trabalho de cada autor.

Outros aspectos a reforçar:
\begin{itemize}
   
\item Deve-se fazer buscas na base de dados WoS ou SCOPUS;
\item é obrigatório declarar um conjunto de perguntas de pesquisa.
\item é preciso declarar o objetivo da pesquisa, que no caso da aqui relatada foi exercitar inicialmente, e relatar, o uso da técnica de análise bibliométrica, para fins didáticos.
\end{itemize}


\section{Coleta de dados\label{MASSA:coleta}}

A coleta de dados feita usando o WoS no dia 03 de agosto de 2021, acessado por meio do Portal de Periódicos da CAPES.

Foram feitas buscas nas coleções \textbf{Science  Citation  Index  Expanded (SCI -EXPANDED)} e \textbf{Social  Sciences  Citation  Index (SSCI)}, que contém registros relativos a vários campos do conhecimento, no qual o SCI-EXPANDED foca mais na área das ciências exatas e naturais, enquanto que o SSCI indexa artigos da área das ciências sociais. Observe que os artigos nessas duas coleções são indexados desde 1945. 

Foi usada a \query\  de busca ilustrada nas linhas 1 a 9 da listagem \ref{query20210803-2}.

\lstinputlisting[numbers=left,basicstyle=\normalsize\ttfamily,caption={\query\  de busca sobre simulação multiagente de fenômenos socials, com ênfase em métodos experimentais.},label=query20210803-2]
{experiments/jhcf/PesqBibliogr/SimulacaoMultiagente/WoS-20210803/classico-mais-citacoes/query.txt}

\subsection{Explicação para os termos de busca usados\label{MASSA:query}}

A busca consistiu de quatro cláusulas disjuntivas, unidas por uma conjunção \textit{and}, aplicadas à busca por tópico (O termo de busca pode aparecer no Título, no Abstract, na Author Keywords, ou nas Keywords Plus da referência)

Os termos \texttt{experimental}, \texttt{numeric*}, \texttt{statist*}, \texttt{hypothes*}, 
\texttt{empiric*}
e \texttt{inferen} (linhas 1 e 2 da query) foram usados na primeira cláusula da \query\  para recuperar artigos que tenham em seu título, palavras-chave e resumo, termos relacionados a métodos experimentais,
métodos numéricos,
métodos estatísticos,
teste de hipóteses,
métodos empíricos e métodos inferenciais.

O termo / cláusula  \texttt{simul*}, na linha 4, foi usado em conjunção com os demais para recuperar apenas trabalhos que explicitem o uso da simulação.
Foi usado um único termo devido à forte adesão ao termo simulação por parte dos pesquisadores que usam simulação. Não existem outros sinônimos frequentes para esse uso.

A cláusula nas linhas 6 e 7 faz união entre o uso dos termos \texttt{agent} e \texttt{multiagent}, \texttt{multi-agent},e  também \texttt{multi and agent}, para cobrir as variadas formas de escrita do conceito.

A $4^{a}$ cláusula, linha 9,  usou os termos \texttt{social} e \texttt{society} para recuperar artigos que tratem de temas ligados à sociedade.
Os termos \texttt{group} e \texttt{behavi*} visam recuperar estudos que tratam de questões comportamentais e grupais.

Os 8.115 registros obtidos encontram-se no github do projeto, em \url{https://github.com/jhcf/Comput-Experim-20212/experiments/jhcf/PesqBibliogr/SimulacaoMultiagente/ WoS-20210803/classico-mais-citacoes/8115recs.txt}. 

Foram utilizadas as opções \textit{Exportar registros para arquivo de texto sem formatação} e \textit{export full record / Gravar Conteúdo: Seleção personalizada, com todos os 29 campos disponíveis, inclusive referências citadas} no WoS, para que as citações também fosse usadas em análises da citações (estrutura intelectual do conhecimento). Os 8115 registros foram recuperados em nove blocos de até 1.000 registros por vez (1-1000, 1001-2000, 2001-3000, ..., 8001-8115).

\section{Análise dos dados}

\subsection{Filtragem de registros}
Antes da análise, é possível aplicar filtros sobre os registros obtidos.

Foi aplicado um filtro ao \dataset\   inicial, com 8.115 registros, que continham pŕevias de artigos, artigos de conferência, capítulos de livro etc. Foram mantidos apenas os registros de artigos publicados em revistas científicas\footnote{A suposição é que que o conhecimento de maior qualidade sobre o tema está nas publicações em revistas.}. Após a aplicação desse filtro, 5.787 registros foram mantidos no \dataset, que será doravante chamado MultiAgentSimulationSociety/Artigos, ou MASSA@jhcf.

\subsection{Análise descritiva do \dataset\   MASSA@jhcf}

A análise bibliométrica descritiva faz uma descrição inicial do \dataset\  . Para explicação detalhada de como são calculadas as diversas taxas geradas pelo Bibliometrix veja a documentação do \textit{package} a partir da página \url{https://cran.r-project.org/web/packages/bibliometrix/index.html}. A análise bibliométrica descritiva é gerada pela função \texttt{biblioAnalysis}.

As informações mais gerais sobre o \dataset\   MASSA@jhcf são as seguintes:
\begin{description}
    \item [\textit{Timespan}] Os artigos que atenderam aos critérios de busca e filtragem foram publicados a partir de 1990, até 2021. Ou seja, não foram encontrados registros entre 1945 e 1989.
    \item [\textit{Sources (Journals, Books, etc)}] São 2.319 fontes de informação que publicaram os documentos recuperados no \dataset\   MASSA@jhcf. Ou seja, em média, cada \textit{scientific journal} publicou $5.787/2.319=2,5$ artigos. \footnote{Note que a média, enquanto medida de tendência central, pode não ser a que melhor reflete a tendência a quantidade de artigos publicados por revista.}
    \item [\textit{Average years from publication}] A média do tempo de publicação dos artigos no \dataset\   MASSA@jhcf é de 7,36 anos.
    \item [\textit{Average citations per documents}] Cada artigo no \dataset\   MASSA@jhcf foi citado, em média 20,7 vezes\footnote{Note que a média, enquanto medida de tendência central, pode não ser a que melhor reflete a tendência de  citações a artigos.}.
    \item [\textit{Average citations per year per doc}] Após publicado, cada um dos 5.787 artigos do \dataset\   MASSA@jhcf  foi citado 2,262 vezes por ano, em média.
    \item [\textit{References}] O \dataset\   MASSA@jhcf contém 201.464 referências citadas (tags CR).
    \item [\textit{Keywords Plus (ID)}] 13.735 distintas palavras-chave do tipo Keywords Plus (ID)\footnote{\textit{KeyWords Plus} são ``termos de índice gerados automaticamente a partir dos títulos de artigos citados. Os termos do KeyWords Plus devem aparecer mais de uma vez na bibliografia e são ordenados de frases com várias palavras a termos únicos. O KeyWords Plus aumenta o número de resultados tradicional de palavras-chave ou títulos.'' Fonte: \url{https://images.webofknowledge.com/WOKRS410B4/help/pt_BR/WOS/hp_full_record.html}} foram encontradas no \dataset\   MASSA@jhcf. 
    \item [\textit{Author's Keywords (DE)}] 15.704 distintas palavras-chave indicadas pelos autores foram encontradas no \dataset\  .
    \item [\textit{Authors}] 19.410 distintos nomes de autores foram encontrados no \dataset\  \footnote{Um mesmo autor pode ter uma ou mais diferentes grafias no \dataset\  , e serão reconhecidos dois ou mais autores diferentes, embora de fato sejam apenas um. Isso significa que a quantidade de \textbf{nomes de autores} equivale à quantidade de \textbf{autores}. Adicionalmente, é possível que distintos autores sejam reconhecidos com o mesmo nome, isso é, que sejam homônimos. Ou seja, o \dataset\   em geral conterá erros de contagem na quantidade de autores reais.}.
    \item [\textit{Author Appearances}] Os 19.410 distintos (nomes de) autores foram encontrados 23.470 vezes, como autores de artigos.
    \item [\textit{Authors of single-authored documents}] Dentre os 19.410 distintos (nomes de) autores encontrados, 375 deles editaram artigos individualmente, isso é, sem co-autores.
    \item [\textit{Authors of multi-authored documents}] Dentre os 19.410 distintos (nomes de) autores encontrados, 19.035 deles editaram artigos com um ou mais co-autores"
    \item [\textit{Single-authored documents}] Dentre os 5.787 documentos presentes no \dataset\   MASSA, 409 foram escritos por um único autor, e os 5.378 restantes foram elaborados em co-autoria.
    \item [\textit{Documents per Author}] Dentre os 19.410 distintos (nomes de) autores, cada um publicou em média 0,298 artigos.
    \item [\textit{Authors per Document}] Cada um dos 5.787 documentos presentes no \dataset\   MASSA foi autorado com 3,35 autores em média ($19.410 / 5.787 = 3,35$).
    \item [\textit{Co-Authors per Documents}] As 23.470 aparições de (nomes de) autores (``Author Appearances''), sem distribuem, em média 4,06 vezes para os 5.787 documentos do \dataset\   MASSA@jhcf.
    \item [\textit{Collaboration Index}] Os 19.035 (nomes de) autores que editaram artigos com um ou mais co-autores, colaboraram em media 3,54 vezes para editar os 5.378 artigos elaborados em co-autoria, gerando, assim, um índice de colaboração 3,54. 
\end{description}

\subsection{Evolução da Produção Científica}

\begin{figure}
    \centering
    \includegraphics[width=1\textwidth]{experiments/jhcf/PesqBibliogr/SimulacaoMultiagente/WoS-20210803/classico-mais-citacoes/Dataset/AnnualScientificProduction-2021-08-05.png}
    \caption{Evolução da produção científica no \dataset\   MASSA@jhcf.}
    \label{fig:evol:anual:MASSA@jhcf}
\end{figure}

A figura \ref{fig:evol:anual:MASSA@jhcf} apresenta a evolução da produção científica mundial no tema de interesse, segundo o \dataset\   MASSA@jhcf. A curva mostra uma tendência de crescimento aproximadamente exponencial da quantidade de publicações, desde a primeira identificada em 1990.

O \textit{Annual Growth Rate} do \dataset\   é de 17,06\%, bem maior que a taxa média de crescimento da publicação científica mundial, de cerca de 3,3\% anuais, em 2016, como ilustra o estudo em \url{https://www.researchgate.net/publication/333972683_Dynamics_of_scientific_production_in_the_world_in_Europe_and_in_France_2000-2016}, página 23.

\subsection{Interpretação do Crescimento} A maior taxa de crescimento do \dataset\   MASSA@jhcf, bem como o seu grande volume, sugerem que o assunto em pauta desperta intenso interesse, inclusive de ordem econômica.

\subsection{Evolução das Citações}

\begin{figure}
    \centering
    \includegraphics[width=1\textwidth]{experiments/jhcf/PesqBibliogr/SimulacaoMultiagente/WoS-20210803/classico-mais-citacoes/Dataset/AverageArticleCitationPerYear-2021-08-09.png}
    \caption{Evolução das citações ao \dataset\   MASSA@jhcf.}
    \label{fig:evol:anual:citacoes:MASSA@jhcf}
\end{figure}

A figura \ref{fig:evol:anual:citacoes:MASSA@jhcf} apresenta a evolução da média de citações aos 5.787 artigos no \dataset\   MASSA@jhcf. 
Nota-se grande estabilidade na média anual de citações, onde os artigos publicados em 1992 possuem cerca de 2 citações médias, e em 2015 (17 anos depois) o valou alterou-se apenas para três. O pico que aparece no ano de 2008 deve-se, possivelmente, à presença de um artigo do \dataset, publicado em 2008, que possui um número surpreendente grande de citações. \footnote{Note que o cálculo do número  médio de citações, nesse caso, utiliza os valores computados no tag "TC (Times Cited)", já presentes no \dataset\   obtido. Ou seja, o gráfico baseia-se no número de citações globais (externas ao \dataset\   MASSA@jhcf), e não no número de citações locais (citações a um artigo do \dataset\   feitas por alguns dos outros artigos dentro do próprio \dataset).}.

\subsection{Interpretação das Citações}
Mesmo perante um crescimento aproximadamente exponencial no volume de publicações, a ocorrência de um crescimento nas citações médias ao longo dos anos sugere que os artigos do \dataset\   possuem uma tendência de crescimento no tamanho da bibliografia citada, bem como também despertam grande interesse dos cientistas nas demais áreas do conhecimento (já que se trata de citações globais).

\subsection{\textit{Three-Field Plots (Sankey diagram)}}

As \textit{Three-Field Plots (Sankey diagram)} (plotagens do tipo ``Três Campos'') apresentam afinidades entre três conjuntos de atributos agregados que ocorrem no \dataset. Uma plotagem do tipo Sankey busca mostrar os principais fluxos entre diferentes conjuntos de itens. \footnote{Para uma introdução ver \url{https://en.wikipedia.org/wiki/Sankey_diagram}. Para obter detalhes sobre a forma de geração e utilização desse gráfico, inclusive de forma interativa, veja o vídeo em \url{https://www.youtube.com/watch?v=jBb1iha6-sg}.} 

\begin{figure}
    \centering
    \includegraphics[angle=0,width=1\textwidth]{experiments/jhcf/PesqBibliogr/SimulacaoMultiagente/WoS-20210803/classico-mais-citacoes/Dataset/ThreeFieldPlot-AU-CR-DE-20-20-20.png}
    \caption{Plotagem ``Três Campos'' (Sankey plot) do \dataset\   MASSA@jhcf: 20 Autores, Citações e Palavras-Chave mais proeminentes.}
    \label{fig:MASSA@jhcf:ThreeFieldPlot}
\end{figure}

A figura \ref{fig:MASSA@jhcf:ThreeFieldPlot} apresenta a plotagem do tipo ``Três Campos'' do \dataset\   MASSA@jhcf, vinculando, ao centro, os 20 Autores mais proeminentes (AU), à esquerda, as 20 Citações mais frequentes (CR - Cited Records), e à direita, as 20 Palavras-Chave mais frequentes empregadas pelos autores.

\subsection{Interpretação da figura \ref{fig:MASSA@jhcf:ThreeFieldPlot}}
Os vinte autores mais relevantes, em relação aos artigos mais relevantes citados, e as palavras-chave mais relevantes são aparentemente de origem asiática, mais especificamente chinesa, com base nos sobrenomes. De outra formal, a mesma origem chinesa parece não se aplicar aos trabalhos mais citados, aparentemente europeus ou norte-americanos. Isso sugere estar ocorrendo uma migração recente da produção científica, do ocidente para o oriente. 

Adicionalmente, dentre as palavras-chave (DE) não relacionadas diretamente aos termos de busca, emergem os termos \textbf{distributed control}, \textbf{event-triggered control}, \textbf{consensus} e \textbf{opinion dynamics}. Isso sugere foco das pesquisas por autores de origem chinesa no uso de simulação multiagente voltada à compreensão dos fenômenos de controle social distribuído, formação de consenso e dinâmica da opinião (pública?).

Ainda sobre a interpretação da plotagem da figura \ref{fig:MASSA@jhcf:ThreeFieldPlot}, observa-se que os artigos mais citados encontram-se publicados pelo menos 10 anos atrás, sugerindo que não houve, nos últimos 10 anos, nenhum trabalho que tenha produzido uma mudança de paradigma no tema.
A fim de melhor evidenciar as citações mais relevantes segundo o peso dos autores e palavras-chave, o gráfico da figura \ref{fig:MASSA@jhcf:ThreeFieldPlot:10-20-20} plota apenas as 10 referências citadas, para 20 autores e palavras-chave mais proeminentes.

\begin{figure}
    \centering
    \includegraphics[angle=0,width=1\textwidth]{experiments/jhcf/PesqBibliogr/SimulacaoMultiagente/WoS-20210803/classico-mais-citacoes/Dataset/ThreeFieldPlot-AU-CR-DE-20-10-20.png}
    \caption{Plotagem ``Três Campos'' (Sankey plot) do \dataset\   MASSA@jhcf: 10 Autores, 20 Citações e Palavras-Chave mais proeminentes.}
    \label{fig:MASSA@jhcf:ThreeFieldPlot:10-20-20}
\end{figure}

Breves comentários sobre cada um desses trabalhos serão tratados em seção posterior.

\begin{itemize}
    \item  \cite{olfati-saber_consensus_2004} apresentam discussões teóricas sobre a formação de consenso em sistemas multi-agentes com topologias variáveis;
    \item  \cite{reynolds_flocks_1987} apresenta modelos multi-agentes para simulação gráfica do movimento de rebanhos ou agregados de animais.
    \item \cite{vicsek_novel_1995} analisam a emergência de fenômenos de transição de fase em simulações de de partículas com comportamento autônomo com interação biologicamente motivada.
    \item \cite{barabasi_emergence_1999} investigam a emergência da distribuição livre de escala (\textit{scale-free}\footnote{Ver introdução em \url{https://en.wikipedia.org/wiki/Scale-free_network}.}) em redes que evoluem com base em ligação preferencial.
    \item \cite{watts_collective_1998} exploram o surgimento de redes do tipo mundo pequeno (\textit{small world}\footnote{Ver introdução em \url{https://en.wikipedia.org/wiki/Small-world_network}.}) formadas a partir da reorganização aleatória de redes biológicas, genéticas e outras formas de redes auto-organizadas.
    \item \cite{castellano_statistical_2009} exploram de que forma as técnicas de análise e simulação já usadas na física-estatística podem ser usadas para explicar vários fenômenos sociais, tais como comportamento de multidões, dispersão social, comportamento de multidões etc. Eles apresentam as afinidades entre os dados gerados pelos modelos simulados e dados empíricos obtidos junto a sistemas sociais reais. 
    \item \cite{hegselmann_opinion_2002} exploram a emergência de fenômenos de consenso, polarização e fragmentação da opinião na simulação de sociedades artificiais.
    \item \cite{bonabeau_agent-based_2002} apresenta os potenciais e campos de aplicação da técnicas de simulação baseada em agentes.
    \item \cite{wilensky_netlogo_1999} apresentam a linguagem e ambiente de simulação NetLogo.
    \item \cite{grimm_standard_2006} apresenta o protocolo ODD, proposto para padronizar a descrição de modelos de simulação multiagente.
\end{itemize}

Nenhum desses 10 documentos citados está contido no \dataset\   recuperado.

%\subsection{Análises Bibliométricas: Fontes de Informação}

%\begin{figure}
%    \centering
%    \includegraphics[angle=0,width=1\textwidth]{}
%    \caption{Plotagem ``Três Campos'' (Sankey plot) do dataset MASSA@jhcf: 20 Autores, Citações e Palavras-Chave mais proeminentes.}
%    \label{fig:MASSA@jhcf:ThreeFieldPlot}
%\end{figure}

\section{Refinamento da Coleta de Dados}

No dia 03 de fevereiro de 2022, no decorrer das análises mais refinadas do \dataset\ MASSA@jhcf, identificou-se um grupo de artigos que não se encaixavam no tema de interesse, e que eram voltados para pesquisas no campo da biologia experimental e nanotecnologia. Isso sugeriu que a \query\  de busca precisaria ser reformulada, para excluir artigos que não se enquadrassem na temática desejada.
O conjunto das palavras-chave que refletia essa dissonância ficou evidente na análise da estrutura intelectual do conhecimento, do tipo \textbf{Rede de Co-ocorrências de Palavras-chave}, ilustrada no cluster em roxo, à esquerda da figura \ref{fig:MASSA@jhcf:redecoocorr-150-termos}.

\begin{figure}[htp]
    \centering
    \includegraphics[clip=true,trim={9cm 0cm 7cm 0cm },width=0.6\textwidth]{experiments/jhcf/PesqBibliogr/SimulacaoMultiagente/WoS-20210803/classico-mais-citacoes/Structure-Informetric/Conceptual/Co-occurrence Network-Keywords-Plus-150-termos.png}
    \caption{Rede de co-ocorrência de palavras, com 150 termos, aplicada ao \dataset\   MASSA@jhcf.}
    \label{fig:MASSA@jhcf:redecoocorr-150-termos}
\end{figure}

As seguintes 30 palavras foram identificadas nesse \textit{cluster}:
in-vitro,
adsorption,
mechanism,
water,
force-field,
molecular-dynamics,
binding,
simulations,
nanoparticles,
bubbles,
derivatives,
temperature,
in-vivo,
mathematical-model,
oscillations,
scattering,
cancer,
contrast agents,
expression,
protein,
activation,
delivery,
surface,
removal,
acid,
agent,
reduction,
aqueous-solution,
degradation,
expectations.

Ficou evidente, pela interpretação do significado da maioria desses termos, que tais artigos não tratavam de simulação de fenômenos sociais. Isso sugere que a query está com problemas de precisão, isso é, muitos registros recuperados não atendem à necessidade de informação do pesquisador. 

Algumas dessas palavras foram então escolhidas para servir como indicativas de artigos fora do escopo, e introduzidas a partir da \query\  original, gerando uma nova \query, aprimorada e ilustrada nas linhas 1 a 13 da listagem \ref{query20220203}.

\lstinputlisting[numbers=left,basicstyle=\normalsize\ttfamily,caption={\query\  de busca sobre simulação multiagente de fenômenos socials, com ênfase em métodos experimentais, com escopo negativo de artigos que tratam de experimentos biológicos em vitro.},label=query20220203]
{experiments/jhcf/PesqBibliogr/SimulacaoMultiagente/WoS-20220203/query-Refinada.txt}

Além das justificativas para os termos usados entre as linhas 1 a 9, já descritas em \ref{MASSA:query},  justifica-se na listagem \ref{query20220203}, a inclusão da cláusula \textit{not (
 adsoption or molecular-dynamics or force-field
 or in-vitro or nanopartic* or in-vivo
 or aqueous-solution or protein or surface)}, entre as linhas 10 e 13 da \query, pois elas irão remover artigos não se enquadram no escopo da busca desejada, por usarem uma ou mais desses termos no título, resumo ou palavras-chave do artigo.
 
Usando a nova \query\ de busca, foram recuperados 6.935 documentos, que se encontram em
\url{https://github.com/jhcf/Comput-Experim-20212/experiments/jhcf/PesqBibliogr/SimulacaoMultiagente/ WoS-20220203/wos6935recs.txt}. Isso sugere que aproximadamente 1.000 registros não se enquadravam na necessidade de busca.
Uma nova análise dos dados recuperados é apresentada a seguir.

\section{Nova Análise dos Dados}

\subsection{Nova filtragem de registros}

Sobre os 6.935 documentos recuperados, foram  aplicados os seguintes filtros:
\begin{itemize}
    \item Remoção dos registros de documentos que não são artigos \textit{full paper}, isso é, artigos completos publicados em revistas;
%    \item Remoção dos registros de artigos científicos que não fazem parte do \textit{core} da bibliografa, segundo a Lei de Bradford.
\end{itemize}

Após os filtros aplicados (apenas um)  obteve-se um total de 4.647 registros, que doravante serão chamados de forma coletiva, de \dataset\   MASSA2@jhcf.

\subsection{Análise descritiva do \dataset\   MASSA2@jhcf}

\subsubsection{Dados Sumários Gerais}

\begin{table}[]
    \centering
\csvautotabular[separator=semicolon
%,filter not strcmp={\csvcolii}{}
]{experiments/jhcf/PesqBibliogr/SimulacaoMultiagente/WoS-20220203/Descritiva/MASSA2-Main-Information.csv}
    \caption{Principais dados descritivos do \dataset\   MASSA2@jhcf.}
    \label{tab:MASSA2:Main}
\end{table}

Nota-se, com os resultados da tabela \ref{tab:MASSA2:Main}, que o \dataset\   abrange um período de 32 anos de publicações (1991 a 2022), evidenciando  a publicação dos 4.647 artigos em 1.910 revistas distintas. Esses artigos tem idade média de publicação de 7.8.

Adicionalmente, o \dataset\ apresenta 157.507 referências citadas, com uma média de (157.507/4.647 = ?) 33,89 referências citadas por artigo.

14.229 autores distintos produziram os artigos, com uma média de 3,73 autores por documento.

\subsubsection{Evolução anual da produção científica}

No tema de simulação multiagente de fenômenos sociais, a evolução anual da produção científica mundial é sumarizada no gráfico da figura \ref{fig:MASSA2:Evolucao}.

\begin{figure}
    \centering
    \includegraphics[width=1\textwidth]{experiments/jhcf/PesqBibliogr/SimulacaoMultiagente/WoS-20220203/Descritiva/MASSA2-Annual-Scientific-Production.png}
    \caption{Evolução da Produção Científica Anual, segundo o \dataset\ MASSA2@jhcf.}
    \label{fig:MASSA2:Annual-Scientific-Production}
\end{figure}

Entre 1991 e 2005 o crescimento de publicações era quase linear. As publicações mostram-se em ascendência forte a partir dos últimos seis anos (2015). Esse crescimento tem sido visto em várias outras áreas de conhecimento.

\subsubsection{Média de citações anuais por artigo}

O gráfico da figura \ref{fig:MASSA2:Media:Citacoes} apresenta a evolução das citações anuais médias, para os artigos do \dataset\ MASSA2@jhcf. Observa-se que há um crescimento discreto da média, onde os artigos mais recentes tendem a ser mais citados, como esperado. A redução da média no ano de 2021 deve-se, provavelmente, à insuficiência de indexação e de citação para os artigos mais recentes, tendo em vista que p ano de 2021 foi encerrado há menos de dois meses. 

\begin{figure}
    \centering
    \includegraphics[width=1\textwidth]{experiments/jhcf/PesqBibliogr/SimulacaoMultiagente/WoS-20220203/Descritiva/MASSA2-Average-Citations-per-Year.png}
    \caption{Média de citações para cada artigo do \dataset\ MASSA2@jhcf, conforme o ano de publicação}
    \label{fig:MASSA2:Media:Citacoes}
\end{figure}

Para que melhor se compreenda como foi produzido o gráfico, a tabela \ref{tab:MASSA2:Media:Citacoes} apresenta parcialmente os dados de citação anual para os artigos do \dataset\ MASSA2@jhcf. A título de exemplo, nota-se que no \dataset\ foram encontrados 9 artigos publicados no ano de 1991, tendo sido cada artigo citado, em média, aproximadamente 31,4 vezes. Dado que esses artigos já tem 31 anos citáveis, obtém-se uma média de 1,01 citações anuais, aproximadamente.

\begin{table}[]
    \centering
\csvautotabular[separator=semicolon
%,filter not strcmp={\csvcolii}{}
]{experiments/jhcf/PesqBibliogr/SimulacaoMultiagente/WoS-20220203/Descritiva/MASSA2-Average-Citations-per-Year.csv}
    \caption{Dados parciais de citação anual para os artigos do \dataset\   MASSA2@jhcf.}
    \label{tab:MASSA2:Media:Citacoes}
\end{table}

\subsubsection{Diagramas de Sankey (\textit{three fields plots})} 

A fim de apresentar mais alguns dados sumários gerais sobre  o \dataset, as figuras \ref{fig:MASSA2:Sankey:CR-AU-DE} e \ref{fig:MASSA2:Sankey:SO:DE:AU_UN} apresentam plotagens do tipo 
\textit{three fields plots}, também conhecidas pelo nome de Diagramas de Sankey \citep{riehmann_interactive_2005}, que possibilitam várias combinações de afinidades mais evidentes entre as diversas colunas dos registros do \dataset.

A primeira plotagem, figura \ref{fig:MASSA2:Sankey:CR-AU-DE}, apresenta as afinidades mais evidentes entre 15 Autores (centro), 15 Palavras-chave (direita) e 15 Referências citadas (esquerda). Ao centro, observa-se que os autores mais evidentes, segundo a técnica apresentada, tem origem asiática, a julgar pelos nomes. 

As 15 palavras-chave mais evidentes sugerem que o \dataset\ possui artigos que refletem a busca sobre o tema desejado, mas que há muitas palavras distintas que representam o mesmo conceito, como as a seguir listadas:
\begin{enumerate}
    \item multi-agent systems;
    \item multiagent system;
    \item multiagent system; 
    \item agent-based model;  
    \item agent-based models;
    \item agent-based modeling;
    \item agent-based simulation;
\end{enumerate}

As cinco palavras a seguir sugerem, o que pode ser comprovado com o aprofundamento desse estudo bibliográfico, que o estado da arte no tema busca atualmente respostas, ou possui fundamentos nas seguintes questões:
\begin{description}
    \item [event-triggered control] Como eventos disparadores exercem  controle sobre o comportamento (coletivo) de grupos sociais?
    \item [consensus] Como usar simulação multi-agente para entender o surgimento de consenso em grupos sociais?
    \item [opinion dynamics] Como usar simulação multi-agente para entender a dinâmica de opiniões que se formam em grupos sociais?
    \item [social networks] Como os métodos da análise de redes sociais podem ser usados no tema da simulação multiagente?
    \item [reinforcement learning] Como usar os métodos e técnicas de aprendizagem por reforço em simulação multiagente?
    \item [game theory] Como usar os métodos e técnicas da teoria dos jogos em simulação multiagente?
\end{description}

Algumas das referências citadas, apresentadas à esquerda do gráfico, devem evidenciar a pertinência das questões acima sugeridas, a ser comprovado até o final do estudo. 

\begin{figure}
    \centering
    \includegraphics[angle=90,width=1\textwidth,height=0.9\textheight]{experiments/jhcf/PesqBibliogr/SimulacaoMultiagente/WoS-20220203/Descritiva/MASSA2-Three-Fields-Plot-CR-AU-DE.png}
    \caption{Diagrama Sankey, relacionando as afinidades mais evidentes entre Autores (centro), Palavras-chave (direita) e Referências citadas (esquerda).}
    \label{fig:MASSA2:Sankey:CR-AU-DE}
\end{figure}

A segunda plotagem, figura \ref{fig:MASSA2:Sankey:SO:DE:AU_UN}, apresenta as afinidades mais evidentes entre 15 revistas (esquerda), 15 palavras-chave (centro) e 15 instituições de filiação dos autores (direita). Com base na técnica usada, fica evidente a proeminência dos seguintes \textit{journals} sobre os demais, sendo apresentado um breve trecho do foco de cada revista, extraído da página online da revista:
\begin{itemize}
    \item JASSS: The Journal of Artificial Societies and Social Simulation. \textit{\small  is an interdisciplinary journal for the exploration and understanding of social processes by means of computer simulation. Since its first issue in 1998, it has been a world-wide leading reference for readers interested in social simulation and the application of computer simulation in the social sciences.}. Fonte \url{https://www.jasss.org/admin/about.html};
    \item PHYSICA A-STATISTICAL MECHANICS AND ITS APPLICATIONS. \textit{\small ... publishes research in the field of statistical mechanics and its applications. Statistical mechanics sets out to explain the behaviour of macroscopic systems, or the large scale, by studying the statistical properties of the microscopic or nanoscopic constituents. Applications of the concepts and techniques of statistical mechanics include: applications to physical and physiochemical systems such as solids, liquids and gases, interfaces, glasses, colloids, complex fluids, polymers, complex networks, applications to economic and social systems (e.g. socio-economic networks, financial time series, agent based models, systemic risk, market dynamics, computational social science, science of science, evolutionary game theory, cultural and political complexity), and traffic and transportation (e.g. vehicular traffic, pedestrian and evacuation dynamics, network traffic, swarms and other forms of collective transport in biology, models of intracellular transport, self-driven particles), as well as biological systems (biological signalling and noise, biological fluctuations, cellular systems and biophysics); and other interdisciplinary applications such as artificial intelligence (e.g. deep learning, genetic algorithms or links between theory of information and thermodynamics/statistical physics.).}. Fonte: \url{https://www.journals.elsevier.com/physica-a-statistical-mechanics-and-its-applications};
    \item IEEE ACCESS. \textit{\small ... is a multidisciplinary, all-electronic archival journal, continuously presenting the results of original research or development across all IEEE’s fields of interest. Supported by article processing charges (APC), its hallmarks are a rapid peer review and publication process of 4 to 6 weeks, with open access to all readers.}. Fonte: \url{https://ieeeaccess.ieee.org/about-ieee-access/learn-more-about-ieee-access/}
\end{itemize} 

\begin{figure}
    \centering
    \includegraphics[angle=90,width=1\textwidth,height=0.9\textheight]{experiments/jhcf/PesqBibliogr/SimulacaoMultiagente/WoS-20220203/Descritiva/MASSA2-Three-Fields-Plot-SO:DE:AU_UN.png}
    \caption{Diagrama Sankey, relacionando as afinidades mais evidentes entre revistas (esquerda), palavras-chave (centro) e instituição de filiação dos autores (direita).}
    \label{fig:MASSA2:Sankey:SO:DE:AU_UN}
\end{figure}

Observa-se, com base no escopo declarado de cada uma das revistas, que a revista JASSS é bem enquadrada no escopo da busca, enquanto que a revista IEEE Access não tem relação direta com o tema. Já a revista Physica A aborda o tema de forma mais ampla do que o buscado, com ênfase em métodos da mecânica estatística, que não são os únicos possíveis de serem empregados.
Os nomes das demais 8 revistas, apresentadas no diagrama, são os seguintes:
\begin{enumerate}
    \item NEUROCOMPUTING \textit{\small  ... welcomes theoretical contributions aimed at winning further understanding of neural networks and learning systems, including, but not restricted to, architectures, learning methods, analysis of network dynamics, theories of learning, self-organization, biological neural network modelling, sensorimotor transformations and interdisciplinary topics with artificial intelligence, artificial life, cognitive science, computational learning theory, fuzzy logic, genetic algorithms, information theory, machine learning, neurobiology and pattern recognition.}. Fonte: \url{https://www.journals.elsevier.com/neurocomputing};
    \item IEEE TRANSACTIONS ON AUTOMATIC CONTROL \textit{\small publishes high-quality papers on the theory, design, and applications of control engineering.  Two types of contributions are regularly considered: 
1) Papers:  Presentation of significant research, development, or application of control concepts. 
2) Technical Notes and Correspondence:  Brief technical notes, comments on published areas or established control topics, corrections to papers and notes published in the Transactions.
In addition, special papers (tutorials, surveys, and perspectives on the theory and applications of control systems topics) are solicited. }. Fonte: \url{https://ieeexplore.ieee.org/xpl/aboutJournal.jsp?punumber=9};
    \item AUTONOMOUS AGENTS AND MULTI-AGENT SYSTEMS \textit{\small is the official journal of the International Foundation for Autonomous Agents and Multi-Agent Systems. It provides a leading forum for disseminating significant original research results in the foundations, theory, development, analysis, and applications of autonomous agents and multi-agent systems. Coverage in Autonomous Agents and Multi-Agent Systems includes, but is not limited to:
Agent decision-making architectures and their evaluation, including: cognitive models; knowledge representation; logics for agency; ontological reasoning; planning (single and multi-agent); reasoning (single and multi-agent)
Cooperation and teamwork, including: distributed problem solving; human-robot/agent interaction; multi-user/multi-virtual-agent interaction; coalition formation; coordination
Agent communication languages, including: their semantics, pragmatics, and implementation; agent communication protocols and conversations; agent commitments; speech act theory
Ontologies for agent systems, agents and the semantic web, agents and semantic web services, Grid-based systems, and service-oriented computing
Agent societies and societal issues, including: artificial social systems; environments, organizations and institutions; ethical and legal issues; privacy, safety and security; trust, reliability and reputation
Agent-based system development, including: agent development techniques, tools and environments; agent programming languages; agent specification or validation languages
Agent-based simulation, including: emergent behavior; participatory simulation; simulation techniques, tools and environments; social simulation
Agreement technologies, including: argumentation; collective decision making; judgment aggregation and belief merging; negotiation; norms
Economi c paradigms, including: auction and mechanism design; bargaining and negotiation; economically-motivated agents; game theory (cooperative and non-cooperative); social choice and voting
Learning agents, including: computational architectures for learning agents; evolution, adaptation; multi-agent learning.
Robotic agents, including: integrated perception, cognition, and action; cognitive robotics; robot planning (including action and motion planning); multi-robot systems.
Virtual agents, including: agents in games and virtual environments; companion and coaching agents; modeling personality, emotions; multimodal interaction; verbal and non-verbal expressiveness
Significant, novel applications of agent technology
Comprehensive reviews and authoritative tutorials of research and practice in agent systems
Comprehensive and authoritative reviews of books dealing with agents and multi-agent systems.
Official journal of the International Foundation for Autonomous Agents and Multi-Agent Systems.
Covers the foundations, theory, development, analysis, and applications of autonomous agents and multi-agent systems.
Presents comprehensive reviews and authoritative tutorials of research and practice in agent systems.}. Fonte: \url{https://www.springer.com/journal/10458};
    \item INTERNATIONAL JOURNAL OF MODERN PHYSICS C \textit{\small is a journal dedicated to Computational Physics and aims at publishing both review and research articles on the use of computers to advance knowledge in physical sciences and the use of physical analogies in computation. Topics covered include: algorithms; computational biophysics; computational fluid dynamics; statistical physics; complex systems; computer and information science; condensed matter physics, materials science; socio- and econophysics; data analysis and computation in experimental physics; environmental physics; traffic modelling; physical computation including neural nets, cellular automata and genetic algorithms.}. Fonte: \url{https://www.worldscientific.com/page/ijmpc/aims-scope};
    \item COMPLEXITY \textit{\small The purpose of Complexity is to report important advances in the scientific study of complex systems. Complex systems are characterized by interactions between their components that produce new information — present in neither the initial nor boundary conditions — which limit their predictability. Given the amount of information processing required to study complexity, the use of computers has been central to complex systems research. Concepts relevant to Complexity include:
    Adaptability, robustness, and resilience;
    Complex networks;
    Criticality;
    Evolution and emergent behaviour;
    Nonlinear dynamics;
    Pattern formation;
    Self-organization.
Methods used within the scientific study of complex systems frequently include:
    Agent-based modelling;
    Analytical methods;
    Cellular automata;
    Computational methods;
    Data science;
    Game theory;
    Machine learning;
    Statistical mechanics.
Applications of complex systems may be related to the following disciplines, among others:
Computational social science;    Digital epidemiology;
    Ecology;
    Economics;
    Engineering;
    Socio-technical systems;
    Statistical linguistics;
    Systems biology;
    Urban systems.}. Fonte: \url{https://www.hindawi.com/journals/complexity/about/};
    \item ECOLOGICAL MODELLING \textit{\small publishes new mathematical models and systems analysis for describing ecological processes, and novel applications of models for environmental management.
We welcome research on process-based models embedded in theory with explicit causative agents and innovative applications of existing models. And because applications can help refine models and propose new directions for research, the journal publishes both to help foster reproducibility and utility.Human activity and well-being are dependent on and integrated with the functioning of ecosystems and the services they provide. We aim to understand these basic ecosystem functions using mathematical and conceptual modelling, systems analysis, thermodynamics, computer simulations, and ecological theory, and look to a wide spectrum of applications ranging from basic ecology to human ecology to socio-ecological systems. The journal welcomes original research articles, review articles, viewpoint articles and short communications.}. Fonte: \url{https://www.journals.elsevier.com/ecological-modelling};
    \item JOURNAL OF THEORETICAL BIOLOGY \textit{\small is the leading forum for theoretical perspectives that give insight into biological processes. It covers a very wide range of topics and is of interest to biologists in many areas of research, including:
Brain and Neuroscience;
Cancer Growth and Treatment;
Cell Biology;
Developmental Biology;
Ecology;
Evolution;
Immunology;
Infectious and non-infectious Diseases;
Mathematical, Computational, Biophysical and Statistical Modeling;
Microbiology, Molecular Biology, and Biochemistry;
Networks and Complex Systems;
Physiology;
Pharmacodynamics;
Animal Behavior and Game Theory}. Fonte: \url{https://www.journals.elsevier.com/journal-of-theoretical-biology};
    \item JOURNAL OF STATISTICAL MECHANICS-THEORY AND EXPERIMENT \textit{\small is targeted to a broad community interested in different aspects of statistical physics, which are roughly defined by the fields represented in the conferences called 'Statistical Physics'. Submissions from experimentalists working on all the topics which have some 'connection to statistical physics are also strongly encouraged.
The journal covers different topics which correspond to the following keyword sections:
Quantum statistical physics, condensed matter, integrable systems;
Classical statistical mechanics, equilibrium and non-equilibrium;
Disordered systems, classical and quantum;
Interdisciplinary statistical mechanics;
Biological modelling and information}. Fonte: \url{https://iopscience.iop.org/journal/1742-5468/page/about_the_journal}.
\end{enumerate}
Considerando a possibilidade de pertinência da questão ecológica ao tema dos sistemas multi-agentes, todos os \textit{journals} identificados apresentam pertinência às perguntas de pesquisa formuladas. 

Acerca das instituições de filiação dos autores, nota-se que 1/3 delas é localizada na china, 1/3 nos EUA, e o restante na Europa e Austrália. É provável que muitos pesquisadores de origem chinesa trabalhem em universidades fora da china, no tema do \dataset.


\subsection{Medidas bibliométricas}

As medidas bibliométricas propriamente ditas, relativas ao \dataset\ MASSA2@jhcf, serão exploradas nesta subseção, e são organizadas em três conjuntos:
\begin{description}
    \item [Relativas às Fontes de Informação] Uma vez que foram consideradas apenas as publicações em revistas, todas as fontes de informação mensuradas serão revistas científicas, ou \textit{journals}. As principais medidas são de impacto das fontes, mensuradas com base no número de citações que os artigos publicados nas revistas obtiveram de outras publicações, possivelmente feitas em outras fontes de informação, como outras revistas, seções de livros, artigos de conferência etc. As citações são registradas pelas organizações que fazem indexação de artigos, como a Web of Science e SCOPUS;
    \item [Relativas aos Autores] Sempre que um artigo publicado por um ou mais autores e também indexado por uma organização (Web of Science,  SCOPUS etc), é citado em um outro artigo também indexado por essa mesma organização, então é feita a anotação de uma citação ao mesmo, e o impacto potencial desse autor sobre a ciência é atestado pelo valor mais alto da citação do conjunto de seus artigos indexados. Várias métricas (índice H, G, M etc) podem ser derivadas dessa medida (quantidade de citações), e são exploradas tanto em relação aos autores como em relação às revistas onde esses artigos foram publicados;
    \item [Relativas aos Documentos] Cada citação adicional a  um documento (artigo de revista, de conferência, livro, ou  capítulo de livro) é um indicador do impacto do documento em si, que evidencia a sua importância. Além das citações, a ocorrência de palavras dentro dos documentos, inclusive ordenada pelo tempo, também produz indicadores numéricos (métricas) relevantes para analisar a importância do documento em relação a outros. 
\end{description}

Essas medidas serão apresentadas a seguir.

\subsubsection{Bibliometrias aplicadas aos documentos (Artigos científicos) no \dataset}

\paragraph{Citações globais aos artigos no \dataset}

Cada registro recuperado no \dataset\ apresenta um conjunto de informações, dentre as quais pode constar a quantidade de vezes que uma citação ao mesmo foi registrada no índice do WoS, desde que no momento da extração seja feita essa solicitação (\textit{TC - Times Cited}).
A tabela \ref{tab:MASSA2:GlobalCitations} apresenta a lista dos 25 artigos do \dataset, que foram mais citados, ordenados de forma decrescente pelo número global de citações do artigo, nos índices da WoS. Para ada artigo é apresentada a referencia abreviada, o DOI e a quantidade de vezes que ele foi citado globalmente (no índice do WoS). Para recuperar a página do artigo deve-se abrir uma url prefixada com \url{http://doi.org/}, e informar o valor do DOI indicado, por exemplo \url{http://doi.org/10.1109/TAC.2008.2010897} levará à página do artigo mais citado, cujo título é ``Flocking of Multi-Agents With a Virtual Leader''.

\begin{table}[]

    \centering
\footnotesize
\csvreader[tabular = |r|l|l|r|,
separator=semicolon
%,filter not strcmp={\csvcolii}{},
, table head = \hline\hline \# & Artigo (Referência Abreviada) & DOI (Digital Object Identifier) & Cit.\\ \hline\hline,
table foot = \hline\hline
]{experiments/jhcf/PesqBibliogr/SimulacaoMultiagente/WoS-20220203/Metricas/Documentos/MASSA2-Most-Global-Cited-Documents.csv}{Paper=\paper, DOI=\doi,Total Citations=\totcit}{ \thecsvrow & {\tiny\paper} & {\tiny \doi} & \totcit}

    \caption{25 artigos mais citados no \dataset\ MASSA2@jhcf.}
    \label{tab:MASSA2:GlobalCitations}
\end{table}


\paragraph{Referências aos (outros) artigos, capítulos de livros etc (documentos) citados pelos artigos no \dataset}

\paragraph{Uso de palavras dentro dos artigos no \dataset}





\part{Estudos Empíricos Exploratórios\label{part:estudos:exploratorios}}

%
\chapter{Análise Bibliográfica sobre Simulação Multiagente e Fenômenos Sociais, por Jorge Fernandes\label{chap:bibliometria:jhcf}}

\section{Planejamento do estudo}
O planejamento o  desenho do estudo deve descrever as motivações, questões de interesse, escopo, limitações e objetivos do trabalho.

O planejamento do estudo deve motivar o tema escolhido e o interesse do autor.

No caso do meu trabalho, as perguntas que o nortearam foram:
\begin{itemize}
    \item Qual a base de conhecimentos científicos produzida em torno do tema simulação multiagente voltada à compreensão de fenômenos sociais, com ênfase em métodos experimentais? 
    \item Como a simulação multiagente tem sido usada para compreender fenômenos sociais, com ênfase em métodos experimentais? 
    \item Quais os principais termos e conceitos ligados à frente de pesquisa no tema simulação multiagente de fenômenos sociais, com ênfase em métodos experimentais? 
    \item Qual a estrutura social da comunidade, se é que existe, que pesquisa sobre o tema simulação multiagente de fenômenos sociais, com ênfase em métodos experimentais?
\end{itemize}

\subsection{O que já existe de pesquisa bibliométrica sobre esse tema?}

\cite{gore_classifying_2016} fizeram uma pesquisa que visava aprofundar a questão da simulação multiagente em relação à computação experimental.

A pesquisa é base para um posterior aprofundamento no campo da Cientometria, como fez \cite{chavalarias_whats_2017}.

\subsection{Uso do Bibliometrix e Biblioshiny}
Serão usadas a ferramenta e o \textit{workflow} proposto pelos autores do pacote Bibliometrix, conforme indica a figura ~\ref{fig:bibliometrix:workflow}.

\subsection{Limitações} O exercício relatado foi feito em uma semana, envolvendo entre 5 a 10 horas de trabalho de cada autor.

Outros aspectos a reforçar:
\begin{itemize}
   
\item Deve-se fazer buscas na base de dados WoS ou SCOPUS;
\item é obrigatório declarar um conjunto de perguntas de pesquisa.
\item é preciso declarar o objetivo da pesquisa, que no caso da aqui relatada foi exercitar inicialmente, e relatar, o uso da técnica de análise bibliométrica, para fins didáticos.
\end{itemize}


\section{Coleta de dados\label{MASSA:coleta}}

A coleta de dados feita usando o WoS no dia 03 de agosto de 2021, acessado por meio do Portal de Periódicos da CAPES.

Foram feitas buscas nas coleções \textbf{Science  Citation  Index  Expanded (SCI -EXPANDED)} e \textbf{Social  Sciences  Citation  Index (SSCI)}, que contém registros relativos a vários campos do conhecimento, no qual o SCI-EXPANDED foca mais na área das ciências exatas e naturais, enquanto que o SSCI indexa artigos da área das ciências sociais. Observe que os artigos nessas duas coleções são indexados desde 1945. 

Foi usada a \query\  de busca ilustrada nas linhas 1 a 9 da listagem \ref{query20210803-2}.

\lstinputlisting[numbers=left,basicstyle=\normalsize\ttfamily,caption={\query\  de busca sobre simulação multiagente de fenômenos socials, com ênfase em métodos experimentais.},label=query20210803-2]
{experiments/jhcf/PesqBibliogr/SimulacaoMultiagente/WoS-20210803/classico-mais-citacoes/query.txt}

\subsection{Explicação para os termos de busca usados\label{MASSA:query}}

A busca consistiu de quatro cláusulas disjuntivas, unidas por uma conjunção \textit{and}, aplicadas à busca por tópico (O termo de busca pode aparecer no Título, no Abstract, na Author Keywords, ou nas Keywords Plus da referência)

Os termos \texttt{experimental}, \texttt{numeric*}, \texttt{statist*}, \texttt{hypothes*}, 
\texttt{empiric*}
e \texttt{inferen} (linhas 1 e 2 da query) foram usados na primeira cláusula da \query\  para recuperar artigos que tenham em seu título, palavras-chave e resumo, termos relacionados a métodos experimentais,
métodos numéricos,
métodos estatísticos,
teste de hipóteses,
métodos empíricos e métodos inferenciais.

O termo / cláusula  \texttt{simul*}, na linha 4, foi usado em conjunção com os demais para recuperar apenas trabalhos que explicitem o uso da simulação.
Foi usado um único termo devido à forte adesão ao termo simulação por parte dos pesquisadores que usam simulação. Não existem outros sinônimos frequentes para esse uso.

A cláusula nas linhas 6 e 7 faz união entre o uso dos termos \texttt{agent} e \texttt{multiagent}, \texttt{multi-agent},e  também \texttt{multi and agent}, para cobrir as variadas formas de escrita do conceito.

A $4^{a}$ cláusula, linha 9,  usou os termos \texttt{social} e \texttt{society} para recuperar artigos que tratem de temas ligados à sociedade.
Os termos \texttt{group} e \texttt{behavi*} visam recuperar estudos que tratam de questões comportamentais e grupais.

Os 8.115 registros obtidos encontram-se no github do projeto, em \url{https://github.com/jhcf/Comput-Experim-20212/experiments/jhcf/PesqBibliogr/SimulacaoMultiagente/ WoS-20210803/classico-mais-citacoes/8115recs.txt}. 

Foram utilizadas as opções \textit{Exportar registros para arquivo de texto sem formatação} e \textit{export full record / Gravar Conteúdo: Seleção personalizada, com todos os 29 campos disponíveis, inclusive referências citadas} no WoS, para que as citações também fosse usadas em análises da citações (estrutura intelectual do conhecimento). Os 8115 registros foram recuperados em nove blocos de até 1.000 registros por vez (1-1000, 1001-2000, 2001-3000, ..., 8001-8115).

\section{Análise dos dados}

\subsection{Filtragem de registros}
Antes da análise, é possível aplicar filtros sobre os registros obtidos.

Foi aplicado um filtro ao \dataset\   inicial, com 8.115 registros, que continham pŕevias de artigos, artigos de conferência, capítulos de livro etc. Foram mantidos apenas os registros de artigos publicados em revistas científicas\footnote{A suposição é que que o conhecimento de maior qualidade sobre o tema está nas publicações em revistas.}. Após a aplicação desse filtro, 5.787 registros foram mantidos no \dataset, que será doravante chamado MultiAgentSimulationSociety/Artigos, ou MASSA@jhcf.

\subsection{Análise descritiva do \dataset\   MASSA@jhcf}

A análise bibliométrica descritiva faz uma descrição inicial do \dataset\  . Para explicação detalhada de como são calculadas as diversas taxas geradas pelo Bibliometrix veja a documentação do \textit{package} a partir da página \url{https://cran.r-project.org/web/packages/bibliometrix/index.html}. A análise bibliométrica descritiva é gerada pela função \texttt{biblioAnalysis}.

As informações mais gerais sobre o \dataset\   MASSA@jhcf são as seguintes:
\begin{description}
    \item [\textit{Timespan}] Os artigos que atenderam aos critérios de busca e filtragem foram publicados a partir de 1990, até 2021. Ou seja, não foram encontrados registros entre 1945 e 1989.
    \item [\textit{Sources (Journals, Books, etc)}] São 2.319 fontes de informação que publicaram os documentos recuperados no \dataset\   MASSA@jhcf. Ou seja, em média, cada \textit{scientific journal} publicou $5.787/2.319=2,5$ artigos. \footnote{Note que a média, enquanto medida de tendência central, pode não ser a que melhor reflete a tendência a quantidade de artigos publicados por revista.}
    \item [\textit{Average years from publication}] A média do tempo de publicação dos artigos no \dataset\   MASSA@jhcf é de 7,36 anos.
    \item [\textit{Average citations per documents}] Cada artigo no \dataset\   MASSA@jhcf foi citado, em média 20,7 vezes\footnote{Note que a média, enquanto medida de tendência central, pode não ser a que melhor reflete a tendência de  citações a artigos.}.
    \item [\textit{Average citations per year per doc}] Após publicado, cada um dos 5.787 artigos do \dataset\   MASSA@jhcf  foi citado 2,262 vezes por ano, em média.
    \item [\textit{References}] O \dataset\   MASSA@jhcf contém 201.464 referências citadas (tags CR).
    \item [\textit{Keywords Plus (ID)}] 13.735 distintas palavras-chave do tipo Keywords Plus (ID)\footnote{\textit{KeyWords Plus} são ``termos de índice gerados automaticamente a partir dos títulos de artigos citados. Os termos do KeyWords Plus devem aparecer mais de uma vez na bibliografia e são ordenados de frases com várias palavras a termos únicos. O KeyWords Plus aumenta o número de resultados tradicional de palavras-chave ou títulos.'' Fonte: \url{https://images.webofknowledge.com/WOKRS410B4/help/pt_BR/WOS/hp_full_record.html}} foram encontradas no \dataset\   MASSA@jhcf. 
    \item [\textit{Author's Keywords (DE)}] 15.704 distintas palavras-chave indicadas pelos autores foram encontradas no \dataset\  .
    \item [\textit{Authors}] 19.410 distintos nomes de autores foram encontrados no \dataset\  \footnote{Um mesmo autor pode ter uma ou mais diferentes grafias no \dataset\  , e serão reconhecidos dois ou mais autores diferentes, embora de fato sejam apenas um. Isso significa que a quantidade de \textbf{nomes de autores} equivale à quantidade de \textbf{autores}. Adicionalmente, é possível que distintos autores sejam reconhecidos com o mesmo nome, isso é, que sejam homônimos. Ou seja, o \dataset\   em geral conterá erros de contagem na quantidade de autores reais.}.
    \item [\textit{Author Appearances}] Os 19.410 distintos (nomes de) autores foram encontrados 23.470 vezes, como autores de artigos.
    \item [\textit{Authors of single-authored documents}] Dentre os 19.410 distintos (nomes de) autores encontrados, 375 deles editaram artigos individualmente, isso é, sem co-autores.
    \item [\textit{Authors of multi-authored documents}] Dentre os 19.410 distintos (nomes de) autores encontrados, 19.035 deles editaram artigos com um ou mais co-autores"
    \item [\textit{Single-authored documents}] Dentre os 5.787 documentos presentes no \dataset\   MASSA, 409 foram escritos por um único autor, e os 5.378 restantes foram elaborados em co-autoria.
    \item [\textit{Documents per Author}] Dentre os 19.410 distintos (nomes de) autores, cada um publicou em média 0,298 artigos.
    \item [\textit{Authors per Document}] Cada um dos 5.787 documentos presentes no \dataset\   MASSA foi autorado com 3,35 autores em média ($19.410 / 5.787 = 3,35$).
    \item [\textit{Co-Authors per Documents}] As 23.470 aparições de (nomes de) autores (``Author Appearances''), sem distribuem, em média 4,06 vezes para os 5.787 documentos do \dataset\   MASSA@jhcf.
    \item [\textit{Collaboration Index}] Os 19.035 (nomes de) autores que editaram artigos com um ou mais co-autores, colaboraram em media 3,54 vezes para editar os 5.378 artigos elaborados em co-autoria, gerando, assim, um índice de colaboração 3,54. 
\end{description}

\subsection{Evolução da Produção Científica}

\begin{figure}
    \centering
    \includegraphics[width=1\textwidth]{experiments/jhcf/PesqBibliogr/SimulacaoMultiagente/WoS-20210803/classico-mais-citacoes/Dataset/AnnualScientificProduction-2021-08-05.png}
    \caption{Evolução da produção científica no \dataset\   MASSA@jhcf.}
    \label{fig:evol:anual:MASSA@jhcf}
\end{figure}

A figura \ref{fig:evol:anual:MASSA@jhcf} apresenta a evolução da produção científica mundial no tema de interesse, segundo o \dataset\   MASSA@jhcf. A curva mostra uma tendência de crescimento aproximadamente exponencial da quantidade de publicações, desde a primeira identificada em 1990.

O \textit{Annual Growth Rate} do \dataset\   é de 17,06\%, bem maior que a taxa média de crescimento da publicação científica mundial, de cerca de 3,3\% anuais, em 2016, como ilustra o estudo em \url{https://www.researchgate.net/publication/333972683_Dynamics_of_scientific_production_in_the_world_in_Europe_and_in_France_2000-2016}, página 23.

\subsection{Interpretação do Crescimento} A maior taxa de crescimento do \dataset\   MASSA@jhcf, bem como o seu grande volume, sugerem que o assunto em pauta desperta intenso interesse, inclusive de ordem econômica.

\subsection{Evolução das Citações}

\begin{figure}
    \centering
    \includegraphics[width=1\textwidth]{experiments/jhcf/PesqBibliogr/SimulacaoMultiagente/WoS-20210803/classico-mais-citacoes/Dataset/AverageArticleCitationPerYear-2021-08-09.png}
    \caption{Evolução das citações ao \dataset\   MASSA@jhcf.}
    \label{fig:evol:anual:citacoes:MASSA@jhcf}
\end{figure}

A figura \ref{fig:evol:anual:citacoes:MASSA@jhcf} apresenta a evolução da média de citações aos 5.787 artigos no \dataset\   MASSA@jhcf. 
Nota-se grande estabilidade na média anual de citações, onde os artigos publicados em 1992 possuem cerca de 2 citações médias, e em 2015 (17 anos depois) o valou alterou-se apenas para três. O pico que aparece no ano de 2008 deve-se, possivelmente, à presença de um artigo do \dataset, publicado em 2008, que possui um número surpreendente grande de citações. \footnote{Note que o cálculo do número  médio de citações, nesse caso, utiliza os valores computados no tag "TC (Times Cited)", já presentes no \dataset\   obtido. Ou seja, o gráfico baseia-se no número de citações globais (externas ao \dataset\   MASSA@jhcf), e não no número de citações locais (citações a um artigo do \dataset\   feitas por alguns dos outros artigos dentro do próprio \dataset).}.

\subsection{Interpretação das Citações}
Mesmo perante um crescimento aproximadamente exponencial no volume de publicações, a ocorrência de um crescimento nas citações médias ao longo dos anos sugere que os artigos do \dataset\   possuem uma tendência de crescimento no tamanho da bibliografia citada, bem como também despertam grande interesse dos cientistas nas demais áreas do conhecimento (já que se trata de citações globais).

\subsection{\textit{Three-Field Plots (Sankey diagram)}}

As \textit{Three-Field Plots (Sankey diagram)} (plotagens do tipo ``Três Campos'') apresentam afinidades entre três conjuntos de atributos agregados que ocorrem no \dataset. Uma plotagem do tipo Sankey busca mostrar os principais fluxos entre diferentes conjuntos de itens. \footnote{Para uma introdução ver \url{https://en.wikipedia.org/wiki/Sankey_diagram}. Para obter detalhes sobre a forma de geração e utilização desse gráfico, inclusive de forma interativa, veja o vídeo em \url{https://www.youtube.com/watch?v=jBb1iha6-sg}.} 

\begin{figure}
    \centering
    \includegraphics[angle=0,width=1\textwidth]{experiments/jhcf/PesqBibliogr/SimulacaoMultiagente/WoS-20210803/classico-mais-citacoes/Dataset/ThreeFieldPlot-AU-CR-DE-20-20-20.png}
    \caption{Plotagem ``Três Campos'' (Sankey plot) do \dataset\   MASSA@jhcf: 20 Autores, Citações e Palavras-Chave mais proeminentes.}
    \label{fig:MASSA@jhcf:ThreeFieldPlot}
\end{figure}

A figura \ref{fig:MASSA@jhcf:ThreeFieldPlot} apresenta a plotagem do tipo ``Três Campos'' do \dataset\   MASSA@jhcf, vinculando, ao centro, os 20 Autores mais proeminentes (AU), à esquerda, as 20 Citações mais frequentes (CR - Cited Records), e à direita, as 20 Palavras-Chave mais frequentes empregadas pelos autores.

\subsection{Interpretação da figura \ref{fig:MASSA@jhcf:ThreeFieldPlot}}
Os vinte autores mais relevantes, em relação aos artigos mais relevantes citados, e as palavras-chave mais relevantes são aparentemente de origem asiática, mais especificamente chinesa, com base nos sobrenomes. De outra formal, a mesma origem chinesa parece não se aplicar aos trabalhos mais citados, aparentemente europeus ou norte-americanos. Isso sugere estar ocorrendo uma migração recente da produção científica, do ocidente para o oriente. 

Adicionalmente, dentre as palavras-chave (DE) não relacionadas diretamente aos termos de busca, emergem os termos \textbf{distributed control}, \textbf{event-triggered control}, \textbf{consensus} e \textbf{opinion dynamics}. Isso sugere foco das pesquisas por autores de origem chinesa no uso de simulação multiagente voltada à compreensão dos fenômenos de controle social distribuído, formação de consenso e dinâmica da opinião (pública?).

Ainda sobre a interpretação da plotagem da figura \ref{fig:MASSA@jhcf:ThreeFieldPlot}, observa-se que os artigos mais citados encontram-se publicados pelo menos 10 anos atrás, sugerindo que não houve, nos últimos 10 anos, nenhum trabalho que tenha produzido uma mudança de paradigma no tema.
A fim de melhor evidenciar as citações mais relevantes segundo o peso dos autores e palavras-chave, o gráfico da figura \ref{fig:MASSA@jhcf:ThreeFieldPlot:10-20-20} plota apenas as 10 referências citadas, para 20 autores e palavras-chave mais proeminentes.

\begin{figure}
    \centering
    \includegraphics[angle=0,width=1\textwidth]{experiments/jhcf/PesqBibliogr/SimulacaoMultiagente/WoS-20210803/classico-mais-citacoes/Dataset/ThreeFieldPlot-AU-CR-DE-20-10-20.png}
    \caption{Plotagem ``Três Campos'' (Sankey plot) do \dataset\   MASSA@jhcf: 10 Autores, 20 Citações e Palavras-Chave mais proeminentes.}
    \label{fig:MASSA@jhcf:ThreeFieldPlot:10-20-20}
\end{figure}

Breves comentários sobre cada um desses trabalhos serão tratados em seção posterior.

\begin{itemize}
    \item  \cite{olfati-saber_consensus_2004} apresentam discussões teóricas sobre a formação de consenso em sistemas multi-agentes com topologias variáveis;
    \item  \cite{reynolds_flocks_1987} apresenta modelos multi-agentes para simulação gráfica do movimento de rebanhos ou agregados de animais.
    \item \cite{vicsek_novel_1995} analisam a emergência de fenômenos de transição de fase em simulações de de partículas com comportamento autônomo com interação biologicamente motivada.
    \item \cite{barabasi_emergence_1999} investigam a emergência da distribuição livre de escala (\textit{scale-free}\footnote{Ver introdução em \url{https://en.wikipedia.org/wiki/Scale-free_network}.}) em redes que evoluem com base em ligação preferencial.
    \item \cite{watts_collective_1998} exploram o surgimento de redes do tipo mundo pequeno (\textit{small world}\footnote{Ver introdução em \url{https://en.wikipedia.org/wiki/Small-world_network}.}) formadas a partir da reorganização aleatória de redes biológicas, genéticas e outras formas de redes auto-organizadas.
    \item \cite{castellano_statistical_2009} exploram de que forma as técnicas de análise e simulação já usadas na física-estatística podem ser usadas para explicar vários fenômenos sociais, tais como comportamento de multidões, dispersão social, comportamento de multidões etc. Eles apresentam as afinidades entre os dados gerados pelos modelos simulados e dados empíricos obtidos junto a sistemas sociais reais. 
    \item \cite{hegselmann_opinion_2002} exploram a emergência de fenômenos de consenso, polarização e fragmentação da opinião na simulação de sociedades artificiais.
    \item \cite{bonabeau_agent-based_2002} apresenta os potenciais e campos de aplicação da técnicas de simulação baseada em agentes.
    \item \cite{wilensky_netlogo_1999} apresentam a linguagem e ambiente de simulação NetLogo.
    \item \cite{grimm_standard_2006} apresenta o protocolo ODD, proposto para padronizar a descrição de modelos de simulação multiagente.
\end{itemize}

Nenhum desses 10 documentos citados está contido no \dataset\   recuperado.

%\subsection{Análises Bibliométricas: Fontes de Informação}

%\begin{figure}
%    \centering
%    \includegraphics[angle=0,width=1\textwidth]{}
%    \caption{Plotagem ``Três Campos'' (Sankey plot) do dataset MASSA@jhcf: 20 Autores, Citações e Palavras-Chave mais proeminentes.}
%    \label{fig:MASSA@jhcf:ThreeFieldPlot}
%\end{figure}

\section{Refinamento da Coleta de Dados}

No dia 03 de fevereiro de 2022, no decorrer das análises mais refinadas do \dataset\ MASSA@jhcf, identificou-se um grupo de artigos que não se encaixavam no tema de interesse, e que eram voltados para pesquisas no campo da biologia experimental e nanotecnologia. Isso sugeriu que a \query\  de busca precisaria ser reformulada, para excluir artigos que não se enquadrassem na temática desejada.
O conjunto das palavras-chave que refletia essa dissonância ficou evidente na análise da estrutura intelectual do conhecimento, do tipo \textbf{Rede de Co-ocorrências de Palavras-chave}, ilustrada no cluster em roxo, à esquerda da figura \ref{fig:MASSA@jhcf:redecoocorr-150-termos}.

\begin{figure}[htp]
    \centering
    \includegraphics[clip=true,trim={9cm 0cm 7cm 0cm },width=0.6\textwidth]{experiments/jhcf/PesqBibliogr/SimulacaoMultiagente/WoS-20210803/classico-mais-citacoes/Structure-Informetric/Conceptual/Co-occurrence Network-Keywords-Plus-150-termos.png}
    \caption{Rede de co-ocorrência de palavras, com 150 termos, aplicada ao \dataset\   MASSA@jhcf.}
    \label{fig:MASSA@jhcf:redecoocorr-150-termos}
\end{figure}

As seguintes 30 palavras foram identificadas nesse \textit{cluster}:
in-vitro,
adsorption,
mechanism,
water,
force-field,
molecular-dynamics,
binding,
simulations,
nanoparticles,
bubbles,
derivatives,
temperature,
in-vivo,
mathematical-model,
oscillations,
scattering,
cancer,
contrast agents,
expression,
protein,
activation,
delivery,
surface,
removal,
acid,
agent,
reduction,
aqueous-solution,
degradation,
expectations.

Ficou evidente, pela interpretação do significado da maioria desses termos, que tais artigos não tratavam de simulação de fenômenos sociais. Isso sugere que a query está com problemas de precisão, isso é, muitos registros recuperados não atendem à necessidade de informação do pesquisador. 

Algumas dessas palavras foram então escolhidas para servir como indicativas de artigos fora do escopo, e introduzidas a partir da \query\  original, gerando uma nova \query, aprimorada e ilustrada nas linhas 1 a 13 da listagem \ref{query20220203}.

\lstinputlisting[numbers=left,basicstyle=\normalsize\ttfamily,caption={\query\  de busca sobre simulação multiagente de fenômenos socials, com ênfase em métodos experimentais, com escopo negativo de artigos que tratam de experimentos biológicos em vitro.},label=query20220203]
{experiments/jhcf/PesqBibliogr/SimulacaoMultiagente/WoS-20220203/query-Refinada.txt}

Além das justificativas para os termos usados entre as linhas 1 a 9, já descritas em \ref{MASSA:query},  justifica-se na listagem \ref{query20220203}, a inclusão da cláusula \textit{not (
 adsoption or molecular-dynamics or force-field
 or in-vitro or nanopartic* or in-vivo
 or aqueous-solution or protein or surface)}, entre as linhas 10 e 13 da \query, pois elas irão remover artigos não se enquadram no escopo da busca desejada, por usarem uma ou mais desses termos no título, resumo ou palavras-chave do artigo.
 
Usando a nova \query\ de busca, foram recuperados 6.935 documentos, que se encontram em
\url{https://github.com/jhcf/Comput-Experim-20212/experiments/jhcf/PesqBibliogr/SimulacaoMultiagente/ WoS-20220203/wos6935recs.txt}. Isso sugere que aproximadamente 1.000 registros não se enquadravam na necessidade de busca.
Uma nova análise dos dados recuperados é apresentada a seguir.

\section{Nova Análise dos Dados}

\subsection{Nova filtragem de registros}

Sobre os 6.935 documentos recuperados, foram  aplicados os seguintes filtros:
\begin{itemize}
    \item Remoção dos registros de documentos que não são artigos \textit{full paper}, isso é, artigos completos publicados em revistas;
%    \item Remoção dos registros de artigos científicos que não fazem parte do \textit{core} da bibliografa, segundo a Lei de Bradford.
\end{itemize}

Após os filtros aplicados (apenas um)  obteve-se um total de 4.647 registros, que doravante serão chamados de forma coletiva, de \dataset\   MASSA2@jhcf.

\subsection{Análise descritiva do \dataset\   MASSA2@jhcf}

\subsubsection{Dados Sumários Gerais}

\begin{table}[]
    \centering
\csvautotabular[separator=semicolon
%,filter not strcmp={\csvcolii}{}
]{experiments/jhcf/PesqBibliogr/SimulacaoMultiagente/WoS-20220203/Descritiva/MASSA2-Main-Information.csv}
    \caption{Principais dados descritivos do \dataset\   MASSA2@jhcf.}
    \label{tab:MASSA2:Main}
\end{table}

Nota-se, com os resultados da tabela \ref{tab:MASSA2:Main}, que o \dataset\   abrange um período de 32 anos de publicações (1991 a 2022), evidenciando  a publicação dos 4.647 artigos em 1.910 revistas distintas. Esses artigos tem idade média de publicação de 7.8.

Adicionalmente, o \dataset\ apresenta 157.507 referências citadas, com uma média de (157.507/4.647 = ?) 33,89 referências citadas por artigo.

14.229 autores distintos produziram os artigos, com uma média de 3,73 autores por documento.

\subsubsection{Evolução anual da produção científica}

No tema de simulação multiagente de fenômenos sociais, a evolução anual da produção científica mundial é sumarizada no gráfico da figura \ref{fig:MASSA2:Evolucao}.

\begin{figure}
    \centering
    \includegraphics[width=1\textwidth]{experiments/jhcf/PesqBibliogr/SimulacaoMultiagente/WoS-20220203/Descritiva/MASSA2-Annual-Scientific-Production.png}
    \caption{Evolução da Produção Científica Anual, segundo o \dataset\ MASSA2@jhcf.}
    \label{fig:MASSA2:Annual-Scientific-Production}
\end{figure}

Entre 1991 e 2005 o crescimento de publicações era quase linear. As publicações mostram-se em ascendência forte a partir dos últimos seis anos (2015). Esse crescimento tem sido visto em várias outras áreas de conhecimento.

\subsubsection{Média de citações anuais por artigo}

O gráfico da figura \ref{fig:MASSA2:Media:Citacoes} apresenta a evolução das citações anuais médias, para os artigos do \dataset\ MASSA2@jhcf. Observa-se que há um crescimento discreto da média, onde os artigos mais recentes tendem a ser mais citados, como esperado. A redução da média no ano de 2021 deve-se, provavelmente, à insuficiência de indexação e de citação para os artigos mais recentes, tendo em vista que p ano de 2021 foi encerrado há menos de dois meses. 

\begin{figure}
    \centering
    \includegraphics[width=1\textwidth]{experiments/jhcf/PesqBibliogr/SimulacaoMultiagente/WoS-20220203/Descritiva/MASSA2-Average-Citations-per-Year.png}
    \caption{Média de citações para cada artigo do \dataset\ MASSA2@jhcf, conforme o ano de publicação}
    \label{fig:MASSA2:Media:Citacoes}
\end{figure}

Para que melhor se compreenda como foi produzido o gráfico, a tabela \ref{tab:MASSA2:Media:Citacoes} apresenta parcialmente os dados de citação anual para os artigos do \dataset\ MASSA2@jhcf. A título de exemplo, nota-se que no \dataset\ foram encontrados 9 artigos publicados no ano de 1991, tendo sido cada artigo citado, em média, aproximadamente 31,4 vezes. Dado que esses artigos já tem 31 anos citáveis, obtém-se uma média de 1,01 citações anuais, aproximadamente.

\begin{table}[]
    \centering
\csvautotabular[separator=semicolon
%,filter not strcmp={\csvcolii}{}
]{experiments/jhcf/PesqBibliogr/SimulacaoMultiagente/WoS-20220203/Descritiva/MASSA2-Average-Citations-per-Year.csv}
    \caption{Dados parciais de citação anual para os artigos do \dataset\   MASSA2@jhcf.}
    \label{tab:MASSA2:Media:Citacoes}
\end{table}

\subsubsection{Diagramas de Sankey (\textit{three fields plots})} 

A fim de apresentar mais alguns dados sumários gerais sobre  o \dataset, as figuras \ref{fig:MASSA2:Sankey:CR-AU-DE} e \ref{fig:MASSA2:Sankey:SO:DE:AU_UN} apresentam plotagens do tipo 
\textit{three fields plots}, também conhecidas pelo nome de Diagramas de Sankey \citep{riehmann_interactive_2005}, que possibilitam várias combinações de afinidades mais evidentes entre as diversas colunas dos registros do \dataset.

A primeira plotagem, figura \ref{fig:MASSA2:Sankey:CR-AU-DE}, apresenta as afinidades mais evidentes entre 15 Autores (centro), 15 Palavras-chave (direita) e 15 Referências citadas (esquerda). Ao centro, observa-se que os autores mais evidentes, segundo a técnica apresentada, tem origem asiática, a julgar pelos nomes. 

As 15 palavras-chave mais evidentes sugerem que o \dataset\ possui artigos que refletem a busca sobre o tema desejado, mas que há muitas palavras distintas que representam o mesmo conceito, como as a seguir listadas:
\begin{enumerate}
    \item multi-agent systems;
    \item multiagent system;
    \item multiagent system; 
    \item agent-based model;  
    \item agent-based models;
    \item agent-based modeling;
    \item agent-based simulation;
\end{enumerate}

As cinco palavras a seguir sugerem, o que pode ser comprovado com o aprofundamento desse estudo bibliográfico, que o estado da arte no tema busca atualmente respostas, ou possui fundamentos nas seguintes questões:
\begin{description}
    \item [event-triggered control] Como eventos disparadores exercem  controle sobre o comportamento (coletivo) de grupos sociais?
    \item [consensus] Como usar simulação multi-agente para entender o surgimento de consenso em grupos sociais?
    \item [opinion dynamics] Como usar simulação multi-agente para entender a dinâmica de opiniões que se formam em grupos sociais?
    \item [social networks] Como os métodos da análise de redes sociais podem ser usados no tema da simulação multiagente?
    \item [reinforcement learning] Como usar os métodos e técnicas de aprendizagem por reforço em simulação multiagente?
    \item [game theory] Como usar os métodos e técnicas da teoria dos jogos em simulação multiagente?
\end{description}

Algumas das referências citadas, apresentadas à esquerda do gráfico, devem evidenciar a pertinência das questões acima sugeridas, a ser comprovado até o final do estudo. 

\begin{figure}
    \centering
    \includegraphics[angle=90,width=1\textwidth,height=0.9\textheight]{experiments/jhcf/PesqBibliogr/SimulacaoMultiagente/WoS-20220203/Descritiva/MASSA2-Three-Fields-Plot-CR-AU-DE.png}
    \caption{Diagrama Sankey, relacionando as afinidades mais evidentes entre Autores (centro), Palavras-chave (direita) e Referências citadas (esquerda).}
    \label{fig:MASSA2:Sankey:CR-AU-DE}
\end{figure}

A segunda plotagem, figura \ref{fig:MASSA2:Sankey:SO:DE:AU_UN}, apresenta as afinidades mais evidentes entre 15 revistas (esquerda), 15 palavras-chave (centro) e 15 instituições de filiação dos autores (direita). Com base na técnica usada, fica evidente a proeminência dos seguintes \textit{journals} sobre os demais, sendo apresentado um breve trecho do foco de cada revista, extraído da página online da revista:
\begin{itemize}
    \item JASSS: The Journal of Artificial Societies and Social Simulation. \textit{\small  is an interdisciplinary journal for the exploration and understanding of social processes by means of computer simulation. Since its first issue in 1998, it has been a world-wide leading reference for readers interested in social simulation and the application of computer simulation in the social sciences.}. Fonte \url{https://www.jasss.org/admin/about.html};
    \item PHYSICA A-STATISTICAL MECHANICS AND ITS APPLICATIONS. \textit{\small ... publishes research in the field of statistical mechanics and its applications. Statistical mechanics sets out to explain the behaviour of macroscopic systems, or the large scale, by studying the statistical properties of the microscopic or nanoscopic constituents. Applications of the concepts and techniques of statistical mechanics include: applications to physical and physiochemical systems such as solids, liquids and gases, interfaces, glasses, colloids, complex fluids, polymers, complex networks, applications to economic and social systems (e.g. socio-economic networks, financial time series, agent based models, systemic risk, market dynamics, computational social science, science of science, evolutionary game theory, cultural and political complexity), and traffic and transportation (e.g. vehicular traffic, pedestrian and evacuation dynamics, network traffic, swarms and other forms of collective transport in biology, models of intracellular transport, self-driven particles), as well as biological systems (biological signalling and noise, biological fluctuations, cellular systems and biophysics); and other interdisciplinary applications such as artificial intelligence (e.g. deep learning, genetic algorithms or links between theory of information and thermodynamics/statistical physics.).}. Fonte: \url{https://www.journals.elsevier.com/physica-a-statistical-mechanics-and-its-applications};
    \item IEEE ACCESS. \textit{\small ... is a multidisciplinary, all-electronic archival journal, continuously presenting the results of original research or development across all IEEE’s fields of interest. Supported by article processing charges (APC), its hallmarks are a rapid peer review and publication process of 4 to 6 weeks, with open access to all readers.}. Fonte: \url{https://ieeeaccess.ieee.org/about-ieee-access/learn-more-about-ieee-access/}
\end{itemize} 

\begin{figure}
    \centering
    \includegraphics[angle=90,width=1\textwidth,height=0.9\textheight]{experiments/jhcf/PesqBibliogr/SimulacaoMultiagente/WoS-20220203/Descritiva/MASSA2-Three-Fields-Plot-SO:DE:AU_UN.png}
    \caption{Diagrama Sankey, relacionando as afinidades mais evidentes entre revistas (esquerda), palavras-chave (centro) e instituição de filiação dos autores (direita).}
    \label{fig:MASSA2:Sankey:SO:DE:AU_UN}
\end{figure}

Observa-se, com base no escopo declarado de cada uma das revistas, que a revista JASSS é bem enquadrada no escopo da busca, enquanto que a revista IEEE Access não tem relação direta com o tema. Já a revista Physica A aborda o tema de forma mais ampla do que o buscado, com ênfase em métodos da mecânica estatística, que não são os únicos possíveis de serem empregados.
Os nomes das demais 8 revistas, apresentadas no diagrama, são os seguintes:
\begin{enumerate}
    \item NEUROCOMPUTING \textit{\small  ... welcomes theoretical contributions aimed at winning further understanding of neural networks and learning systems, including, but not restricted to, architectures, learning methods, analysis of network dynamics, theories of learning, self-organization, biological neural network modelling, sensorimotor transformations and interdisciplinary topics with artificial intelligence, artificial life, cognitive science, computational learning theory, fuzzy logic, genetic algorithms, information theory, machine learning, neurobiology and pattern recognition.}. Fonte: \url{https://www.journals.elsevier.com/neurocomputing};
    \item IEEE TRANSACTIONS ON AUTOMATIC CONTROL \textit{\small publishes high-quality papers on the theory, design, and applications of control engineering.  Two types of contributions are regularly considered: 
1) Papers:  Presentation of significant research, development, or application of control concepts. 
2) Technical Notes and Correspondence:  Brief technical notes, comments on published areas or established control topics, corrections to papers and notes published in the Transactions.
In addition, special papers (tutorials, surveys, and perspectives on the theory and applications of control systems topics) are solicited. }. Fonte: \url{https://ieeexplore.ieee.org/xpl/aboutJournal.jsp?punumber=9};
    \item AUTONOMOUS AGENTS AND MULTI-AGENT SYSTEMS \textit{\small is the official journal of the International Foundation for Autonomous Agents and Multi-Agent Systems. It provides a leading forum for disseminating significant original research results in the foundations, theory, development, analysis, and applications of autonomous agents and multi-agent systems. Coverage in Autonomous Agents and Multi-Agent Systems includes, but is not limited to:
Agent decision-making architectures and their evaluation, including: cognitive models; knowledge representation; logics for agency; ontological reasoning; planning (single and multi-agent); reasoning (single and multi-agent)
Cooperation and teamwork, including: distributed problem solving; human-robot/agent interaction; multi-user/multi-virtual-agent interaction; coalition formation; coordination
Agent communication languages, including: their semantics, pragmatics, and implementation; agent communication protocols and conversations; agent commitments; speech act theory
Ontologies for agent systems, agents and the semantic web, agents and semantic web services, Grid-based systems, and service-oriented computing
Agent societies and societal issues, including: artificial social systems; environments, organizations and institutions; ethical and legal issues; privacy, safety and security; trust, reliability and reputation
Agent-based system development, including: agent development techniques, tools and environments; agent programming languages; agent specification or validation languages
Agent-based simulation, including: emergent behavior; participatory simulation; simulation techniques, tools and environments; social simulation
Agreement technologies, including: argumentation; collective decision making; judgment aggregation and belief merging; negotiation; norms
Economi c paradigms, including: auction and mechanism design; bargaining and negotiation; economically-motivated agents; game theory (cooperative and non-cooperative); social choice and voting
Learning agents, including: computational architectures for learning agents; evolution, adaptation; multi-agent learning.
Robotic agents, including: integrated perception, cognition, and action; cognitive robotics; robot planning (including action and motion planning); multi-robot systems.
Virtual agents, including: agents in games and virtual environments; companion and coaching agents; modeling personality, emotions; multimodal interaction; verbal and non-verbal expressiveness
Significant, novel applications of agent technology
Comprehensive reviews and authoritative tutorials of research and practice in agent systems
Comprehensive and authoritative reviews of books dealing with agents and multi-agent systems.
Official journal of the International Foundation for Autonomous Agents and Multi-Agent Systems.
Covers the foundations, theory, development, analysis, and applications of autonomous agents and multi-agent systems.
Presents comprehensive reviews and authoritative tutorials of research and practice in agent systems.}. Fonte: \url{https://www.springer.com/journal/10458};
    \item INTERNATIONAL JOURNAL OF MODERN PHYSICS C \textit{\small is a journal dedicated to Computational Physics and aims at publishing both review and research articles on the use of computers to advance knowledge in physical sciences and the use of physical analogies in computation. Topics covered include: algorithms; computational biophysics; computational fluid dynamics; statistical physics; complex systems; computer and information science; condensed matter physics, materials science; socio- and econophysics; data analysis and computation in experimental physics; environmental physics; traffic modelling; physical computation including neural nets, cellular automata and genetic algorithms.}. Fonte: \url{https://www.worldscientific.com/page/ijmpc/aims-scope};
    \item COMPLEXITY \textit{\small The purpose of Complexity is to report important advances in the scientific study of complex systems. Complex systems are characterized by interactions between their components that produce new information — present in neither the initial nor boundary conditions — which limit their predictability. Given the amount of information processing required to study complexity, the use of computers has been central to complex systems research. Concepts relevant to Complexity include:
    Adaptability, robustness, and resilience;
    Complex networks;
    Criticality;
    Evolution and emergent behaviour;
    Nonlinear dynamics;
    Pattern formation;
    Self-organization.
Methods used within the scientific study of complex systems frequently include:
    Agent-based modelling;
    Analytical methods;
    Cellular automata;
    Computational methods;
    Data science;
    Game theory;
    Machine learning;
    Statistical mechanics.
Applications of complex systems may be related to the following disciplines, among others:
Computational social science;    Digital epidemiology;
    Ecology;
    Economics;
    Engineering;
    Socio-technical systems;
    Statistical linguistics;
    Systems biology;
    Urban systems.}. Fonte: \url{https://www.hindawi.com/journals/complexity/about/};
    \item ECOLOGICAL MODELLING \textit{\small publishes new mathematical models and systems analysis for describing ecological processes, and novel applications of models for environmental management.
We welcome research on process-based models embedded in theory with explicit causative agents and innovative applications of existing models. And because applications can help refine models and propose new directions for research, the journal publishes both to help foster reproducibility and utility.Human activity and well-being are dependent on and integrated with the functioning of ecosystems and the services they provide. We aim to understand these basic ecosystem functions using mathematical and conceptual modelling, systems analysis, thermodynamics, computer simulations, and ecological theory, and look to a wide spectrum of applications ranging from basic ecology to human ecology to socio-ecological systems. The journal welcomes original research articles, review articles, viewpoint articles and short communications.}. Fonte: \url{https://www.journals.elsevier.com/ecological-modelling};
    \item JOURNAL OF THEORETICAL BIOLOGY \textit{\small is the leading forum for theoretical perspectives that give insight into biological processes. It covers a very wide range of topics and is of interest to biologists in many areas of research, including:
Brain and Neuroscience;
Cancer Growth and Treatment;
Cell Biology;
Developmental Biology;
Ecology;
Evolution;
Immunology;
Infectious and non-infectious Diseases;
Mathematical, Computational, Biophysical and Statistical Modeling;
Microbiology, Molecular Biology, and Biochemistry;
Networks and Complex Systems;
Physiology;
Pharmacodynamics;
Animal Behavior and Game Theory}. Fonte: \url{https://www.journals.elsevier.com/journal-of-theoretical-biology};
    \item JOURNAL OF STATISTICAL MECHANICS-THEORY AND EXPERIMENT \textit{\small is targeted to a broad community interested in different aspects of statistical physics, which are roughly defined by the fields represented in the conferences called 'Statistical Physics'. Submissions from experimentalists working on all the topics which have some 'connection to statistical physics are also strongly encouraged.
The journal covers different topics which correspond to the following keyword sections:
Quantum statistical physics, condensed matter, integrable systems;
Classical statistical mechanics, equilibrium and non-equilibrium;
Disordered systems, classical and quantum;
Interdisciplinary statistical mechanics;
Biological modelling and information}. Fonte: \url{https://iopscience.iop.org/journal/1742-5468/page/about_the_journal}.
\end{enumerate}
Considerando a possibilidade de pertinência da questão ecológica ao tema dos sistemas multi-agentes, todos os \textit{journals} identificados apresentam pertinência às perguntas de pesquisa formuladas. 

Acerca das instituições de filiação dos autores, nota-se que 1/3 delas é localizada na china, 1/3 nos EUA, e o restante na Europa e Austrália. É provável que muitos pesquisadores de origem chinesa trabalhem em universidades fora da china, no tema do \dataset.


\subsection{Medidas bibliométricas}

As medidas bibliométricas propriamente ditas, relativas ao \dataset\ MASSA2@jhcf, serão exploradas nesta subseção, e são organizadas em três conjuntos:
\begin{description}
    \item [Relativas às Fontes de Informação] Uma vez que foram consideradas apenas as publicações em revistas, todas as fontes de informação mensuradas serão revistas científicas, ou \textit{journals}. As principais medidas são de impacto das fontes, mensuradas com base no número de citações que os artigos publicados nas revistas obtiveram de outras publicações, possivelmente feitas em outras fontes de informação, como outras revistas, seções de livros, artigos de conferência etc. As citações são registradas pelas organizações que fazem indexação de artigos, como a Web of Science e SCOPUS;
    \item [Relativas aos Autores] Sempre que um artigo publicado por um ou mais autores e também indexado por uma organização (Web of Science,  SCOPUS etc), é citado em um outro artigo também indexado por essa mesma organização, então é feita a anotação de uma citação ao mesmo, e o impacto potencial desse autor sobre a ciência é atestado pelo valor mais alto da citação do conjunto de seus artigos indexados. Várias métricas (índice H, G, M etc) podem ser derivadas dessa medida (quantidade de citações), e são exploradas tanto em relação aos autores como em relação às revistas onde esses artigos foram publicados;
    \item [Relativas aos Documentos] Cada citação adicional a  um documento (artigo de revista, de conferência, livro, ou  capítulo de livro) é um indicador do impacto do documento em si, que evidencia a sua importância. Além das citações, a ocorrência de palavras dentro dos documentos, inclusive ordenada pelo tempo, também produz indicadores numéricos (métricas) relevantes para analisar a importância do documento em relação a outros. 
\end{description}

Essas medidas serão apresentadas a seguir.

\subsubsection{Bibliometrias aplicadas aos documentos (Artigos científicos) no \dataset}

\paragraph{Citações globais aos artigos no \dataset}

Cada registro recuperado no \dataset\ apresenta um conjunto de informações, dentre as quais pode constar a quantidade de vezes que uma citação ao mesmo foi registrada no índice do WoS, desde que no momento da extração seja feita essa solicitação (\textit{TC - Times Cited}).
A tabela \ref{tab:MASSA2:GlobalCitations} apresenta a lista dos 25 artigos do \dataset, que foram mais citados, ordenados de forma decrescente pelo número global de citações do artigo, nos índices da WoS. Para ada artigo é apresentada a referencia abreviada, o DOI e a quantidade de vezes que ele foi citado globalmente (no índice do WoS). Para recuperar a página do artigo deve-se abrir uma url prefixada com \url{http://doi.org/}, e informar o valor do DOI indicado, por exemplo \url{http://doi.org/10.1109/TAC.2008.2010897} levará à página do artigo mais citado, cujo título é ``Flocking of Multi-Agents With a Virtual Leader''.

\begin{table}[]

    \centering
\footnotesize
\csvreader[tabular = |r|l|l|r|,
separator=semicolon
%,filter not strcmp={\csvcolii}{},
, table head = \hline\hline \# & Artigo (Referência Abreviada) & DOI (Digital Object Identifier) & Cit.\\ \hline\hline,
table foot = \hline\hline
]{experiments/jhcf/PesqBibliogr/SimulacaoMultiagente/WoS-20220203/Metricas/Documentos/MASSA2-Most-Global-Cited-Documents.csv}{Paper=\paper, DOI=\doi,Total Citations=\totcit}{ \thecsvrow & {\tiny\paper} & {\tiny \doi} & \totcit}

    \caption{25 artigos mais citados no \dataset\ MASSA2@jhcf.}
    \label{tab:MASSA2:GlobalCitations}
\end{table}


\paragraph{Referências aos (outros) artigos, capítulos de livros etc (documentos) citados pelos artigos no \dataset}

\paragraph{Uso de palavras dentro dos artigos no \dataset}

\chapter{Análise Bibliográfica sobre Processamento de Linguagem Natural, por Lucas de Almeida Bandeira Macedo}

\section{Planejamento do estudo}

Com a vinda de assistentes virtuais, como a Alexa (Amazon), Cortana (Microsoft) ou Siri (Apple), as pessoas costumam se perguntar cada vez mais: "como que esse programa está entendendo o que eu falo?".

Mas não só de assistentes virtuais vive o Processamento de Linguagem Natural (também conhecido como NLP - Natural Language Processing), afinal, qualquer texto ou fala pode ser interpretado por uma máquina e devidamente classificado. Por exemplo, uma aplicação famosa é o "classificador de sentimentos", em que um modelo treinado consegue classificar textos entre sentimentos "positivos" ou "negativos". Com a ascensão do Twitter, uma rede social baseada em pequenos textos de não mais que 280 caracteres, NLP se torna cada vez mais interessante.

Assim, as perguntas que traçam o norte para este estudo são:

\begin{itemize}
    \item Quais os principais conceitos ligados com Processamento de Linguagem Natural?
    \item Como se dá o progresso das pesquisas em NLP ao longo dos anos? As redes sociais influenciaram esse crescimento?
    \item Qual o estado da estrutura social da comunidade de NLP?
\end{itemize}

\subsection{Uso do Bibliometrix e Biblioshiny}

Será usada a ferramenta Bibliometrix, com sua função Biblioshiny, para gerar gráficos e grafos iterativos e personalizáveis, para auxiliar na interpretação da realidade científica do tópico.

\section{Coleta de dados}

A coleta de dados foi feita utilizando o site Web of Science (WoS), no dia 03/02/2022, através do portal periódico da capes.

A pesquisa foi realizada utilizando as edições "Science Citation Index Expanded" e "Conference Proceedings Citation Index – Science", ambas coleções são voltadas para, principalmente, as ciências exatas.

A \textit{string} (ou \textit{query}) de busca inicialmente utilizada foi a seguinte:

\lstinputlisting[numbers=left,basicstyle=\normalsize\ttfamily,caption={Query de busca sobre Procesasmento de Linguagem Natural.},label=queryNLP03022022]
{experiments/ABMHub/PesquisaBibliometrica/NLP/pesquisa_velha.txt}

\subsection{Explicação para os termos de busca usados}

A proposta é apenas pesquisar sobre Processamento de Linguagem Natural, sem muito rigor na aplicação em que essa arquitetura de rede neural é aplicada. Portanto, inicialmente a pesquisa foi apenas "natural language processing".

Porém, uma rápida olhada pelos artigos retornados evidenciou uma grande quantidade de artigos sobre linguísticas, e áreas que não são da computação. Como o objetivo aqui adquirir modelos de Deep Learning, a pesquisa foi ajustada para filtrar apenas por NLP ligadas diretamente a computação e inteligência artificial, evidenciado pelas cláusulas "neural network", "(machine or deep) and learning" e "artificial intelligence". Essa nova pesquisa trouxe melhores resultados, todos evidenciando redes neurais e variadas técnicas de machine learning. O total de registros retornado pela query foi 

\subsubsection{Refinamento da Coleta de Dados}

 Em seguida, em uma análise mais fina, utilizando a \textbf{Rede de Co-ocorrências de Palavras-chave}, podemos evidenciar outras palavras chaves que estavam aparecendo entre os registros da pesquisa, que não deveriam estar aparecendo. É possível observar na imagem \ref{fig:ABMHub:NLPgraph1}, palavras como "câncer" ou "diagnóstico" que estão relacionadas a visão computacional mais que NLP, aparecendo com pesos não-desprezíveis.
 
 \begin{figure}
    \centering
    \includegraphics[angle=0,width=1\textwidth]{experiments/ABMHub/PesquisaBibliometrica/NLP/network.png}
    \caption{Grafo de relação de keywords}
    \label{fig:ABMHub:NLPgraph1}
\end{figure}

Assim, é necessário uma nova iteração da pesquisa, para evitar que registros de visão computacional corrompam a pesquisa de NLP. É delicado fazer isso, pois existem muitas menções a Visão Computacional nos registros de LP, já que ambos são ligados a Deep Learning, então retirar a keyword "Visão Computacional" provavelmente removeria muitos registros que não gostaríamos de remover da pesquisa. Assim, a melhor solução encontrada foi remover palavras que não têm intersecção entre os dois assuntos. Por exemplo, "medical", "cancer" e "diagnosis".

Assim, chegamos na mais recente query:

\lstinputlisting[numbers=left,basicstyle=\normalsize\ttfamily,caption={Query de busca sobre Procesasmento de Linguagem Natural.},label=queryNLP03022022]
{experiments/ABMHub/PesquisaBibliometrica/NLP/pesquisa_nova.txt}



\chapter{Análise Bibliográfica sobre Otimizações algorítmicas para simulações de fenômenos fluídos e óticos por Alexsander Correa de Oliveira}

\section{Planejamento do estudo}
    A indústria de jogos é a que mais cresce dentre todas as formas de entretenimento atuais. Alguns desses jogos podem chegar a investimentos tão grandes que disputam com os filmes mais caros da história. No posto de vista do consumidor, todo tempo e dinheiro gastos são apenas meios para um fim, que é o de ter a melhor experiência possível. Contudo do ponto de vista dos desenvolvedores, esses fatores são consequências de horas e horas de trabalho.
    
    Toda a tecnologia criada para esses jogos tem de ser cada vez mais eficiente, dado a necessidade de tornar os gráficos cada vez mais realistas, e seus mundos ainda mais acreditáveis. As técnicas utilizadas para tal otimização são fortemente baseadas em artigos \emph{state-of-the-art} tanto e física quanto em geometria.
    
    Entre todos aspectos físicos, que hoje em dia são mais prevalentes nos jogos, temos algumas áreas de estudo que pesam mais, principalmente em performance: 
    \begin{itemize}
        \item Corpos fluídos, como água e ar;
        \item Análise de vetores em ótica, para saber como a iluminação afetará um determinado ambiente;
        \item Análise da topologia, com fins de otimizar \emph{path-finding};
    \end{itemize}
    
    Considerando o corpo de estudo, algumas questões surgem:
    \begin{item}
        \item Quais são os principais interesses relacionados ao estudo dos fenômenos naturais voltados a computação?
        \item Quem são os agentes que produzem o maior volume de artigos?
        \item Quais são os corpos de estudo mais relacionados entre si?
    \end{item}
    Todas elas serão analisadas e respondidas no decorrer da seção.
\subsection{Uso do Bibliometrix e Biblioshiny}
     Com o auxílio das ferramentas disponibilizadas pelo Bibliometrix, como o Biblioshiny, serão analisados os artigos encontrados, por meio de gráficos e tabelas.   
\section{Coleta de Dados}
    A coleta de dados foi iniciada no dia 02/02/2022, e usou a base Web of Science, com acesso direto pelo periódico Capes.
    
    Por fins de diminuir o tamanho do \emph{dataset}, só foi utilizada a edição \emph{Science Citation Index Expanded}, que tem o foco voltado para ciências exatas e naturais.
    
    A busca inicial foi feita com a seguinte \emph{query}:
\begin{lstlisting}[basicstyle = \normalsize]
((algorit* ) and (Optimization)) and 
(optics or ((fluid* or aero*) and dynamics))
\end{lstlisting}
\subsection{Explicação para a \emph{Query}}
    A busca foi feita com o objetivo de encontrar apenas técnicas para otimizar algoritmos relacionados a ótica, aerodinâmica e hidrodinâmica.
    
    Os termos \verb|((algorit* ) and (Optimization))| são para encontrar apenas os artigos relacionados a algoritmos computacionais.
    
    Já \verb|(optics or ((fluid* or aero*) and dynamics))| serve para falar que tanto faz um artigo de ótica ou de aerodinâmica ou de hidrodinâmica.
    
    Com a \emph{Query} já montada, os registros foram exportados do WoS com todas as informações disponíveis e no formato de arquivo de texto sem formatação. Foram recuperados desa maneira, 7443 registros no total.
\section{Análise dos dados}
    Uma análise inicial foi feita com o objetivo de retirar artigos indesejados. Para atingir isso, foi utilizado o gráfico \emph{Co-occurrence Network}, que mostra as palavras com maior peso, e o relacionamento entre elas.
    
     \begin{figure}
    \centering
    \includegraphics[width=1\textwidth]{experiments/KvotheKS/PesqBibliogr/AlgoritmosSimulacaoOptica-Dinamica/WoS-20220202/OldQueryDataset/CoOccurrence.png}
    \caption{Rede de co-ocorrência}
    \label{fig:KvotheKS:OldQueryCoOccurrence}
\end{figure}
    
    Destacando um dos lados do grafo e o meio, podemos ver que o foco em computação e otimização foram atingidos. Contudo, como um efeito não desejado, também foram "recebidos" artigos que envolvem I.A, como também ótica de um ponto de vista médico.
    
\subsection{Refinamento dos Dados}
    Para retirar todo produto indesejado foi feita uma nova \emph{query} na mesma base e edição:
    
\begin{lstlisting}[basicstyle = \normalsize]
((algorit* ) and (Optimization)) and 

(optics or ((fluid* or aero*) and dynamics))

not ((genetic* and algorit*) or medic* or (machin* and learn*))
\end{lstlisting}

    Com os novos parâmetros, o objetivo de retirar tudo relacionado a medicina e a maioria de algoritmos genéticos foi atingido. Como resultado da nova busca, foram retornados 4917 elementos. Contudo, considerando apenas o número de artigos, o número cai para 4859.
    
    Como demonstração da melhora do \emph{dataset}, segue o gráfico \ref{fig:KvotheKS:Final_Data_Set}:
    
    \begin{figure}[H]
    \centering
    \includegraphics[width=1.3\textwidth]{experiments/KvotheKS/PesqBibliogr/AlgoritmosSimulacaoOptica-Dinamica/WoS-20220202/Dataset/AU_CR_DE.png}
    \caption{Dataset final}
    \label{fig:KvotheKS:Final_Data_Set}
\end{figure}

\subsection{Análise descritiva do \emph{dataset}}
    As informações iniciais do \emph{dataset} de 4859 registros são as seguintes:
    
\begin{description}
    \item [\textit{Timespan}] Todos os artigos que passaram pelo filtro e pela busca foram feitos de 1985 a 2022.
    \item [\textit{Sources (Journals, Books, etc)}] São 924 fontes de informação registradas.
    \item [\textit{Average years from publication}] A média de tempo para publicação é de 7,95 anos.
    \item [\textit{Average citations per documents}] A média de citações dos artigos é de 17,87 vezes.
    \item [\textit{Average citations per year per doc}] Os artigos, após sua publicação, tiveram em média 1,887 citações anuais.
    \item [\textit{References}] A quantidade total de referências do \emph{dataset} se dá em 127.349.
    \item [\textit{Keywords Plus (ID)}] 7.218 palavras-chave distintas foram encontradas no \emph{dataset}.
    \item [\textit{Author's Keywords (DE)}] 10.367 palavras-chave distintas escritas pelos autores.
    \item [\textit{Authors}] No total, foram 14.247 autores, sendo que boa parte deles tem origem chinesa.
    \item [\textit{Author Appearances}] No total, tiveram 20.024 aparições de autores, sendo que o número de autores distintos é, como mencionado anteriormente, 14.247
    \item [\textit{Authors of single-authored documents}] Dentre o número total de autores, apenas 206 fizeram pelo menos 1 artigo sozinhos.
    \item [\textit{Authors of multi-authored documents}] Se retirarmos do número total de autores, o número de autores que escreveram artigo(s) sozinhos, chegamos em 14.041 autores que escreveram apenas artigos coletivos.
    \item [\textit{Single-authored documents}] Dentro do \emph{dataset} apenas 227 deles são de criação individual.
    \item [\textit{Documents per Author}] Se dividirmos o número total de artigos pela quantidade de autores, chegamos em 0,341 artigos/autor.
    \item [\textit{Authors per Document}] Agora, inversamente se fizermos a quantidade de autores distintos divido pelo número de artigos, chegamos em 2,93 autores(/artigo.
    \item [\textit{Co-Authors per Documents}] Se pegarmos o número total de autores (também os repetidos) e dividirmos pela quantidade de documentos, temos 4.12 autores/artigo
    \item [\textit{Collaboration Index}] Por fim, a quantidade de vezes que autores distintos editaram artigos com um ou mais co-autores é de 3,03.
\end{description}
\subsection{Evolução da Produção Científica}
    Os temas procurados na busca são consideravelmente mais recentes que o esperado. O gráfico \ref{fig:KvotheKS:Annual_Scientific} mostra um crescimento quase que perfeitamente exponencial, sendo ele de 12.45\%.
    \begin{figure}[H]
    \centering
    \includegraphics[width=1\textwidth]{experiments/KvotheKS/PesqBibliogr/AlgoritmosSimulacaoOptica-Dinamica/WoS-20220202/Dataset/Annual_Scientific.png}
    \caption{Produção anual científica}
    \label{fig:KvotheKS:Annual_Scientific}
\end{figure}
\subsection{Interpretação do crescimento}
    Com o avanço dos computadores e uma disponibilidade maior de recursos científicos como um todo, vários temas acabam ganhando força por fatores variados. No caso do meu \emph{dataset}, os estudos vão de análise topológica para robôs a estudo de aerodinâmica para aviões, e no fim acabam em simulações de iluminação.
\subsection{\emph{Clustering Network}}
    Como meio de demonstrar o quão "compacto" estão os resultados do \emph{dataset}, podemos utilizar uma \emph{Clustering Network}, que mostra em, em forma de grafo, quais estudos estão relacionados entre si, e qual peso de cada um.  
\begin{figure}[H]
    \centering
    \includegraphics[width=1\textwidth]{experiments/KvotheKS/PesqBibliogr/AlgoritmosSimulacaoOptica-Dinamica/WoS-20220202/Dataset/Cluster_network.png}
    \caption{Grafo de citações}
    \label{fig:KvotheKS:Cluster_}
\end{figure}
\subsection{Interpretação da rede}
    Analisando a figura \ref{fig:KvotheKS:Cluster_}, podemos ver o quão inter-relacionados os artigos estão. Isso é um resultado óbvio do refinamento de dados feito anteriormente. Também mostra que alguns tópicos, como hidrodinâmica, aparecem em maior peso, por causa das revistas em qual artigos foram publicados. 
\subsection{\emph{Three-Field Plot}}
    Já foi demonstrado um dos \emph{Sankey diagrams} anteriormente \ref{fig:KvotheKS:Final_Data_Set}, onde o resultado mais interessante são as palavras-chave a direita, que mostram realmente quais são os tópicos mais abordados no \emph{dataset}. Porém alguns dados interessantes não foram abordados.
    
\begin{figure}[H]
    \centering
    \includegraphics[width=1.1\textwidth]{experiments/KvotheKS/PesqBibliogr/AlgoritmosSimulacaoOptica-Dinamica/WoS-20220202/Dataset/AU_CO_AU_UN_SO.png}
    \caption{Afiliações, revistas e países}
    \label{fig:KvotheKS:SankeyCountry}
\end{figure}
\subsection{Considerações do peso dos países}
    Os dados interessantes da figura \ref{fig:KvotheKS:SankeyCountry} se dão nos países e universidades. Mais da metade dos artigos são chineses, porém não só há uma diversidade grande de universidades chinesas, mas também há uma falta de diferença entre as estado-unidenses, por onde artigos de vários países acabam passando.
\section{}
%\chapter{Análise Bibliográfica sobre Simulação , por }%\label{}


\section{Planejamento do estudo}


\begin{itemize}
    \item 
\end{itemize}

%% Keywords usadas: (graphic processing unit or GPU) and (lighting or light or shadow*)

\chapter{Análise Bibliográfica sobre , por Gustavo Tomás}

\section{Planejamento do estudo}

O objetivo do trabalho é analisar o impacto das GPUs (Graphic Processing Units) no processamento e simulação da luz. Para isso, foram utilizadas as ferramentas Bibliometrix e Biblioshiny.

\subsection{O que já existe de pesquisa bibliométrica sobre esse tema?}

\subsection{Limitações} O exercício relatado foi feito em cerca de uma semana, entre os dias 02 e 10 de fevereiro de 2022 e a base de dados utilizada foi Web Of Science (WoS).

\section{Coleta de dados}

A coleta de dados foi feita usando o WoS no dia 03/02/2022, por meio do Portal de Periódicos da CAPES. Foram feitas buscas nas coleções Science Citation Index Expanded (SCI-EXPANDED) e Social Sciences Citation Index (SSCI), mas com o foco em registros relativos a área de ciências naturais e exatas. A busca utilizada foi a seguinte:

\begin{verbatim}
(graphic processing unit or GPU) and (lighting or light or shadow*)
\end{verbatim}

Essa busca consiste em dois termos, sendo que o primeiro é composto pela GPU (por extenso ou pela sigla) e o segundo pelas palavras luz ou iluminação ou sombra(s). Dessa forma, foram encontrados 1311 registros, sendo que nesse trabalho foram utilizados os primeiros 1000 registros, disponíveis em \ref{}.

\section{Análise dos dados}

\subsection{Filtragem de registros}
Antes da análise, foram aplicados filtros aos registros, de forma que apenas registros do tipo \textit{article}, de qualquer ano e com qualquer número de citações, fossem analisados. O resultado consiste em 850/1000 registros originais.

\subsection{Análise descritiva do dataset}

As informações mais gerais sobre o \textit{dataset} MASSA@jhcf são as seguintes:
\begin{description}
    \item [\textit{Timespan}] Os artigos que atenderam aos critérios de busca e filtragem foram publicados a partir de 1990, até 2021. Ou seja, não foram encontrados registros entre 1945 e 1989.
    \item [\textit{Sources (Journals, Books, etc)}] São 2.319 fontes de informação que publicaram os documentos recuperados no dataset MASSA@jhcf. Ou seja, em média, cada \textit{scientific journal} publicou $5.787/2.319=2,5$ artigos. \footnote{Note que a média, enquanto medida de tendência central, pode não ser a que melhor reflete a tendência a quantidade de artigos publicados por revista.}
    \item [\textit{Average years from publication}] A média do tempo de publicação dos artigos no dataset MASSA@jhcf é de 7,36 anos.
    \item [\textit{Average citations per documents}] Cada artigo no dataset MASSA@jhcf foi citado, em média 20,7 vezes\footnote{Note que a média, enquanto medida de tendência central, pode não ser a que melhor reflete a tendência de  citações a artigos.}.
    \item [\textit{Average citations per year per doc}] Após publicado, cada um dos 5.787 artigos do dataset MASSA@jhcf  foi citado 2,262 vezes por ano, em média.
    \item [\textit{References}] O dataset MASSA@jhcf contém 201.464 referências citadas (tags CR).
    \item [\textit{Keywords Plus (ID)}] 13.735 distintas palavras-chave do tipo Keywords Plus (ID)\footnote{\textit{KeyWords Plus} são ``termos de índice gerados automaticamente a partir dos títulos de artigos citados. Os termos do KeyWords Plus devem aparecer mais de uma vez na bibliografia e são ordenados de frases com várias palavras a termos únicos. O KeyWords Plus aumenta o número de resultados tradicional de palavras-chave ou títulos.'' Fonte: \url{https://images.webofknowledge.com/WOKRS410B4/help/pt_BR/WOS/hp_full_record.html}} foram encontradas no dataset MASSA@jhcf. 
    \item [\textit{Author's Keywords (DE)}] 15.704 distintas palavras-chave indicadas pelos autores foram encontradas no \textit{dataset}.
    \item [\textit{Authors}] 19.410 distintos nomes de autores foram encontrados no dataset\footnote{Um mesmo autor pode ter uma ou mais diferentes grafias no dataset, e serão reconhecidos dois ou mais autores diferentes, embora de fato sejam apenas um. Isso significa que a quantidade de \textbf{nomes de autores} equivale à quantidade de \textbf{autores}. Adicionalmente, é possível que distintos autores sejam reconhecidos com o mesmo nome, isso é, que sejam homônimos. Ou seja, o dataset em geral conterá erros de contagem na quantidade de autores reais.}.
    \item [\textit{Author Appearances}] Os 19.410 distintos (nomes de) autores foram encontrados 23.470 vezes, como autores de artigos.
    \item [\textit{Authors of single-authored documents}] Dentre os 19.410 distintos (nomes de) autores encontrados, 375 deles editaram artigos individualmente, isso é, sem co-autores.
    \item [\textit{Authors of multi-authored documents}] Dentre os 19.410 distintos (nomes de) autores encontrados, 19.035 deles editaram artigos com um ou mais co-autores"
    \item [\textit{Single-authored documents}] Dentre os 5.787 documentos presentes no dataset MASSA, 409 foram escritos por um único autor, e os 5.378 restantes foram elaborados em co-autoria.
    \item [\textit{Documents per Author}] Dentre os 19.410 distintos (nomes de) autores, cada um publicou em média 0,298 artigos.
    \item [\textit{Authors per Document}] Cada um dos 5.787 documentos presentes no dataset MASSA foi autorado com 3,35 autores em média ($19.410 / 5.787 = 3,35$).
    \item [\textit{Co-Authors per Documents}] As 23.470 aparições de (nomes de) autores (``Author Appearances''), sem distribuem, em média 4,06 vezes para os 5.787 documentos do dataset MASSA@jhcf.
    \item [\textit{Collaboration Index}] Os 19.035 (nomes de) autores que editaram artigos com um ou mais co-autores, colaboraram em media 3,54 vezes para editar os 5.378 artigos elaborados em co-autoria, gerando, assim, um índice de colaboração 3,54. 
\end{description}

\subsection{Evolução da Produção Científica}

\subsection{Interpretação do Crescimento}

\subsection{Evolução das Citações}

\subsection{Interpretação das Citações}

\subsection{\textit{Three-Field Plots (Sankey diagram)}}

\subsection{Interpretação da figura}

\subsection{Análises Bibliométricas: Fontes de Informação}

\subsection{Análises Bibliométricas: Autores}

\subsection{Análises Bibliométricas: Documentos}



\subsection{Minhas impressões iniciais sobre a ciência, por Ítalo Eduardo Dias Frota}

A ciência é um escopo de \gls{Conhecimento} e um processo. A área se apoia na busca e aplicação dos conhecimentos a respeito das esferas sociais e naturais, seguindo uma \gls{Metodologia} bem definida baseada em evidências que descrevam, expliquem e possam prever um \gls{fenomeno}. Entretanto, definir a ciência não é uma tarefa fácil devido a pluralidade de aplicações e abordagens que permeiam a \gls{ComunidadeCientifica}.

Um dos aspectos mais importantes no campo científico é o da metodologia científica, que permite a formulação de hipóteses, experimentos  e verificações. Os resultados obtidos devem ser reproduzidos através de artigos e publicações que auxiliem na propagação dos pontos observados, instigando novas descobertas e indagações. Sendo assim, o processo é extremamente autocorretivo, pois está em constante evolução.

Sem a ciência e o pensamento científico, a humanidade enfrentaria dificuldades em larga escala. As descobertas e previsões documentadas pelos membros da comunidade são de extrema importância para os avanços nas mais diversas áreas, desde a saúde, até a educação e tecnologia. É imprescindível que seja dada a devida atenção às observações coletadas pela ciência, caso contrário, a humanidade estará sempre destinada ao fracasso.



\chapter{Análise Bibliográfica sobre Simulação Multiagente e Fenômenos Sociais, por Jorge Fernandes\label{chap:bibliometria:jhcf}}

\section{Planejamento do estudo}
O planejamento o  desenho do estudo deve descrever as motivações, questões de interesse, escopo, limitações e objetivos do trabalho.

O planejamento do estudo deve motivar o tema escolhido e o interesse do autor.

No caso do meu trabalho, as perguntas que o nortearam foram:
\begin{itemize}
    \item Qual a base de conhecimentos científicos produzida em torno do tema simulação multiagente voltada à compreensão de fenômenos sociais, com ênfase em métodos experimentais? 
    \item Como a simulação multiagente tem sido usada para compreender fenômenos sociais, com ênfase em métodos experimentais? 
    \item Quais os principais termos e conceitos ligados à frente de pesquisa no tema simulação multiagente de fenômenos sociais, com ênfase em métodos experimentais? 
    \item Qual a estrutura social da comunidade, se é que existe, que pesquisa sobre o tema simulação multiagente de fenômenos sociais, com ênfase em métodos experimentais?
\end{itemize}

\subsection{O que já existe de pesquisa bibliométrica sobre esse tema?}

\cite{gore_classifying_2016} fizeram uma pesquisa que visava aprofundar a questão da simulação multiagente em relação à computação experimental.

A pesquisa é base para um posterior aprofundamento no campo da Cientometria, como fez \cite{chavalarias_whats_2017}.

\subsection{Uso do Bibliometrix e Biblioshiny}
Serão usadas a ferramenta e o \textit{workflow} proposto pelos autores do pacote Bibliometrix, conforme indica a figura ~\ref{fig:bibliometrix:workflow}.

\subsection{Limitações} O exercício relatado foi feito em uma semana, envolvendo entre 5 a 10 horas de trabalho de cada autor.

Outros aspectos a reforçar:
\begin{itemize}
   
\item Deve-se fazer buscas na base de dados WoS ou SCOPUS;
\item é obrigatório declarar um conjunto de perguntas de pesquisa.
\item é preciso declarar o objetivo da pesquisa, que no caso da aqui relatada foi exercitar inicialmente, e relatar, o uso da técnica de análise bibliométrica, para fins didáticos.
\end{itemize}


\section{Coleta de dados\label{MASSA:coleta}}

A coleta de dados feita usando o WoS no dia 03 de agosto de 2021, acessado por meio do Portal de Periódicos da CAPES.

Foram feitas buscas nas coleções \textbf{Science  Citation  Index  Expanded (SCI -EXPANDED)} e \textbf{Social  Sciences  Citation  Index (SSCI)}, que contém registros relativos a vários campos do conhecimento, no qual o SCI-EXPANDED foca mais na área das ciências exatas e naturais, enquanto que o SSCI indexa artigos da área das ciências sociais. Observe que os artigos nessas duas coleções são indexados desde 1945. 

Foi usada a \query\  de busca ilustrada nas linhas 1 a 9 da listagem \ref{query20210803-2}.

\lstinputlisting[numbers=left,basicstyle=\normalsize\ttfamily,caption={\query\  de busca sobre simulação multiagente de fenômenos socials, com ênfase em métodos experimentais.},label=query20210803-2]
{experiments/jhcf/PesqBibliogr/SimulacaoMultiagente/WoS-20210803/classico-mais-citacoes/query.txt}

\subsection{Explicação para os termos de busca usados\label{MASSA:query}}

A busca consistiu de quatro cláusulas disjuntivas, unidas por uma conjunção \textit{and}, aplicadas à busca por tópico (O termo de busca pode aparecer no Título, no Abstract, na Author Keywords, ou nas Keywords Plus da referência)

Os termos \texttt{experimental}, \texttt{numeric*}, \texttt{statist*}, \texttt{hypothes*}, 
\texttt{empiric*}
e \texttt{inferen} (linhas 1 e 2 da query) foram usados na primeira cláusula da \query\  para recuperar artigos que tenham em seu título, palavras-chave e resumo, termos relacionados a métodos experimentais,
métodos numéricos,
métodos estatísticos,
teste de hipóteses,
métodos empíricos e métodos inferenciais.

O termo / cláusula  \texttt{simul*}, na linha 4, foi usado em conjunção com os demais para recuperar apenas trabalhos que explicitem o uso da simulação.
Foi usado um único termo devido à forte adesão ao termo simulação por parte dos pesquisadores que usam simulação. Não existem outros sinônimos frequentes para esse uso.

A cláusula nas linhas 6 e 7 faz união entre o uso dos termos \texttt{agent} e \texttt{multiagent}, \texttt{multi-agent},e  também \texttt{multi and agent}, para cobrir as variadas formas de escrita do conceito.

A $4^{a}$ cláusula, linha 9,  usou os termos \texttt{social} e \texttt{society} para recuperar artigos que tratem de temas ligados à sociedade.
Os termos \texttt{group} e \texttt{behavi*} visam recuperar estudos que tratam de questões comportamentais e grupais.

Os 8.115 registros obtidos encontram-se no github do projeto, em \url{https://github.com/jhcf/Comput-Experim-20212/experiments/jhcf/PesqBibliogr/SimulacaoMultiagente/ WoS-20210803/classico-mais-citacoes/8115recs.txt}. 

Foram utilizadas as opções \textit{Exportar registros para arquivo de texto sem formatação} e \textit{export full record / Gravar Conteúdo: Seleção personalizada, com todos os 29 campos disponíveis, inclusive referências citadas} no WoS, para que as citações também fosse usadas em análises da citações (estrutura intelectual do conhecimento). Os 8115 registros foram recuperados em nove blocos de até 1.000 registros por vez (1-1000, 1001-2000, 2001-3000, ..., 8001-8115).

\section{Análise dos dados}

\subsection{Filtragem de registros}
Antes da análise, é possível aplicar filtros sobre os registros obtidos.

Foi aplicado um filtro ao \dataset\   inicial, com 8.115 registros, que continham pŕevias de artigos, artigos de conferência, capítulos de livro etc. Foram mantidos apenas os registros de artigos publicados em revistas científicas\footnote{A suposição é que que o conhecimento de maior qualidade sobre o tema está nas publicações em revistas.}. Após a aplicação desse filtro, 5.787 registros foram mantidos no \dataset, que será doravante chamado MultiAgentSimulationSociety/Artigos, ou MASSA@jhcf.

\subsection{Análise descritiva do \dataset\   MASSA@jhcf}

A análise bibliométrica descritiva faz uma descrição inicial do \dataset\  . Para explicação detalhada de como são calculadas as diversas taxas geradas pelo Bibliometrix veja a documentação do \textit{package} a partir da página \url{https://cran.r-project.org/web/packages/bibliometrix/index.html}. A análise bibliométrica descritiva é gerada pela função \texttt{biblioAnalysis}.

As informações mais gerais sobre o \dataset\   MASSA@jhcf são as seguintes:
\begin{description}
    \item [\textit{Timespan}] Os artigos que atenderam aos critérios de busca e filtragem foram publicados a partir de 1990, até 2021. Ou seja, não foram encontrados registros entre 1945 e 1989.
    \item [\textit{Sources (Journals, Books, etc)}] São 2.319 fontes de informação que publicaram os documentos recuperados no \dataset\   MASSA@jhcf. Ou seja, em média, cada \textit{scientific journal} publicou $5.787/2.319=2,5$ artigos. \footnote{Note que a média, enquanto medida de tendência central, pode não ser a que melhor reflete a tendência a quantidade de artigos publicados por revista.}
    \item [\textit{Average years from publication}] A média do tempo de publicação dos artigos no \dataset\   MASSA@jhcf é de 7,36 anos.
    \item [\textit{Average citations per documents}] Cada artigo no \dataset\   MASSA@jhcf foi citado, em média 20,7 vezes\footnote{Note que a média, enquanto medida de tendência central, pode não ser a que melhor reflete a tendência de  citações a artigos.}.
    \item [\textit{Average citations per year per doc}] Após publicado, cada um dos 5.787 artigos do \dataset\   MASSA@jhcf  foi citado 2,262 vezes por ano, em média.
    \item [\textit{References}] O \dataset\   MASSA@jhcf contém 201.464 referências citadas (tags CR).
    \item [\textit{Keywords Plus (ID)}] 13.735 distintas palavras-chave do tipo Keywords Plus (ID)\footnote{\textit{KeyWords Plus} são ``termos de índice gerados automaticamente a partir dos títulos de artigos citados. Os termos do KeyWords Plus devem aparecer mais de uma vez na bibliografia e são ordenados de frases com várias palavras a termos únicos. O KeyWords Plus aumenta o número de resultados tradicional de palavras-chave ou títulos.'' Fonte: \url{https://images.webofknowledge.com/WOKRS410B4/help/pt_BR/WOS/hp_full_record.html}} foram encontradas no \dataset\   MASSA@jhcf. 
    \item [\textit{Author's Keywords (DE)}] 15.704 distintas palavras-chave indicadas pelos autores foram encontradas no \dataset\  .
    \item [\textit{Authors}] 19.410 distintos nomes de autores foram encontrados no \dataset\  \footnote{Um mesmo autor pode ter uma ou mais diferentes grafias no \dataset\  , e serão reconhecidos dois ou mais autores diferentes, embora de fato sejam apenas um. Isso significa que a quantidade de \textbf{nomes de autores} equivale à quantidade de \textbf{autores}. Adicionalmente, é possível que distintos autores sejam reconhecidos com o mesmo nome, isso é, que sejam homônimos. Ou seja, o \dataset\   em geral conterá erros de contagem na quantidade de autores reais.}.
    \item [\textit{Author Appearances}] Os 19.410 distintos (nomes de) autores foram encontrados 23.470 vezes, como autores de artigos.
    \item [\textit{Authors of single-authored documents}] Dentre os 19.410 distintos (nomes de) autores encontrados, 375 deles editaram artigos individualmente, isso é, sem co-autores.
    \item [\textit{Authors of multi-authored documents}] Dentre os 19.410 distintos (nomes de) autores encontrados, 19.035 deles editaram artigos com um ou mais co-autores"
    \item [\textit{Single-authored documents}] Dentre os 5.787 documentos presentes no \dataset\   MASSA, 409 foram escritos por um único autor, e os 5.378 restantes foram elaborados em co-autoria.
    \item [\textit{Documents per Author}] Dentre os 19.410 distintos (nomes de) autores, cada um publicou em média 0,298 artigos.
    \item [\textit{Authors per Document}] Cada um dos 5.787 documentos presentes no \dataset\   MASSA foi autorado com 3,35 autores em média ($19.410 / 5.787 = 3,35$).
    \item [\textit{Co-Authors per Documents}] As 23.470 aparições de (nomes de) autores (``Author Appearances''), sem distribuem, em média 4,06 vezes para os 5.787 documentos do \dataset\   MASSA@jhcf.
    \item [\textit{Collaboration Index}] Os 19.035 (nomes de) autores que editaram artigos com um ou mais co-autores, colaboraram em media 3,54 vezes para editar os 5.378 artigos elaborados em co-autoria, gerando, assim, um índice de colaboração 3,54. 
\end{description}

\subsection{Evolução da Produção Científica}

\begin{figure}
    \centering
    \includegraphics[width=1\textwidth]{experiments/jhcf/PesqBibliogr/SimulacaoMultiagente/WoS-20210803/classico-mais-citacoes/Dataset/AnnualScientificProduction-2021-08-05.png}
    \caption{Evolução da produção científica no \dataset\   MASSA@jhcf.}
    \label{fig:evol:anual:MASSA@jhcf}
\end{figure}

A figura \ref{fig:evol:anual:MASSA@jhcf} apresenta a evolução da produção científica mundial no tema de interesse, segundo o \dataset\   MASSA@jhcf. A curva mostra uma tendência de crescimento aproximadamente exponencial da quantidade de publicações, desde a primeira identificada em 1990.

O \textit{Annual Growth Rate} do \dataset\   é de 17,06\%, bem maior que a taxa média de crescimento da publicação científica mundial, de cerca de 3,3\% anuais, em 2016, como ilustra o estudo em \url{https://www.researchgate.net/publication/333972683_Dynamics_of_scientific_production_in_the_world_in_Europe_and_in_France_2000-2016}, página 23.

\subsection{Interpretação do Crescimento} A maior taxa de crescimento do \dataset\   MASSA@jhcf, bem como o seu grande volume, sugerem que o assunto em pauta desperta intenso interesse, inclusive de ordem econômica.

\subsection{Evolução das Citações}

\begin{figure}
    \centering
    \includegraphics[width=1\textwidth]{experiments/jhcf/PesqBibliogr/SimulacaoMultiagente/WoS-20210803/classico-mais-citacoes/Dataset/AverageArticleCitationPerYear-2021-08-09.png}
    \caption{Evolução das citações ao \dataset\   MASSA@jhcf.}
    \label{fig:evol:anual:citacoes:MASSA@jhcf}
\end{figure}

A figura \ref{fig:evol:anual:citacoes:MASSA@jhcf} apresenta a evolução da média de citações aos 5.787 artigos no \dataset\   MASSA@jhcf. 
Nota-se grande estabilidade na média anual de citações, onde os artigos publicados em 1992 possuem cerca de 2 citações médias, e em 2015 (17 anos depois) o valou alterou-se apenas para três. O pico que aparece no ano de 2008 deve-se, possivelmente, à presença de um artigo do \dataset, publicado em 2008, que possui um número surpreendente grande de citações. \footnote{Note que o cálculo do número  médio de citações, nesse caso, utiliza os valores computados no tag "TC (Times Cited)", já presentes no \dataset\   obtido. Ou seja, o gráfico baseia-se no número de citações globais (externas ao \dataset\   MASSA@jhcf), e não no número de citações locais (citações a um artigo do \dataset\   feitas por alguns dos outros artigos dentro do próprio \dataset).}.

\subsection{Interpretação das Citações}
Mesmo perante um crescimento aproximadamente exponencial no volume de publicações, a ocorrência de um crescimento nas citações médias ao longo dos anos sugere que os artigos do \dataset\   possuem uma tendência de crescimento no tamanho da bibliografia citada, bem como também despertam grande interesse dos cientistas nas demais áreas do conhecimento (já que se trata de citações globais).

\subsection{\textit{Three-Field Plots (Sankey diagram)}}

As \textit{Three-Field Plots (Sankey diagram)} (plotagens do tipo ``Três Campos'') apresentam afinidades entre três conjuntos de atributos agregados que ocorrem no \dataset. Uma plotagem do tipo Sankey busca mostrar os principais fluxos entre diferentes conjuntos de itens. \footnote{Para uma introdução ver \url{https://en.wikipedia.org/wiki/Sankey_diagram}. Para obter detalhes sobre a forma de geração e utilização desse gráfico, inclusive de forma interativa, veja o vídeo em \url{https://www.youtube.com/watch?v=jBb1iha6-sg}.} 

\begin{figure}
    \centering
    \includegraphics[angle=0,width=1\textwidth]{experiments/jhcf/PesqBibliogr/SimulacaoMultiagente/WoS-20210803/classico-mais-citacoes/Dataset/ThreeFieldPlot-AU-CR-DE-20-20-20.png}
    \caption{Plotagem ``Três Campos'' (Sankey plot) do \dataset\   MASSA@jhcf: 20 Autores, Citações e Palavras-Chave mais proeminentes.}
    \label{fig:MASSA@jhcf:ThreeFieldPlot}
\end{figure}

A figura \ref{fig:MASSA@jhcf:ThreeFieldPlot} apresenta a plotagem do tipo ``Três Campos'' do \dataset\   MASSA@jhcf, vinculando, ao centro, os 20 Autores mais proeminentes (AU), à esquerda, as 20 Citações mais frequentes (CR - Cited Records), e à direita, as 20 Palavras-Chave mais frequentes empregadas pelos autores.

\subsection{Interpretação da figura \ref{fig:MASSA@jhcf:ThreeFieldPlot}}
Os vinte autores mais relevantes, em relação aos artigos mais relevantes citados, e as palavras-chave mais relevantes são aparentemente de origem asiática, mais especificamente chinesa, com base nos sobrenomes. De outra formal, a mesma origem chinesa parece não se aplicar aos trabalhos mais citados, aparentemente europeus ou norte-americanos. Isso sugere estar ocorrendo uma migração recente da produção científica, do ocidente para o oriente. 

Adicionalmente, dentre as palavras-chave (DE) não relacionadas diretamente aos termos de busca, emergem os termos \textbf{distributed control}, \textbf{event-triggered control}, \textbf{consensus} e \textbf{opinion dynamics}. Isso sugere foco das pesquisas por autores de origem chinesa no uso de simulação multiagente voltada à compreensão dos fenômenos de controle social distribuído, formação de consenso e dinâmica da opinião (pública?).

Ainda sobre a interpretação da plotagem da figura \ref{fig:MASSA@jhcf:ThreeFieldPlot}, observa-se que os artigos mais citados encontram-se publicados pelo menos 10 anos atrás, sugerindo que não houve, nos últimos 10 anos, nenhum trabalho que tenha produzido uma mudança de paradigma no tema.
A fim de melhor evidenciar as citações mais relevantes segundo o peso dos autores e palavras-chave, o gráfico da figura \ref{fig:MASSA@jhcf:ThreeFieldPlot:10-20-20} plota apenas as 10 referências citadas, para 20 autores e palavras-chave mais proeminentes.

\begin{figure}
    \centering
    \includegraphics[angle=0,width=1\textwidth]{experiments/jhcf/PesqBibliogr/SimulacaoMultiagente/WoS-20210803/classico-mais-citacoes/Dataset/ThreeFieldPlot-AU-CR-DE-20-10-20.png}
    \caption{Plotagem ``Três Campos'' (Sankey plot) do \dataset\   MASSA@jhcf: 10 Autores, 20 Citações e Palavras-Chave mais proeminentes.}
    \label{fig:MASSA@jhcf:ThreeFieldPlot:10-20-20}
\end{figure}

Breves comentários sobre cada um desses trabalhos serão tratados em seção posterior.

\begin{itemize}
    \item  \cite{olfati-saber_consensus_2004} apresentam discussões teóricas sobre a formação de consenso em sistemas multi-agentes com topologias variáveis;
    \item  \cite{reynolds_flocks_1987} apresenta modelos multi-agentes para simulação gráfica do movimento de rebanhos ou agregados de animais.
    \item \cite{vicsek_novel_1995} analisam a emergência de fenômenos de transição de fase em simulações de de partículas com comportamento autônomo com interação biologicamente motivada.
    \item \cite{barabasi_emergence_1999} investigam a emergência da distribuição livre de escala (\textit{scale-free}\footnote{Ver introdução em \url{https://en.wikipedia.org/wiki/Scale-free_network}.}) em redes que evoluem com base em ligação preferencial.
    \item \cite{watts_collective_1998} exploram o surgimento de redes do tipo mundo pequeno (\textit{small world}\footnote{Ver introdução em \url{https://en.wikipedia.org/wiki/Small-world_network}.}) formadas a partir da reorganização aleatória de redes biológicas, genéticas e outras formas de redes auto-organizadas.
    \item \cite{castellano_statistical_2009} exploram de que forma as técnicas de análise e simulação já usadas na física-estatística podem ser usadas para explicar vários fenômenos sociais, tais como comportamento de multidões, dispersão social, comportamento de multidões etc. Eles apresentam as afinidades entre os dados gerados pelos modelos simulados e dados empíricos obtidos junto a sistemas sociais reais. 
    \item \cite{hegselmann_opinion_2002} exploram a emergência de fenômenos de consenso, polarização e fragmentação da opinião na simulação de sociedades artificiais.
    \item \cite{bonabeau_agent-based_2002} apresenta os potenciais e campos de aplicação da técnicas de simulação baseada em agentes.
    \item \cite{wilensky_netlogo_1999} apresentam a linguagem e ambiente de simulação NetLogo.
    \item \cite{grimm_standard_2006} apresenta o protocolo ODD, proposto para padronizar a descrição de modelos de simulação multiagente.
\end{itemize}

Nenhum desses 10 documentos citados está contido no \dataset\   recuperado.

%\subsection{Análises Bibliométricas: Fontes de Informação}

%\begin{figure}
%    \centering
%    \includegraphics[angle=0,width=1\textwidth]{}
%    \caption{Plotagem ``Três Campos'' (Sankey plot) do dataset MASSA@jhcf: 20 Autores, Citações e Palavras-Chave mais proeminentes.}
%    \label{fig:MASSA@jhcf:ThreeFieldPlot}
%\end{figure}

\section{Refinamento da Coleta de Dados}

No dia 03 de fevereiro de 2022, no decorrer das análises mais refinadas do \dataset\ MASSA@jhcf, identificou-se um grupo de artigos que não se encaixavam no tema de interesse, e que eram voltados para pesquisas no campo da biologia experimental e nanotecnologia. Isso sugeriu que a \query\  de busca precisaria ser reformulada, para excluir artigos que não se enquadrassem na temática desejada.
O conjunto das palavras-chave que refletia essa dissonância ficou evidente na análise da estrutura intelectual do conhecimento, do tipo \textbf{Rede de Co-ocorrências de Palavras-chave}, ilustrada no cluster em roxo, à esquerda da figura \ref{fig:MASSA@jhcf:redecoocorr-150-termos}.

\begin{figure}[htp]
    \centering
    \includegraphics[clip=true,trim={9cm 0cm 7cm 0cm },width=0.6\textwidth]{experiments/jhcf/PesqBibliogr/SimulacaoMultiagente/WoS-20210803/classico-mais-citacoes/Structure-Informetric/Conceptual/Co-occurrence Network-Keywords-Plus-150-termos.png}
    \caption{Rede de co-ocorrência de palavras, com 150 termos, aplicada ao \dataset\   MASSA@jhcf.}
    \label{fig:MASSA@jhcf:redecoocorr-150-termos}
\end{figure}

As seguintes 30 palavras foram identificadas nesse \textit{cluster}:
in-vitro,
adsorption,
mechanism,
water,
force-field,
molecular-dynamics,
binding,
simulations,
nanoparticles,
bubbles,
derivatives,
temperature,
in-vivo,
mathematical-model,
oscillations,
scattering,
cancer,
contrast agents,
expression,
protein,
activation,
delivery,
surface,
removal,
acid,
agent,
reduction,
aqueous-solution,
degradation,
expectations.

Ficou evidente, pela interpretação do significado da maioria desses termos, que tais artigos não tratavam de simulação de fenômenos sociais. Isso sugere que a query está com problemas de precisão, isso é, muitos registros recuperados não atendem à necessidade de informação do pesquisador. 

Algumas dessas palavras foram então escolhidas para servir como indicativas de artigos fora do escopo, e introduzidas a partir da \query\  original, gerando uma nova \query, aprimorada e ilustrada nas linhas 1 a 13 da listagem \ref{query20220203}.

\lstinputlisting[numbers=left,basicstyle=\normalsize\ttfamily,caption={\query\  de busca sobre simulação multiagente de fenômenos socials, com ênfase em métodos experimentais, com escopo negativo de artigos que tratam de experimentos biológicos em vitro.},label=query20220203]
{experiments/jhcf/PesqBibliogr/SimulacaoMultiagente/WoS-20220203/query-Refinada.txt}

Além das justificativas para os termos usados entre as linhas 1 a 9, já descritas em \ref{MASSA:query},  justifica-se na listagem \ref{query20220203}, a inclusão da cláusula \textit{not (
 adsoption or molecular-dynamics or force-field
 or in-vitro or nanopartic* or in-vivo
 or aqueous-solution or protein or surface)}, entre as linhas 10 e 13 da \query, pois elas irão remover artigos não se enquadram no escopo da busca desejada, por usarem uma ou mais desses termos no título, resumo ou palavras-chave do artigo.
 
Usando a nova \query\ de busca, foram recuperados 6.935 documentos, que se encontram em
\url{https://github.com/jhcf/Comput-Experim-20212/experiments/jhcf/PesqBibliogr/SimulacaoMultiagente/ WoS-20220203/wos6935recs.txt}. Isso sugere que aproximadamente 1.000 registros não se enquadravam na necessidade de busca.
Uma nova análise dos dados recuperados é apresentada a seguir.

\section{Nova Análise dos Dados}

\subsection{Nova filtragem de registros}

Sobre os 6.935 documentos recuperados, foram  aplicados os seguintes filtros:
\begin{itemize}
    \item Remoção dos registros de documentos que não são artigos \textit{full paper}, isso é, artigos completos publicados em revistas;
%    \item Remoção dos registros de artigos científicos que não fazem parte do \textit{core} da bibliografa, segundo a Lei de Bradford.
\end{itemize}

Após os filtros aplicados (apenas um)  obteve-se um total de 4.647 registros, que doravante serão chamados de forma coletiva, de \dataset\   MASSA2@jhcf.

\subsection{Análise descritiva do \dataset\   MASSA2@jhcf}

\subsubsection{Dados Sumários Gerais}

\begin{table}[]
    \centering
\csvautotabular[separator=semicolon
%,filter not strcmp={\csvcolii}{}
]{experiments/jhcf/PesqBibliogr/SimulacaoMultiagente/WoS-20220203/Descritiva/MASSA2-Main-Information.csv}
    \caption{Principais dados descritivos do \dataset\   MASSA2@jhcf.}
    \label{tab:MASSA2:Main}
\end{table}

Nota-se, com os resultados da tabela \ref{tab:MASSA2:Main}, que o \dataset\   abrange um período de 32 anos de publicações (1991 a 2022), evidenciando  a publicação dos 4.647 artigos em 1.910 revistas distintas. Esses artigos tem idade média de publicação de 7.8.

Adicionalmente, o \dataset\ apresenta 157.507 referências citadas, com uma média de (157.507/4.647 = ?) 33,89 referências citadas por artigo.

14.229 autores distintos produziram os artigos, com uma média de 3,73 autores por documento.

\subsubsection{Evolução anual da produção científica}

No tema de simulação multiagente de fenômenos sociais, a evolução anual da produção científica mundial é sumarizada no gráfico da figura \ref{fig:MASSA2:Evolucao}.

\begin{figure}
    \centering
    \includegraphics[width=1\textwidth]{experiments/jhcf/PesqBibliogr/SimulacaoMultiagente/WoS-20220203/Descritiva/MASSA2-Annual-Scientific-Production.png}
    \caption{Evolução da Produção Científica Anual, segundo o \dataset\ MASSA2@jhcf.}
    \label{fig:MASSA2:Annual-Scientific-Production}
\end{figure}

Entre 1991 e 2005 o crescimento de publicações era quase linear. As publicações mostram-se em ascendência forte a partir dos últimos seis anos (2015). Esse crescimento tem sido visto em várias outras áreas de conhecimento.

\subsubsection{Média de citações anuais por artigo}

O gráfico da figura \ref{fig:MASSA2:Media:Citacoes} apresenta a evolução das citações anuais médias, para os artigos do \dataset\ MASSA2@jhcf. Observa-se que há um crescimento discreto da média, onde os artigos mais recentes tendem a ser mais citados, como esperado. A redução da média no ano de 2021 deve-se, provavelmente, à insuficiência de indexação e de citação para os artigos mais recentes, tendo em vista que p ano de 2021 foi encerrado há menos de dois meses. 

\begin{figure}
    \centering
    \includegraphics[width=1\textwidth]{experiments/jhcf/PesqBibliogr/SimulacaoMultiagente/WoS-20220203/Descritiva/MASSA2-Average-Citations-per-Year.png}
    \caption{Média de citações para cada artigo do \dataset\ MASSA2@jhcf, conforme o ano de publicação}
    \label{fig:MASSA2:Media:Citacoes}
\end{figure}

Para que melhor se compreenda como foi produzido o gráfico, a tabela \ref{tab:MASSA2:Media:Citacoes} apresenta parcialmente os dados de citação anual para os artigos do \dataset\ MASSA2@jhcf. A título de exemplo, nota-se que no \dataset\ foram encontrados 9 artigos publicados no ano de 1991, tendo sido cada artigo citado, em média, aproximadamente 31,4 vezes. Dado que esses artigos já tem 31 anos citáveis, obtém-se uma média de 1,01 citações anuais, aproximadamente.

\begin{table}[]
    \centering
\csvautotabular[separator=semicolon
%,filter not strcmp={\csvcolii}{}
]{experiments/jhcf/PesqBibliogr/SimulacaoMultiagente/WoS-20220203/Descritiva/MASSA2-Average-Citations-per-Year.csv}
    \caption{Dados parciais de citação anual para os artigos do \dataset\   MASSA2@jhcf.}
    \label{tab:MASSA2:Media:Citacoes}
\end{table}

\subsubsection{Diagramas de Sankey (\textit{three fields plots})} 

A fim de apresentar mais alguns dados sumários gerais sobre  o \dataset, as figuras \ref{fig:MASSA2:Sankey:CR-AU-DE} e \ref{fig:MASSA2:Sankey:SO:DE:AU_UN} apresentam plotagens do tipo 
\textit{three fields plots}, também conhecidas pelo nome de Diagramas de Sankey \citep{riehmann_interactive_2005}, que possibilitam várias combinações de afinidades mais evidentes entre as diversas colunas dos registros do \dataset.

A primeira plotagem, figura \ref{fig:MASSA2:Sankey:CR-AU-DE}, apresenta as afinidades mais evidentes entre 15 Autores (centro), 15 Palavras-chave (direita) e 15 Referências citadas (esquerda). Ao centro, observa-se que os autores mais evidentes, segundo a técnica apresentada, tem origem asiática, a julgar pelos nomes. 

As 15 palavras-chave mais evidentes sugerem que o \dataset\ possui artigos que refletem a busca sobre o tema desejado, mas que há muitas palavras distintas que representam o mesmo conceito, como as a seguir listadas:
\begin{enumerate}
    \item multi-agent systems;
    \item multiagent system;
    \item multiagent system; 
    \item agent-based model;  
    \item agent-based models;
    \item agent-based modeling;
    \item agent-based simulation;
\end{enumerate}

As cinco palavras a seguir sugerem, o que pode ser comprovado com o aprofundamento desse estudo bibliográfico, que o estado da arte no tema busca atualmente respostas, ou possui fundamentos nas seguintes questões:
\begin{description}
    \item [event-triggered control] Como eventos disparadores exercem  controle sobre o comportamento (coletivo) de grupos sociais?
    \item [consensus] Como usar simulação multi-agente para entender o surgimento de consenso em grupos sociais?
    \item [opinion dynamics] Como usar simulação multi-agente para entender a dinâmica de opiniões que se formam em grupos sociais?
    \item [social networks] Como os métodos da análise de redes sociais podem ser usados no tema da simulação multiagente?
    \item [reinforcement learning] Como usar os métodos e técnicas de aprendizagem por reforço em simulação multiagente?
    \item [game theory] Como usar os métodos e técnicas da teoria dos jogos em simulação multiagente?
\end{description}

Algumas das referências citadas, apresentadas à esquerda do gráfico, devem evidenciar a pertinência das questões acima sugeridas, a ser comprovado até o final do estudo. 

\begin{figure}
    \centering
    \includegraphics[angle=90,width=1\textwidth,height=0.9\textheight]{experiments/jhcf/PesqBibliogr/SimulacaoMultiagente/WoS-20220203/Descritiva/MASSA2-Three-Fields-Plot-CR-AU-DE.png}
    \caption{Diagrama Sankey, relacionando as afinidades mais evidentes entre Autores (centro), Palavras-chave (direita) e Referências citadas (esquerda).}
    \label{fig:MASSA2:Sankey:CR-AU-DE}
\end{figure}

A segunda plotagem, figura \ref{fig:MASSA2:Sankey:SO:DE:AU_UN}, apresenta as afinidades mais evidentes entre 15 revistas (esquerda), 15 palavras-chave (centro) e 15 instituições de filiação dos autores (direita). Com base na técnica usada, fica evidente a proeminência dos seguintes \textit{journals} sobre os demais, sendo apresentado um breve trecho do foco de cada revista, extraído da página online da revista:
\begin{itemize}
    \item JASSS: The Journal of Artificial Societies and Social Simulation. \textit{\small  is an interdisciplinary journal for the exploration and understanding of social processes by means of computer simulation. Since its first issue in 1998, it has been a world-wide leading reference for readers interested in social simulation and the application of computer simulation in the social sciences.}. Fonte \url{https://www.jasss.org/admin/about.html};
    \item PHYSICA A-STATISTICAL MECHANICS AND ITS APPLICATIONS. \textit{\small ... publishes research in the field of statistical mechanics and its applications. Statistical mechanics sets out to explain the behaviour of macroscopic systems, or the large scale, by studying the statistical properties of the microscopic or nanoscopic constituents. Applications of the concepts and techniques of statistical mechanics include: applications to physical and physiochemical systems such as solids, liquids and gases, interfaces, glasses, colloids, complex fluids, polymers, complex networks, applications to economic and social systems (e.g. socio-economic networks, financial time series, agent based models, systemic risk, market dynamics, computational social science, science of science, evolutionary game theory, cultural and political complexity), and traffic and transportation (e.g. vehicular traffic, pedestrian and evacuation dynamics, network traffic, swarms and other forms of collective transport in biology, models of intracellular transport, self-driven particles), as well as biological systems (biological signalling and noise, biological fluctuations, cellular systems and biophysics); and other interdisciplinary applications such as artificial intelligence (e.g. deep learning, genetic algorithms or links between theory of information and thermodynamics/statistical physics.).}. Fonte: \url{https://www.journals.elsevier.com/physica-a-statistical-mechanics-and-its-applications};
    \item IEEE ACCESS. \textit{\small ... is a multidisciplinary, all-electronic archival journal, continuously presenting the results of original research or development across all IEEE’s fields of interest. Supported by article processing charges (APC), its hallmarks are a rapid peer review and publication process of 4 to 6 weeks, with open access to all readers.}. Fonte: \url{https://ieeeaccess.ieee.org/about-ieee-access/learn-more-about-ieee-access/}
\end{itemize} 

\begin{figure}
    \centering
    \includegraphics[angle=90,width=1\textwidth,height=0.9\textheight]{experiments/jhcf/PesqBibliogr/SimulacaoMultiagente/WoS-20220203/Descritiva/MASSA2-Three-Fields-Plot-SO:DE:AU_UN.png}
    \caption{Diagrama Sankey, relacionando as afinidades mais evidentes entre revistas (esquerda), palavras-chave (centro) e instituição de filiação dos autores (direita).}
    \label{fig:MASSA2:Sankey:SO:DE:AU_UN}
\end{figure}

Observa-se, com base no escopo declarado de cada uma das revistas, que a revista JASSS é bem enquadrada no escopo da busca, enquanto que a revista IEEE Access não tem relação direta com o tema. Já a revista Physica A aborda o tema de forma mais ampla do que o buscado, com ênfase em métodos da mecânica estatística, que não são os únicos possíveis de serem empregados.
Os nomes das demais 8 revistas, apresentadas no diagrama, são os seguintes:
\begin{enumerate}
    \item NEUROCOMPUTING \textit{\small  ... welcomes theoretical contributions aimed at winning further understanding of neural networks and learning systems, including, but not restricted to, architectures, learning methods, analysis of network dynamics, theories of learning, self-organization, biological neural network modelling, sensorimotor transformations and interdisciplinary topics with artificial intelligence, artificial life, cognitive science, computational learning theory, fuzzy logic, genetic algorithms, information theory, machine learning, neurobiology and pattern recognition.}. Fonte: \url{https://www.journals.elsevier.com/neurocomputing};
    \item IEEE TRANSACTIONS ON AUTOMATIC CONTROL \textit{\small publishes high-quality papers on the theory, design, and applications of control engineering.  Two types of contributions are regularly considered: 
1) Papers:  Presentation of significant research, development, or application of control concepts. 
2) Technical Notes and Correspondence:  Brief technical notes, comments on published areas or established control topics, corrections to papers and notes published in the Transactions.
In addition, special papers (tutorials, surveys, and perspectives on the theory and applications of control systems topics) are solicited. }. Fonte: \url{https://ieeexplore.ieee.org/xpl/aboutJournal.jsp?punumber=9};
    \item AUTONOMOUS AGENTS AND MULTI-AGENT SYSTEMS \textit{\small is the official journal of the International Foundation for Autonomous Agents and Multi-Agent Systems. It provides a leading forum for disseminating significant original research results in the foundations, theory, development, analysis, and applications of autonomous agents and multi-agent systems. Coverage in Autonomous Agents and Multi-Agent Systems includes, but is not limited to:
Agent decision-making architectures and their evaluation, including: cognitive models; knowledge representation; logics for agency; ontological reasoning; planning (single and multi-agent); reasoning (single and multi-agent)
Cooperation and teamwork, including: distributed problem solving; human-robot/agent interaction; multi-user/multi-virtual-agent interaction; coalition formation; coordination
Agent communication languages, including: their semantics, pragmatics, and implementation; agent communication protocols and conversations; agent commitments; speech act theory
Ontologies for agent systems, agents and the semantic web, agents and semantic web services, Grid-based systems, and service-oriented computing
Agent societies and societal issues, including: artificial social systems; environments, organizations and institutions; ethical and legal issues; privacy, safety and security; trust, reliability and reputation
Agent-based system development, including: agent development techniques, tools and environments; agent programming languages; agent specification or validation languages
Agent-based simulation, including: emergent behavior; participatory simulation; simulation techniques, tools and environments; social simulation
Agreement technologies, including: argumentation; collective decision making; judgment aggregation and belief merging; negotiation; norms
Economi c paradigms, including: auction and mechanism design; bargaining and negotiation; economically-motivated agents; game theory (cooperative and non-cooperative); social choice and voting
Learning agents, including: computational architectures for learning agents; evolution, adaptation; multi-agent learning.
Robotic agents, including: integrated perception, cognition, and action; cognitive robotics; robot planning (including action and motion planning); multi-robot systems.
Virtual agents, including: agents in games and virtual environments; companion and coaching agents; modeling personality, emotions; multimodal interaction; verbal and non-verbal expressiveness
Significant, novel applications of agent technology
Comprehensive reviews and authoritative tutorials of research and practice in agent systems
Comprehensive and authoritative reviews of books dealing with agents and multi-agent systems.
Official journal of the International Foundation for Autonomous Agents and Multi-Agent Systems.
Covers the foundations, theory, development, analysis, and applications of autonomous agents and multi-agent systems.
Presents comprehensive reviews and authoritative tutorials of research and practice in agent systems.}. Fonte: \url{https://www.springer.com/journal/10458};
    \item INTERNATIONAL JOURNAL OF MODERN PHYSICS C \textit{\small is a journal dedicated to Computational Physics and aims at publishing both review and research articles on the use of computers to advance knowledge in physical sciences and the use of physical analogies in computation. Topics covered include: algorithms; computational biophysics; computational fluid dynamics; statistical physics; complex systems; computer and information science; condensed matter physics, materials science; socio- and econophysics; data analysis and computation in experimental physics; environmental physics; traffic modelling; physical computation including neural nets, cellular automata and genetic algorithms.}. Fonte: \url{https://www.worldscientific.com/page/ijmpc/aims-scope};
    \item COMPLEXITY \textit{\small The purpose of Complexity is to report important advances in the scientific study of complex systems. Complex systems are characterized by interactions between their components that produce new information — present in neither the initial nor boundary conditions — which limit their predictability. Given the amount of information processing required to study complexity, the use of computers has been central to complex systems research. Concepts relevant to Complexity include:
    Adaptability, robustness, and resilience;
    Complex networks;
    Criticality;
    Evolution and emergent behaviour;
    Nonlinear dynamics;
    Pattern formation;
    Self-organization.
Methods used within the scientific study of complex systems frequently include:
    Agent-based modelling;
    Analytical methods;
    Cellular automata;
    Computational methods;
    Data science;
    Game theory;
    Machine learning;
    Statistical mechanics.
Applications of complex systems may be related to the following disciplines, among others:
Computational social science;    Digital epidemiology;
    Ecology;
    Economics;
    Engineering;
    Socio-technical systems;
    Statistical linguistics;
    Systems biology;
    Urban systems.}. Fonte: \url{https://www.hindawi.com/journals/complexity/about/};
    \item ECOLOGICAL MODELLING \textit{\small publishes new mathematical models and systems analysis for describing ecological processes, and novel applications of models for environmental management.
We welcome research on process-based models embedded in theory with explicit causative agents and innovative applications of existing models. And because applications can help refine models and propose new directions for research, the journal publishes both to help foster reproducibility and utility.Human activity and well-being are dependent on and integrated with the functioning of ecosystems and the services they provide. We aim to understand these basic ecosystem functions using mathematical and conceptual modelling, systems analysis, thermodynamics, computer simulations, and ecological theory, and look to a wide spectrum of applications ranging from basic ecology to human ecology to socio-ecological systems. The journal welcomes original research articles, review articles, viewpoint articles and short communications.}. Fonte: \url{https://www.journals.elsevier.com/ecological-modelling};
    \item JOURNAL OF THEORETICAL BIOLOGY \textit{\small is the leading forum for theoretical perspectives that give insight into biological processes. It covers a very wide range of topics and is of interest to biologists in many areas of research, including:
Brain and Neuroscience;
Cancer Growth and Treatment;
Cell Biology;
Developmental Biology;
Ecology;
Evolution;
Immunology;
Infectious and non-infectious Diseases;
Mathematical, Computational, Biophysical and Statistical Modeling;
Microbiology, Molecular Biology, and Biochemistry;
Networks and Complex Systems;
Physiology;
Pharmacodynamics;
Animal Behavior and Game Theory}. Fonte: \url{https://www.journals.elsevier.com/journal-of-theoretical-biology};
    \item JOURNAL OF STATISTICAL MECHANICS-THEORY AND EXPERIMENT \textit{\small is targeted to a broad community interested in different aspects of statistical physics, which are roughly defined by the fields represented in the conferences called 'Statistical Physics'. Submissions from experimentalists working on all the topics which have some 'connection to statistical physics are also strongly encouraged.
The journal covers different topics which correspond to the following keyword sections:
Quantum statistical physics, condensed matter, integrable systems;
Classical statistical mechanics, equilibrium and non-equilibrium;
Disordered systems, classical and quantum;
Interdisciplinary statistical mechanics;
Biological modelling and information}. Fonte: \url{https://iopscience.iop.org/journal/1742-5468/page/about_the_journal}.
\end{enumerate}
Considerando a possibilidade de pertinência da questão ecológica ao tema dos sistemas multi-agentes, todos os \textit{journals} identificados apresentam pertinência às perguntas de pesquisa formuladas. 

Acerca das instituições de filiação dos autores, nota-se que 1/3 delas é localizada na china, 1/3 nos EUA, e o restante na Europa e Austrália. É provável que muitos pesquisadores de origem chinesa trabalhem em universidades fora da china, no tema do \dataset.


\subsection{Medidas bibliométricas}

As medidas bibliométricas propriamente ditas, relativas ao \dataset\ MASSA2@jhcf, serão exploradas nesta subseção, e são organizadas em três conjuntos:
\begin{description}
    \item [Relativas às Fontes de Informação] Uma vez que foram consideradas apenas as publicações em revistas, todas as fontes de informação mensuradas serão revistas científicas, ou \textit{journals}. As principais medidas são de impacto das fontes, mensuradas com base no número de citações que os artigos publicados nas revistas obtiveram de outras publicações, possivelmente feitas em outras fontes de informação, como outras revistas, seções de livros, artigos de conferência etc. As citações são registradas pelas organizações que fazem indexação de artigos, como a Web of Science e SCOPUS;
    \item [Relativas aos Autores] Sempre que um artigo publicado por um ou mais autores e também indexado por uma organização (Web of Science,  SCOPUS etc), é citado em um outro artigo também indexado por essa mesma organização, então é feita a anotação de uma citação ao mesmo, e o impacto potencial desse autor sobre a ciência é atestado pelo valor mais alto da citação do conjunto de seus artigos indexados. Várias métricas (índice H, G, M etc) podem ser derivadas dessa medida (quantidade de citações), e são exploradas tanto em relação aos autores como em relação às revistas onde esses artigos foram publicados;
    \item [Relativas aos Documentos] Cada citação adicional a  um documento (artigo de revista, de conferência, livro, ou  capítulo de livro) é um indicador do impacto do documento em si, que evidencia a sua importância. Além das citações, a ocorrência de palavras dentro dos documentos, inclusive ordenada pelo tempo, também produz indicadores numéricos (métricas) relevantes para analisar a importância do documento em relação a outros. 
\end{description}

Essas medidas serão apresentadas a seguir.

\subsubsection{Bibliometrias aplicadas aos documentos (Artigos científicos) no \dataset}

\paragraph{Citações globais aos artigos no \dataset}

Cada registro recuperado no \dataset\ apresenta um conjunto de informações, dentre as quais pode constar a quantidade de vezes que uma citação ao mesmo foi registrada no índice do WoS, desde que no momento da extração seja feita essa solicitação (\textit{TC - Times Cited}).
A tabela \ref{tab:MASSA2:GlobalCitations} apresenta a lista dos 25 artigos do \dataset, que foram mais citados, ordenados de forma decrescente pelo número global de citações do artigo, nos índices da WoS. Para ada artigo é apresentada a referencia abreviada, o DOI e a quantidade de vezes que ele foi citado globalmente (no índice do WoS). Para recuperar a página do artigo deve-se abrir uma url prefixada com \url{http://doi.org/}, e informar o valor do DOI indicado, por exemplo \url{http://doi.org/10.1109/TAC.2008.2010897} levará à página do artigo mais citado, cujo título é ``Flocking of Multi-Agents With a Virtual Leader''.

\begin{table}[]

    \centering
\footnotesize
\csvreader[tabular = |r|l|l|r|,
separator=semicolon
%,filter not strcmp={\csvcolii}{},
, table head = \hline\hline \# & Artigo (Referência Abreviada) & DOI (Digital Object Identifier) & Cit.\\ \hline\hline,
table foot = \hline\hline
]{experiments/jhcf/PesqBibliogr/SimulacaoMultiagente/WoS-20220203/Metricas/Documentos/MASSA2-Most-Global-Cited-Documents.csv}{Paper=\paper, DOI=\doi,Total Citations=\totcit}{ \thecsvrow & {\tiny\paper} & {\tiny \doi} & \totcit}

    \caption{25 artigos mais citados no \dataset\ MASSA2@jhcf.}
    \label{tab:MASSA2:GlobalCitations}
\end{table}


\paragraph{Referências aos (outros) artigos, capítulos de livros etc (documentos) citados pelos artigos no \dataset}

\paragraph{Uso de palavras dentro dos artigos no \dataset}

\chapter{Análise Bibliográfica sobre Processamento de Linguagem Natural, por Lucas de Almeida Bandeira Macedo}

\section{Planejamento do estudo}

Com a vinda de assistentes virtuais, como a Alexa (Amazon), Cortana (Microsoft) ou Siri (Apple), as pessoas costumam se perguntar cada vez mais: "como que esse programa está entendendo o que eu falo?".

Mas não só de assistentes virtuais vive o Processamento de Linguagem Natural (também conhecido como NLP - Natural Language Processing), afinal, qualquer texto ou fala pode ser interpretado por uma máquina e devidamente classificado. Por exemplo, uma aplicação famosa é o "classificador de sentimentos", em que um modelo treinado consegue classificar textos entre sentimentos "positivos" ou "negativos". Com a ascensão do Twitter, uma rede social baseada em pequenos textos de não mais que 280 caracteres, NLP se torna cada vez mais interessante.

Assim, as perguntas que traçam o norte para este estudo são:

\begin{itemize}
    \item Quais os principais conceitos ligados com Processamento de Linguagem Natural?
    \item Como se dá o progresso das pesquisas em NLP ao longo dos anos? As redes sociais influenciaram esse crescimento?
    \item Qual o estado da estrutura social da comunidade de NLP?
\end{itemize}

\subsection{Uso do Bibliometrix e Biblioshiny}

Será usada a ferramenta Bibliometrix, com sua função Biblioshiny, para gerar gráficos e grafos iterativos e personalizáveis, para auxiliar na interpretação da realidade científica do tópico.

\section{Coleta de dados}

A coleta de dados foi feita utilizando o site Web of Science (WoS), no dia 03/02/2022, através do portal periódico da capes.

A pesquisa foi realizada utilizando as edições "Science Citation Index Expanded" e "Conference Proceedings Citation Index – Science", ambas coleções são voltadas para, principalmente, as ciências exatas.

A \textit{string} (ou \textit{query}) de busca inicialmente utilizada foi a seguinte:

\lstinputlisting[numbers=left,basicstyle=\normalsize\ttfamily,caption={Query de busca sobre Procesasmento de Linguagem Natural.},label=queryNLP03022022]
{experiments/ABMHub/PesquisaBibliometrica/NLP/pesquisa_velha.txt}

\subsection{Explicação para os termos de busca usados}

A proposta é apenas pesquisar sobre Processamento de Linguagem Natural, sem muito rigor na aplicação em que essa arquitetura de rede neural é aplicada. Portanto, inicialmente a pesquisa foi apenas "natural language processing".

Porém, uma rápida olhada pelos artigos retornados evidenciou uma grande quantidade de artigos sobre linguísticas, e áreas que não são da computação. Como o objetivo aqui adquirir modelos de Deep Learning, a pesquisa foi ajustada para filtrar apenas por NLP ligadas diretamente a computação e inteligência artificial, evidenciado pelas cláusulas "neural network", "(machine or deep) and learning" e "artificial intelligence". Essa nova pesquisa trouxe melhores resultados, todos evidenciando redes neurais e variadas técnicas de machine learning. O total de registros retornado pela query foi 

\subsubsection{Refinamento da Coleta de Dados}

 Em seguida, em uma análise mais fina, utilizando a \textbf{Rede de Co-ocorrências de Palavras-chave}, podemos evidenciar outras palavras chaves que estavam aparecendo entre os registros da pesquisa, que não deveriam estar aparecendo. É possível observar na imagem \ref{fig:ABMHub:NLPgraph1}, palavras como "câncer" ou "diagnóstico" que estão relacionadas a visão computacional mais que NLP, aparecendo com pesos não-desprezíveis.
 
 \begin{figure}
    \centering
    \includegraphics[angle=0,width=1\textwidth]{experiments/ABMHub/PesquisaBibliometrica/NLP/network.png}
    \caption{Grafo de relação de keywords}
    \label{fig:ABMHub:NLPgraph1}
\end{figure}

Assim, é necessário uma nova iteração da pesquisa, para evitar que registros de visão computacional corrompam a pesquisa de NLP. É delicado fazer isso, pois existem muitas menções a Visão Computacional nos registros de LP, já que ambos são ligados a Deep Learning, então retirar a keyword "Visão Computacional" provavelmente removeria muitos registros que não gostaríamos de remover da pesquisa. Assim, a melhor solução encontrada foi remover palavras que não têm intersecção entre os dois assuntos. Por exemplo, "medical", "cancer" e "diagnosis".

Assim, chegamos na mais recente query:

\lstinputlisting[numbers=left,basicstyle=\normalsize\ttfamily,caption={Query de busca sobre Procesasmento de Linguagem Natural.},label=queryNLP03022022]
{experiments/ABMHub/PesquisaBibliometrica/NLP/pesquisa_nova.txt}



\chapter{Análise Bibliográfica sobre Otimizações algorítmicas para simulações de fenômenos fluídos e óticos por Alexsander Correa de Oliveira}

\section{Planejamento do estudo}
    A indústria de jogos é a que mais cresce dentre todas as formas de entretenimento atuais. Alguns desses jogos podem chegar a investimentos tão grandes que disputam com os filmes mais caros da história. No posto de vista do consumidor, todo tempo e dinheiro gastos são apenas meios para um fim, que é o de ter a melhor experiência possível. Contudo do ponto de vista dos desenvolvedores, esses fatores são consequências de horas e horas de trabalho.
    
    Toda a tecnologia criada para esses jogos tem de ser cada vez mais eficiente, dado a necessidade de tornar os gráficos cada vez mais realistas, e seus mundos ainda mais acreditáveis. As técnicas utilizadas para tal otimização são fortemente baseadas em artigos \emph{state-of-the-art} tanto e física quanto em geometria.
    
    Entre todos aspectos físicos, que hoje em dia são mais prevalentes nos jogos, temos algumas áreas de estudo que pesam mais, principalmente em performance: 
    \begin{itemize}
        \item Corpos fluídos, como água e ar;
        \item Análise de vetores em ótica, para saber como a iluminação afetará um determinado ambiente;
        \item Análise da topologia, com fins de otimizar \emph{path-finding};
    \end{itemize}
    
    Considerando o corpo de estudo, algumas questões surgem:
    \begin{item}
        \item Quais são os principais interesses relacionados ao estudo dos fenômenos naturais voltados a computação?
        \item Quem são os agentes que produzem o maior volume de artigos?
        \item Quais são os corpos de estudo mais relacionados entre si?
    \end{item}
    Todas elas serão analisadas e respondidas no decorrer da seção.
\subsection{Uso do Bibliometrix e Biblioshiny}
     Com o auxílio das ferramentas disponibilizadas pelo Bibliometrix, como o Biblioshiny, serão analisados os artigos encontrados, por meio de gráficos e tabelas.   
\section{Coleta de Dados}
    A coleta de dados foi iniciada no dia 02/02/2022, e usou a base Web of Science, com acesso direto pelo periódico Capes.
    
    Por fins de diminuir o tamanho do \emph{dataset}, só foi utilizada a edição \emph{Science Citation Index Expanded}, que tem o foco voltado para ciências exatas e naturais.
    
    A busca inicial foi feita com a seguinte \emph{query}:
\begin{lstlisting}[basicstyle = \normalsize]
((algorit* ) and (Optimization)) and 
(optics or ((fluid* or aero*) and dynamics))
\end{lstlisting}
\subsection{Explicação para a \emph{Query}}
    A busca foi feita com o objetivo de encontrar apenas técnicas para otimizar algoritmos relacionados a ótica, aerodinâmica e hidrodinâmica.
    
    Os termos \verb|((algorit* ) and (Optimization))| são para encontrar apenas os artigos relacionados a algoritmos computacionais.
    
    Já \verb|(optics or ((fluid* or aero*) and dynamics))| serve para falar que tanto faz um artigo de ótica ou de aerodinâmica ou de hidrodinâmica.
    
    Com a \emph{Query} já montada, os registros foram exportados do WoS com todas as informações disponíveis e no formato de arquivo de texto sem formatação. Foram recuperados desa maneira, 7443 registros no total.
\section{Análise dos dados}
    Uma análise inicial foi feita com o objetivo de retirar artigos indesejados. Para atingir isso, foi utilizado o gráfico \emph{Co-occurrence Network}, que mostra as palavras com maior peso, e o relacionamento entre elas.
    
     \begin{figure}
    \centering
    \includegraphics[width=1\textwidth]{experiments/KvotheKS/PesqBibliogr/AlgoritmosSimulacaoOptica-Dinamica/WoS-20220202/OldQueryDataset/CoOccurrence.png}
    \caption{Rede de co-ocorrência}
    \label{fig:KvotheKS:OldQueryCoOccurrence}
\end{figure}
    
    Destacando um dos lados do grafo e o meio, podemos ver que o foco em computação e otimização foram atingidos. Contudo, como um efeito não desejado, também foram "recebidos" artigos que envolvem I.A, como também ótica de um ponto de vista médico.
    
\subsection{Refinamento dos Dados}
    Para retirar todo produto indesejado foi feita uma nova \emph{query} na mesma base e edição:
    
\begin{lstlisting}[basicstyle = \normalsize]
((algorit* ) and (Optimization)) and 

(optics or ((fluid* or aero*) and dynamics))

not ((genetic* and algorit*) or medic* or (machin* and learn*))
\end{lstlisting}

    Com os novos parâmetros, o objetivo de retirar tudo relacionado a medicina e a maioria de algoritmos genéticos foi atingido. Como resultado da nova busca, foram retornados 4917 elementos. Contudo, considerando apenas o número de artigos, o número cai para 4859.
    
    Como demonstração da melhora do \emph{dataset}, segue o gráfico \ref{fig:KvotheKS:Final_Data_Set}:
    
    \begin{figure}[H]
    \centering
    \includegraphics[width=1.3\textwidth]{experiments/KvotheKS/PesqBibliogr/AlgoritmosSimulacaoOptica-Dinamica/WoS-20220202/Dataset/AU_CR_DE.png}
    \caption{Dataset final}
    \label{fig:KvotheKS:Final_Data_Set}
\end{figure}

\subsection{Análise descritiva do \emph{dataset}}
    As informações iniciais do \emph{dataset} de 4859 registros são as seguintes:
    
\begin{description}
    \item [\textit{Timespan}] Todos os artigos que passaram pelo filtro e pela busca foram feitos de 1985 a 2022.
    \item [\textit{Sources (Journals, Books, etc)}] São 924 fontes de informação registradas.
    \item [\textit{Average years from publication}] A média de tempo para publicação é de 7,95 anos.
    \item [\textit{Average citations per documents}] A média de citações dos artigos é de 17,87 vezes.
    \item [\textit{Average citations per year per doc}] Os artigos, após sua publicação, tiveram em média 1,887 citações anuais.
    \item [\textit{References}] A quantidade total de referências do \emph{dataset} se dá em 127.349.
    \item [\textit{Keywords Plus (ID)}] 7.218 palavras-chave distintas foram encontradas no \emph{dataset}.
    \item [\textit{Author's Keywords (DE)}] 10.367 palavras-chave distintas escritas pelos autores.
    \item [\textit{Authors}] No total, foram 14.247 autores, sendo que boa parte deles tem origem chinesa.
    \item [\textit{Author Appearances}] No total, tiveram 20.024 aparições de autores, sendo que o número de autores distintos é, como mencionado anteriormente, 14.247
    \item [\textit{Authors of single-authored documents}] Dentre o número total de autores, apenas 206 fizeram pelo menos 1 artigo sozinhos.
    \item [\textit{Authors of multi-authored documents}] Se retirarmos do número total de autores, o número de autores que escreveram artigo(s) sozinhos, chegamos em 14.041 autores que escreveram apenas artigos coletivos.
    \item [\textit{Single-authored documents}] Dentro do \emph{dataset} apenas 227 deles são de criação individual.
    \item [\textit{Documents per Author}] Se dividirmos o número total de artigos pela quantidade de autores, chegamos em 0,341 artigos/autor.
    \item [\textit{Authors per Document}] Agora, inversamente se fizermos a quantidade de autores distintos divido pelo número de artigos, chegamos em 2,93 autores(/artigo.
    \item [\textit{Co-Authors per Documents}] Se pegarmos o número total de autores (também os repetidos) e dividirmos pela quantidade de documentos, temos 4.12 autores/artigo
    \item [\textit{Collaboration Index}] Por fim, a quantidade de vezes que autores distintos editaram artigos com um ou mais co-autores é de 3,03.
\end{description}
\subsection{Evolução da Produção Científica}
    Os temas procurados na busca são consideravelmente mais recentes que o esperado. O gráfico \ref{fig:KvotheKS:Annual_Scientific} mostra um crescimento quase que perfeitamente exponencial, sendo ele de 12.45\%.
    \begin{figure}[H]
    \centering
    \includegraphics[width=1\textwidth]{experiments/KvotheKS/PesqBibliogr/AlgoritmosSimulacaoOptica-Dinamica/WoS-20220202/Dataset/Annual_Scientific.png}
    \caption{Produção anual científica}
    \label{fig:KvotheKS:Annual_Scientific}
\end{figure}
\subsection{Interpretação do crescimento}
    Com o avanço dos computadores e uma disponibilidade maior de recursos científicos como um todo, vários temas acabam ganhando força por fatores variados. No caso do meu \emph{dataset}, os estudos vão de análise topológica para robôs a estudo de aerodinâmica para aviões, e no fim acabam em simulações de iluminação.
\subsection{\emph{Clustering Network}}
    Como meio de demonstrar o quão "compacto" estão os resultados do \emph{dataset}, podemos utilizar uma \emph{Clustering Network}, que mostra em, em forma de grafo, quais estudos estão relacionados entre si, e qual peso de cada um.  
\begin{figure}[H]
    \centering
    \includegraphics[width=1\textwidth]{experiments/KvotheKS/PesqBibliogr/AlgoritmosSimulacaoOptica-Dinamica/WoS-20220202/Dataset/Cluster_network.png}
    \caption{Grafo de citações}
    \label{fig:KvotheKS:Cluster_}
\end{figure}
\subsection{Interpretação da rede}
    Analisando a figura \ref{fig:KvotheKS:Cluster_}, podemos ver o quão inter-relacionados os artigos estão. Isso é um resultado óbvio do refinamento de dados feito anteriormente. Também mostra que alguns tópicos, como hidrodinâmica, aparecem em maior peso, por causa das revistas em qual artigos foram publicados. 
\subsection{\emph{Three-Field Plot}}
    Já foi demonstrado um dos \emph{Sankey diagrams} anteriormente \ref{fig:KvotheKS:Final_Data_Set}, onde o resultado mais interessante são as palavras-chave a direita, que mostram realmente quais são os tópicos mais abordados no \emph{dataset}. Porém alguns dados interessantes não foram abordados.
    
\begin{figure}[H]
    \centering
    \includegraphics[width=1.1\textwidth]{experiments/KvotheKS/PesqBibliogr/AlgoritmosSimulacaoOptica-Dinamica/WoS-20220202/Dataset/AU_CO_AU_UN_SO.png}
    \caption{Afiliações, revistas e países}
    \label{fig:KvotheKS:SankeyCountry}
\end{figure}
\subsection{Considerações do peso dos países}
    Os dados interessantes da figura \ref{fig:KvotheKS:SankeyCountry} se dão nos países e universidades. Mais da metade dos artigos são chineses, porém não só há uma diversidade grande de universidades chinesas, mas também há uma falta de diferença entre as estado-unidenses, por onde artigos de vários países acabam passando.
\section{}
%\chapter{Análise Bibliográfica sobre Simulação , por }%\label{}


\section{Planejamento do estudo}


\begin{itemize}
    \item 
\end{itemize}

%% Keywords usadas: (graphic processing unit or GPU) and (lighting or light or shadow*)

\chapter{Análise Bibliográfica sobre , por Gustavo Tomás}

\section{Planejamento do estudo}

O objetivo do trabalho é analisar o impacto das GPUs (Graphic Processing Units) no processamento e simulação da luz. Para isso, foram utilizadas as ferramentas Bibliometrix e Biblioshiny.

\subsection{O que já existe de pesquisa bibliométrica sobre esse tema?}

\subsection{Limitações} O exercício relatado foi feito em cerca de uma semana, entre os dias 02 e 10 de fevereiro de 2022 e a base de dados utilizada foi Web Of Science (WoS).

\section{Coleta de dados}

A coleta de dados foi feita usando o WoS no dia 03/02/2022, por meio do Portal de Periódicos da CAPES. Foram feitas buscas nas coleções Science Citation Index Expanded (SCI-EXPANDED) e Social Sciences Citation Index (SSCI), mas com o foco em registros relativos a área de ciências naturais e exatas. A busca utilizada foi a seguinte:

\begin{verbatim}
(graphic processing unit or GPU) and (lighting or light or shadow*)
\end{verbatim}

Essa busca consiste em dois termos, sendo que o primeiro é composto pela GPU (por extenso ou pela sigla) e o segundo pelas palavras luz ou iluminação ou sombra(s). Dessa forma, foram encontrados 1311 registros, sendo que nesse trabalho foram utilizados os primeiros 1000 registros, disponíveis em \ref{}.

\section{Análise dos dados}

\subsection{Filtragem de registros}
Antes da análise, foram aplicados filtros aos registros, de forma que apenas registros do tipo \textit{article}, de qualquer ano e com qualquer número de citações, fossem analisados. O resultado consiste em 850/1000 registros originais.

\subsection{Análise descritiva do dataset}

As informações mais gerais sobre o \textit{dataset} MASSA@jhcf são as seguintes:
\begin{description}
    \item [\textit{Timespan}] Os artigos que atenderam aos critérios de busca e filtragem foram publicados a partir de 1990, até 2021. Ou seja, não foram encontrados registros entre 1945 e 1989.
    \item [\textit{Sources (Journals, Books, etc)}] São 2.319 fontes de informação que publicaram os documentos recuperados no dataset MASSA@jhcf. Ou seja, em média, cada \textit{scientific journal} publicou $5.787/2.319=2,5$ artigos. \footnote{Note que a média, enquanto medida de tendência central, pode não ser a que melhor reflete a tendência a quantidade de artigos publicados por revista.}
    \item [\textit{Average years from publication}] A média do tempo de publicação dos artigos no dataset MASSA@jhcf é de 7,36 anos.
    \item [\textit{Average citations per documents}] Cada artigo no dataset MASSA@jhcf foi citado, em média 20,7 vezes\footnote{Note que a média, enquanto medida de tendência central, pode não ser a que melhor reflete a tendência de  citações a artigos.}.
    \item [\textit{Average citations per year per doc}] Após publicado, cada um dos 5.787 artigos do dataset MASSA@jhcf  foi citado 2,262 vezes por ano, em média.
    \item [\textit{References}] O dataset MASSA@jhcf contém 201.464 referências citadas (tags CR).
    \item [\textit{Keywords Plus (ID)}] 13.735 distintas palavras-chave do tipo Keywords Plus (ID)\footnote{\textit{KeyWords Plus} são ``termos de índice gerados automaticamente a partir dos títulos de artigos citados. Os termos do KeyWords Plus devem aparecer mais de uma vez na bibliografia e são ordenados de frases com várias palavras a termos únicos. O KeyWords Plus aumenta o número de resultados tradicional de palavras-chave ou títulos.'' Fonte: \url{https://images.webofknowledge.com/WOKRS410B4/help/pt_BR/WOS/hp_full_record.html}} foram encontradas no dataset MASSA@jhcf. 
    \item [\textit{Author's Keywords (DE)}] 15.704 distintas palavras-chave indicadas pelos autores foram encontradas no \textit{dataset}.
    \item [\textit{Authors}] 19.410 distintos nomes de autores foram encontrados no dataset\footnote{Um mesmo autor pode ter uma ou mais diferentes grafias no dataset, e serão reconhecidos dois ou mais autores diferentes, embora de fato sejam apenas um. Isso significa que a quantidade de \textbf{nomes de autores} equivale à quantidade de \textbf{autores}. Adicionalmente, é possível que distintos autores sejam reconhecidos com o mesmo nome, isso é, que sejam homônimos. Ou seja, o dataset em geral conterá erros de contagem na quantidade de autores reais.}.
    \item [\textit{Author Appearances}] Os 19.410 distintos (nomes de) autores foram encontrados 23.470 vezes, como autores de artigos.
    \item [\textit{Authors of single-authored documents}] Dentre os 19.410 distintos (nomes de) autores encontrados, 375 deles editaram artigos individualmente, isso é, sem co-autores.
    \item [\textit{Authors of multi-authored documents}] Dentre os 19.410 distintos (nomes de) autores encontrados, 19.035 deles editaram artigos com um ou mais co-autores"
    \item [\textit{Single-authored documents}] Dentre os 5.787 documentos presentes no dataset MASSA, 409 foram escritos por um único autor, e os 5.378 restantes foram elaborados em co-autoria.
    \item [\textit{Documents per Author}] Dentre os 19.410 distintos (nomes de) autores, cada um publicou em média 0,298 artigos.
    \item [\textit{Authors per Document}] Cada um dos 5.787 documentos presentes no dataset MASSA foi autorado com 3,35 autores em média ($19.410 / 5.787 = 3,35$).
    \item [\textit{Co-Authors per Documents}] As 23.470 aparições de (nomes de) autores (``Author Appearances''), sem distribuem, em média 4,06 vezes para os 5.787 documentos do dataset MASSA@jhcf.
    \item [\textit{Collaboration Index}] Os 19.035 (nomes de) autores que editaram artigos com um ou mais co-autores, colaboraram em media 3,54 vezes para editar os 5.378 artigos elaborados em co-autoria, gerando, assim, um índice de colaboração 3,54. 
\end{description}

\subsection{Evolução da Produção Científica}

\subsection{Interpretação do Crescimento}

\subsection{Evolução das Citações}

\subsection{Interpretação das Citações}

\subsection{\textit{Three-Field Plots (Sankey diagram)}}

\subsection{Interpretação da figura}

\subsection{Análises Bibliométricas: Fontes de Informação}

\subsection{Análises Bibliométricas: Autores}

\subsection{Análises Bibliométricas: Documentos}



\subsection{Minhas impressões iniciais sobre a ciência, por Ítalo Eduardo Dias Frota}

A ciência é um escopo de \gls{Conhecimento} e um processo. A área se apoia na busca e aplicação dos conhecimentos a respeito das esferas sociais e naturais, seguindo uma \gls{Metodologia} bem definida baseada em evidências que descrevam, expliquem e possam prever um \gls{fenomeno}. Entretanto, definir a ciência não é uma tarefa fácil devido a pluralidade de aplicações e abordagens que permeiam a \gls{ComunidadeCientifica}.

Um dos aspectos mais importantes no campo científico é o da metodologia científica, que permite a formulação de hipóteses, experimentos  e verificações. Os resultados obtidos devem ser reproduzidos através de artigos e publicações que auxiliem na propagação dos pontos observados, instigando novas descobertas e indagações. Sendo assim, o processo é extremamente autocorretivo, pois está em constante evolução.

Sem a ciência e o pensamento científico, a humanidade enfrentaria dificuldades em larga escala. As descobertas e previsões documentadas pelos membros da comunidade são de extrema importância para os avanços nas mais diversas áreas, desde a saúde, até a educação e tecnologia. É imprescindível que seja dada a devida atenção às observações coletadas pela ciência, caso contrário, a humanidade estará sempre destinada ao fracasso.


%\chapter{Pesquisas Bibliográficas: Respostas d(a/o)s Estudantes}

\section{Tarefa 4: Análise Bibliométrica com apoio de R/R Studio e Bibliometrix}

A tarefa vale 20 pontos, e consiste em produzir individualmente uma análise bibliométrica inicial, abordando um tema de computação que  interessa ao autor. 

A análise bibliométrica, o produto da tarefa, deve estar apresentada num capítulo da parte \ref{part:estudos:exploratorios}, no relatório da turma no Overleaf, no diretório ``2-Analise-Exploratoria-Dados/tarefas/2.2-Pesquisa-Bibliografica/estudantes/'', em um arquivo de nome "tarefa-\githubusername.tex. Ver um exemplo com a complexidade referencial, no arquivo ``tarefa-jhcf.tex'', no diretório ``2-Analise-Exploratoria-Dados/tarefas/2.2-Pesquisa-Bibliografica/estudantes/'', que pode ser lido no capítulo \ref{chap:bibliometria:jhcf}. 

A análise deve conter texto, figuras, tabelas  etc. Todas as figuras, tabelas, dados etc, incluídas no texto, devem estar montadas no diretório /experiments/<githubusername>/AnaliseBibliometrica/\textbf{<tema-pesquisa>}. 

\textbf{<tema-pesquisa>} é um nome curto, em formato CamelCase, que você dará ao diretório onde os dados, figuras, etc, do seu estudo, estarão montados. Não use caracteres acentuados em nomes de arquivos no Overleaf.

Veja um exemplo de dados e figuras em 
/experiments / jhcf / PesqBiblogr / SimulacaoMultiagente.


Para a compilação da tarefa é necessário fazer o input do arquivo de texto em "2-Analise-Exploratoria-Dados / tarefas / 2.1-Pesquisa-Bibliografica / estudantes / main.tex".

O \textit{dataset} de análise bibliométrica deve conter, minimamente, 250 registros bibliográficos.


A análise precisa ser realizada e descrita em cinco etapas, e deve seguir as orientações feitas em \ref{metodo:analise:bibliografica}, e no detalhamento proposto por \citet{aria_bibliometrix_2017}:
\begin{enumerate}
    \item \textit{Study design} (Planejamento do estudo);

    \item  \textit{Data collection} (Coleta de dados);

    \item \textit{Data analysis} (Análise dos dados);

    \item \textit{Data visualization} (Visualização dos dados representados de forma gráfica, em vários formatos, vários tipos de diagrama);

    \item  \textit{Interpretation} (Interpretação, traçar conclusões, reflexões, sugestões de aprofundamento).
\end{enumerate}

Na edição do \LaTeX~ deve-se atentar aos seguintes aspectos:
\begin{enumerate}
    \item Todos os dados, inclusive as imagens, usadas na produção da análise, deve estar inseridos no diretório de experimentos individuais do estudante;
    \item Todas as figuras e gráficos inseridos na análise devem ser individualmente rotulados com label, sem conflitar com os labels já existentes, devem ter um título (caption) descritivo do que apresenta a figura e o nome do dataset usado, e também a figura/gráfico deve ser explicitamente descritas e citadas, usando referencia (ref);
    \item As figuras deve ser automaticamente dimensionadas, e eventualmente rotacionadas,  para caber na largura e (ou) altura do texto. 
\end{enumerate}


\chapter{Análise Bibliográfica sobre Simulação Multiagente e Fenômenos Sociais, por Jorge Fernandes\label{chap:bibliometria:jhcf}}

\section{Planejamento do estudo}
O planejamento o  desenho do estudo deve descrever as motivações, questões de interesse, escopo, limitações e objetivos do trabalho.

O planejamento do estudo deve motivar o tema escolhido e o interesse do autor.

No caso do meu trabalho, as perguntas que o nortearam foram:
\begin{itemize}
    \item Qual a base de conhecimentos científicos produzida em torno do tema simulação multiagente voltada à compreensão de fenômenos sociais, com ênfase em métodos experimentais? 
    \item Como a simulação multiagente tem sido usada para compreender fenômenos sociais, com ênfase em métodos experimentais? 
    \item Quais os principais termos e conceitos ligados à frente de pesquisa no tema simulação multiagente de fenômenos sociais, com ênfase em métodos experimentais? 
    \item Qual a estrutura social da comunidade, se é que existe, que pesquisa sobre o tema simulação multiagente de fenômenos sociais, com ênfase em métodos experimentais?
\end{itemize}

\subsection{O que já existe de pesquisa bibliométrica sobre esse tema?}

\cite{gore_classifying_2016} fizeram uma pesquisa que visava aprofundar a questão da simulação multiagente em relação à computação experimental.

A pesquisa é base para um posterior aprofundamento no campo da Cientometria, como fez \cite{chavalarias_whats_2017}.

\subsection{Uso do Bibliometrix e Biblioshiny}
Serão usadas a ferramenta e o \textit{workflow} proposto pelos autores do pacote Bibliometrix, conforme indica a figura ~\ref{fig:bibliometrix:workflow}.

\subsection{Limitações} O exercício relatado foi feito em uma semana, envolvendo entre 5 a 10 horas de trabalho de cada autor.

Outros aspectos a reforçar:
\begin{itemize}
   
\item Deve-se fazer buscas na base de dados WoS ou SCOPUS;
\item é obrigatório declarar um conjunto de perguntas de pesquisa.
\item é preciso declarar o objetivo da pesquisa, que no caso da aqui relatada foi exercitar inicialmente, e relatar, o uso da técnica de análise bibliométrica, para fins didáticos.
\end{itemize}


\section{Coleta de dados\label{MASSA:coleta}}

A coleta de dados feita usando o WoS no dia 03 de agosto de 2021, acessado por meio do Portal de Periódicos da CAPES.

Foram feitas buscas nas coleções \textbf{Science  Citation  Index  Expanded (SCI -EXPANDED)} e \textbf{Social  Sciences  Citation  Index (SSCI)}, que contém registros relativos a vários campos do conhecimento, no qual o SCI-EXPANDED foca mais na área das ciências exatas e naturais, enquanto que o SSCI indexa artigos da área das ciências sociais. Observe que os artigos nessas duas coleções são indexados desde 1945. 

Foi usada a \query\  de busca ilustrada nas linhas 1 a 9 da listagem \ref{query20210803-2}.

\lstinputlisting[numbers=left,basicstyle=\normalsize\ttfamily,caption={\query\  de busca sobre simulação multiagente de fenômenos socials, com ênfase em métodos experimentais.},label=query20210803-2]
{experiments/jhcf/PesqBibliogr/SimulacaoMultiagente/WoS-20210803/classico-mais-citacoes/query.txt}

\subsection{Explicação para os termos de busca usados\label{MASSA:query}}

A busca consistiu de quatro cláusulas disjuntivas, unidas por uma conjunção \textit{and}, aplicadas à busca por tópico (O termo de busca pode aparecer no Título, no Abstract, na Author Keywords, ou nas Keywords Plus da referência)

Os termos \texttt{experimental}, \texttt{numeric*}, \texttt{statist*}, \texttt{hypothes*}, 
\texttt{empiric*}
e \texttt{inferen} (linhas 1 e 2 da query) foram usados na primeira cláusula da \query\  para recuperar artigos que tenham em seu título, palavras-chave e resumo, termos relacionados a métodos experimentais,
métodos numéricos,
métodos estatísticos,
teste de hipóteses,
métodos empíricos e métodos inferenciais.

O termo / cláusula  \texttt{simul*}, na linha 4, foi usado em conjunção com os demais para recuperar apenas trabalhos que explicitem o uso da simulação.
Foi usado um único termo devido à forte adesão ao termo simulação por parte dos pesquisadores que usam simulação. Não existem outros sinônimos frequentes para esse uso.

A cláusula nas linhas 6 e 7 faz união entre o uso dos termos \texttt{agent} e \texttt{multiagent}, \texttt{multi-agent},e  também \texttt{multi and agent}, para cobrir as variadas formas de escrita do conceito.

A $4^{a}$ cláusula, linha 9,  usou os termos \texttt{social} e \texttt{society} para recuperar artigos que tratem de temas ligados à sociedade.
Os termos \texttt{group} e \texttt{behavi*} visam recuperar estudos que tratam de questões comportamentais e grupais.

Os 8.115 registros obtidos encontram-se no github do projeto, em \url{https://github.com/jhcf/Comput-Experim-20212/experiments/jhcf/PesqBibliogr/SimulacaoMultiagente/ WoS-20210803/classico-mais-citacoes/8115recs.txt}. 

Foram utilizadas as opções \textit{Exportar registros para arquivo de texto sem formatação} e \textit{export full record / Gravar Conteúdo: Seleção personalizada, com todos os 29 campos disponíveis, inclusive referências citadas} no WoS, para que as citações também fosse usadas em análises da citações (estrutura intelectual do conhecimento). Os 8115 registros foram recuperados em nove blocos de até 1.000 registros por vez (1-1000, 1001-2000, 2001-3000, ..., 8001-8115).

\section{Análise dos dados}

\subsection{Filtragem de registros}
Antes da análise, é possível aplicar filtros sobre os registros obtidos.

Foi aplicado um filtro ao \dataset\   inicial, com 8.115 registros, que continham pŕevias de artigos, artigos de conferência, capítulos de livro etc. Foram mantidos apenas os registros de artigos publicados em revistas científicas\footnote{A suposição é que que o conhecimento de maior qualidade sobre o tema está nas publicações em revistas.}. Após a aplicação desse filtro, 5.787 registros foram mantidos no \dataset, que será doravante chamado MultiAgentSimulationSociety/Artigos, ou MASSA@jhcf.

\subsection{Análise descritiva do \dataset\   MASSA@jhcf}

A análise bibliométrica descritiva faz uma descrição inicial do \dataset\  . Para explicação detalhada de como são calculadas as diversas taxas geradas pelo Bibliometrix veja a documentação do \textit{package} a partir da página \url{https://cran.r-project.org/web/packages/bibliometrix/index.html}. A análise bibliométrica descritiva é gerada pela função \texttt{biblioAnalysis}.

As informações mais gerais sobre o \dataset\   MASSA@jhcf são as seguintes:
\begin{description}
    \item [\textit{Timespan}] Os artigos que atenderam aos critérios de busca e filtragem foram publicados a partir de 1990, até 2021. Ou seja, não foram encontrados registros entre 1945 e 1989.
    \item [\textit{Sources (Journals, Books, etc)}] São 2.319 fontes de informação que publicaram os documentos recuperados no \dataset\   MASSA@jhcf. Ou seja, em média, cada \textit{scientific journal} publicou $5.787/2.319=2,5$ artigos. \footnote{Note que a média, enquanto medida de tendência central, pode não ser a que melhor reflete a tendência a quantidade de artigos publicados por revista.}
    \item [\textit{Average years from publication}] A média do tempo de publicação dos artigos no \dataset\   MASSA@jhcf é de 7,36 anos.
    \item [\textit{Average citations per documents}] Cada artigo no \dataset\   MASSA@jhcf foi citado, em média 20,7 vezes\footnote{Note que a média, enquanto medida de tendência central, pode não ser a que melhor reflete a tendência de  citações a artigos.}.
    \item [\textit{Average citations per year per doc}] Após publicado, cada um dos 5.787 artigos do \dataset\   MASSA@jhcf  foi citado 2,262 vezes por ano, em média.
    \item [\textit{References}] O \dataset\   MASSA@jhcf contém 201.464 referências citadas (tags CR).
    \item [\textit{Keywords Plus (ID)}] 13.735 distintas palavras-chave do tipo Keywords Plus (ID)\footnote{\textit{KeyWords Plus} são ``termos de índice gerados automaticamente a partir dos títulos de artigos citados. Os termos do KeyWords Plus devem aparecer mais de uma vez na bibliografia e são ordenados de frases com várias palavras a termos únicos. O KeyWords Plus aumenta o número de resultados tradicional de palavras-chave ou títulos.'' Fonte: \url{https://images.webofknowledge.com/WOKRS410B4/help/pt_BR/WOS/hp_full_record.html}} foram encontradas no \dataset\   MASSA@jhcf. 
    \item [\textit{Author's Keywords (DE)}] 15.704 distintas palavras-chave indicadas pelos autores foram encontradas no \dataset\  .
    \item [\textit{Authors}] 19.410 distintos nomes de autores foram encontrados no \dataset\  \footnote{Um mesmo autor pode ter uma ou mais diferentes grafias no \dataset\  , e serão reconhecidos dois ou mais autores diferentes, embora de fato sejam apenas um. Isso significa que a quantidade de \textbf{nomes de autores} equivale à quantidade de \textbf{autores}. Adicionalmente, é possível que distintos autores sejam reconhecidos com o mesmo nome, isso é, que sejam homônimos. Ou seja, o \dataset\   em geral conterá erros de contagem na quantidade de autores reais.}.
    \item [\textit{Author Appearances}] Os 19.410 distintos (nomes de) autores foram encontrados 23.470 vezes, como autores de artigos.
    \item [\textit{Authors of single-authored documents}] Dentre os 19.410 distintos (nomes de) autores encontrados, 375 deles editaram artigos individualmente, isso é, sem co-autores.
    \item [\textit{Authors of multi-authored documents}] Dentre os 19.410 distintos (nomes de) autores encontrados, 19.035 deles editaram artigos com um ou mais co-autores"
    \item [\textit{Single-authored documents}] Dentre os 5.787 documentos presentes no \dataset\   MASSA, 409 foram escritos por um único autor, e os 5.378 restantes foram elaborados em co-autoria.
    \item [\textit{Documents per Author}] Dentre os 19.410 distintos (nomes de) autores, cada um publicou em média 0,298 artigos.
    \item [\textit{Authors per Document}] Cada um dos 5.787 documentos presentes no \dataset\   MASSA foi autorado com 3,35 autores em média ($19.410 / 5.787 = 3,35$).
    \item [\textit{Co-Authors per Documents}] As 23.470 aparições de (nomes de) autores (``Author Appearances''), sem distribuem, em média 4,06 vezes para os 5.787 documentos do \dataset\   MASSA@jhcf.
    \item [\textit{Collaboration Index}] Os 19.035 (nomes de) autores que editaram artigos com um ou mais co-autores, colaboraram em media 3,54 vezes para editar os 5.378 artigos elaborados em co-autoria, gerando, assim, um índice de colaboração 3,54. 
\end{description}

\subsection{Evolução da Produção Científica}

\begin{figure}
    \centering
    \includegraphics[width=1\textwidth]{experiments/jhcf/PesqBibliogr/SimulacaoMultiagente/WoS-20210803/classico-mais-citacoes/Dataset/AnnualScientificProduction-2021-08-05.png}
    \caption{Evolução da produção científica no \dataset\   MASSA@jhcf.}
    \label{fig:evol:anual:MASSA@jhcf}
\end{figure}

A figura \ref{fig:evol:anual:MASSA@jhcf} apresenta a evolução da produção científica mundial no tema de interesse, segundo o \dataset\   MASSA@jhcf. A curva mostra uma tendência de crescimento aproximadamente exponencial da quantidade de publicações, desde a primeira identificada em 1990.

O \textit{Annual Growth Rate} do \dataset\   é de 17,06\%, bem maior que a taxa média de crescimento da publicação científica mundial, de cerca de 3,3\% anuais, em 2016, como ilustra o estudo em \url{https://www.researchgate.net/publication/333972683_Dynamics_of_scientific_production_in_the_world_in_Europe_and_in_France_2000-2016}, página 23.

\subsection{Interpretação do Crescimento} A maior taxa de crescimento do \dataset\   MASSA@jhcf, bem como o seu grande volume, sugerem que o assunto em pauta desperta intenso interesse, inclusive de ordem econômica.

\subsection{Evolução das Citações}

\begin{figure}
    \centering
    \includegraphics[width=1\textwidth]{experiments/jhcf/PesqBibliogr/SimulacaoMultiagente/WoS-20210803/classico-mais-citacoes/Dataset/AverageArticleCitationPerYear-2021-08-09.png}
    \caption{Evolução das citações ao \dataset\   MASSA@jhcf.}
    \label{fig:evol:anual:citacoes:MASSA@jhcf}
\end{figure}

A figura \ref{fig:evol:anual:citacoes:MASSA@jhcf} apresenta a evolução da média de citações aos 5.787 artigos no \dataset\   MASSA@jhcf. 
Nota-se grande estabilidade na média anual de citações, onde os artigos publicados em 1992 possuem cerca de 2 citações médias, e em 2015 (17 anos depois) o valou alterou-se apenas para três. O pico que aparece no ano de 2008 deve-se, possivelmente, à presença de um artigo do \dataset, publicado em 2008, que possui um número surpreendente grande de citações. \footnote{Note que o cálculo do número  médio de citações, nesse caso, utiliza os valores computados no tag "TC (Times Cited)", já presentes no \dataset\   obtido. Ou seja, o gráfico baseia-se no número de citações globais (externas ao \dataset\   MASSA@jhcf), e não no número de citações locais (citações a um artigo do \dataset\   feitas por alguns dos outros artigos dentro do próprio \dataset).}.

\subsection{Interpretação das Citações}
Mesmo perante um crescimento aproximadamente exponencial no volume de publicações, a ocorrência de um crescimento nas citações médias ao longo dos anos sugere que os artigos do \dataset\   possuem uma tendência de crescimento no tamanho da bibliografia citada, bem como também despertam grande interesse dos cientistas nas demais áreas do conhecimento (já que se trata de citações globais).

\subsection{\textit{Three-Field Plots (Sankey diagram)}}

As \textit{Three-Field Plots (Sankey diagram)} (plotagens do tipo ``Três Campos'') apresentam afinidades entre três conjuntos de atributos agregados que ocorrem no \dataset. Uma plotagem do tipo Sankey busca mostrar os principais fluxos entre diferentes conjuntos de itens. \footnote{Para uma introdução ver \url{https://en.wikipedia.org/wiki/Sankey_diagram}. Para obter detalhes sobre a forma de geração e utilização desse gráfico, inclusive de forma interativa, veja o vídeo em \url{https://www.youtube.com/watch?v=jBb1iha6-sg}.} 

\begin{figure}
    \centering
    \includegraphics[angle=0,width=1\textwidth]{experiments/jhcf/PesqBibliogr/SimulacaoMultiagente/WoS-20210803/classico-mais-citacoes/Dataset/ThreeFieldPlot-AU-CR-DE-20-20-20.png}
    \caption{Plotagem ``Três Campos'' (Sankey plot) do \dataset\   MASSA@jhcf: 20 Autores, Citações e Palavras-Chave mais proeminentes.}
    \label{fig:MASSA@jhcf:ThreeFieldPlot}
\end{figure}

A figura \ref{fig:MASSA@jhcf:ThreeFieldPlot} apresenta a plotagem do tipo ``Três Campos'' do \dataset\   MASSA@jhcf, vinculando, ao centro, os 20 Autores mais proeminentes (AU), à esquerda, as 20 Citações mais frequentes (CR - Cited Records), e à direita, as 20 Palavras-Chave mais frequentes empregadas pelos autores.

\subsection{Interpretação da figura \ref{fig:MASSA@jhcf:ThreeFieldPlot}}
Os vinte autores mais relevantes, em relação aos artigos mais relevantes citados, e as palavras-chave mais relevantes são aparentemente de origem asiática, mais especificamente chinesa, com base nos sobrenomes. De outra formal, a mesma origem chinesa parece não se aplicar aos trabalhos mais citados, aparentemente europeus ou norte-americanos. Isso sugere estar ocorrendo uma migração recente da produção científica, do ocidente para o oriente. 

Adicionalmente, dentre as palavras-chave (DE) não relacionadas diretamente aos termos de busca, emergem os termos \textbf{distributed control}, \textbf{event-triggered control}, \textbf{consensus} e \textbf{opinion dynamics}. Isso sugere foco das pesquisas por autores de origem chinesa no uso de simulação multiagente voltada à compreensão dos fenômenos de controle social distribuído, formação de consenso e dinâmica da opinião (pública?).

Ainda sobre a interpretação da plotagem da figura \ref{fig:MASSA@jhcf:ThreeFieldPlot}, observa-se que os artigos mais citados encontram-se publicados pelo menos 10 anos atrás, sugerindo que não houve, nos últimos 10 anos, nenhum trabalho que tenha produzido uma mudança de paradigma no tema.
A fim de melhor evidenciar as citações mais relevantes segundo o peso dos autores e palavras-chave, o gráfico da figura \ref{fig:MASSA@jhcf:ThreeFieldPlot:10-20-20} plota apenas as 10 referências citadas, para 20 autores e palavras-chave mais proeminentes.

\begin{figure}
    \centering
    \includegraphics[angle=0,width=1\textwidth]{experiments/jhcf/PesqBibliogr/SimulacaoMultiagente/WoS-20210803/classico-mais-citacoes/Dataset/ThreeFieldPlot-AU-CR-DE-20-10-20.png}
    \caption{Plotagem ``Três Campos'' (Sankey plot) do \dataset\   MASSA@jhcf: 10 Autores, 20 Citações e Palavras-Chave mais proeminentes.}
    \label{fig:MASSA@jhcf:ThreeFieldPlot:10-20-20}
\end{figure}

Breves comentários sobre cada um desses trabalhos serão tratados em seção posterior.

\begin{itemize}
    \item  \cite{olfati-saber_consensus_2004} apresentam discussões teóricas sobre a formação de consenso em sistemas multi-agentes com topologias variáveis;
    \item  \cite{reynolds_flocks_1987} apresenta modelos multi-agentes para simulação gráfica do movimento de rebanhos ou agregados de animais.
    \item \cite{vicsek_novel_1995} analisam a emergência de fenômenos de transição de fase em simulações de de partículas com comportamento autônomo com interação biologicamente motivada.
    \item \cite{barabasi_emergence_1999} investigam a emergência da distribuição livre de escala (\textit{scale-free}\footnote{Ver introdução em \url{https://en.wikipedia.org/wiki/Scale-free_network}.}) em redes que evoluem com base em ligação preferencial.
    \item \cite{watts_collective_1998} exploram o surgimento de redes do tipo mundo pequeno (\textit{small world}\footnote{Ver introdução em \url{https://en.wikipedia.org/wiki/Small-world_network}.}) formadas a partir da reorganização aleatória de redes biológicas, genéticas e outras formas de redes auto-organizadas.
    \item \cite{castellano_statistical_2009} exploram de que forma as técnicas de análise e simulação já usadas na física-estatística podem ser usadas para explicar vários fenômenos sociais, tais como comportamento de multidões, dispersão social, comportamento de multidões etc. Eles apresentam as afinidades entre os dados gerados pelos modelos simulados e dados empíricos obtidos junto a sistemas sociais reais. 
    \item \cite{hegselmann_opinion_2002} exploram a emergência de fenômenos de consenso, polarização e fragmentação da opinião na simulação de sociedades artificiais.
    \item \cite{bonabeau_agent-based_2002} apresenta os potenciais e campos de aplicação da técnicas de simulação baseada em agentes.
    \item \cite{wilensky_netlogo_1999} apresentam a linguagem e ambiente de simulação NetLogo.
    \item \cite{grimm_standard_2006} apresenta o protocolo ODD, proposto para padronizar a descrição de modelos de simulação multiagente.
\end{itemize}

Nenhum desses 10 documentos citados está contido no \dataset\   recuperado.

%\subsection{Análises Bibliométricas: Fontes de Informação}

%\begin{figure}
%    \centering
%    \includegraphics[angle=0,width=1\textwidth]{}
%    \caption{Plotagem ``Três Campos'' (Sankey plot) do dataset MASSA@jhcf: 20 Autores, Citações e Palavras-Chave mais proeminentes.}
%    \label{fig:MASSA@jhcf:ThreeFieldPlot}
%\end{figure}

\section{Refinamento da Coleta de Dados}

No dia 03 de fevereiro de 2022, no decorrer das análises mais refinadas do \dataset\ MASSA@jhcf, identificou-se um grupo de artigos que não se encaixavam no tema de interesse, e que eram voltados para pesquisas no campo da biologia experimental e nanotecnologia. Isso sugeriu que a \query\  de busca precisaria ser reformulada, para excluir artigos que não se enquadrassem na temática desejada.
O conjunto das palavras-chave que refletia essa dissonância ficou evidente na análise da estrutura intelectual do conhecimento, do tipo \textbf{Rede de Co-ocorrências de Palavras-chave}, ilustrada no cluster em roxo, à esquerda da figura \ref{fig:MASSA@jhcf:redecoocorr-150-termos}.

\begin{figure}[htp]
    \centering
    \includegraphics[clip=true,trim={9cm 0cm 7cm 0cm },width=0.6\textwidth]{experiments/jhcf/PesqBibliogr/SimulacaoMultiagente/WoS-20210803/classico-mais-citacoes/Structure-Informetric/Conceptual/Co-occurrence Network-Keywords-Plus-150-termos.png}
    \caption{Rede de co-ocorrência de palavras, com 150 termos, aplicada ao \dataset\   MASSA@jhcf.}
    \label{fig:MASSA@jhcf:redecoocorr-150-termos}
\end{figure}

As seguintes 30 palavras foram identificadas nesse \textit{cluster}:
in-vitro,
adsorption,
mechanism,
water,
force-field,
molecular-dynamics,
binding,
simulations,
nanoparticles,
bubbles,
derivatives,
temperature,
in-vivo,
mathematical-model,
oscillations,
scattering,
cancer,
contrast agents,
expression,
protein,
activation,
delivery,
surface,
removal,
acid,
agent,
reduction,
aqueous-solution,
degradation,
expectations.

Ficou evidente, pela interpretação do significado da maioria desses termos, que tais artigos não tratavam de simulação de fenômenos sociais. Isso sugere que a query está com problemas de precisão, isso é, muitos registros recuperados não atendem à necessidade de informação do pesquisador. 

Algumas dessas palavras foram então escolhidas para servir como indicativas de artigos fora do escopo, e introduzidas a partir da \query\  original, gerando uma nova \query, aprimorada e ilustrada nas linhas 1 a 13 da listagem \ref{query20220203}.

\lstinputlisting[numbers=left,basicstyle=\normalsize\ttfamily,caption={\query\  de busca sobre simulação multiagente de fenômenos socials, com ênfase em métodos experimentais, com escopo negativo de artigos que tratam de experimentos biológicos em vitro.},label=query20220203]
{experiments/jhcf/PesqBibliogr/SimulacaoMultiagente/WoS-20220203/query-Refinada.txt}

Além das justificativas para os termos usados entre as linhas 1 a 9, já descritas em \ref{MASSA:query},  justifica-se na listagem \ref{query20220203}, a inclusão da cláusula \textit{not (
 adsoption or molecular-dynamics or force-field
 or in-vitro or nanopartic* or in-vivo
 or aqueous-solution or protein or surface)}, entre as linhas 10 e 13 da \query, pois elas irão remover artigos não se enquadram no escopo da busca desejada, por usarem uma ou mais desses termos no título, resumo ou palavras-chave do artigo.
 
Usando a nova \query\ de busca, foram recuperados 6.935 documentos, que se encontram em
\url{https://github.com/jhcf/Comput-Experim-20212/experiments/jhcf/PesqBibliogr/SimulacaoMultiagente/ WoS-20220203/wos6935recs.txt}. Isso sugere que aproximadamente 1.000 registros não se enquadravam na necessidade de busca.
Uma nova análise dos dados recuperados é apresentada a seguir.

\section{Nova Análise dos Dados}

\subsection{Nova filtragem de registros}

Sobre os 6.935 documentos recuperados, foram  aplicados os seguintes filtros:
\begin{itemize}
    \item Remoção dos registros de documentos que não são artigos \textit{full paper}, isso é, artigos completos publicados em revistas;
%    \item Remoção dos registros de artigos científicos que não fazem parte do \textit{core} da bibliografa, segundo a Lei de Bradford.
\end{itemize}

Após os filtros aplicados (apenas um)  obteve-se um total de 4.647 registros, que doravante serão chamados de forma coletiva, de \dataset\   MASSA2@jhcf.

\subsection{Análise descritiva do \dataset\   MASSA2@jhcf}

\subsubsection{Dados Sumários Gerais}

\begin{table}[]
    \centering
\csvautotabular[separator=semicolon
%,filter not strcmp={\csvcolii}{}
]{experiments/jhcf/PesqBibliogr/SimulacaoMultiagente/WoS-20220203/Descritiva/MASSA2-Main-Information.csv}
    \caption{Principais dados descritivos do \dataset\   MASSA2@jhcf.}
    \label{tab:MASSA2:Main}
\end{table}

Nota-se, com os resultados da tabela \ref{tab:MASSA2:Main}, que o \dataset\   abrange um período de 32 anos de publicações (1991 a 2022), evidenciando  a publicação dos 4.647 artigos em 1.910 revistas distintas. Esses artigos tem idade média de publicação de 7.8.

Adicionalmente, o \dataset\ apresenta 157.507 referências citadas, com uma média de (157.507/4.647 = ?) 33,89 referências citadas por artigo.

14.229 autores distintos produziram os artigos, com uma média de 3,73 autores por documento.

\subsubsection{Evolução anual da produção científica}

No tema de simulação multiagente de fenômenos sociais, a evolução anual da produção científica mundial é sumarizada no gráfico da figura \ref{fig:MASSA2:Evolucao}.

\begin{figure}
    \centering
    \includegraphics[width=1\textwidth]{experiments/jhcf/PesqBibliogr/SimulacaoMultiagente/WoS-20220203/Descritiva/MASSA2-Annual-Scientific-Production.png}
    \caption{Evolução da Produção Científica Anual, segundo o \dataset\ MASSA2@jhcf.}
    \label{fig:MASSA2:Annual-Scientific-Production}
\end{figure}

Entre 1991 e 2005 o crescimento de publicações era quase linear. As publicações mostram-se em ascendência forte a partir dos últimos seis anos (2015). Esse crescimento tem sido visto em várias outras áreas de conhecimento.

\subsubsection{Média de citações anuais por artigo}

O gráfico da figura \ref{fig:MASSA2:Media:Citacoes} apresenta a evolução das citações anuais médias, para os artigos do \dataset\ MASSA2@jhcf. Observa-se que há um crescimento discreto da média, onde os artigos mais recentes tendem a ser mais citados, como esperado. A redução da média no ano de 2021 deve-se, provavelmente, à insuficiência de indexação e de citação para os artigos mais recentes, tendo em vista que p ano de 2021 foi encerrado há menos de dois meses. 

\begin{figure}
    \centering
    \includegraphics[width=1\textwidth]{experiments/jhcf/PesqBibliogr/SimulacaoMultiagente/WoS-20220203/Descritiva/MASSA2-Average-Citations-per-Year.png}
    \caption{Média de citações para cada artigo do \dataset\ MASSA2@jhcf, conforme o ano de publicação}
    \label{fig:MASSA2:Media:Citacoes}
\end{figure}

Para que melhor se compreenda como foi produzido o gráfico, a tabela \ref{tab:MASSA2:Media:Citacoes} apresenta parcialmente os dados de citação anual para os artigos do \dataset\ MASSA2@jhcf. A título de exemplo, nota-se que no \dataset\ foram encontrados 9 artigos publicados no ano de 1991, tendo sido cada artigo citado, em média, aproximadamente 31,4 vezes. Dado que esses artigos já tem 31 anos citáveis, obtém-se uma média de 1,01 citações anuais, aproximadamente.

\begin{table}[]
    \centering
\csvautotabular[separator=semicolon
%,filter not strcmp={\csvcolii}{}
]{experiments/jhcf/PesqBibliogr/SimulacaoMultiagente/WoS-20220203/Descritiva/MASSA2-Average-Citations-per-Year.csv}
    \caption{Dados parciais de citação anual para os artigos do \dataset\   MASSA2@jhcf.}
    \label{tab:MASSA2:Media:Citacoes}
\end{table}

\subsubsection{Diagramas de Sankey (\textit{three fields plots})} 

A fim de apresentar mais alguns dados sumários gerais sobre  o \dataset, as figuras \ref{fig:MASSA2:Sankey:CR-AU-DE} e \ref{fig:MASSA2:Sankey:SO:DE:AU_UN} apresentam plotagens do tipo 
\textit{three fields plots}, também conhecidas pelo nome de Diagramas de Sankey \citep{riehmann_interactive_2005}, que possibilitam várias combinações de afinidades mais evidentes entre as diversas colunas dos registros do \dataset.

A primeira plotagem, figura \ref{fig:MASSA2:Sankey:CR-AU-DE}, apresenta as afinidades mais evidentes entre 15 Autores (centro), 15 Palavras-chave (direita) e 15 Referências citadas (esquerda). Ao centro, observa-se que os autores mais evidentes, segundo a técnica apresentada, tem origem asiática, a julgar pelos nomes. 

As 15 palavras-chave mais evidentes sugerem que o \dataset\ possui artigos que refletem a busca sobre o tema desejado, mas que há muitas palavras distintas que representam o mesmo conceito, como as a seguir listadas:
\begin{enumerate}
    \item multi-agent systems;
    \item multiagent system;
    \item multiagent system; 
    \item agent-based model;  
    \item agent-based models;
    \item agent-based modeling;
    \item agent-based simulation;
\end{enumerate}

As cinco palavras a seguir sugerem, o que pode ser comprovado com o aprofundamento desse estudo bibliográfico, que o estado da arte no tema busca atualmente respostas, ou possui fundamentos nas seguintes questões:
\begin{description}
    \item [event-triggered control] Como eventos disparadores exercem  controle sobre o comportamento (coletivo) de grupos sociais?
    \item [consensus] Como usar simulação multi-agente para entender o surgimento de consenso em grupos sociais?
    \item [opinion dynamics] Como usar simulação multi-agente para entender a dinâmica de opiniões que se formam em grupos sociais?
    \item [social networks] Como os métodos da análise de redes sociais podem ser usados no tema da simulação multiagente?
    \item [reinforcement learning] Como usar os métodos e técnicas de aprendizagem por reforço em simulação multiagente?
    \item [game theory] Como usar os métodos e técnicas da teoria dos jogos em simulação multiagente?
\end{description}

Algumas das referências citadas, apresentadas à esquerda do gráfico, devem evidenciar a pertinência das questões acima sugeridas, a ser comprovado até o final do estudo. 

\begin{figure}
    \centering
    \includegraphics[angle=90,width=1\textwidth,height=0.9\textheight]{experiments/jhcf/PesqBibliogr/SimulacaoMultiagente/WoS-20220203/Descritiva/MASSA2-Three-Fields-Plot-CR-AU-DE.png}
    \caption{Diagrama Sankey, relacionando as afinidades mais evidentes entre Autores (centro), Palavras-chave (direita) e Referências citadas (esquerda).}
    \label{fig:MASSA2:Sankey:CR-AU-DE}
\end{figure}

A segunda plotagem, figura \ref{fig:MASSA2:Sankey:SO:DE:AU_UN}, apresenta as afinidades mais evidentes entre 15 revistas (esquerda), 15 palavras-chave (centro) e 15 instituições de filiação dos autores (direita). Com base na técnica usada, fica evidente a proeminência dos seguintes \textit{journals} sobre os demais, sendo apresentado um breve trecho do foco de cada revista, extraído da página online da revista:
\begin{itemize}
    \item JASSS: The Journal of Artificial Societies and Social Simulation. \textit{\small  is an interdisciplinary journal for the exploration and understanding of social processes by means of computer simulation. Since its first issue in 1998, it has been a world-wide leading reference for readers interested in social simulation and the application of computer simulation in the social sciences.}. Fonte \url{https://www.jasss.org/admin/about.html};
    \item PHYSICA A-STATISTICAL MECHANICS AND ITS APPLICATIONS. \textit{\small ... publishes research in the field of statistical mechanics and its applications. Statistical mechanics sets out to explain the behaviour of macroscopic systems, or the large scale, by studying the statistical properties of the microscopic or nanoscopic constituents. Applications of the concepts and techniques of statistical mechanics include: applications to physical and physiochemical systems such as solids, liquids and gases, interfaces, glasses, colloids, complex fluids, polymers, complex networks, applications to economic and social systems (e.g. socio-economic networks, financial time series, agent based models, systemic risk, market dynamics, computational social science, science of science, evolutionary game theory, cultural and political complexity), and traffic and transportation (e.g. vehicular traffic, pedestrian and evacuation dynamics, network traffic, swarms and other forms of collective transport in biology, models of intracellular transport, self-driven particles), as well as biological systems (biological signalling and noise, biological fluctuations, cellular systems and biophysics); and other interdisciplinary applications such as artificial intelligence (e.g. deep learning, genetic algorithms or links between theory of information and thermodynamics/statistical physics.).}. Fonte: \url{https://www.journals.elsevier.com/physica-a-statistical-mechanics-and-its-applications};
    \item IEEE ACCESS. \textit{\small ... is a multidisciplinary, all-electronic archival journal, continuously presenting the results of original research or development across all IEEE’s fields of interest. Supported by article processing charges (APC), its hallmarks are a rapid peer review and publication process of 4 to 6 weeks, with open access to all readers.}. Fonte: \url{https://ieeeaccess.ieee.org/about-ieee-access/learn-more-about-ieee-access/}
\end{itemize} 

\begin{figure}
    \centering
    \includegraphics[angle=90,width=1\textwidth,height=0.9\textheight]{experiments/jhcf/PesqBibliogr/SimulacaoMultiagente/WoS-20220203/Descritiva/MASSA2-Three-Fields-Plot-SO:DE:AU_UN.png}
    \caption{Diagrama Sankey, relacionando as afinidades mais evidentes entre revistas (esquerda), palavras-chave (centro) e instituição de filiação dos autores (direita).}
    \label{fig:MASSA2:Sankey:SO:DE:AU_UN}
\end{figure}

Observa-se, com base no escopo declarado de cada uma das revistas, que a revista JASSS é bem enquadrada no escopo da busca, enquanto que a revista IEEE Access não tem relação direta com o tema. Já a revista Physica A aborda o tema de forma mais ampla do que o buscado, com ênfase em métodos da mecânica estatística, que não são os únicos possíveis de serem empregados.
Os nomes das demais 8 revistas, apresentadas no diagrama, são os seguintes:
\begin{enumerate}
    \item NEUROCOMPUTING \textit{\small  ... welcomes theoretical contributions aimed at winning further understanding of neural networks and learning systems, including, but not restricted to, architectures, learning methods, analysis of network dynamics, theories of learning, self-organization, biological neural network modelling, sensorimotor transformations and interdisciplinary topics with artificial intelligence, artificial life, cognitive science, computational learning theory, fuzzy logic, genetic algorithms, information theory, machine learning, neurobiology and pattern recognition.}. Fonte: \url{https://www.journals.elsevier.com/neurocomputing};
    \item IEEE TRANSACTIONS ON AUTOMATIC CONTROL \textit{\small publishes high-quality papers on the theory, design, and applications of control engineering.  Two types of contributions are regularly considered: 
1) Papers:  Presentation of significant research, development, or application of control concepts. 
2) Technical Notes and Correspondence:  Brief technical notes, comments on published areas or established control topics, corrections to papers and notes published in the Transactions.
In addition, special papers (tutorials, surveys, and perspectives on the theory and applications of control systems topics) are solicited. }. Fonte: \url{https://ieeexplore.ieee.org/xpl/aboutJournal.jsp?punumber=9};
    \item AUTONOMOUS AGENTS AND MULTI-AGENT SYSTEMS \textit{\small is the official journal of the International Foundation for Autonomous Agents and Multi-Agent Systems. It provides a leading forum for disseminating significant original research results in the foundations, theory, development, analysis, and applications of autonomous agents and multi-agent systems. Coverage in Autonomous Agents and Multi-Agent Systems includes, but is not limited to:
Agent decision-making architectures and their evaluation, including: cognitive models; knowledge representation; logics for agency; ontological reasoning; planning (single and multi-agent); reasoning (single and multi-agent)
Cooperation and teamwork, including: distributed problem solving; human-robot/agent interaction; multi-user/multi-virtual-agent interaction; coalition formation; coordination
Agent communication languages, including: their semantics, pragmatics, and implementation; agent communication protocols and conversations; agent commitments; speech act theory
Ontologies for agent systems, agents and the semantic web, agents and semantic web services, Grid-based systems, and service-oriented computing
Agent societies and societal issues, including: artificial social systems; environments, organizations and institutions; ethical and legal issues; privacy, safety and security; trust, reliability and reputation
Agent-based system development, including: agent development techniques, tools and environments; agent programming languages; agent specification or validation languages
Agent-based simulation, including: emergent behavior; participatory simulation; simulation techniques, tools and environments; social simulation
Agreement technologies, including: argumentation; collective decision making; judgment aggregation and belief merging; negotiation; norms
Economi c paradigms, including: auction and mechanism design; bargaining and negotiation; economically-motivated agents; game theory (cooperative and non-cooperative); social choice and voting
Learning agents, including: computational architectures for learning agents; evolution, adaptation; multi-agent learning.
Robotic agents, including: integrated perception, cognition, and action; cognitive robotics; robot planning (including action and motion planning); multi-robot systems.
Virtual agents, including: agents in games and virtual environments; companion and coaching agents; modeling personality, emotions; multimodal interaction; verbal and non-verbal expressiveness
Significant, novel applications of agent technology
Comprehensive reviews and authoritative tutorials of research and practice in agent systems
Comprehensive and authoritative reviews of books dealing with agents and multi-agent systems.
Official journal of the International Foundation for Autonomous Agents and Multi-Agent Systems.
Covers the foundations, theory, development, analysis, and applications of autonomous agents and multi-agent systems.
Presents comprehensive reviews and authoritative tutorials of research and practice in agent systems.}. Fonte: \url{https://www.springer.com/journal/10458};
    \item INTERNATIONAL JOURNAL OF MODERN PHYSICS C \textit{\small is a journal dedicated to Computational Physics and aims at publishing both review and research articles on the use of computers to advance knowledge in physical sciences and the use of physical analogies in computation. Topics covered include: algorithms; computational biophysics; computational fluid dynamics; statistical physics; complex systems; computer and information science; condensed matter physics, materials science; socio- and econophysics; data analysis and computation in experimental physics; environmental physics; traffic modelling; physical computation including neural nets, cellular automata and genetic algorithms.}. Fonte: \url{https://www.worldscientific.com/page/ijmpc/aims-scope};
    \item COMPLEXITY \textit{\small The purpose of Complexity is to report important advances in the scientific study of complex systems. Complex systems are characterized by interactions between their components that produce new information — present in neither the initial nor boundary conditions — which limit their predictability. Given the amount of information processing required to study complexity, the use of computers has been central to complex systems research. Concepts relevant to Complexity include:
    Adaptability, robustness, and resilience;
    Complex networks;
    Criticality;
    Evolution and emergent behaviour;
    Nonlinear dynamics;
    Pattern formation;
    Self-organization.
Methods used within the scientific study of complex systems frequently include:
    Agent-based modelling;
    Analytical methods;
    Cellular automata;
    Computational methods;
    Data science;
    Game theory;
    Machine learning;
    Statistical mechanics.
Applications of complex systems may be related to the following disciplines, among others:
Computational social science;    Digital epidemiology;
    Ecology;
    Economics;
    Engineering;
    Socio-technical systems;
    Statistical linguistics;
    Systems biology;
    Urban systems.}. Fonte: \url{https://www.hindawi.com/journals/complexity/about/};
    \item ECOLOGICAL MODELLING \textit{\small publishes new mathematical models and systems analysis for describing ecological processes, and novel applications of models for environmental management.
We welcome research on process-based models embedded in theory with explicit causative agents and innovative applications of existing models. And because applications can help refine models and propose new directions for research, the journal publishes both to help foster reproducibility and utility.Human activity and well-being are dependent on and integrated with the functioning of ecosystems and the services they provide. We aim to understand these basic ecosystem functions using mathematical and conceptual modelling, systems analysis, thermodynamics, computer simulations, and ecological theory, and look to a wide spectrum of applications ranging from basic ecology to human ecology to socio-ecological systems. The journal welcomes original research articles, review articles, viewpoint articles and short communications.}. Fonte: \url{https://www.journals.elsevier.com/ecological-modelling};
    \item JOURNAL OF THEORETICAL BIOLOGY \textit{\small is the leading forum for theoretical perspectives that give insight into biological processes. It covers a very wide range of topics and is of interest to biologists in many areas of research, including:
Brain and Neuroscience;
Cancer Growth and Treatment;
Cell Biology;
Developmental Biology;
Ecology;
Evolution;
Immunology;
Infectious and non-infectious Diseases;
Mathematical, Computational, Biophysical and Statistical Modeling;
Microbiology, Molecular Biology, and Biochemistry;
Networks and Complex Systems;
Physiology;
Pharmacodynamics;
Animal Behavior and Game Theory}. Fonte: \url{https://www.journals.elsevier.com/journal-of-theoretical-biology};
    \item JOURNAL OF STATISTICAL MECHANICS-THEORY AND EXPERIMENT \textit{\small is targeted to a broad community interested in different aspects of statistical physics, which are roughly defined by the fields represented in the conferences called 'Statistical Physics'. Submissions from experimentalists working on all the topics which have some 'connection to statistical physics are also strongly encouraged.
The journal covers different topics which correspond to the following keyword sections:
Quantum statistical physics, condensed matter, integrable systems;
Classical statistical mechanics, equilibrium and non-equilibrium;
Disordered systems, classical and quantum;
Interdisciplinary statistical mechanics;
Biological modelling and information}. Fonte: \url{https://iopscience.iop.org/journal/1742-5468/page/about_the_journal}.
\end{enumerate}
Considerando a possibilidade de pertinência da questão ecológica ao tema dos sistemas multi-agentes, todos os \textit{journals} identificados apresentam pertinência às perguntas de pesquisa formuladas. 

Acerca das instituições de filiação dos autores, nota-se que 1/3 delas é localizada na china, 1/3 nos EUA, e o restante na Europa e Austrália. É provável que muitos pesquisadores de origem chinesa trabalhem em universidades fora da china, no tema do \dataset.


\subsection{Medidas bibliométricas}

As medidas bibliométricas propriamente ditas, relativas ao \dataset\ MASSA2@jhcf, serão exploradas nesta subseção, e são organizadas em três conjuntos:
\begin{description}
    \item [Relativas às Fontes de Informação] Uma vez que foram consideradas apenas as publicações em revistas, todas as fontes de informação mensuradas serão revistas científicas, ou \textit{journals}. As principais medidas são de impacto das fontes, mensuradas com base no número de citações que os artigos publicados nas revistas obtiveram de outras publicações, possivelmente feitas em outras fontes de informação, como outras revistas, seções de livros, artigos de conferência etc. As citações são registradas pelas organizações que fazem indexação de artigos, como a Web of Science e SCOPUS;
    \item [Relativas aos Autores] Sempre que um artigo publicado por um ou mais autores e também indexado por uma organização (Web of Science,  SCOPUS etc), é citado em um outro artigo também indexado por essa mesma organização, então é feita a anotação de uma citação ao mesmo, e o impacto potencial desse autor sobre a ciência é atestado pelo valor mais alto da citação do conjunto de seus artigos indexados. Várias métricas (índice H, G, M etc) podem ser derivadas dessa medida (quantidade de citações), e são exploradas tanto em relação aos autores como em relação às revistas onde esses artigos foram publicados;
    \item [Relativas aos Documentos] Cada citação adicional a  um documento (artigo de revista, de conferência, livro, ou  capítulo de livro) é um indicador do impacto do documento em si, que evidencia a sua importância. Além das citações, a ocorrência de palavras dentro dos documentos, inclusive ordenada pelo tempo, também produz indicadores numéricos (métricas) relevantes para analisar a importância do documento em relação a outros. 
\end{description}

Essas medidas serão apresentadas a seguir.

\subsubsection{Bibliometrias aplicadas aos documentos (Artigos científicos) no \dataset}

\paragraph{Citações globais aos artigos no \dataset}

Cada registro recuperado no \dataset\ apresenta um conjunto de informações, dentre as quais pode constar a quantidade de vezes que uma citação ao mesmo foi registrada no índice do WoS, desde que no momento da extração seja feita essa solicitação (\textit{TC - Times Cited}).
A tabela \ref{tab:MASSA2:GlobalCitations} apresenta a lista dos 25 artigos do \dataset, que foram mais citados, ordenados de forma decrescente pelo número global de citações do artigo, nos índices da WoS. Para ada artigo é apresentada a referencia abreviada, o DOI e a quantidade de vezes que ele foi citado globalmente (no índice do WoS). Para recuperar a página do artigo deve-se abrir uma url prefixada com \url{http://doi.org/}, e informar o valor do DOI indicado, por exemplo \url{http://doi.org/10.1109/TAC.2008.2010897} levará à página do artigo mais citado, cujo título é ``Flocking of Multi-Agents With a Virtual Leader''.

\begin{table}[]

    \centering
\footnotesize
\csvreader[tabular = |r|l|l|r|,
separator=semicolon
%,filter not strcmp={\csvcolii}{},
, table head = \hline\hline \# & Artigo (Referência Abreviada) & DOI (Digital Object Identifier) & Cit.\\ \hline\hline,
table foot = \hline\hline
]{experiments/jhcf/PesqBibliogr/SimulacaoMultiagente/WoS-20220203/Metricas/Documentos/MASSA2-Most-Global-Cited-Documents.csv}{Paper=\paper, DOI=\doi,Total Citations=\totcit}{ \thecsvrow & {\tiny\paper} & {\tiny \doi} & \totcit}

    \caption{25 artigos mais citados no \dataset\ MASSA2@jhcf.}
    \label{tab:MASSA2:GlobalCitations}
\end{table}


\paragraph{Referências aos (outros) artigos, capítulos de livros etc (documentos) citados pelos artigos no \dataset}

\paragraph{Uso de palavras dentro dos artigos no \dataset}

\chapter{Análise Bibliográfica sobre Processamento de Linguagem Natural, por Lucas de Almeida Bandeira Macedo}

\section{Planejamento do estudo}

Com a vinda de assistentes virtuais, como a Alexa (Amazon), Cortana (Microsoft) ou Siri (Apple), as pessoas costumam se perguntar cada vez mais: "como que esse programa está entendendo o que eu falo?".

Mas não só de assistentes virtuais vive o Processamento de Linguagem Natural (também conhecido como NLP - Natural Language Processing), afinal, qualquer texto ou fala pode ser interpretado por uma máquina e devidamente classificado. Por exemplo, uma aplicação famosa é o "classificador de sentimentos", em que um modelo treinado consegue classificar textos entre sentimentos "positivos" ou "negativos". Com a ascensão do Twitter, uma rede social baseada em pequenos textos de não mais que 280 caracteres, NLP se torna cada vez mais interessante.

Assim, as perguntas que traçam o norte para este estudo são:

\begin{itemize}
    \item Quais os principais conceitos ligados com Processamento de Linguagem Natural?
    \item Como se dá o progresso das pesquisas em NLP ao longo dos anos? As redes sociais influenciaram esse crescimento?
    \item Qual o estado da estrutura social da comunidade de NLP?
\end{itemize}

\subsection{Uso do Bibliometrix e Biblioshiny}

Será usada a ferramenta Bibliometrix, com sua função Biblioshiny, para gerar gráficos e grafos iterativos e personalizáveis, para auxiliar na interpretação da realidade científica do tópico.

\section{Coleta de dados}

A coleta de dados foi feita utilizando o site Web of Science (WoS), no dia 03/02/2022, através do portal periódico da capes.

A pesquisa foi realizada utilizando as edições "Science Citation Index Expanded" e "Conference Proceedings Citation Index – Science", ambas coleções são voltadas para, principalmente, as ciências exatas.

A \textit{string} (ou \textit{query}) de busca inicialmente utilizada foi a seguinte:

\lstinputlisting[numbers=left,basicstyle=\normalsize\ttfamily,caption={Query de busca sobre Procesasmento de Linguagem Natural.},label=queryNLP03022022]
{experiments/ABMHub/PesquisaBibliometrica/NLP/pesquisa_velha.txt}

\subsection{Explicação para os termos de busca usados}

A proposta é apenas pesquisar sobre Processamento de Linguagem Natural, sem muito rigor na aplicação em que essa arquitetura de rede neural é aplicada. Portanto, inicialmente a pesquisa foi apenas "natural language processing".

Porém, uma rápida olhada pelos artigos retornados evidenciou uma grande quantidade de artigos sobre linguísticas, e áreas que não são da computação. Como o objetivo aqui adquirir modelos de Deep Learning, a pesquisa foi ajustada para filtrar apenas por NLP ligadas diretamente a computação e inteligência artificial, evidenciado pelas cláusulas "neural network", "(machine or deep) and learning" e "artificial intelligence". Essa nova pesquisa trouxe melhores resultados, todos evidenciando redes neurais e variadas técnicas de machine learning. O total de registros retornado pela query foi 

\subsubsection{Refinamento da Coleta de Dados}

 Em seguida, em uma análise mais fina, utilizando a \textbf{Rede de Co-ocorrências de Palavras-chave}, podemos evidenciar outras palavras chaves que estavam aparecendo entre os registros da pesquisa, que não deveriam estar aparecendo. É possível observar na imagem \ref{fig:ABMHub:NLPgraph1}, palavras como "câncer" ou "diagnóstico" que estão relacionadas a visão computacional mais que NLP, aparecendo com pesos não-desprezíveis.
 
 \begin{figure}
    \centering
    \includegraphics[angle=0,width=1\textwidth]{experiments/ABMHub/PesquisaBibliometrica/NLP/network.png}
    \caption{Grafo de relação de keywords}
    \label{fig:ABMHub:NLPgraph1}
\end{figure}

Assim, é necessário uma nova iteração da pesquisa, para evitar que registros de visão computacional corrompam a pesquisa de NLP. É delicado fazer isso, pois existem muitas menções a Visão Computacional nos registros de LP, já que ambos são ligados a Deep Learning, então retirar a keyword "Visão Computacional" provavelmente removeria muitos registros que não gostaríamos de remover da pesquisa. Assim, a melhor solução encontrada foi remover palavras que não têm intersecção entre os dois assuntos. Por exemplo, "medical", "cancer" e "diagnosis".

Assim, chegamos na mais recente query:

\lstinputlisting[numbers=left,basicstyle=\normalsize\ttfamily,caption={Query de busca sobre Procesasmento de Linguagem Natural.},label=queryNLP03022022]
{experiments/ABMHub/PesquisaBibliometrica/NLP/pesquisa_nova.txt}



\chapter{Análise Bibliográfica sobre Otimizações algorítmicas para simulações de fenômenos fluídos e óticos por Alexsander Correa de Oliveira}

\section{Planejamento do estudo}
    A indústria de jogos é a que mais cresce dentre todas as formas de entretenimento atuais. Alguns desses jogos podem chegar a investimentos tão grandes que disputam com os filmes mais caros da história. No posto de vista do consumidor, todo tempo e dinheiro gastos são apenas meios para um fim, que é o de ter a melhor experiência possível. Contudo do ponto de vista dos desenvolvedores, esses fatores são consequências de horas e horas de trabalho.
    
    Toda a tecnologia criada para esses jogos tem de ser cada vez mais eficiente, dado a necessidade de tornar os gráficos cada vez mais realistas, e seus mundos ainda mais acreditáveis. As técnicas utilizadas para tal otimização são fortemente baseadas em artigos \emph{state-of-the-art} tanto e física quanto em geometria.
    
    Entre todos aspectos físicos, que hoje em dia são mais prevalentes nos jogos, temos algumas áreas de estudo que pesam mais, principalmente em performance: 
    \begin{itemize}
        \item Corpos fluídos, como água e ar;
        \item Análise de vetores em ótica, para saber como a iluminação afetará um determinado ambiente;
        \item Análise da topologia, com fins de otimizar \emph{path-finding};
    \end{itemize}
    
    Considerando o corpo de estudo, algumas questões surgem:
    \begin{item}
        \item Quais são os principais interesses relacionados ao estudo dos fenômenos naturais voltados a computação?
        \item Quem são os agentes que produzem o maior volume de artigos?
        \item Quais são os corpos de estudo mais relacionados entre si?
    \end{item}
    Todas elas serão analisadas e respondidas no decorrer da seção.
\subsection{Uso do Bibliometrix e Biblioshiny}
     Com o auxílio das ferramentas disponibilizadas pelo Bibliometrix, como o Biblioshiny, serão analisados os artigos encontrados, por meio de gráficos e tabelas.   
\section{Coleta de Dados}
    A coleta de dados foi iniciada no dia 02/02/2022, e usou a base Web of Science, com acesso direto pelo periódico Capes.
    
    Por fins de diminuir o tamanho do \emph{dataset}, só foi utilizada a edição \emph{Science Citation Index Expanded}, que tem o foco voltado para ciências exatas e naturais.
    
    A busca inicial foi feita com a seguinte \emph{query}:
\begin{lstlisting}[basicstyle = \normalsize]
((algorit* ) and (Optimization)) and 
(optics or ((fluid* or aero*) and dynamics))
\end{lstlisting}
\subsection{Explicação para a \emph{Query}}
    A busca foi feita com o objetivo de encontrar apenas técnicas para otimizar algoritmos relacionados a ótica, aerodinâmica e hidrodinâmica.
    
    Os termos \verb|((algorit* ) and (Optimization))| são para encontrar apenas os artigos relacionados a algoritmos computacionais.
    
    Já \verb|(optics or ((fluid* or aero*) and dynamics))| serve para falar que tanto faz um artigo de ótica ou de aerodinâmica ou de hidrodinâmica.
    
    Com a \emph{Query} já montada, os registros foram exportados do WoS com todas as informações disponíveis e no formato de arquivo de texto sem formatação. Foram recuperados desa maneira, 7443 registros no total.
\section{Análise dos dados}
    Uma análise inicial foi feita com o objetivo de retirar artigos indesejados. Para atingir isso, foi utilizado o gráfico \emph{Co-occurrence Network}, que mostra as palavras com maior peso, e o relacionamento entre elas.
    
     \begin{figure}
    \centering
    \includegraphics[width=1\textwidth]{experiments/KvotheKS/PesqBibliogr/AlgoritmosSimulacaoOptica-Dinamica/WoS-20220202/OldQueryDataset/CoOccurrence.png}
    \caption{Rede de co-ocorrência}
    \label{fig:KvotheKS:OldQueryCoOccurrence}
\end{figure}
    
    Destacando um dos lados do grafo e o meio, podemos ver que o foco em computação e otimização foram atingidos. Contudo, como um efeito não desejado, também foram "recebidos" artigos que envolvem I.A, como também ótica de um ponto de vista médico.
    
\subsection{Refinamento dos Dados}
    Para retirar todo produto indesejado foi feita uma nova \emph{query} na mesma base e edição:
    
\begin{lstlisting}[basicstyle = \normalsize]
((algorit* ) and (Optimization)) and 

(optics or ((fluid* or aero*) and dynamics))

not ((genetic* and algorit*) or medic* or (machin* and learn*))
\end{lstlisting}

    Com os novos parâmetros, o objetivo de retirar tudo relacionado a medicina e a maioria de algoritmos genéticos foi atingido. Como resultado da nova busca, foram retornados 4917 elementos. Contudo, considerando apenas o número de artigos, o número cai para 4859.
    
    Como demonstração da melhora do \emph{dataset}, segue o gráfico \ref{fig:KvotheKS:Final_Data_Set}:
    
    \begin{figure}[H]
    \centering
    \includegraphics[width=1.3\textwidth]{experiments/KvotheKS/PesqBibliogr/AlgoritmosSimulacaoOptica-Dinamica/WoS-20220202/Dataset/AU_CR_DE.png}
    \caption{Dataset final}
    \label{fig:KvotheKS:Final_Data_Set}
\end{figure}

\subsection{Análise descritiva do \emph{dataset}}
    As informações iniciais do \emph{dataset} de 4859 registros são as seguintes:
    
\begin{description}
    \item [\textit{Timespan}] Todos os artigos que passaram pelo filtro e pela busca foram feitos de 1985 a 2022.
    \item [\textit{Sources (Journals, Books, etc)}] São 924 fontes de informação registradas.
    \item [\textit{Average years from publication}] A média de tempo para publicação é de 7,95 anos.
    \item [\textit{Average citations per documents}] A média de citações dos artigos é de 17,87 vezes.
    \item [\textit{Average citations per year per doc}] Os artigos, após sua publicação, tiveram em média 1,887 citações anuais.
    \item [\textit{References}] A quantidade total de referências do \emph{dataset} se dá em 127.349.
    \item [\textit{Keywords Plus (ID)}] 7.218 palavras-chave distintas foram encontradas no \emph{dataset}.
    \item [\textit{Author's Keywords (DE)}] 10.367 palavras-chave distintas escritas pelos autores.
    \item [\textit{Authors}] No total, foram 14.247 autores, sendo que boa parte deles tem origem chinesa.
    \item [\textit{Author Appearances}] No total, tiveram 20.024 aparições de autores, sendo que o número de autores distintos é, como mencionado anteriormente, 14.247
    \item [\textit{Authors of single-authored documents}] Dentre o número total de autores, apenas 206 fizeram pelo menos 1 artigo sozinhos.
    \item [\textit{Authors of multi-authored documents}] Se retirarmos do número total de autores, o número de autores que escreveram artigo(s) sozinhos, chegamos em 14.041 autores que escreveram apenas artigos coletivos.
    \item [\textit{Single-authored documents}] Dentro do \emph{dataset} apenas 227 deles são de criação individual.
    \item [\textit{Documents per Author}] Se dividirmos o número total de artigos pela quantidade de autores, chegamos em 0,341 artigos/autor.
    \item [\textit{Authors per Document}] Agora, inversamente se fizermos a quantidade de autores distintos divido pelo número de artigos, chegamos em 2,93 autores(/artigo.
    \item [\textit{Co-Authors per Documents}] Se pegarmos o número total de autores (também os repetidos) e dividirmos pela quantidade de documentos, temos 4.12 autores/artigo
    \item [\textit{Collaboration Index}] Por fim, a quantidade de vezes que autores distintos editaram artigos com um ou mais co-autores é de 3,03.
\end{description}
\subsection{Evolução da Produção Científica}
    Os temas procurados na busca são consideravelmente mais recentes que o esperado. O gráfico \ref{fig:KvotheKS:Annual_Scientific} mostra um crescimento quase que perfeitamente exponencial, sendo ele de 12.45\%.
    \begin{figure}[H]
    \centering
    \includegraphics[width=1\textwidth]{experiments/KvotheKS/PesqBibliogr/AlgoritmosSimulacaoOptica-Dinamica/WoS-20220202/Dataset/Annual_Scientific.png}
    \caption{Produção anual científica}
    \label{fig:KvotheKS:Annual_Scientific}
\end{figure}
\subsection{Interpretação do crescimento}
    Com o avanço dos computadores e uma disponibilidade maior de recursos científicos como um todo, vários temas acabam ganhando força por fatores variados. No caso do meu \emph{dataset}, os estudos vão de análise topológica para robôs a estudo de aerodinâmica para aviões, e no fim acabam em simulações de iluminação.
\subsection{\emph{Clustering Network}}
    Como meio de demonstrar o quão "compacto" estão os resultados do \emph{dataset}, podemos utilizar uma \emph{Clustering Network}, que mostra em, em forma de grafo, quais estudos estão relacionados entre si, e qual peso de cada um.  
\begin{figure}[H]
    \centering
    \includegraphics[width=1\textwidth]{experiments/KvotheKS/PesqBibliogr/AlgoritmosSimulacaoOptica-Dinamica/WoS-20220202/Dataset/Cluster_network.png}
    \caption{Grafo de citações}
    \label{fig:KvotheKS:Cluster_}
\end{figure}
\subsection{Interpretação da rede}
    Analisando a figura \ref{fig:KvotheKS:Cluster_}, podemos ver o quão inter-relacionados os artigos estão. Isso é um resultado óbvio do refinamento de dados feito anteriormente. Também mostra que alguns tópicos, como hidrodinâmica, aparecem em maior peso, por causa das revistas em qual artigos foram publicados. 
\subsection{\emph{Three-Field Plot}}
    Já foi demonstrado um dos \emph{Sankey diagrams} anteriormente \ref{fig:KvotheKS:Final_Data_Set}, onde o resultado mais interessante são as palavras-chave a direita, que mostram realmente quais são os tópicos mais abordados no \emph{dataset}. Porém alguns dados interessantes não foram abordados.
    
\begin{figure}[H]
    \centering
    \includegraphics[width=1.1\textwidth]{experiments/KvotheKS/PesqBibliogr/AlgoritmosSimulacaoOptica-Dinamica/WoS-20220202/Dataset/AU_CO_AU_UN_SO.png}
    \caption{Afiliações, revistas e países}
    \label{fig:KvotheKS:SankeyCountry}
\end{figure}
\subsection{Considerações do peso dos países}
    Os dados interessantes da figura \ref{fig:KvotheKS:SankeyCountry} se dão nos países e universidades. Mais da metade dos artigos são chineses, porém não só há uma diversidade grande de universidades chinesas, mas também há uma falta de diferença entre as estado-unidenses, por onde artigos de vários países acabam passando.
\section{}
%\chapter{Análise Bibliográfica sobre Simulação , por }%\label{}


\section{Planejamento do estudo}


\begin{itemize}
    \item 
\end{itemize}

%% Keywords usadas: (graphic processing unit or GPU) and (lighting or light or shadow*)

\chapter{Análise Bibliográfica sobre , por Gustavo Tomás}

\section{Planejamento do estudo}

O objetivo do trabalho é analisar o impacto das GPUs (Graphic Processing Units) no processamento e simulação da luz. Para isso, foram utilizadas as ferramentas Bibliometrix e Biblioshiny.

\subsection{O que já existe de pesquisa bibliométrica sobre esse tema?}

\subsection{Limitações} O exercício relatado foi feito em cerca de uma semana, entre os dias 02 e 10 de fevereiro de 2022 e a base de dados utilizada foi Web Of Science (WoS).

\section{Coleta de dados}

A coleta de dados foi feita usando o WoS no dia 03/02/2022, por meio do Portal de Periódicos da CAPES. Foram feitas buscas nas coleções Science Citation Index Expanded (SCI-EXPANDED) e Social Sciences Citation Index (SSCI), mas com o foco em registros relativos a área de ciências naturais e exatas. A busca utilizada foi a seguinte:

\begin{verbatim}
(graphic processing unit or GPU) and (lighting or light or shadow*)
\end{verbatim}

Essa busca consiste em dois termos, sendo que o primeiro é composto pela GPU (por extenso ou pela sigla) e o segundo pelas palavras luz ou iluminação ou sombra(s). Dessa forma, foram encontrados 1311 registros, sendo que nesse trabalho foram utilizados os primeiros 1000 registros, disponíveis em \ref{}.

\section{Análise dos dados}

\subsection{Filtragem de registros}
Antes da análise, foram aplicados filtros aos registros, de forma que apenas registros do tipo \textit{article}, de qualquer ano e com qualquer número de citações, fossem analisados. O resultado consiste em 850/1000 registros originais.

\subsection{Análise descritiva do dataset}

As informações mais gerais sobre o \textit{dataset} MASSA@jhcf são as seguintes:
\begin{description}
    \item [\textit{Timespan}] Os artigos que atenderam aos critérios de busca e filtragem foram publicados a partir de 1990, até 2021. Ou seja, não foram encontrados registros entre 1945 e 1989.
    \item [\textit{Sources (Journals, Books, etc)}] São 2.319 fontes de informação que publicaram os documentos recuperados no dataset MASSA@jhcf. Ou seja, em média, cada \textit{scientific journal} publicou $5.787/2.319=2,5$ artigos. \footnote{Note que a média, enquanto medida de tendência central, pode não ser a que melhor reflete a tendência a quantidade de artigos publicados por revista.}
    \item [\textit{Average years from publication}] A média do tempo de publicação dos artigos no dataset MASSA@jhcf é de 7,36 anos.
    \item [\textit{Average citations per documents}] Cada artigo no dataset MASSA@jhcf foi citado, em média 20,7 vezes\footnote{Note que a média, enquanto medida de tendência central, pode não ser a que melhor reflete a tendência de  citações a artigos.}.
    \item [\textit{Average citations per year per doc}] Após publicado, cada um dos 5.787 artigos do dataset MASSA@jhcf  foi citado 2,262 vezes por ano, em média.
    \item [\textit{References}] O dataset MASSA@jhcf contém 201.464 referências citadas (tags CR).
    \item [\textit{Keywords Plus (ID)}] 13.735 distintas palavras-chave do tipo Keywords Plus (ID)\footnote{\textit{KeyWords Plus} são ``termos de índice gerados automaticamente a partir dos títulos de artigos citados. Os termos do KeyWords Plus devem aparecer mais de uma vez na bibliografia e são ordenados de frases com várias palavras a termos únicos. O KeyWords Plus aumenta o número de resultados tradicional de palavras-chave ou títulos.'' Fonte: \url{https://images.webofknowledge.com/WOKRS410B4/help/pt_BR/WOS/hp_full_record.html}} foram encontradas no dataset MASSA@jhcf. 
    \item [\textit{Author's Keywords (DE)}] 15.704 distintas palavras-chave indicadas pelos autores foram encontradas no \textit{dataset}.
    \item [\textit{Authors}] 19.410 distintos nomes de autores foram encontrados no dataset\footnote{Um mesmo autor pode ter uma ou mais diferentes grafias no dataset, e serão reconhecidos dois ou mais autores diferentes, embora de fato sejam apenas um. Isso significa que a quantidade de \textbf{nomes de autores} equivale à quantidade de \textbf{autores}. Adicionalmente, é possível que distintos autores sejam reconhecidos com o mesmo nome, isso é, que sejam homônimos. Ou seja, o dataset em geral conterá erros de contagem na quantidade de autores reais.}.
    \item [\textit{Author Appearances}] Os 19.410 distintos (nomes de) autores foram encontrados 23.470 vezes, como autores de artigos.
    \item [\textit{Authors of single-authored documents}] Dentre os 19.410 distintos (nomes de) autores encontrados, 375 deles editaram artigos individualmente, isso é, sem co-autores.
    \item [\textit{Authors of multi-authored documents}] Dentre os 19.410 distintos (nomes de) autores encontrados, 19.035 deles editaram artigos com um ou mais co-autores"
    \item [\textit{Single-authored documents}] Dentre os 5.787 documentos presentes no dataset MASSA, 409 foram escritos por um único autor, e os 5.378 restantes foram elaborados em co-autoria.
    \item [\textit{Documents per Author}] Dentre os 19.410 distintos (nomes de) autores, cada um publicou em média 0,298 artigos.
    \item [\textit{Authors per Document}] Cada um dos 5.787 documentos presentes no dataset MASSA foi autorado com 3,35 autores em média ($19.410 / 5.787 = 3,35$).
    \item [\textit{Co-Authors per Documents}] As 23.470 aparições de (nomes de) autores (``Author Appearances''), sem distribuem, em média 4,06 vezes para os 5.787 documentos do dataset MASSA@jhcf.
    \item [\textit{Collaboration Index}] Os 19.035 (nomes de) autores que editaram artigos com um ou mais co-autores, colaboraram em media 3,54 vezes para editar os 5.378 artigos elaborados em co-autoria, gerando, assim, um índice de colaboração 3,54. 
\end{description}

\subsection{Evolução da Produção Científica}

\subsection{Interpretação do Crescimento}

\subsection{Evolução das Citações}

\subsection{Interpretação das Citações}

\subsection{\textit{Three-Field Plots (Sankey diagram)}}

\subsection{Interpretação da figura}

\subsection{Análises Bibliométricas: Fontes de Informação}

\subsection{Análises Bibliométricas: Autores}

\subsection{Análises Bibliométricas: Documentos}



\subsection{Minhas impressões iniciais sobre a ciência, por Ítalo Eduardo Dias Frota}

A ciência é um escopo de \gls{Conhecimento} e um processo. A área se apoia na busca e aplicação dos conhecimentos a respeito das esferas sociais e naturais, seguindo uma \gls{Metodologia} bem definida baseada em evidências que descrevam, expliquem e possam prever um \gls{fenomeno}. Entretanto, definir a ciência não é uma tarefa fácil devido a pluralidade de aplicações e abordagens que permeiam a \gls{ComunidadeCientifica}.

Um dos aspectos mais importantes no campo científico é o da metodologia científica, que permite a formulação de hipóteses, experimentos  e verificações. Os resultados obtidos devem ser reproduzidos através de artigos e publicações que auxiliem na propagação dos pontos observados, instigando novas descobertas e indagações. Sendo assim, o processo é extremamente autocorretivo, pois está em constante evolução.

Sem a ciência e o pensamento científico, a humanidade enfrentaria dificuldades em larga escala. As descobertas e previsões documentadas pelos membros da comunidade são de extrema importância para os avanços nas mais diversas áreas, desde a saúde, até a educação e tecnologia. É imprescindível que seja dada a devida atenção às observações coletadas pela ciência, caso contrário, a humanidade estará sempre destinada ao fracasso.


\chapter{Slides: Pesquisa Bibliográfica com Bibliometrix\label{bibliometrix}}
	
\includepdf[pages=-]{2-Analise-Exploratoria-Dados/aulas/2.2-Pesquisa-Bibliografica/R-RStudio-Bibliometrix.pdf}

\part{Simulação como Estratégia Experimental\label{part:simulacao:estrategia}}

\chapter{Slides: Fundamentos da Computação Experimental\label{fundamentos:ce}}

\includepdf[pages=-]{3-Computacao-Experimental/aulas/3.1-Componentes-Experimento/slides-experimentos.pdf}

\chapter{Tarefa: Fundamentos da Experimentação: Respostas d(a/o)s Estudantes e Professor}

\section{Especificação da Tarefa: Reflexão sobre os Fundamentos da Computação Experimental}

\subsection{Motivação}

Veja os slides disponíveis em \ref{fundamentos:ce} e a(s) correspondente(s) aula(s) gravada(s), feita(s) pelo professor, em fevereiro de 2022. Elas abordam os componentes de um experimento. A atividade de um cientista da computação utiliza todos esses componentes? Como e por que sim? Como e por que não? 

\subsection{Pergunta a ser respondida por você}

Com base no apresentado e discutido, crie em \textbf{3-Computacao-Experimental / tarefas / 3.1-Componentes-Experimento / estudantes} o texto de uma seção contendo o seu nome, e escreva nela um texto com pelo menos 250 palavras, que ofereça uma resposta ou reflexão sobre a seguinte questão central:
\begin{quote}
A atividade profissional de um cientista da computação utiliza todos os componentes de um experimento científico?
\end{quote}


\subsection{Três perguntas basilares, para iniciar}

Algumas perguntas basilares que permitem o desenvolvimento pleno de uma resposta são a seguir apresentadas:
\begin{itemize}
    \item A computação é uma atividade científica? 
    \item A computação é uma ciência experimental? 
    \item A computação é uma ciência empírica? 
    \item Por que sim e por que não? 
Justifique. 
\end{itemize}

\subsection{O que precisa ser feito na tarefa}

Na sua resposta à questão central, você precisa necessariamente referenciar pelo menos três termos do glossário, sendo:
\begin{itemize}
    \item Dois termos de glossário já existentes;
    \item Um novo termo que você vai criar em \textbf{ 1-Introducao / tarefas / 1.1-Glossario / estudantes / tarefa-\githubusername}, contendo uma nova definição e exemplo de algum termo usado nos slides disponíveis em \ref{fundamentos:ce}.
\end{itemize}

\subsection{Exemplo Inicial}

Veja como exemplo inicial, o texto de provocação feito pelo professor em \ref{tarefa-jhcf-componentes-eperimento}, em resposta às três perguntas basilares, e
desenvolva seus próprios argumentos.



\section{Respostas de Jorge H C Fernandes\label{tarefa-jhcf-componentes-eperimento}}

\subsection{A computação é uma atividade científica? Justifique. }

Para responder se a computação é uma atividade científica é preciso:
\begin{itemize}
    \item Definir o que é uma atividade científica;
    \item Definir o que é computação;
    \item Analisar se a definição de computação sem enquadra na definição de atividade científica. 
\end{itemize}

\subsubsection{O que é uma atividade científica?} Uma atividade científica é aquela que está circunscrita ao ciclo de produção do conhecimento científico. Segundo \cite[p.2]{barton_graphical_1999}, são quatro as fases de um ciclo de produção do conhecimento científico, ilustradas na figura \ref{fig:ciclo:barton}:
\begin{enumerate}
    \item ideação ou hipotetização; 
    \item  planejamento de experimentação;
    \item  experimentação;
    \item Análise dos dados (empíricos).
\end{enumerate}

\begin{figure}
    \centering
    \includegraphics[page=31,clip=true,width=\textwidth]{3-Computacao-Experimental/aulas/3.1-Componentes-Experimento/slides-experimentos.pdf}
    \caption{Ciclo do processo de investigação científica. Fonte: \cite{barton_graphical_1999}\label{fig:ciclo:barton}}
\end{figure}

\subsubsection{O que é a computação?}
A computação, entendida como sendo igual ao processamento de dados, é a ação feita pelos computadores, que envolve o processamento de dados de entrada, gerando dados de saída.

A computação, segundo essa definição estrita, ou seja \textit{strictu sensu}, não seguiria o ciclo de atividade científica, e portanto, não poderia ser considerada uma atividade científica, ou de produção de conhecimento, pois não está em busca da verdade.

Entretanto, a computação que é efetiva precisa sempre processar dados que serão utilizados ou interpretados por algo externo ao computador.
Assim sendo, a atividade de computação é uma ação eminentemente empírica, isso é, produz dados, para uma finalidade qualquer, coerente com a realidade que foi antecipada ou planejada por alguém, o idealizador do programa e o codificador do computador.

Embora não seja uma atividade científica \textit{strictu sensu}, a computação, de forma \textit{lato sensu}, precisa estar inserida em um ciclo produtivo de geração de dados que terão utilidade prática para alguém, sendo dessa forma aproximadamente equivalente a uma atividade de experimentação, no sentido expresso por Barton.

Se considerarmos as ações humanas  antecedentes e sucessoras à computação, elas também apresentam congruência com as demais fases do ciclo de investigação científica, como demonstrado a seguir.

\begin{itemize}
    \item Toda computação
é precedida por um ato criativo de ideação ou formulação de hipóteses feitas por seres humanos, acerca da possibilidade de se criar uma computação que produza um efeito prático desejado junto aos seus usuários. Essa ideação ou formulação, que não precisa ser feita por programadores, visa a produção futura de algum conhecimento novo, ou seja, almeja aproximar-se de alguma verdade. Logo, chega-se à seguinte proposição:
\textbf{
\begin{quote}
    Ideação de um modelo computacional útil e empírico $\cong$ Ideação ou hipotetização científica
\end{quote}
}
\item Uma vez concebidas ideias ou hipóteses sobre um uso útil de uma computação de uma determinada natureza, faz-se necessário o desenvolvimento de um código ou modelo formal, materializado na forma de programas de computador, \textit{firmware} ou hardware, que nada mais é que um plano de execução de um processamento de dados. Assim sendo, pode-se fazer a seguinte proposição:

\textbf{
\begin{quote}
    Codificar  software/firmware/hardware $\cong$ Parte do planejamento de uma experimentação
\end{quote}
}

\item uma vez desenvolvido o modelo computacional, o seu teste em um sistema computacional real ocorre por interação do computador, que opera o modelo, com entidades externas ao computador, chamadas de testadores, que vão trocar dados entre si. O \textit{output} computacional é uma fonte de dados empíricos, similar à ação de experimentação.
O planejamento dos vários casos de teste, para validação do modelo, equivale à ação de controle das variáveis de um experimento científico. Logo, é coerente a proposição a seguir:
\textbf{
\begin{quote}
    Teste sistemático de um sistema computacional $\cong$ Parte do planejamento e execução de experimentos
\end{quote}
}

\item Por fim, concluída a experimentação, os dados empíricos gerados no \textit{output} são analisados e interpretados pelos usuários finais, ou por quem os representa, 
que usarão o computador com o objetivo de analisar se há coerência efetiva entre a realidade externa ao computador, com a ``realidade'' expressa pelo modelo computacional, na produção de \textit{outputs} coerentes com os \textit{inputs} recebidos.
Ou seja, as ações de uso do sistema computacional produzido de forma metodicamente organizada nas etapas anteriores, equivalem a uma ação sistemática de aquisição de novo conhecimento. Em última instância, o novo conhecimento é útil se o usuário fica ``feliz'', se o sistema computacional é coerente, preciso, de boa usabilidade, flexibilidade etc. Concluí-se com a seguinte proposição:
\textbf{
\begin{quote}
 Uso de um sistema de computação $\cong$ Análise dos dados (empíricos)    
\end{quote}
}
\end{itemize}

Depreende-se, pelos argumentos expostos, que a computação, \textit{lato sensu}, é uma atividade que se aproxima bastante da atividade científica.

\subsection{A computação é uma ciência experimental? Justifique. }

Com base nos argumentos anteriores, a computação \textit{lato sensu}, quando usada no sentido de ideação, codificação, teste e uso de computadores aproxima-se da produção do conhecimento científico pelo método experimental.

\subsection{A computação é uma ciência empírica? Justifique. }

Com base nos argumentos anteriores, a computação \textit{lato sensu}, quando usada no sentido de ideação, codificação, teste e uso de computadores aproxima-se da produção do conhecimento científico baseado no consumo e geração de dados, sendo assim, uma atividade eminentemente empírica.





\section{Respostas de João Pedro Felix}

Partindo do princípio que a ciência tem por objetivo descrever, explicar e prever fenômenos (\gls{fenomeno}) a partir de dados (\gls{Dado}), tudo isso através de um estudo metodicamente organizado, fica claro que a posição da área de ciência da computação enquanto ciência fica em cheque. Ainda assim, é interessante analisarmos se a atividade profissional de um cientista da computação utiliza todos os componentes de um experimento científico.

Um experimento científico nasce dos seguintes componentes:

Uma classe de fenômenos de interesse. No caso da computação poderíamos citar qualquer problema que possa ser resolvido computacionalmente, como traçar a melhor rota para fugir do trânsito.

Hipóteses e modelos. O processo de definir um modelo computacional, isto é, as bases para que o programa que será desenvolvido de fato solucione problema que caracteriza o \gls{fenomeno} de interesse, pode se encaixar neste componente.

Causa e efeito. A \gls{Causalidade} diz respeito a influência que um evento tem de produzir outro evento. É evidente os impactos gerados pela computação no mundo inteiro, sem ela, este texto por exemplo não teria sido elaborado.

Procedimentos Repetíveis. A própria natureza da computação é baseada no fato de que toda vez que um programa for executado sob as mesmas circunstâncias, este deve apresentar os mesmos resultados. Um exemplo desta abordagem seria a da comparação de eficiência entre dois algoritmos de ordenação. Ambos algoritmos podem ser disponibilizados e uma análise pode ser refeita por qualquer pessoa que detenha o conhecimento necessário. Os resultados serão os mesmos, desde que feitos corretamente.

Controle de fatores. Ainda no exemplo anterior, seria possível desconsiderar fatores na análise como a variação na frequência do processador utilizado durante a execução do algoritmo. Seria possível até realizar a análise ignorando completamente as interferências dos componentes físicos do sistema.

Coleta de dados. A computação tem como principio básico o processamento de dados de entrada e a geração de dados de saída. Isso a caracteriza como uma atividade claramente empírica. Naturalmente, isso se dá através da coleta de dados.

Análise Lógica. A comprovação de um sistema computacional pode se dar de diferentes maneiras, desde uma argumentação formal matemática, até uma análise empírica por parte dos usuários durante o uso do sistema.

Com tudo isso em mente, podemos concluir que, embora a computação não se encaixe muito bem na definição de ciência, a computação trata-se de uma atividade empírica que se aproxima bastante do processo de produção do conhecimento científico via método experimental.


%
Veja os slides (disponíveis em \ref{fundamentos:ce}) e a aulas do professor, em 12 e 19 de agosto de 2021.

Com base no apresentado e discutido crie abaixo uma subseção contendo o seu nome, e escreva nela um parágrafo com pelo menos 50 palavras que ofereça uma resposta ou reflexão sobre as seguintes questões:

\begin{itemize}
    \item A computação é uma atividade científica? 
    \item A computação é uma ciência experimental? 
    \item A computação é uma ciência empírica? 
    \item Por que sim e por que não? 
Justifique. 
\end{itemize}

Na sua resposta você precisa necessariamente referenciar um termo no glossário, que você criou com uma definição e exemplo, conforme orientação na tarefa T3.1.

\section{Resposta de Jorge H C Fernandes}

\subsection{A computação é uma atividade científica? Justifique. }

Para responder se a computação é uma atividade científica é preciso:
\begin{itemize}
    \item Definir o que é uma atividade científica;
    \item Definir o que é computação;
    \item Analisar se a definição de computação sem enquadra na definição de atividade científica. 
\end{itemize}

\subsubsection{O que é uma atividade científica?} Uma atividade científica é aquela que está circunscrita ao ciclo de produção do conhecimento científico. Segundo \cite[p.2]{barton_graphical_1999}, são quatro as fases de um ciclo de produção do conhecimento científico, ilustradas na figura \ref{fig:ciclo:barton}:
\begin{enumerate}
    \item ideação ou hipotetização; 
    \item  planejamento de experimentação;
    \item  experimentação;
    \item Análise dos dados (empíricos).
\end{enumerate}

\begin{figure}
    \centering
    \includegraphics[page=7,clip=true,width=\textwidth]{aulas/3-Simulacao/slides_Aula_12082021_ComputacaoExperimental_ConceitosFundamentais.pdf}
    \caption{Ciclo do processo de investigação científica. Fonte: \cite{barton_graphical_1999}\label{fig:ciclo:barton}}
\end{figure}

\subsubsection{O que é a computação?}
A computação, entendida como sendo igual ao processamento de dados, é a ação feita pelos computadores, que envolve o processamento de dados de entrada, gerando dados de saída.

A computação, segundo essa definição estrita, ou seja \textit{strictu sensu}, não seguiria o ciclo de atividade científica, e portanto, não poderia ser considerada uma atividade científica, ou de produção de conhecimento, pois não está em busca da verdade.

Entretanto, a computação que é efetiva precisa sempre processar dados que serão utilizados ou interpretados por algo externo ao computador.
Assim sendo, a atividade de computação é uma ação eminentemente empírica, isso é, produz dados, para uma finalidade qualquer, coerente com a realidade que foi antecipada ou planejada por alguém, o idealizador do programa e o codificador do computador.

Assim sendo, embora não seja uma atividade científica \textit{strictu sensu}, a computação, de forma \textit{lato sensu}, precisa estar inserida em um ciclo produtivo de geração de dados que terão utilidade prática para alguém, sendo dessa forma aproximadamente equivalente a uma atividade de experimentação, no sentido expresso por Barton.

Se considerarmos as ações humanas  antecedentes e sucessoras à computação, elas também apresentam congruência com as demais fases do ciclo de investigação científica, como demonstrado a seguir.

\begin{itemize}
    \item Toda computação
é precedida por um ato criativo de ideação ou formulação de hipóteses feitas por seres humanos, acerca da possibilidade de se criar uma computação que produza um efeito prático desejado junto aos seus usuários. Essa ideação ou formulação, que não precisa ser feita por programadores, visa a produção futura de algum conhecimento novo, ou seja, almeja aproximar-se de alguma verdade. Logo, chega-se à seguinte proposição:
\textbf{
\begin{quote}
    Ideação de um modelo computacional útil e empírico $\cong$ Ideação ou hipotetização científica
\end{quote}
}
\item Uma vez concebidas ideias ou hipóteses sobre um uso útil de uma computação de uma determinada natureza, faz-se necessário o desenvolvimento de um código ou modelo formal, materializado na forma de programas de computador, \textit{firmware} ou hardware, que nada mais é que um plano de execução de um processamento de dados. Assim sendo, pode-se fazer a seguinte proposição:

\textbf{
\begin{quote}
    Codificar software/firmware/hardware $\cong$ Planejamento de experimentação
\end{quote}
}

\item uma vez desenvolvido o modelo computacional, o seu teste em um sistema computacional real ocorre por interação do computador, que opera o modelo, com entidades externas ao computador, chamadas de testadores, que vão trocar dados entre si. O \textit{output} computacional é uma fonte de dados empíricos, similar à ação de experimentação.
O planejamento dos vários casos de teste, para validação do modelo, equivale à ação de controle das variáveis de um experimento científico. Logo, é coerente a proposição a seguir:
\textbf{
\begin{quote}
    Teste sistemático de um sistema computacional $\cong$ Experimentação
\end{quote}
}

\item Por fim, concluída a experimentação, os dados empíricos gerados no \textit{output} são analisados e interpretados pelos usuários finais, ou por quem os representa, 
que usarão o computador com o objetivo de analisar se há coerência efetiva entre a realidade externa ao computador, com a ``realidade'' expressa pelo modelo computacional, na produção de \textit{outputs} coerentes com os \textit{inputs} recebidos.
Ou seja, as ações de uso do sistema computacional produzido de forma metodicamente organizada nas etapas anteriores, equivalem a uma ação sistemática de aquisição de novo conhecimento. Em última instância, o novo conhecimento é útil se o usuário fica ``feliz'', se o sistema computacional é coerente, preciso, de boa usabilidade, flexibilidade etc. Concluí-se com a seguinte proposição:
\textbf{
\begin{quote}
 Uso de um sistema de computação $\cong$ Análise dos dados (empíricos)    
\end{quote}
}
\end{itemize}

Depreende-se, pelos argumentos expostos, que a computação, \textit{lato sensu}, é uma atividade que se aproxima bastante da atividade científica.

\subsection{A computação é uma ciência experimental? Justifique. }

Com base nos argumentos anteriores, a computação \textit{lato sensu}, quando usada no sentido de ideação, codificação, teste e uso de computadores aproxima-se da produção do conhecimento científico pelo método experimental.

\subsection{A computação é uma ciência empírica? Justifique. }

Com base nos argumentos anteriores, a computação \textit{lato sensu}, quando usada no sentido de ideação, codificação, teste e uso de computadores aproxima-se da produção do conhecimento científico baseado no consumo e geração de dados, sendo assim, uma atividade eminentemente empírica.

\subsection{Citação a todos os termos do glossário gerado pela turma, no escopo do exercício de início da etapa experimental da disciplina}

\begin{enumerate}
\item \gls{EA};
\item \gls{Glossario};
\end{enumerate}




%
Veja os slides (disponíveis em \ref{fundamentos:ce}) e a aulas do professor, em 12 e 19 de agosto de 2021.

Com base no apresentado e discutido crie abaixo uma subseção contendo o seu nome, e escreva nela um parágrafo com pelo menos 50 palavras que ofereça uma resposta ou reflexão sobre as seguintes questões:

\begin{itemize}
    \item A computação é uma atividade científica? 
    \item A computação é uma ciência experimental? 
    \item A computação é uma ciência empírica? 
    \item Por que sim e por que não? 
Justifique. 
\end{itemize}

Na sua resposta você precisa necessariamente referenciar um termo no glossário, que você criou com uma definição e exemplo, conforme orientação na tarefa T3.1.

\section{Resposta de Jorge H C Fernandes}

\subsection{A computação é uma atividade científica? Justifique. }

Para responder se a computação é uma atividade científica é preciso:
\begin{itemize}
    \item Definir o que é uma atividade científica;
    \item Definir o que é computação;
    \item Analisar se a definição de computação sem enquadra na definição de atividade científica. 
\end{itemize}

\subsubsection{O que é uma atividade científica?} Uma atividade científica é aquela que está circunscrita ao ciclo de produção do conhecimento científico. Segundo \cite[p.2]{barton_graphical_1999}, são quatro as fases de um ciclo de produção do conhecimento científico, ilustradas na figura \ref{fig:ciclo:barton}:
\begin{enumerate}
    \item ideação ou hipotetização; 
    \item  planejamento de experimentação;
    \item  experimentação;
    \item Análise dos dados (empíricos).
\end{enumerate}

\begin{figure}
    \centering
    \includegraphics[page=7,clip=true,width=\textwidth]{aulas/3-Simulacao/slides_Aula_12082021_ComputacaoExperimental_ConceitosFundamentais.pdf}
    \caption{Ciclo do processo de investigação científica. Fonte: \cite{barton_graphical_1999}\label{fig:ciclo:barton}}
\end{figure}

\subsubsection{O que é a computação?}
A computação, entendida como sendo igual ao processamento de dados, é a ação feita pelos computadores, que envolve o processamento de dados de entrada, gerando dados de saída.

A computação, segundo essa definição estrita, ou seja \textit{strictu sensu}, não seguiria o ciclo de atividade científica, e portanto, não poderia ser considerada uma atividade científica, ou de produção de conhecimento, pois não está em busca da verdade.

Entretanto, a computação que é efetiva precisa sempre processar dados que serão utilizados ou interpretados por algo externo ao computador.
Assim sendo, a atividade de computação é uma ação eminentemente empírica, isso é, produz dados, para uma finalidade qualquer, coerente com a realidade que foi antecipada ou planejada por alguém, o idealizador do programa e o codificador do computador.

Assim sendo, embora não seja uma atividade científica \textit{strictu sensu}, a computação, de forma \textit{lato sensu}, precisa estar inserida em um ciclo produtivo de geração de dados que terão utilidade prática para alguém, sendo dessa forma aproximadamente equivalente a uma atividade de experimentação, no sentido expresso por Barton.

Se considerarmos as ações humanas  antecedentes e sucessoras à computação, elas também apresentam congruência com as demais fases do ciclo de investigação científica, como demonstrado a seguir.

\begin{itemize}
    \item Toda computação
é precedida por um ato criativo de ideação ou formulação de hipóteses feitas por seres humanos, acerca da possibilidade de se criar uma computação que produza um efeito prático desejado junto aos seus usuários. Essa ideação ou formulação, que não precisa ser feita por programadores, visa a produção futura de algum conhecimento novo, ou seja, almeja aproximar-se de alguma verdade. Logo, chega-se à seguinte proposição:
\textbf{
\begin{quote}
    Ideação de um modelo computacional útil e empírico $\cong$ Ideação ou hipotetização científica
\end{quote}
}
\item Uma vez concebidas ideias ou hipóteses sobre um uso útil de uma computação de uma determinada natureza, faz-se necessário o desenvolvimento de um código ou modelo formal, materializado na forma de programas de computador, \textit{firmware} ou hardware, que nada mais é que um plano de execução de um processamento de dados. Assim sendo, pode-se fazer a seguinte proposição:

\textbf{
\begin{quote}
    Codificar software/firmware/hardware $\cong$ Planejamento de experimentação
\end{quote}
}

\item uma vez desenvolvido o modelo computacional, o seu teste em um sistema computacional real ocorre por interação do computador, que opera o modelo, com entidades externas ao computador, chamadas de testadores, que vão trocar dados entre si. O \textit{output} computacional é uma fonte de dados empíricos, similar à ação de experimentação.
O planejamento dos vários casos de teste, para validação do modelo, equivale à ação de controle das variáveis de um experimento científico. Logo, é coerente a proposição a seguir:
\textbf{
\begin{quote}
    Teste sistemático de um sistema computacional $\cong$ Experimentação
\end{quote}
}

\item Por fim, concluída a experimentação, os dados empíricos gerados no \textit{output} são analisados e interpretados pelos usuários finais, ou por quem os representa, 
que usarão o computador com o objetivo de analisar se há coerência efetiva entre a realidade externa ao computador, com a ``realidade'' expressa pelo modelo computacional, na produção de \textit{outputs} coerentes com os \textit{inputs} recebidos.
Ou seja, as ações de uso do sistema computacional produzido de forma metodicamente organizada nas etapas anteriores, equivalem a uma ação sistemática de aquisição de novo conhecimento. Em última instância, o novo conhecimento é útil se o usuário fica ``feliz'', se o sistema computacional é coerente, preciso, de boa usabilidade, flexibilidade etc. Concluí-se com a seguinte proposição:
\textbf{
\begin{quote}
 Uso de um sistema de computação $\cong$ Análise dos dados (empíricos)    
\end{quote}
}
\end{itemize}

Depreende-se, pelos argumentos expostos, que a computação, \textit{lato sensu}, é uma atividade que se aproxima bastante da atividade científica.

\subsection{A computação é uma ciência experimental? Justifique. }

Com base nos argumentos anteriores, a computação \textit{lato sensu}, quando usada no sentido de ideação, codificação, teste e uso de computadores aproxima-se da produção do conhecimento científico pelo método experimental.

\subsection{A computação é uma ciência empírica? Justifique. }

Com base nos argumentos anteriores, a computação \textit{lato sensu}, quando usada no sentido de ideação, codificação, teste e uso de computadores aproxima-se da produção do conhecimento científico baseado no consumo e geração de dados, sendo assim, uma atividade eminentemente empírica.

\subsection{Citação a todos os termos do glossário gerado pela turma, no escopo do exercício de início da etapa experimental da disciplina}

\begin{enumerate}
\item \gls{EA};
\item \gls{Glossario};
\end{enumerate}




\chapter{Apresentação do Protocolo ODD\label{oddprotocol}}
	
%\includepdf[pages=-]{aulas/3-Simulacao/ProtocoloODD.pdf}

\chapter{O Modelo CE20211 e o Protocolo ODD: Respostas d(a/o)s Estudantes}

%
\section{Introdução}
CE20211 é o nome provisório do modelo de simulação que será desenvolvido de forma conjunta pela turma, para a realização de experimentos didáticos no âmbito da disciplina-turma Computação Experimental, no semestre 2021.1

Este capítulo apresenta uma descrição padronizada do modelo, adotando-se o protocolo ODD \citep{grimm_standard_2006}.

Para suporte e detalhamento, ver os slides em \ref{oddprotocol}, juntamente com as aulas gravadas.

\section{Overview}

\subsection{Propósito do Modelo - Geral}\label{sec:proposito}

\begin{quote}
    Descreva aqui o propósito do modelo de simulação do processo de produção científica no domínio de simulação multiagente de sociedades.
\end{quote}

O propósito do modelo de simulação será auxiliar na compreensão de como o comportamento social dos cientistas (seres humanos) – em particular:
\begin{itemize}
    \item 
envolvimento em projetos de pesquisa científica como investigadores \footnote{
Um projeto científico é um empreendimento temporário, executado com recursos limitados, com participação de cientistas, aspirantes a cientistas, técnicos - que tem por objetivo produzir algo novo (no caso, um novo conjunto de conhecimentos científicos – dados, relatórios, artigos, publicações)}
\item filiação a Organizações, a \replaced[id==JHCF]{Territórios}{países}, 
\item atuação em áreas das ciências humanas ou naturais, ou culturais ou tecnológicas, 
\end{itemize}

afeta o seu desempenho, em particular a produtividade científica (dos cientistas), mensurada por índices de impacto, tais como:
\begin{itemize}
\item Índice H
\item Índice G
\item outros 
\end{itemize}


Exercício 3.2: 2 pontos: Contribua com entre 20 e 50 palavras em complementação ao propósito do modelo declarado acima, segundo a ideia de declaração de propósito descrita no protocolo ODD \citep{grimm_standard_2006}.
Na sua contribuição, faça referência a pelo menos uma fonte de informação na Internet, usando uma URL ou referência no Zotero.

\subsubsection{Complementação ao propósito do Modelo segundo aluno Guilherme Mendel}
Acredito que talvez seja interessante observar como esses fenômenos sociais influenciam cientistas de diferentes \href{https://pt.wikipedia.org/wiki/Gera\%C3\%A7\%C3\%A3o}{gerações} ou territórios. Cientistas ativos no século 20 podem apresentar resultados diferentes dos cientistas ativos atualmente, por exemplo.

    
\subsubsection{Complementação ao propósito do Modelo segundo aluna Jaqueline Gutierri Coelho}
Minha complementação ao propósito do modelo trata-se da discrepância de métodos científicos(experimentações científicas) utilizados por cada cientista, pois cada cientista possui a intervenção dos meios em que foi criado, e do local onde foi realizado seus estudos em sua formação. \citep{vernotte_risk-driven_2015}

\subsubsection{Complementação ao propósito do Modelo segundo aluna Caroline Ferreira Pinto}
Complemento o propósito do modelo descrito anteriormente dizendo que, além dos fatores citados acima, a história de vida de um cientista também interfere em suas pesquisas, já que cada um deles possui uma preferência em relação à áreas de estudo ou uma familiaridade com certo assunto. \href{https://comportese.com/2014/05/01/o-comportamento-do-cientista}{Fonte}

\subsubsection{Complementação ao propósito do Modelo segundo aluno Otávio Souza de Oliveira}

Complementando o modelo, um dos principais aspectos que mobilizam um pesquisador na produção científica é a \href{https://www.scielo.br/j/pci/a/ww5zR3KhYCk65bPkWJyTQtf/?lang=pt}{internet} que facilita nas pesquisas, referências, na busca por informações. Mas por sua vez com a quantidade enorme de informações já divulgadas, podem atrapalhar o crescimento desse pesquisador no quesito reconhecimento e assim desmotivá-lo.

\subsubsection{Complementação ao propósito do Modelo segundo aluna Kailany Ketulhe Gomes Rocha}

A minha complementação ao modelo descrito na seção \ref{sec:proposito}, seria entender os recursos disponíveis e utilizados pelos cientistas nas suas produções, entendendo as limitações de suas pesquisas. A escassez de
\href{https://conexao.ufrj.br/2019/02/escassez-de-verbas-corroi-pesquisa-cientifica-nacional/}{recurso} pode influenciar no desenvolvimento de uma pesquisa e na manutenção da mesma.

\subsubsection{Complementação ao propósito do Modelo segundo o aluno Álvaro Veloso Cavalcanti Luz}

Como complementação para o modelo, é importante que sejam notados aspectos que afetam a produtividade científica. Por exemplo, um desses aspectos é a inconsistência do apoio financeiro à área, visto que os investimentos públicos para pesquisa são constantemente alvos de \href{https://www.redebrasilatual.com.br/educacao/2021/04/educacao-e-a-area-mais-atingida-pelos-cortes-orcamentarios-de-bolsonaro/} {cortes e imposições de tetos de gastos} no Brasil atual.

\subsubsection{Complementação ao propósito do Modelo segundo aluna Giovana Pinho Garcia}
Algo muito importante que afeta a produtividade de um cientista é o financiamento por parte do governo ou de algum patrocinador. Essa falta de financiamento pode afetar muito o desempenho de cientistas principalmente de baixa renda. \href{https://www.ufg.br/n/129177-baixo-investimento-em-ciencia-e-tecnologia-eleva-a-desigualdade-social}{Fonte}.

\subsubsection{Complementação ao propósito do Modelo segundo aluno Vitor Vasconcelos de Oliveira}
Um fato válido e importante para o modelo retratar, seria o aspecto do acesso à educação no ambiente de criação dos cientistas. Muitos lugares no mundo não proporcionam um acesso adequado a educação e ao conhecimento, isso pode gerar discrepância na forma que o conhecimento científico é dividido pelo mundo.
\href{https://www.senado.gov.br/noticias/Jornal/emdiscussao/inovacao/pesquisa-ciencia-tecnologia-e-inovacao-educacao/caminho-para-o-conhecimento-cientifico-e-a-inovacao-tecnologica-no-pais-ensino-de-ciencias-nas-escolas-e-desafio-para-alunos-e-professores-do-brasil.aspx}{Fonte}

\subsubsection{Complementação ao propósito do Modelo segundo aluno Leonardo Rodrigues de Souza}
Como complementação para o modelo, é importante entender como sustentar áreas da ciência que não trazem um grande retorno financeiro ao mercado, já que esse tipo de pesquisa costumam trabalhar escassez de recursos, ou mesmo serem cortadas rapidamente. \href{https://exame.com/ciencia/crise-na-ciencia-nao-se-deve-so-a-falta-de-recursos/}{Fonte}

\subsubsection{Complementação ao propósito do Modelo segundo aluno Rafael Gonçalves de Paulo}
A capacidade produtiva de um pesquisador depende da sua qualidade de vida, que, pelo pesquisador estar inserido num ambiente político e social, depende das políticas públicas de onde vive. Incorporar essa lógica no modelo pode ser interessante como complementação ao modelo. \href{https://www.revista.ueg.br/index.php/mirante/article/view/7103}{Fonte}.

\subsubsection{Complementação ao propósito do Modelo segundo aluno João Gabriel Ferreira Saraiva}
Uma coisa que ocorre muito no nosso país, é a falta de conhecimento das pessoas sobre os trabalhos científicos desenvolvidos, principalmente nas universidades, é um desses desafios, o que faz com que os impactos gerados por essas pesquisas não sejam de conhecimento da população. \href{https://www.periodicosdeminas.ufmg.br/entenda-os-atuais-desafios-das-pesquisas-cientificas/}{Fonte}

\subsubsection{Complementação ao propósito do Modelo segundo aluno João Pedro Sadéri da Silva}
Para a complementação desse modelo, a falta de matérias adequados e a ausência de infraestrutura para o \href{https://www.blogs.unicamp.br/socialmente/2011/08/23/o-que-e-e-para-que-serve-um-experimento/}{experimento} pode gerar impacto na forma como é obtido resultados dessa pesquisa.

\subsubsection{Complementação ao propósito do Modelo segundo aluno Lucas Vinicius Magalhães Pinheiro}
Um outro fator que impacta na produção científica de cientistas é a questão financeira. Mais em foco, no âmbito de quais tópicos agregam interesse de investidores, tendo assim um agregado de tópicos que angariam mais atenção. Assim, a liberdade criativa do pesquisador é limitada pelo potencial financeiro de sua pesquisa. \href{https://revistapesquisa.fapesp.br/conflito-de-interesses-um-desafio-inevitavel/}{Fonte}

\subsubsection{Complementação ao propósito do Modelo segundo aluno Thales Lima Menezes}

Visando complementar o propósito do modelo sugerido acima proponho avaliar a presença em redes sociais dos cientistas; o objetivo é observar a habilidade dos cientistas em compartilhar o conteúdo de suas pesquisas com a população não acadêmica, podendo incluir contato com possíveis investidores. \href{https://vejasp.abril.com.br/cultura-lazer/influencers-ciencia-atila-iamarino/}{Fonte}

\subsubsection{Complementação ao propósito do Modelo segundo aluno Heitor de Lima Belém}https://pt.overleaf.com/project/60f1998bd62a4772843288a0
Para complementar o modelo, é necessário avaliar aspectos que afetem a produtividade científica. A proximidade da sociedade com as universidades é um deles. Com o apoio social, é mais provável que mais investimentos sejam feitos na área de produção científica de modo que haja maior desenvolvimento e engajamento em pesquisas. \href{https://www.periodicosdeminas.ufmg.br/entenda-os-atuais-desafios-das-pesquisas-cientificas/}{Fonte}

\subsubsection{Complementação ao propósito do Modelo segundo aluno João Francisco Gomes Targino}
Como um dos fatores que impacta na produção científica, o desequilíbrio de gênero é um deles, pois a hegemonia masculina prevalece nos ganhos de verbas e bolsas, essa discrepância acontece em quase todas as áreas mas principalmente na de Engenharia, Ciências Exatas e da Terra.
\href{https://sites.usp.br/psicousp/desequilibrio-de-genero-afeta-mulheres-cientistas-brasil/}{Fonte}

\subsubsection{Complementação ao propósito do Modelo segundo aluno João Victor Pinheiro de Souza}

Um dos autores que influência a produção científica é a descrença na ciência, onde notícias falsa e teorias da conspiração são propagada com muita velocidade pelas redes sociais, o conhecimento científico virou alvo frequente de ataques de grupos de crenças, interesse politico e econômicos contrariados.
\href{https://revistapesquisa.fapesp.br/resistencia-a-ciencia/}{Fonte}

\subsubsection{Complementação ao propósito do Modelo segundo aluno Matheus Arruda Aguiar}

É importante ressaltar que fenômenos educacionais impactam consideravelmente na produção cientifica, dado que cada instituição de ensino é direcionada na visão pedagógica dos coordenadores e não uma instrução teórica, crítica despertando o interesse cientifico. \href{https://monografias.brasilescola.uol.com.br/pedagogia/influencia-globalizacao-praticas-educativas-e-reformulacao-conteudos.htm}{Fonte-Contemplação}

\subsubsection{Complementação ao propósito do Modelo segundo aluno Felipe Gomes Paradas}

Um dos fatores que mais impactam na produção científica é a jornada dupla para mulheres, enquanto cientistas ainda precisam prestar o papel de donas de casa e mães, o que afeta diretamente a produtividade. \url{https://agenciabrasil.ebc.com.br/radioagencia-nacional/economia/audio/2021-03/dupla-jornada-e-salarios-menores-realidade-que-ainda-afeta-mulheres}

\subsubsection{Complementação ao propósito do Modelo segundo aluno José Fortes Neto}

O modelo poderia ser complementado estudando os fatores externos que influenciam na qualidade da pesquisa, principalmente fatores atrelados ao apoio da sociedade aos trabalhos científicos e como eles são vistos. \href{https://www.periodicosdeminas.ufmg.br/entenda-os-atuais-desafios-das-pesquisas-cientificas/}{Fonte}


\subsubsection{Complementação ao propósito do Modelo segundo aluno Adelson Jhonata Silva de Sousa}
O propósito do modelo proposto pode ser complementando procurando entender os fatores que levam a fuga dos cérebros brasileiros, ou seja, a fuga dos pesquisadores brasileiros para continuar e terminar as suas pesquisas em outros países. \href{https://www.bbc.com/portuguese/brasil-51110626}{Fonte}


\subsubsection{Complementação ao propósito do Modelo segundo aluno João Gabriel Lima Neves}

O modelo poderia ser complementado fazendo uma analise de como aspectos burocráticos no processo de realização de pesquisas por profissionais afetam a suas pesquisas e a praticidade delas. Obstáculos existentes no processo de pesquisa em campo, que pressupõe uma articulação constante entre teoria e pratica, podem dificultar a utilização dos resultados dessas pesquisas.\href{https://www.scielo.br/j/icse/a/pZ8djC8vGSWbkQzwmKr4ByH/?lang=pt}{Fonte} 



\subsubsection{Complementação ao propósito do Modelo segundo aluna Alice da Silva de Lima}
Em complemento, uma vez que o presente trabalho está sendo realizado em um período pandêmico (COVID-19), o modelo também deve auxiliar na análise do número publicações do ano de 2020 em comparação aos anos anteriores, verificar se existe alguma anormalidade, isto é, um \href{https://medium.com/ensina-ai/outlier-o-ponto-fora-da-curva-1f28f3d9c23}{outlier} nessa métrica. É interessante observar essa questão geograficamente.

\subsubsection{Complementação ao propósito do Modelo segundo aluno Carlos Eduardo de Oliveira Ribeiro}
Um dos aspectos que pode ser observado tanto para o incentivo quanto para a desmotivação de uma produção científica é a existência da \href{https://pt.wikipedia.org/wiki/Pseudoci\%C3\%AAncia}{pseudociência} sobre um determinado assunto. A desmotivação por não conseguir o alcance como o de uma \textit{fake news} e incentivo para refutar aquilo que está circulando sem embasamento científico.

\subsubsection{Complementação ao propósito do Modelo segundo aluno Henrique Mendes de Freitas Mariano}
O modelo poderia cobrir sobre como a psicologia dos indivíduos que produzem ciência afeta na produção cientifica, pois dependendo do estado da psicologia do cientista sua produção pode ser afetada, assim diminuindo a sua contribuição para a ciência. \href{https://www.scielo.br/j/psoc/a/6X46nvFMKpmcLKv7HnYx76R/?lang=pt}{Fonte}.


\subsubsection{Complementação ao propósito do Modelo segundo o aluno Luthiery Costa Cavalcante}
Seria interessante ao propósito do modelo considerar como a presença massiva de determinado setor da indústria/comércio em uma cidade, região ou país, pode afetar o investimento e a produção científica dos indivíduos dessa localidade. \href{https://www.scielo.br/j/ss/a/FgDGxNb9L74g4fMch8kqM5F/?lang=pt}{Este artigo} aborda especificamente influências negativas da indústria na ciência local.

\subsubsection{Complementação ao propósito do Modelo segundo o aluno Gabriel Nazareno Halabi}
Um aspecto a ser observado como complemento ao modelo seriam as diferenças que se encontra entre o financiamento público e privado no desenvolvimento de produções científicas, considerando que há grandes diferenças no foco de investimento em diferentes países. \href{https://www.scielo.br/j/ep/a/WgdZnSMrX49LLTJMffmsqNK/?lang=pt}{Fonte}

\subsubsection{Complementação ao propósito do Modelo segundo aluno Paulo Maurício Costa Lopes}
Em complemento ao protocolo ODD seria interessante fazer um estudo das expectativas dos pesquisadores antes e depois do desenvolvimento de produções cientificas como também o senso critico deles antes e pós produção cientifica, \href{ https://noticias.unb.br/publicacoes/67-ensino/1938-iniciacao-cientifica-da-unb-e-premiada}{Texto exemplo}. 


\subsection{Redeclaração do propósito do modelo, após contribuições da turma}\label{sec:proposito:contrib}

\begin{quote}
    Segue nova descrição do propósito do modelo de simulação do processo de produção científica no domínio de simulação multiagente de sociedades, incorporando contribuições da turma, com acréscimos, esclarecimentos etc.
\end{quote}

O propósito do modelo de simulação será auxiliar na compreensão de como o comportamento dos cientistas 
é afetado por fatores diversos de ordem geral, como: 
\begin{itemize}
    \item Humana/Psicológica (pessoas que tem um ciclo/história de vida de nascimento, estudo acadêmico, ingresso, atividade e saída da carreira científica, experimentam infância, juventude, maturidade, envelhecimento, formação de família, filhos, tem aspirações geracionais, evolução da moral,  cognição\cite{sprouts_kohlberg_2019}, competitividade etc);
    \item Geográfica (clima, barreiras físicas-geográficas, vias de ligação terrestre, aérea, marítima com outros territórios etc, formando uma malha - grafo - com suporte a determinados tipos de fluxos de matéria, energia, pessoas e informação) etc;
    \item Cultural (tudo o que determina o comportamento coletivo, hábitos coletivos, sistema de valores e princípios coletivos, sistemas filosóficos, educação científica, tecnológica,  cultural e artística, integração com outros povos, isolamento cultural, equilíbrio de papéis e protagonismo entre gêneros, ausência de discriminação racial, étnica, sectarismo, multiculturalismo, multilinguísmo/plurilinguísmo,  visibilidade social dos cientistas, valorização do conhecimento científico versus o conhecimento dogmático/mitológico/religioso, de carreira acadêmica, valorização de pesquisa na indústria, comércio e serviços, atuação em determinadas estruturas sociais, em redes etc); 
    \item Histórica (fatos de alto impacto social e eventual permanência que determinam o curso/estágios de evolução do comportamento coletivo/cultural, experiência em conflitos, guerras, pandemias, apego a tradições etc); 
    \item Econômica (domínio do território na economia mundial, financiamento, investimentos e recursos para pesquisa, vida acadêmica, trabalho, integração com mercado, indução de \textit{startups}, equilíbrio de oportunidades entre territórios, empreendedorismo, simplicidade e agilidade no uso de recursos públicos etc); e
    \item Tecnológica (acesso a recursos e ferramentas, ambientes e infraestruturas de pesquisa e suporte à convivência acadêmica);
\end{itemize} 
e em particular:
\begin{itemize}
    \item 
envolvimento dos cientistas em projetos de pesquisa científica como investigadores \footnote{
Um projeto científico é um empreendimento temporário, executado com recursos limitados, com participação de cientistas, aspirantes a cientistas, técnicos - que tem por objetivo produzir algo novo (no caso, um novo conjunto de conhecimentos científicos – dados, relatórios, artigos, publicações)}, por meio de 
\item filiação a Organizações;
\item atuação em  \replaced[id==JHCF]{Territórios}{países}, 
\item pesquisa em áreas das ciências humanas ou naturais, exatas, ou tecnológicas, ou culturais;
\end{itemize}

afetando a produção de dados, relatórios, artigos científicos que resultam em publicações que permitem mensurar o desempenho individual, em particular a produtividade científica (dos cientistas), usando-se índices de impacto, tais como:
\begin{itemize}
\item Índice H
\item Índice G
\item outros 
\end{itemize}


\section{Variáveis de estado}

A partir de \citet{grimm_standard_2006}, as variáveis de estado podem ser elicitadas a partir de respostas às seguintes perguntas:
\begin{itemize}
\item ``What is the structure of the model system?'' 
\item ``For example, what kind of low-level entities (e.g., individuals, habitat units) are described in the model?'' 
\item ``How are they described?'' 
\item ``What hierarchical levels exist?'' 
\item ``How are the abiotic and biotic environments described?'' 
\item  ``What is the temporal and spatial resolution and extent of the model system?''
\end{itemize}

O modelo, a princípio, será composto por quatro níveis hierárquicos:
\begin{description}
    \item [Cientistas] agente representando as pessoas que realizam produção científica;
    \item [Organizações Cientificas] agrega agentes de vários tipos, representando as entidades que promovem e organizam a produção científica;
    \item [Territórios] agrega os agentes que representam os locais geográficos e onde a sociedade abriga as organizações e os cientistas;
    \item [Consumidores da Ciência Mundial] agrega os agentes que consomem o que é produzido pela ciência em alcance mundial;  
\end{description}

Cada um dos agentes nesses níveis é apresentado e detalhado a seguir.

\subsection{Nível dos Cientistas}

\subsubsection{Agente Cientista}
Apenas os agentes do tipo cientista compõem esse nível. 
Um agente cientista é um indivíduo, cientista, pesquisador etc, caracterizado por:
    \begin{description}
        \item [número de identificação] Ter uma identificação única na vida real, definida pelo ORCID ou similar. 
        \item [idade] ter uma idade em anos, que o caracteriza em um estágio da vida.
        \item [sexo] Ter um sexo, que em uma estrutura dominante machista, pode determinar certas limitações à atividade da pesquisadora;
        \item [Comportamentos individuais] Além de se adaptar ao ambiente de produção científica em função de seus atributos, um cientista tem dois atributos de comportamento que parecem impactar a produção mundial: (i) Metodologias que emprega e (ii) Linguagens que articula (português, inglês etc);
        \item atuação em uma rede social global, formada por vários outros cientistas, exercendo diferentes papéis, numa estrutura acadêmica globalizada, evidenciada pelas colaborações acadêmicas \cite{sampaio_as_2015}; 
        \item [Organização científica] Organização de duração indeterminada à qual o pesquisador é vinculado, ou atua. Um cientista pode ser vinculado a uma ou ais organizações científicas. Nunca trabalha de forma isolada. Organizações científicas são agentes com alguma autonomia comportamental, por serem estruturas coletivas sujeitas a uma série de riscos. São detalhadas no nível das Organizações Científicas. As organizações científicas se organizam na forma de um rede que evolui lentamente a partir das redes interpessoais feitas pelos cientistas que atuam nas organizações; 
        
        \item [áreas de conhecimento] Um cientista atua em um conjunto de áreas de conhecimento, usualmente uma, mas as vezes mais que uma. As áreas do conhecimento são agentes com alguma autonomia comportamental, por serem estruturas coletivas sujeitas a uma série de riscos, tanto humanos quanto naturais. As áreas do conhecimento se organizam na forma de uma árvore, ou taxonomia;
        
        \item [Projetos] Organizações projetizadas, com duração determinada, nas quais atua um pesquisador, com uma dedicação maior ou menor. Projetos são agentes com alguma autonomia comportamental, por serem estruturas coletivas sujeitas a uma série de riscos. Os cientistas em um projeto se organizam na forma de um rede temporária; 
    \end{description}

O ciclo de vida de um pesquisador tem uma média de 25-50 anos, com uma certa normalidade.

\subsection{Nível das Organizações Científicas}

É no nível das organizações científicas que ocorre a produção do conhecimento.

O modelo contém quatro diferentes agentes que representam as organizações científicas:
\begin{description}
    \item [Organização de Pesquisa], agente que representa uma universidade ou entidade de filiação de natureza perene, que é presente em um ou mais pontos geográficos, e que agrega um conjunto com diversos cientistas. Uma organização de pesquisa pode agregar em uma estrutura estritamente hierárquica outras organizações de pesquisa. Esse tipo de agente é detalhado a seguir.;
    \item [Projeto de Pesquisa], agente que representa um projeto, uma organização temporária, isso é, de natureza transitória, e que agrega uma hierarquia com diversos cientistas. Esse tipo de agente é detalhado a seguir.;
    \item [Revista Científica], agente que representa um veículo de publicação, de natureza perene, que não está vinculada a um ponto geográfico específico, mas atuam publicando artigos em uma ou mais áreas de pesquisa. Esse tipo de agente é detalhado a seguir.;
    \item [Área de Pesquisa], agente que representa a uma classe de fenômenos naturais ou humanos que possuem coesão entre si. As áreas podem agregar de forma estritamente hierárquica outras áreas de pesquisa. Esse tipo de agente é detalhado a seguir.
\end{description}

\subsubsection{Organizações de pesquisa}
As organizações de pesquisa representam uma universidade ou entidade que tem natureza perene, e que é presente em um ou mais territórios geográficos. A duração de uma Organização de Pesquisa excede em muito o ciclo de vida de um pesquisador: 100 a 200 anos.

Uma organizações de pesquisa agrega um conjunto com diversos cientistas, que são assim filiados à organização. 

Uma organização de pesquisa, como uma universidade, pode agregar em uma estrutura estritamente hierárquica outras organizações de pesquisa, como institutos, faculdades e centros de pesquisa.

Uma organização pode ocupar um único campus, presente em um único território geográfico. Se uma organização científica agrega outras, ela pode ocupar vários campi, onde estariam presentes as suas subunidades.

\subsubsection{Projeto de Pesquisa}
Os projetos de pesquisa são organizações científicas temporária, isso é, de natureza transitória. Tem duração média de um ano, com uma distribuição de frequência de duração não normal.

Os projetos de pesquisa são abrigados nas organizações de pesquisa às quais estão vinculados os vários pesquisadores agregados ao projeto. 

Os pesquisadores agregam-se ao projeto obedecendo a uma hierarquia estrita, onde há o coordenador do projeto, no topo da hierarquia, e ao qual se subordinam outros cientistas, usualmente de menor experiência.
A hierarquia possui vários níveis de profundidade, três ou quatro, e usualmente os cientistas mais maduros e de maior idade ocupam as posições superiores.

\subsubsection{Revista Científica}
Uma Revista Científica representa um veículo de publicação, de natureza perene, que não está vinculada a um ponto geográfico específico, e que  atua selecionando e publicando artigos a ela submetidos, e que estão alinhados às áreas de pesquisa e políticas científicas a elas alinhadas.

Uma revista científica possui um cientista que atua como editor-chefe, e que coordena as atividades de uma banca de revisores, que também são cientistas que atuam nas áreas de conhecimento de interesse da revista.

Embora perene, uma revista científica tem um tempo de vida usualmente menor que uma organização de pesquisa, obedecendo a um ciclo de vida similar a de um pesquisador: 25 a 50 anos de existência.

\subsubsection{Área de Pesquisa}
Uma Área de Pesquisa é um agente que representa a uma classe de fenômenos naturais ou humanos que possuem coesão entre si, e que são classificados pelos cientistas como constituindo uma área. 

As dinâmicas dos fenômenos naturais ou humanos pertinentes a uma área, promovidas pela introdução de tecnologias, podem mudar a conformação de pesquisa de uma área, quando ocorrem as mudanças de paradigma nas áreas.

As áreas podem agregar de forma estritamente hierárquica outras áreas de pesquisa, que representam refinamentos a área mais elevada na hierarquia.

Uma área de pesquisa não se vincula a um território geográfico. É um conjunto de alcance mundial.

São características próprias e transitórias das áreas do conhecimento:
    \begin{itemize}
        \item Inserção social (quanto  sociedade demanda da presença de profissionais que conhecem os fenômenos da área);
        \item Metodologia dominante, com ênfase no uso de dados empíricos, ênfase teórica ou prática, ênfase em processos sociais versus processos mecânicos, ou exatos; ênfase qualitativa ou quantitativa;
        \item Estabilidade (quantidade de anos decorridos desde que essa área passou a ser de conhecimento da humanidade). Quão estáveis são os métodos de pesquisa empregados na área;
    \end{itemize}

\subsection{Nível dos Territórios}

Um Território Geográfico é um agente que representa de forma agregada todos os fenômenos que ocorrem em um espaço delimitado do planeta terra.
Pode-se associar de forma prática o conceito de territórios ao de cidades, pois se considerarmos o elevado nível de introdução de tecnologias subjacente à existência cidades, é nas cidades de porte médio, grande e de metrópole que se torna essencial a presença de organizações científicas.

OS territórios tem duração perene, isso é, não tem data para acabar, mas tem taxas de crescimento que usualmente estão vinculadas às propriedades geográficas, tais como clima, aspectos econômicos, demográficos, políticos.

\subsection{Ambiente mundial de Consumo da Ciência}
    
O ambiente mundial de consumo científico agrega os agentes leitores, consumidores da produção científica mundial, que podemos classificar em grupos como:
    \begin{itemize}
            \item Outros cientistas;
            \item Empresas de tecnologia, engenharia, serviços (consultoria), indústria criativa (games) etc;
            \item Estado e Governos, nos vários segmentos de políticas públicas.
            \item cidadão comum;
        \end{itemize}

No modelo CE20211, além dos cientistas já modelados na forma de agente, serão considerados os agentes Empresa e Governo.

\subsubsection{Empresa}
As Empresas são agentes que demandam o conhecimento científico para melhorar suas atividades de desenvolvimento tecnológico, engenharia, serviços (consultoria), comércio etc
As empresas são vinculadas aos territórios, e demandam conhecimento vinculado ao território onde vivem.

\subsubsection{Estado e Governo}
O agente Estado/Governo 


\paragraph{Exercício} Exercício 3.3: 2 pontos:
Contribua com 20 a 50 palavras formulando novas propostas de variáveis de estado que podem estar presentes nos níveis de modelagem de
cientistas, territórios, meta-populações e ambiente mundial de produção e consumo do conhecimento científico.
Faça referência a pelo menos uma fonte de informação na Internet. Use uma URL ou referência no Zotero.

\subsection{Proposta do aluno Leonardo Rodrigues de Souza para variáveis de estado}

Seria interessante registrar o valor pago aos pesquisadores de cada país, para poder realçar a dificuldade enfrentada por cientistas em diferentes países. Assim, entender porque países de primeiro mundo possuem uma alta taxa de produção científica. \ref{https://www.bbc.com/portuguese/brasil-57289688}{Fonte}.

\subsection{Proposta do aluno Guilherme Mendel para variáveis de estado}
Seria interessante também registrar a localização física das instituições científicas (país, por exemplo), para facilitar a \href{https://pt.wikipedia.org/wiki/An\%C3\%A1lise}{análise} dos resultados por território.

\subsection{Proposta do aluno Lucas Vinicius Magalhães Pinheiro para variáveis de estado}
Para analisar o ponto mencionado anteriormente, é útil uma variável que mencione o investimento/bolsa mensal recebido pela cientista durante a pesquisa. Assim é possível checar a dificuldade da pesquisa e a resistência governamental e social da pesquisa. \href{https://www.ufrgs.br/jornal/cortes-no-investimento-em-ciencia-prejudicam-resposta-a-pandemia-no-brasil/}{Fonte}

\subsection{Proposta da aluna Jaqueline Gutierri Coelho para variáveis de estado}
Também seria interessante registrar aspectos do cotidiano dos cientistas, tais como rotina e problemas de estresse no dia a dia, pois estes influenciam diretamente nas \href{https://journals.openedition.org/eces/507}{pesquisas} realizadas pelos cientistas. 

\subsection{Proposta da aluna Kailany Ketulhe Gomes Rocha para variáveis de estado}

Seria de grande proveito o registro do mapeamento das revistas e eventos em que os cientistas publicam seus estudos, entendendo esse \href{https://www.scielo.br/j/pusp/a/6MzXfkS6XScLfrbBj9WDLd/?lang=pt}{mapeamento} podemos compreender melhor o trabalho realizado pelos cientistas.

\subsection{Proposta da aluna Caroline Ferreira Pinto para variáveis de estado}
Creio que seria interessante registrar o conjunto das áreas de conhecimento dos cientistas, visto que cada um deles possui um comportamento, uma história própria de vida, uma forma de pesquisa e também uma familiaridade com determinado tema. 
\href{https://comportese.com/2014/05/01/o-comportamento-do-cientista}{Fonte}

\subsection{Proposta do aluno Thales Lima Menezes para variáveis de estado}
Visando complementar o propósito do modelo sugerido acima, proponho avaliar a quantidade de trabalhos publicados por ano de cada cientista do modelo; dessa forma podemos comparar a experiência do pesquisador em escrever trabalhos na sua área. \href{https://www.upf.br/biblioteca/noticia/brasil-lidera-ranking-de-paises-com-maior-quantidade-de-publicacoes-cientificas-em-acesso-aberto}{Fonte}

\subsection{Proposta do aluno Rafael Gonçalves de Paulo para variáveis de estado}
Apesar de não necessariamente onerar o cientista diretamente, o custo de exercer a ciência (materiais, energia, licenças, compra e manutenção de ferramentas) limita a produção científica. Proponho adicionar o custo médio de produção de experimentos e pesquisas como variável de área do conhecimento. \href{https://www.nature.com/articles/495426a}{Fonte}. \href{https://intpolicydigest.org/the-cost-of-science/}{Fonte}.

\subsection{Proposta do aluno João  Pedro Sadéri da  Silva para variáveis de estado}
Seria interessante obter a relação de como os pesquisadores em um determinado estudo se conheceram e quais os conhecimentos que esses trouxeram para a \href{https://pt.wikipedia.org/wiki/Pesquisa}{pesquisa}.


\subsection{Proposta do aluno Otávio Souza de Oliveira para variáveis de estado}

Seria interessante registrar quais os principais \href{https://blog.even3.com.br/pesquisas-cientificas-no-brasil/}{fatores} que afetam a baixa produção científica no Brasil, e como aumentar essa produtividade em comparação a países do hemisfério norte.

\subsection{Proposta do aluno Carlos Eduardo de Oliveira Ribeiro para variáveis de estado}
Seria interessante analisar quais são os desafios e \href{https://www.nber.org/system/files/working_papers/w26752/w26752.pdf}{incentivos} que o pesquisador tem para realizar uma produção cientifica. Alguns países sofrem muito com a falta de recursos e cortes nos orçamentos que impedem o cientista de tomar a iniciativa para produzir algo.

\subsection{Proposta da aluna Giovana Pinho Garcia para variáveis de estado}
Também seria interessante registrar a renda dos pesquisadores. Já que esse pode ser um grande fator limitador, principalmente, para países que não possuem financiamento e incentivo a pesquisadores e a ciência. \href{https://www.ufg.br/n/129177-baixo-investimento-em-ciencia-e-tecnologia-eleva-a-desigualdade-social}{Fonte}

\subsection{Proposta do aluno João Francisco Gomes Targino para variáveis de estado}
Seria bom para a produtividade científica a distribuição igualitária de bolsas em todas as áreas, pois hoje em dia as condições de mulheres conseguirem bolsas de pesquisa estão muito ligadas a Ciências da vida principalmente pelo esteriótipo de cuidadora e etc, enquanto nas outras áreas a chance fica cada vez menor.
\href{https://sites.usp.br/psicousp/desequilibrio-de-genero-afeta-mulheres-cientistas-brasil/}{Fonte}

\subsection{Proposta do aluno Álvaro Veloso Cavalcanti Luz para variáveis de estado}
Seria produtivo para o projeto que fosse traçado um perfil dos pesquisadores e de suas dificuldades diárias quando trabalhando com a produção científica, tais como dificuldades financeiras e limitações de espaços e materiais.\href{https://www.labnetwork.com.br/especiais/o-desafio-de-fazer-pesquisa-cientifica-no-brasil/}{Fonte}

\subsection{Proposta do aluno Vitor Vasconcelos de Oliveira para variáveis de estado}
Uma boa proposta de variáveis de estado seria a distribuição da educação pelo mundo, ou seja os países e seus respectivos acessos/distribuição da educação em seus territórios.
\href{https://agenciadenoticias.ibge.gov.br/agencia-noticias/2012-agencia-de-noticias/noticias/22842-acesso-a-educacao-ainda-e-desigual}{Fonte}

\subsection{Proposta do aluno João Gabriel Ferreira Saraiva para variáveis de estado}
A aproximação da população das universidades, para evidenciar e divulgar melhor os projetos. Dessa forma, o apoio da sociedade contribuirá para que mais investimentos sejam destinados as essas áreas.
\href{Dessa forma, o apoio da sociedade contribuirá para que mais investimentos sejam destinados as essas áreas.}{Fonte}

\subsection{Proposta do aluno Felipe Gomes Paradas para variáveis de estado}
Uma variável importante de se analisar que muito afeta a produção científica é o \href{https://www.gov.br/economia/pt-br/assuntos/planejamento-e-orcamento/orcamento}{orçamento} da universidade em que o pesquisador faz parte, caso esse seja de uma universidade pública.


\subsection{Proposta do aluno João Victor Pinheiro de Souza para variáveis de estado}

Com programas e iniciativas que ajuda a identificar notícias falsa e teorias da conspiração e informa a população sobre elas, ajudando a desmitificar a ciência seria uma boa proposta de variáveis de estado.
\href{https://revistapesquisa.fapesp.br/resistencia-a-ciencia/}{Fonte}

\subsection{Proposta do aluno Matheus Arruda Aguiar para variáveis de estado}
Uma proposta interessante a ser considerada é de que o ensino médio não seja apenas profissionalizante e/ou ideológico, mas de conhecimento teórico, crítico e prático. A instrução politécnica deve ser transmitida com os fundamentos científicos gerais de todos os processos de produção para que cada aluno atue em qualquer área tendo esse embasamento como suporte. \href{https://monografias.brasilescola.uol.com.br/pedagogia/influencia-globalizacao-praticas-educativas-e-reformulacao-conteudos.htm}{Fonte-Contemplação}


\subsection{Proposta do aluno Heitor de Lima Belém para variáveis de estado}
É interessante avaliar os desafios enfrentados no processo de produção científica pela \href{https://www.scielo.br/j/ep/a/WgdZnSMrX49LLTJMffmsqNK/?lang=pt}{perspectiva dos pesquisadores}, para entender quais são os fatores mais relevantes ao iniciar uma pesquisa.

\subsection{Proposta do aluno Luthiery Costa Cavalcante para variáveis de estado}
Uma variável que considero importante para a análise de produção científica, no que diz respeito às universidades, é a separação em universidade pública ou privada, visto que as públicas tendem a concentrar o foco e investimento em pesquisa, em comparação com as privadas, particularmente no caso do Brasil. Tal tendência pode ser observada em: \href{https://ruf.folha.uol.com.br/2019/ranking-de-universidades/pesquisa/}.

\subsection{Proposta do aluno João Gabriel Lima Neves para variáveis de estado}
Uma variável que seria interessante para analisar como a burocracia afeta o processo de produção cientifica seria a porcentagem de tempo de trabalho que um profissional gasta para resolver problemas de natureza burocrática. \href{https://humanas.blog.scielo.org/blog/2020/04/07/como-aliviar-a-burocracia-na-pesquisa-cientifica/#.YTFdi3VKhhE}{Fonte}

\subsection{Proposta do aluno José Fortes Neto para variáveis de estado}

É útil saber a opinião da população acerca do trabalho científico desenvolvido no Brasil e em paralelo saber a opinião dos cientistas acerca do apoio recebido da população e estudar como esse apoio afeta a evolução das pesquisas. \href{https://www.periodicosdeminas.ufmg.br/entenda-os-atuais-desafios-das-pesquisas-cientificas/}{Fonte}

\subsection{Proposta do aluno Adelson Jhonata Silva de Sousa para variáveis de estado}
Uma variável que acho válido analisar e estudar seria qual o valor social e econômico do cientista em seu país e o porque ocorre a fuga dos cientistas para outros países. O que os outros países oferecem que o seu país de origem não oferece. \href{https://www.bbc.com/portuguese/brasil-51110626}{Fonte}

\subsection{Proposta da aluna Alice da Silva de Lima para variáveis de estado}
Uma variável de estado interessante a ser considerada é a respeito das instituições de filiação dos autores. Sabe-se que a grande maioria destas são instituições de ensino, mas e as as instituições que configuram outra modalidade? Quais são elas?
É importante analisar quais os outros tipos de instituição que estão contribuindo para o \href{https://www.ipea.gov.br/cts/pt/central-de-conteudo/artigos/artigos/116-a-ciencia-e-a-tecnologia-como-estrategia-de-desenvolvimento}{desenvolvimento científico}.

\subsection{Proposta do aluno Henrique Mendes de Freitas Mariano para variáveis de estado}
Creio que seja importante ressaltar para a análise de produção científica a variável de estado saúde mental, pois nessa pandemia muitas pessoas perderam seus entes queridos e importantes, com isso muitas pessoas depositaram suas esperanças nos cientistas para nosso cotidiano voltar a ser o que era, assim toda pressão para a resolução dos problemas foi jogada para os cientistas, dessa forma afetando a saúde mental dos cientistas. \href{https://bioinfo.imd.ufrn.br/transcricaoemdia/opiniao/eg-010}{Fonte}.

\subsection{Proposta do aluno Gabriel Nazareno Halabi para variáveis de estado}

Uma variável relevante a ser analisada pelo modelo seria a valores de investimento regional em pesquisas e produções científicas de ambas instituições públicas e privadas a fim de identificar os fatores que influenciam as concentrações de produções nacionais. \href{https://jornal.usp.br/universidade/levantamento-mostra-quem-financia-a-pesquisa-no-brasil-e-na-usp/}{Fonte}

\subsection{Proposta do aluno Paulo Mauricio Costa Lopes}
Uma variável de estado que poderia ser utilizada seria numero de produções acadêmicas feitas nos primeiros anos do pesquisador (durante a formação acadêmica ou nos primeiros anos apos se formar) e a área de foco do pesquisador (se ela mudou ou se manteve a mesma).\href{https://periodicoscientificos.ufmt.br/ojs/index.php/repad/article/view/8571/6115}{Texto exemplo}.


\section{Escalas da simulação}

As seguintes escalas serão desenvolvidas:
No nível de indivíduo:
\begin{itemize}
    \item Ciclo de vida do cientista, de 50 anos, com passos de evolução ocorrendo a cada semana; 
    \item Escolha do território, durante o seu ciclo de vida, entre 20 e 30 anos, ocorre o pico de mudança de território, com uma ou duas mudanças de local, durante a vida, e em média, com um desvio padrão. A cada ano há uma chance de haver mudança em território;
    \item Escolha e (ou) migração de área do conhecimento, no início da carreira o cientista escolhe uma área, e com uma frequência pequena, muda  de área ainda nos primeiros cinco anos da escolha. Depois é mais difícil mudar.
    \item Desistência da carreira. Ocorre por razões de limitação do número de pessoas que ao amadurecerem poderão continuar na pŕodução da ciência. A cada ano, u conjunto de pessoas desiste da carreira de cientista, por falta de oportunidades, emprego. Ciclo anual de saída do sistema;
    \item Reintegração. Depois de um tempo, com a vida estabilizada, volta ao sistema de pós-graduação, e eventualmente ingressa como professor universitário
\end{itemize}

No nível da Organização científica existem algumas opções:
\begin{enumerate}
    \item Existe um conjunto fixo de organizações científicas, ocupando um conjunto fixo de locais geográficos, gerados no início da simulação, ou.
    \item Cada década, em média, são criadas x\% (descobrir esse valor empírico) novas organizações científicas;
    \item Cada década, em média, cada organização científica coloca x (descobrir esse valor empírico) novos campus em um território geográfico
\end{enumerate}

Cada Organização científica está vinculada a um ou mais territórios geográficos, e cada território geográfico é similar a uma cidade, e se relaciona com outros territórios geográficos (outras cidades) por meio de vias de troca. Estruturam-se os territórios na forma de um grafo. Com arestas mais ou menos fortes, interligando as cidades conforme estabelecem relações econômicas entre si. Alguns territórios são pequenos, outros são grandes, e usualmente os pequenos se organizam na periferia dos grandes. Os territórios são fixos e predeterminados no começo da simulação, com um tamanho inicial, que pode se modificar ao longo de seu ciclo de vida, que tem duração de milênios, e que portanto está foram do escopo da simulação. 


No nível de áreas do conhecimento, considerar-se-á estáveis as áreas e subáreas, dispostas na forma de uma árvore. 
No começo da simulação existirá um conjunto gerado de áreas e subáreas, conforme um conjunto dos seus atributos. As áreas são comuns ao mundo inteiro.

No nível do ambiente mundial de produção e consumo 

\section{Processos, Escalonamento e Parâmetros para o modelo CE20111}

\subsection{Modelo Integrado de Processos da Simulação}

O mapa conceitual da figura \ref{fig:CE20211:ConceitosCentrais} apresenta o modelo integrado do estado e das relações estabelecidas entre os agentes dos vários tipos.

\begin{figure}
    \centering
    \includegraphics[page=1,angle=90,width=1\textwidth]{aulas/3-Simulacao/CE20211/img/CE20211Conceitos.pdf}
    \caption{Conceitos e atributos centrais da simulação CE20211.}
    \label{fig:CE20211:ConceitosCentrais}
\end{figure}

Os parágrafos a seguir expressam as principais características do modelo, agrupados por temas.

\subsubsection{Sobre Territórios Geográficos} Os territórios são espaços geográficos onde ocorre a produção do conhecimento, e o conhecimento científico produzido em um território geralmente é voltado para a compreensão de problemas que ocorrem nesse território, pois a ciência possui um senso de utilidade, entre outros.
O quadro \ref{tab:ce20211:territorios} apresenta as sentenças centrais sobre a modelagem dos territórios.

\begin{table}
\resizebox{\textwidth}{!}{%
\begin{tabular}{|l|l|l|}
\hline
\textbf{Conceito}           & \textbf{Predicado} & \textbf{Conceito}           \\ \hline
Território Geográfico       & possui atributos   & Culturais                   \\ \hline
Território Geográfico       & possui atributos   & Demográficos                \\ \hline
Território Geográfico       & possui atributos   & Econômicos                  \\ \hline
Território Geográfico       & possui atributos   & Geográficos Físicos         \\ \hline
Território Geográfico       & possui atributos   & Históricos                  \\ \hline
Território Geográfico       & possui atributos   & Políticos                   \\ \hline
Território Geográfico       & é parte de uma     & Rede Mundial de Territórios \\ \hline
Rede Mundial de Territórios & agrega em rede vários & Território Geográfico                         \\ \hline
Rede Mundial de Territórios & representa         & Vínculos Entre territórios  \\ \hline
Vínculos Entre territórios  & podem ser          & Físicos                     \\ \hline
Vínculos Entre territórios  & podem ser          & Informacionais              \\ \hline
Vínculos Entre territórios  & podem ser             & Similaridades Histórico- Culturais- Políticos \\ \hline
Vínculos Entre territórios  & podem ser             & Trocas Comerciais- Industriais                \\ \hline
\end{tabular}%
}
    \caption{Sentenças lógicas sobre Territórios no modelo CE20211.}
    \label{tab:ce20211:territorios}
\end{table}

Com base no quadro \ref{tab:ce20211:territorios}, os agentes do tipo território geográfico possuem uma população (atributo demográfico), que possui comportamentos (cultura), realiza trocas de recursos entre si (economia), uma sequência de fatos importantes, determinantes dos rumos da população e do próprio território (história), uma rede de interesses, onde algumas pessoas assumem papel mais central e outras mais periféricas (política) e que se integram na forma de uma rede (um grafo), que interliga o território e sua população a outros territórios, de forma heterogênea.

A figura \ref{fig:malha} apresenta um exemplo da rede de influência das cidades (territórios) no Brasil, em 2018, produzida pelo IBGE \citep{coordenacao_de_geografia_do_ibge_regioes_2020}.

\begin{figure}
    \framebox{
    \includegraphics[page=1,width=\textwidth,clip=true,trim=1.2cm 2.5cm 2cm 4cm]{aulas/3-Simulacao/CE20211/img/liv101728_p14.pdf}
    }
    \caption{Rede de integração entre os municípios brasileiros, em 2018. Fonte: \cite[p. 14]{coordenacao_de_geografia_do_ibge_regioes_2020}}
    \label{fig:malha}
\end{figure}

A partir da página do \cite{ibge_regioes_2021} é possível se obter um \textit{dataset} contendo dados detalhados de modelagem da rede graficamente apresentada na figura \ref{fig:malha}, de modo a se obter um conjunto de parâmetros de grafo capazes de gerar de forma artificial, uma malha territorial simulada.

\subsection{Escalonamento e Parâmetros}

Os principais parâmetros para o modelo de simulação CE20111 são definidos a partir de respostas quantitativas e algorítmicas a processos, com base na tabela ``Frases de ligação do modelo'' da planilha CE20211.ods.

A sequência de apresentação dos processos segue a lógica de inicialização e escalonamento do modelo.

Em refinamento à sequência de escalonamento do processo, um conjunto de respostas a questões pertinentes a simulação de cada processo, devem ser feitas por grupos de até três alunos, com base no especificado a partir da tarefa 3.4 - elaboração do modelo conceitual.
%(já está aberta) e 3.5 - implementação do algoritmo em Python, a partir do esqueleto da simulação (na semana que vem).
%Na terceira semana, o modelo será integrado e testado.
%Após o modelo integrado e testando serão feitas análises inciais. e uma vez o modelo minimante funcional, cada grupo vai fazer o desenho das experimentações com o modelo implementado.


\begin{description}

\item [Surgem os territórios] 

\item [Surgem as linguagens] 

\item [Surgem as áreas do Conhecimento] 

\item [Surgem as organizações e \textit{campi} nos territórios] 

\item [Cientistas pesquisam em áreas] 

\item [Surgem as revistas científicas]

\item [Cientistas trabalham em organizações] 

\item [Cientistas atuam em área, organização, e  território] 

\item [Surgem os Projetos] 
\item [Cientistas se engajam em projetos] 
\item [Projetos geram Produtos Científicos] 
\item[Cientistas empregam linguagens] 

\item [Cientistas escrevem artigos] 

\item [Cientistas fazem citações em artigos] 

\item [Revisores avaliam artigos submetidos por cientistas]  

\item [Artigos são publicados por revistas] 
\item [Artigos são referenciados] 
\item [Ocorre o consumo da Ciência] 
\end{description}

Para cada um dos processos listados, a seguir é feita uma lista de questões, buscando respostas quantitativas e algorítmicas que serão base para implementar o algoritmo e calibrar o modelo. Cada conjunto de questões apresenta um código (\#1, \#2 etc). Associado a cada conjunto, são feitas recomendações sobre como pode ser feita a busca por informações que indiquem valores para parâmetros e estrutura para algoritmos, usando-se, onde pertinente, esqueleto de codificação na linguagem Python.   

\subsubsection{Surgem os territórios (Questões \#14)}

Qual a estrutura de um conjunto de territórios representantes das principais cidades do mundo, onde ocorre a produção científica? De que maneira representar os atributos sintéticos dos territórios, bem como os vínculos físicos, informacionais, similaridades históricas, culturais, políticas, comerciais, industriais e de serviços entre os territórios? Qual a estrutura e comportamento de um modelo numérico quantitativo e(ou) algorítmico para simular a situação? 

\paragraph{Respostas às questões \#14} podem ser buscadas junto a bases de dados geográficas, tais como OpenStreetMap e GoogleMaps, de alcance mundial. 

Uma fonte de informação mais fácil, apenas sobre a organização das cidades no Brasil, inclusive a estrutura da malha de afinidades entre as cidades, que nada mais é que um grafo, foi desenvolvida pelo \cite{ibge_regioes_2021}. 
A partir da mensuração de parâmetros chave desse grafo, tais como:
\begin{itemize}
    \item distribuição de frequência dos tamanhos dos vértices (usando como tamanho a população das cidades);
    \item peso das arestas do grafo (a partir da intensidade de afinidades interurbanos)
    \item índices de clusterização dos grafos;
    \item índices de centralidade;
    \item e outras medidas de grafos;
\end{itemize} 
podem ser encontrados valores, que inseridos em bibliotecas de geração de grafos, tais como networkx \citep{hagberg_exploring_2008}, poderão criar uma malha territorial artificial.

\paragraph{Essa questão será respondida pelo grupo} Jorge Fernandes

\subsubsection{Surgem as linguagens (Questões \#8)} Quantas linguagens de escrita e publicação de artigos são usadas no mundo? Qual a distribuição de frequência delas? Qual o vocabulário de cada uma delas? Como sintetizar um vocabulário para um conjunto de linguagens a serem usadas na escrita dos artigos? Qual a frequência de uso de cada uma delas? Qual a estrutura e comportamento de um modelo numérico quantitativo e(ou) algorítmico para simular essa situação?

\paragraph{Respostas às questões \#8} tem como ponto de partida a ligação visceral entre linguagem e território, estudada na área de linguística, usando como ponto de partida o trabalho de \cite{magadan_language_2020}.
De outra forma, é importante ter-se uma estimativa da quantidade e distribuição de frequência de uso de linguagens dispersas ao redor do mundo, que pode ser obtida a partir de \cite{infoplease_languages_2021}. 
Por fim, é necessário escolher pelo menos cinco distintas linguagens que podem ser usadas para escrita de artigos, sendo uma predominante, como ocorre hoje com o inglês, e as outras com frequência reduzida, conforme a curva atualmente usada na publicação mundial.
Para sintetizar artificialmente as linguagens e vocabulário de termos a ser usado na escrita de artigos, pode-se usar \cite{vulgarlang_fantasy_2021}.
Durante a construção sintética dos artigos, é preciso oferecer um serviço gerador de sentenças a ser usado pelos autores.

\paragraph{Essa questão será respondida pelo grupo}

\begin{itemize}
    \item - João Pedro Sadéri da Silva - 170126021
    \item - Matheus Arruda Aguiar - 180127659
\end{itemize}

\subsubsection{Surgem as áreas do Conhecimento (Questões \#15)} 
Quantas áreas de conhecimento existem? Como as áreas do conhecimento se organizam hierarquicamente? Qual a forma da distribuição de frequência dessas áreas nas organizações? Nos territórios? Quais as características metodológicas predominantes nessas áreas? Qual a estrutura e comportamento de um modelo numérico quantitativo e(ou) algorítmico para simular a criação das áreas do conhecimento, sua organização hierárquica e sua vinculação aos territórios? 


\paragraph{Respostas às questões \#15} pode ser obtidas a partir da utilização de uma estrutura de organização do conhecimento já existente, e utilizando-se como referências quantitativas e qualitativas a própria estrutura de áreas do conhecimento, programas de pós-graduação e docentes atuantes em pós-graduação, geradas pela CAPES - Coordenação de Acompanhamento de Pessoal de Ensino Superior, agência do Ministério da Educação. 

Para conhecer a hierarquia das áreas do conhecimento veja \cite{capes_tabela_2021}. A partir dessa página, é possível encontra-se a organização do conhecimento usado na pós-graduação brasileira, que apresenta 3 colégios, subdivididos em nove grandes áreas do conhecimento, subdivididos em 49 áreas do conhecimento. Essa organização tem grandes similaridades com o que é usado no mundo inteiro.

Para conhecer as características metodológicas das pesquisas conduzidas nas 49 (quarenta e nove) áreas do conhecimento organizadas pela CAPES, veja os documentos produzidos pelos especialistas referentes à avaliação de cada uma das  \cite{capes_documentos_2018}. Uma vez que trata-se de uma tarefa muito complexa analisar as metodologias de cada uma das áreas de avaliação, considere pertinente avaliar apenas uma área pertencente a cada uma das 9 (nove) grandes áreas, para adotar como base para o restante das áreas.

Para conhecer a complexidade, impacto social e intensidade de pesquisa (quantidade de pesquisadores) conduzida em cada uma das 49 áreas, tome por base:
\begin{description}
    \item [Para estimar a complexidade] use a planilha da situação da pós-graduação brasileira no ano de 2020, em \cite{capes_plataforma_2021} e compare a quantidade de programas em cada área;
    \item [Para estimar a intensidade de pesquisa] use a planilha da quantidade de docentes nas pós-graduações, por área do conhecimento, em \cite{capes_docentes_2020}. 
    \item [Para estimar o impacto social] use a quantidade de bolsas distribuídas em cada uma das 9 grandes áreas, em \cite{capes_geocapes_2021}.
\end{description}

\paragraph{Essa questão será respondida pelo grupo} 
Trio:
\begin{itemize}
    \item João Francisco Gomes Targino - 18/0102991
    \item João Gabriel Ferreira Saraiva - 18/0103016
    \item João Victor Pinheiro de Souza - 18/0103407
\end{itemize}

\subsubsection{Surgem as organizações e \textit{campi} nos territórios (Questões \#11)}
Como as organizações e seus campi são distribuídos nos territórios? Quantos campi cada organização possui? Qual o formato da curva? Distribuição de frequência? Como as características dos territórios influenciam a existência ou não de campi? Qual a estrutura e comportamento de um modelo numérico quantitativo e(ou) algorítmico para simular a situação de criação de organizações, campi e alocação nos territórios?

\paragraph{Respostas às questões \#11}
podem ser obtidas de forma aproximada a partir da comparação entre a população e riqueza dos territórios, e a presença de organizações de ensino que possuem programas de pesquisa neles. Cabe, para obter-se uma estimativa da distribuição das organizações nos territórios, traçar uma correlação entre as cidades brasileiras e  presença dessas organizações com programas de pós-graduação nos municípios brasileiros no ano de 2019, como disponível em \cite{capes_programas_2021}. 

Para identificar a distribuição de organizações em campi, pode-se utilizar o mesmo arquivo para identificar, para cada organização de ensino nas tabelas em  \cite{capes_programas_2021}, a quantidade de municípios em que elas estão presentes, bem como a afinidade por proximidade entre os municípios.

\paragraph{Análise de gráficos relacionados ao tema}
Para o tema "Como surgem as organizações nos territórios", nós procuramos primeiro buscar em quais territórios a produção científica é mais frequente, uma vez que produção científica é indicador de organização científica ativa; depois procuramos quais tipos de organizações produzem mais, e por último buscamos analisar o ritmo de produção com o PIB per capita regional. As fontes para coleta de dados foram escolhidas com base na recomendação inicial do professor e com base numa busca no recurso \textit{https://dados.gov.br/dataset}, em procura de dados relativos aos municípios que dessem um \textit{insight} em porque os campi são distribuídos da forma que são.

A ideia com o primeiro gráfico, produção científica por território, era de termos uma visualização dos territórios nacionais que compunham a maior parte da produção nacional de pesquisa. Nossa suposição era de que os maiores centros urbanos fossem responsáveis pelas maiores quantidades de pesquisa (principalmente por serem as maiores concentrações de pessoas). Como esperado foi encontrado uma correlação entre centros urbanos e maior produção, então pudemos chegar à conclusão de que lugares com maior concentração populacional instigam a criação de centros de pesquisa. 

Em seguida, no gráfico 2, produção científica por tipo de instituição, vemos uma predominância de instituições federais na área de pesquisa brasileira, seguidas por estaduais, privadas e por ultimo municipais. Então é possível concluir que a iniciativa pública (mais de 75\% das produções disponíveis no \textit{dataset}) é agente indispensável para a pesquisa nacional. Isso somado ao fato concluído no primeiro gráfico mostra a necessidade de criação de centros de ensino públicos em lugares com concentração populacional.

E no gráfico 3, PIB per capita por produção científica, nossa expectativa era de encontrar uma relação demonstrando que em lugares com maior PIB, a produção seria mais intensificada. Entretanto o resultado encontrado foi diferente. O gráfico que encontramos mostra uma distribuição relativamente uniforme de produção nos níveis de produção baixa, e nos pontos de produção alta não se encontra relação disso com PIB (os dois pontos com maiores produções estão na média da escala de PIB utilizada). Então não foi possível encontrar correlação entre estes dois valores. Aparenta ocorrer uma correlação positiva fraca entre PIB per capita e produção científica, mas não temos com o \textit{dataset} utilizado dados o suficiente para ter confiança nessa intuição.

\paragraph{Essa questão será respondida pelo grupo} Trio:
\begin{itemize}
    \item Lucas Vinicius ~ 17/0061001
    \item Rafael Gonçalves ~ 17/0043959
    \item Leonardo Rodrigues ~ 17/0060543
\end{itemize}

\subsubsection{Surgem os cientistas, que pesquisam em áreas (Questões \#3)} 
Como surgem os cientistas? Como vem à existência? Que idade, sexo, atitudes comportamentais e outros fatores relevantes eles exibem? Como pode ser criado um modelo numérico quantitativo que indique um número que sumarize os quatro fatores, determinantes da chance de um cientista se interessar por pesquisar numa área de pesquisa específica, em função dos atributos da área de pesquisa, como sua inserção no mercado, relevância social, empregabilidade, maturidade etc? Qual a estrutura e comportamento de um modelo numérico quantitativo e(ou) algorítmico para simular essa situação?

\paragraph{Respostas às questões \#3} pode ser obtidas no arquivo de docentes participantes dos programas de pós-graduação no Brasil, no ano de 2019, como disponível em \cite{capes_docentes_2020}. No ano de 2019 existiam 107.113 docentes vinculados a programas de pós-graduação \textit{Strictu senso} (mestrado e doutorado) no Brasil. As planilhas apresentam entre outros dados, ano de nascimento, área e ano de formação no doutorado e(ou) mestrado, e área de atuação dos mesmos.
Crie, a partir desse dados, um estimador de atitudes (chance de escolher uma área, idade em que atinge o ápice da carreira científica) etc.

\paragraph{Essa questão será respondida pelo grupo} 
\begin{itemize}
    \item Caroline - 16/0067766
    \item Kailany - 17/0147720
\end{itemize}

\subsubsection{Surgem as revistas científicas (Questões \#9)}
Quantas revistas científicas existem? Quantas existem por área do conhecimento? Quantas revistas focam nos problemas existentes em cada território? são distribuições normais,  exponenciais, logarítmicas? Qual a estrutura e comportamento de um modelo numérico quantitativo e(ou) algorítmico para simular essa situação?

\paragraph{Respostas às questões \#9} 
podem ser obtidas por aproximação através da consulta à tabela de classificação de revistas científicas por área de avaliação da CAPES, disponível em \cite{capes_qualis_2016}. Na tabela de qualificação de revistas no ano de 2016 (arquivo 
\texttt{classificações\_publicadas\_todas\_as\_areas\_avaliacao1522078273541.xls}), que contém 131.275 registros, é possível identificar aspectos como:
\begin{itemize}
    \item a quantidade aproximada de revistas científicas em língua inglesa, bem como as revistas em língua portuguesa;  
    \item a qualidade, impacto ou pontuação aproximada das revistas, em cada área do conhecimento (use a maior pontuação) como indicador de qualidade ou atratividade da revista;
    \item as revistas mais importantes para cada área do conhecimento;
    \item uma proporção entre revistas do território (em língua portuguesa) e revistas que tem olhar para fora do território brasileiro (em língua inglesa)
\end{itemize}
Com base nesses números, deve-se criar um serviço de revistas, conforme a probabilidade das mesmas publicarem trabalhos em uma determinada área (ou em duas ou mais áreas).

\paragraph{Essa questão será respondida pelo grupo} 
Dupla:
\begin{itemize}
    \item Caroline - 16/0067766
    \item Kailany - 17/0147720
\end{itemize}

\subsubsection{Cientistas trabalham em organizações (Questões \#12)}
Como os cientistas trabalham por organização científica, no mundo? Como assumem o papel de editor-chefe e revisor numa revista e como isso impacta sua presença na área? Como a distribuição de frequência ocorre? Qual a estrutura e comportamento de um modelo numérico quantitativo e(ou) algorítmico para simular a situação de emprego de cientistas por organizações?

\paragraph{Respostas às questões \#12} pode ser obtidas de forma aproximada no arquivo de docentes participantes dos programas de pós-graduação no Brasil, no ano de 2019, como disponível em \cite{capes_docentes_2020}. No ano de 2019 existiam 107.113 docentes vinculados a programas de pós-graduação \textit{Strictu senso} (mestrado e doutorado) no Brasil. As planilhas apresentam entre outros dados, a organização de ensino e o município eo qual estão filiados.
Crie, a partir desse dados, um estimador de emprego de cientista em organização e campus (chance de trabalhar em uma determinada organização em um determinado campus).
Adicionalmente, a chance de um pesquisador vir a ser editor-chefe ou revisor de uma revista está fortemente correlacionada ao nível de avaliação da organização de pesquisa na qual ele trabalha. Nesse caso, perceba que em \cite{capes_programas_2021} é indicado o nível de avaliação de cada programa, em nota que varia de 3 a 7. Use a nota do programa ao qual o docente está vinculado como sendo um valor que aumenta a chance inicial dele ser convidado a participar como editor de uma revista.

\paragraph{Essa questão será respondida pelo grupo}
Ausente

\subsubsection{Cientistas atuam numa área, numa organização, num território (Questões \#2)}
Como um cientista amadurece e escolhe ou tem oportunidade de ingressar numa área de pesquisa para trabalhar, em uma organização situada em um determinado território, em função de atributos cruzados, pessoais, organizacionais, territoriais e da área do conhecimento atuante? Qual a estrutura e comportamento de um modelo numérico quantitativo e(ou) algorítmico para simular essa situação?

\paragraph{Respostas às questões \#2} podem ser obtidas de forma aproximada a partir do arquivo de docentes participantes dos programas de pós-graduação no Brasil, no ano de 2019, como disponível em \cite{capes_docentes_2020}. No ano de 2019 existiam 107.113 docentes vinculados a programas de pós-graduação \textit{Strictu senso} (mestrado e doutorado) no Brasil. As planilhas apresentam entre outros dados, a formação no cientista doutorado e(ou) mestrado, a atual área de atuação dos mesmos, bem como o país e a sigla da IES (Instituição de Ensino Superior) onde ele foi titulado..
Crie, a partir desse dados, um estimador de atitudes (chance de mudar de área, organização e território) etc.

\paragraph{Essa questão será respondida pelo grupo}


\subsubsection{Surgem os Projetos (Questões \#16)} Como surgem os projetos para pesquisa científica? Como surgem os projetos por área do conhecimento, e por organização, e por território? De que maneira fatores econômicos, políticos, culturais, territoriais, por área do conhecimento, determinam a alocação de recursos que viabilizam ou inviabilizam o surgimento, duração e continuidade de um projeto?

\paragraph{Respostas às questões \#16} podem ser obtidas de forma aproximada fazendo-se uma equiparação entre programas de pós-graduação e projetos de pesquisa. Cada programa de pós-graduação pode ser visto como sendo um projeto de pesquisa, e a nota que os programas apresentam, bem como a quantidade de outros programas que ocupam o mesmo território, são  indicadores de oportunidade econômica para materialização de projetos.
A partir do arquivo dos programas de pós-graduação no Brasil, no ano de 2019, como disponível em \cite{capes_programas_2021} crie um estimador do duração e continuidade de um projeto.

\paragraph{Essa questão será respondida pelo grupo}
\begin{itemize}
    \item Alice da Silva de Lima - 18/0112601
    \item Carlos Eduardo de Oliveira Ribeiro - 18/0099094
    \item Giovana Pinho Garcia - 18/0101374
\end{itemize}

\subsubsection{Cientistas se engajam em projetos (Questões \#6)} Como ocorre o engajamento de cientistas em projetos? Em quantos projetos de pesquisa cada cientista se engaja ao longo do tempo? Quantos cientistas se engajam em cada projeto? Qual a duração dese engajamento, qual a distribuição de frequência desse engajamento? Qual a estrutura e comportamento de um modelo numérico quantitativo e(ou) algorítmico para simular essa situação?

\paragraph{Respostas às questões \#6} podem ser obtidas de forma aproximada fazendo-se uma equiparação entre programas de pós-graduação e projetos de pesquisa. Cada programa de pós-graduação pode ser visto como sendo um projeto de pesquisa, e a rede de colaborações de autoria que os membros dos programas de pós-graduação desenvolvem representam, de forma aproximada, o conjunto total de todas as redes de relacionamento presentes em todos os  projetos realizados na organização. Os cientistas com maior centralidade na rede, e que atuam na área ou próximo dela, são os que tem mais chances de serem os coordenadores dos projetos. Os demais cientistas da mesma organização, atuantes na mesma área do projeto, agregam-se à rede do projeto em posições subordinadas ao coordenador.
Com base nas orientações, crie um gerador de engajamento de cientistas em um projeto criado numa organização, campi/território, e área do conhecimento.

\paragraph{Essa questão será respondida pelo grupo} 
Trio:
\begin{itemize}
    \item Henrique - 17/0012280
    \item Luthiery - 17/0040631
    \item Thales - 17/0045919
\end{itemize}


\subsubsection{Projetos geram Produtos Científicos (Questões \#17)} Como os cientistas, engajados em redes nos projetos, em uma ou mais organizações envolvidas, em uma ou mais áreas do conhecimento envolvidas, em um ou mais territórios envolvidos, com limitações maiores ou menores nos recursos alocados, geram os \textit{datasets} e relatórios típicos de um projeto? Os projetos podem fracassar na geração de seus produtos? Qual a estrutura e comportamento de um modelo numérico quantitativo e(ou) algorítmico para simular essa situação? 

\paragraph{Respostas às questões \#17} 
podem ser obtidas a partir de uma fórmula heurística que estime a chance de um projeto ser interrompido, de produzir um dataset, de produzir um relatório, com base nos seguintes princípios:
\begin{itemize}
    \item A chance de um projeto ser interrompido está diretamente ligada ao valor de recursos envolvidos com o projeto, e à qualificação de seu coordenador e equipe.
    \item a chance de um projeto ativo (não interrompido) gerar um dataset está ligado ao valor de recursos envolvidos com o projeto, à qualificação de seu coordenador e equipe, combinados com o viés da metodologia da área de pesquisa envolvida, se mais quantitativo (mais dados) ou mais qualitativo (menos dados);
    \item a chance de um projeto ativo (não interrompido) gerar um relatório está ligada à disponibilidade e riqueza de datasets, ao valor de recursos envolvidos com o projeto, à qualificação de seu coordenador e equipe;
\end{itemize}

Os parâmetros quantitativos de volume e (ou) quantidade de produtos (datasets e relatórios) também deve ser calibrada com base na taxa de produtividade por área de conhecimento, que pode ser estimada com base nos Detalhes da Produção Intelectual Bibliográfica de Programas de Pós-Graduação Stricto Sensu no Brasil [2013 a 2016] \cite{capes_detalhes_2017}.

\paragraph{Essa questão será respondida pelo grupo}
Ausente

\subsubsection{Cientistas empregam linguagens (Questões \#1)}
Como - e de que forma quantitativa - ocorre o emprego de uma ou mais linguagens (no mundo real, português, inglês, francês etc), na escrita de um artigo científico pelos cientistas autores? Como essa escolha ou limitação afeta a chance de sucesso de publicações desse artigo em uma revista cientifica? Como o critério de linguagem de publicação afeta o desempenho de uma revista? Como o uso de uma linguagem afeta a difusão de conhecimento gerado (artigo escrito) naquela linguagem? Qual a estrutura e comportamento de um modelo numérico quantitativo e(ou) algorítmico para simular essa situação?

\paragraph{Respostas às questões \#1} podem ter como ponto de partida conceitual a discussão conduzida por \cite{ramirez-castaneda_disadvantages_2020}, sobre as desvantagens pelas quais passam os pesquisadores que não tem o inglês como língua materna. As referências citadas nesse artigo permitem maior aprofundamento. Em especial, o livro de \cite{gordin_scientific_2015} apresenta em contraponto, um conjunto de argumentos favoráveis à dominação do inglês como língua científica mundial.
Assim sendo, os responsáveis por responder a essa pergunta devem modelar de forma algorítmica e quantitativa a dificuldade que um cientista que não fala inglês tem para escrever um artigo de boa qualidade em inglês.

\paragraph{Essa questão será respondida pelo grupo}
Ausente

\subsubsection{Cientistas escrevem artigos (Questões \#5)} De que forma os cientistas são estimulados, durante os projetos ou após o término dos projetos, a escrever os artigos científicos que darão visibilidade ao conhecimento gerado nos projetos?  Como pode ser sintetizada a escrita de um artigo científico (ainda não publicado), a partir de um ou mais relatórios e dados (\textit{datasets}) de pesquisa? Qual a estrutura e comportamento de um modelo numérico quantitativo e(ou) algorítmico para simular essa situação de escrita de um artigo, sabendo-se que um artigo é escrito baseado na existência de um relatório científico, que possui um conjunto de autores, onde o título, resumo, corpo do texto, e palavras-chave são escritos no vocabulário de uma linguagem específica, e que poderá ou não vir a ser publicado?

\paragraph{Respostas às questões \#5} podem ter seu fundamento prático embasado no trabalho de \cite{gu_developing_2019}, que realizou uma análise de cluster de padrões de publicações de acadêmicos, e gerou uma tipologia de pesquisadores baseada em seis tipos, com as seguintes frequências:
\begin{description}
    \item [singleton] (8\%); 
    \item [small‐team low performer] (16\%)
    \item [small‐team high performer] (17\%);
    \item [big‐team strategist] (22\%)
    \item [free‐style follower] (21\%); e
    \item [life‐time warrior] (17\%)
\end{description}
Com base na leitura desse artigo, e na compreensão dos determinantes gerados pelos atributos dos pesquisadores analisados, construa um estimador da chance de um membro qualquer do projeto, escrever um artigo, em co-autoria com os demais. O seu modelo também deve considerar que são poucas as chances de escrita, se não houverem produtos subjacentes, especialmente relatórios, mas também os datasets no projeto.
Uma vez que o artigo seja escrito por um ou mais autores, ele deve conter um título, resumo e palavras-chave, que serão usados no processo de criação da referência bibliográfica e indexação.

\paragraph{Essa questão será respondida pelo grupo} 
Dupla:
\begin{itemize}
    \item Gabriel Nazareno Halabi - 15/0010290
    \item Paulo Maurício Costa Lopes - 18/0112520
\end{itemize}

\subsubsection{Cientistas fazem citações em artigos (Questões \#7)} Qual a estrutura e comportamento numérico das citações que um artigo faz durante sua escrita? Quantas citações cada artigo faz? Como ocorre a distribuição de frequência nas citações? Como os artigos são escolhidos para serem citados? Como sintetizar esse processo de citação? Qual a estrutura e comportamento de um modelo numérico quantitativo e(ou) algorítmico para simular essa situação?

\paragraph{Respostas às questões \#7} deve ser elaboradas a partir do estudo do padrão de citações presente em bases de dados bibliográficas com citações, geradas na Web of Science ou SCOPUS, como a disponível em  \texttt{experiments / jhcf / PesquisaBibliogr / Computacao Experimental / WoS-20210803 / 8155recs.txt}. A partir desse dataset devem ser geradas as distribuições de frequência de citações por artigo. Já acerca dos artigos citados, podem ser citados apenas artigos já publicados e \textbf{referenciados}, e dessa forma, os primeiros artigos da simulação terão nenhuma, ou poucas citações, obtendo uma situação de \textit{steady-state} após um período de milhares de publicações. Observe que as referências aos artigos são essenciais para que eles possam ser encontrados e lidos.

\paragraph{Essa questão será respondida pelo grupo}

\begin{itemize}
 \item Vitor Vasconcelos de Oliveira - 18/0114778
 \item Álvaro Veloso Cavalcanti Luz - 18/0115391 
 \item Gabriel Cesário Silva Martins - 18/0100912 
\end{itemize}


\subsubsection{Revisores avaliam artigos submetidos por cientistas (Questões \#13)}  Como os cientistas escolhem as revistas para submeter à publicação? Quais fatores determinam suas preferências? Como os revisores das revistas avaliam os artigos submetidos à publicação pelos seus cientistas pares? Quantos revisores existem em cada \textit{scientific journal}? Como ocorre essa distribuição de frequência? Quais as taxas de rejeição ou aceitação de artigos por revisor? Quanto tempo os revisores passar para avaliar um artigo? Que critérios são usados para aceitar ou rejeitar artigos? Quantos dias leva para ser tomada a decisão de aceitação ou rejeição? Qual a taxa de rejeição de artigos por revista e por área? Qual a estrutura e comportamento de um modelo numérico quantitativo e(ou) algorítmico para simular a situação? 

\paragraph{Respostas às questões \#13} devem se fundamentar nas questões teóricas que são subjacentes ao uso da revisão por pares na publicação científica. O trabalho de \cite{kovanis_evaluating_2017} apresenta um excelente discussão sobre a questão.
A partir de uma leitura rápida desse artigo, e de outros, desenvolva um estimador da chance de um artigo ser aprovado ou rejeitado considerando o que é encaminhado pelos três revisores que usualmente avaliam um artigo submetido a uma revista, em um período de tempo de algumas semanas. Observe que a distribuição do artigo pelos revisores é feita por afinidade da área do artigo com a área de atuação do revisor, o que pode envolver também a questão da afinidade do território entre os autores e o revisor. A chance de submissão de um artigo a uma revista aumenta conforme aumentam os índices-H das revistas, inicialmente zerados. Investigue também as taxas médias de rejeição a artigos, estimando faixas de rejeição, que vão crescendo à medida em que os índices-H da revista vão também crescendo, bem como vão crescendo se cresce a fila de artigos aceitos e ainda não publicados.

\paragraph{Essa questão será respondida pelo grupo}
Ausente

\subsubsection{Artigos são publicados por revistas (Questões \#10)} Quantos artigos são publicados por cada revista, a cada número e volume? Quantos volumes por ano? Quais as formas das distribuições de frequência? Qual a estrutura e comportamento de um modelo numérico quantitativo e(ou) algorítmico para simular essa situação?

\paragraph{Respostas às questões \#10} devem se fundamentar na disponibilidade que as revistas tem para publicar artigos. Toda revista tem pelo menos cinco filas de artigos:
\begin{itemize}
    \item Uma fila de artigos submetidos a serem avaliados;
    \item Uma fila de artigos em avaliação;
    \item Uma fila de artigos rejeitados;
    \item Uma fila de artigos aceitos;
    \item uma fila de artigos publicados;
\end{itemize} 
As revistas possuem recursos limitados de avaliadores, que só conseguem avaliar um número máximo de artigos numa taxa de tempo, bem como uma fila de artigos publicados que também só publica x artigos por numero por volume ao longo de um ano.
Os artigos rejeitados podem ser submetidos pelos seus autores a outras revistas. Os aceitos para publicação tem que aguardar na fila o momento de serem publicados, até que se atinja um equilíbrio entre submissão, rejeição/aceitação e publicação. O modelo a ser desenvolvido deve equilibrar essas questões, dentro de parâmetros próximos aos valores reais praticados por revistas. Sugere-se uma investigação em três revistas reais, uma de alto desempenho e impacto, uma de médio e uma de baixo, para estimar os parâmetros da simulação.  

\paragraph{Essa questão será respondida pelo grupo}
Ausente

\subsubsection{Artigos são referenciados (Questões \#4)} Como pode ser sintetizada a referenciação e inexação de um artigo, em uma referência bibliográfica? Qual a estrutura e comportamento de um modelo numérico quantitativo e(ou) algorítmico para simular essa situação sabendo-se que um artigo publicado em uma revista científica é representado por uma referência bibliográfica contendo data de publicação, DOI, ano, volume e número, páginas inicial e final, autor de correspondência, além dos atributos inerentes ao artigo? Será que todos os artigos são indexados de forma equivalente, independentemente da linguagem e território vinculados? Qual a estrutura e comportamento de um modelo numérico quantitativo e(ou) algorítmico para simular essa situação?

\paragraph{Respostas às questões \#4} devem se fundamentar no fato de que o consumo da ciência ocorre por meio do acesso aos registros bibliográficos que são produzidos pelos sistemas de indexação, como Web of Science, Scopus e Google Scholar. Após ler o trabalho de \cite{gusenbauer_google_2019} desenvolva um pelo menos três diferentes agentes indexadores dos artigos, que com uma certa taxa de sucesso, maior ou menor, visita as revistas e gera registros bibliográficos no formato BIBTEX, que serão disponibilizados para busca, recuperação e leitura pelos consumidores da ciência.

\paragraph{Essa questão será respondida pelo grupo}
Ausente

\subsubsection{Ocorre o consumo da ciência (Questões \#18)}
Quem consome os produtos essenciais da ciência, os artigos científicos publicados? De que modo ocorre o consumo de produtos científicos pelos Estados e Governos? Pela Indústria, Comércio e Serviços? Pela Sociedade em geral? Pelos próprios cientistas? Como o consumo varia por área do conhecimento, território, visibilidade da revista? e como esse consumo determina a disponibilidade de recursos para o ciclo de vida do sistema de organizações, revistas, projetos, cientistas e áreas do conhecimento? 

\paragraph{Respostas às questões \#18} dependem da compreensão de que:
\begin{itemize}
    \item O Estado/Governo consome ciência para elaborar melhores políticas públicas, de ordem econômica, social, ambiental, educacional, de saúde, de segurança e defesa etc. O maior ou menor nível de organização política de um Estado/Governo é determinante do maior ou menor nível de consumo de ciência pelo Estado, e consequentemente o maior ou menor aporte de dinheiro para o funcionamento das organizações científicas daquele território;
    \item O setor privado consome ciência para desenvolver tecnologia, a fim de melhorar indústria, comércio e serviços. Se a ciência é produzida para fora do território, há menor interesse privado no consumo da ciência, pois ela não será útil para a solução dos problemas. Nesse caso, não terão interesse em estimular o aporte do governo na ciência.
\end{itemize}
Com base nessa compreensão, você deve fazer a leitura do artigo de \cite{schepelmann_evidence-based_2021}, para desenvolver melhor percepção da questão, e desenvolver uma abordagem na qual cada território atua como agente do Estado/Governo e setor Privado, investindo recursos na promoção de pesquisas junto às organizações de pesquisa presentes no seu território, ou próximas a ele.  

\paragraph{Essa questão será respondida pelo grupo}
\begin{itemize}
    \item José Fortes Neto - 160128331
    \item Adelson Jhonata Silva de Sousa - 180114913
    \item Heitor de Lima Belém - 160123950
\end{itemize}

\subsection{Conceitos no Design da Simulação da Produção Mundial da Ciência}


\subsection{Detalhes da Simulação da Produção Mundial da Ciência}

\subsubsection{Submodelos}

\paragraph{Como surgem os territórios? por Jorge H c Fernandes}

Os parâmetros matemáticos, estatísticos e computacionais necessários ao refinamento deste submodelo da simulação são sumarizados no apêndice \ref{surgeosterritorios} deste documento.

A análise é gerada por meio do notebook R Markdown, 
%a seguir listado.
%\lstinputlisting[language=R]{
em \url{experiments/jhcf/ProducaoDaCiencia/R/Territorios/ComoSurgemOsTerritorios.Rmd}


\paragraph{Surgem as Linguagens Por João Pedro Sadéri da Silva e Matheus Arruda Aguiar}

Os parâmetros matemáticos, estatísticos e computacionais necessários ao refinamento desse submodelo de simulação são sumarizados no apêndice \ref{surgemaslinguagens} deste documento.

A análise é gerada por meio do notebook R Markdown, em \url{experiments/jpsaderi/ProducaoDaCiencia/R/Linguagens/SurgeAsLinguagens.Rmd}

\paragraph{Consumo e Investimentos na Ciência Por Adelson Jhonata Silva de Sousa, Heitor de Lima Belém e José Fortes Neto}

Os parâmetros matemáticos, estatísticos e computacionais necessários ao refinamento desse submodelo de simulação são sumarizados no apêndice \ref{investimentoCiencia} deste documento.

A análise é gerada por meio do notebook R Markdown, em \url{experiments/jhonatasousa58/PesquisaDaCiencia/investimentoCiencia.Rmd}

\paragraph{Como surgem as áreas do conhecimento Por João Gabriel Ferreira Saraiva, João Francisco Gomes Targino e João Victor Pinheiro de Souza}

Os parâmetros matemáticos, estatísticos e computacionais necessários ao refinamento desse submodelo de simulação são sumarizados no apêndice \ref{SurgemAreas} deste documento.

A análise é gerada por meio do notebook R Markdown, em \url{experiments/Joaofsrs/AreasDoConhecimento/R/Areas_Conhecimento.Rmd}


%	\appendix
%    \include{textoslides}

\part{Estudo Empíricos de Submodelos: Respostas d(a/o)s Estudantes\label{part:submodelos:empiricos}}

%\input{EstudoEmpiricoSubmodelos}

\part{Projetos de Simulações: Respostas d(a/o)s Estudantes\label{part:projetos:simulacoes}}

%\input{ProjetoSimulacaoSubmodelos}

\part{Implementação de Simulações e Análises de Dados: Respostas d(a/o)s Estudantes\label{part:implement:simulacoes}}

%\input{ImplementacaoSimulacaoSubmodelos}

\bibliographystyle{plainnat}
\bibliography{RESIC}

\end{document}.
