\subsection{Minhas impressões iniciais sobre a ciência, por Allann Gois Hoffmann}

A ciência é o estudo de todos os fenômenos que podem ser observados, tanto individualmente ou em grupo, ou ferramentas de observação que realizam as analises e que por fim produzem registros sobre o fato. Outro ponto importante é que a ciência esta sempre sendo amparada pela filosofia e possui ramos em diversas áreas de estudo e com diversos métodos de pesquisa, sendo um deles o \gls{MetodoExperimental} \citep{rabelo_o_2011}.

Mesmo com essa definição \citet{schwartzman_ciencia_1984} mostra que a definição de ciência não é algo simples e as vezes é definida de maneira errada. Apesar de como a ciência é compreendida, a sua utilidade para o desenvolvimento tecnológico e econômico é sempre reconhecida por meio de produções cientificas ou outro tipo de publicação, e independente de qual método foi utilizado.