\subsection{Minhas impressões iniciais sobre a ciência, por Alexsander Correa de Oliveira}

A ciência da computação, enquanto ciência, é uma das áreas mais subestimadas, pois computação sempre toma a frente, tanto em fundos, quanto em visibilidade. Isso tudo é apenas uma falta de conhecimento sobre como ela é estruturada, pois é sem dúvida uma das ciências mais diferentes. A aplicação de grandes \emph{break-troughs} é não só facilmente reproduzível (\gls{ReprodutibilidadeCientifica}) mas também é quase imediata, como pode ser observado em algoritmos como busca binária e as buscas em grafos criadas por Dijkstra. 
Além do mais, os problemas computacionais tendem a ser mais facilmente apresentáveis para um publico \emph{pop} \citep{wikipedia_scientific_2021}, como um exemplo podemos apresentar um dos problemas do milênio, $P = NP$, em que existem claras aplicações em jogos como xadrez. Essa facilidade de compreensão superficial, muita das vezes, acaba por gerar também o efeito já citado, o de subestimarem a real força da ciência da computação enquanto meio intelectual. 
