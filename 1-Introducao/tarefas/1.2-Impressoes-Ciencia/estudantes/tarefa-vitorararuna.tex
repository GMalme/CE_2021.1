\subsection{Minhas impressões iniciais sobre a ciência, por Vitor de Oliveira Araujo Araruna}

A ciência é vista como um estudo organizado de quaisquer fenômenos que ocorrem no universo, com o objetivo de explicar ou entender algo. Visto que o conhecimento é um capital cultural, todos nós deveríamos ter acesso a tudo que já é conhecido ou foi descoberto, porém não é assim que funciona na prática. Com isso, percebe-se que o estudo científico tem se tornado cada vez mais praticado por pessoas, empresas, órgãos e todas as outras atividades humanas. Tais estudos podem ser realizados em diversas variáveis \citep{li_variable_2018} , que são características/áreas de interesse observadas nos mais variados tipos de amostras e populações. É importante citar que a observabilidade \citep{arantes_observabilidade_2021} é outro conceito importante no que diz respeito ao estudo científico, uma vez que ela pode ser definida como uma medida que descreve quão bem podem os estados de um sistema serem inferidos a partir do conhecimento de suas saídas externas. Além desses conceitos, a ciência e o estudo científico contém diversas formas de como pode ser produzido e analisado o objeto a ser estudado.
