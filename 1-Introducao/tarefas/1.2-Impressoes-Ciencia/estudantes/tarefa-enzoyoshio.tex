\subsection{Minhas impressões iniciais sobre a ciência, por Enzo Yoshio Niho}

A  ciência  pode  ser  definida  como  o  estudo  metodicamente  organizado  de quaisquer fenômenos que ocorrem no universo, com finalidade de descrever,explicar e prever o comportamento e a estrutura de tais fenômenos, amparado pelo registros (dados) produzidos acerca de tais fenômenos \citep{fernandes_consideracoes_2021}. Com isso, a ciência é muito importante para o desenvolvimento em várias áreas, inclusive a tecnológica, que é a área de nosso interesse. Também é possível observar que a ciência é metódica e muito rigorosa em cima de pesquisas e estudos acerca dos variados temas que ela aborda.

Como a ciência é bastante metódica e busca sempre manter as suas informações coesas, ela precisa de alguns princípios para se manter coesa, como \gls{ComunidadeCientifica} e \gls{ReprodutibilidadeCientifica}. Mesmo com princípios bem estabelecidos, a ciência pode e provavelmente vai receber revoluções e novas definições com o passar do tempo, por isso, por mais que seja importante seguir alguns princípios impostos pela ciência atual, é necessário entender também que ela é passível de mudanças com o tempo, dependendo de suas necessidades. Acredito que a ciência seja muito importante para a população mundial pois a tecnologia ajuda muitas pessoas. A ciência pode inclusive estudar a si mesma, podendo ser vistas como \gls{fenomeno}, no qual o estudo e registro bibliográfico é muito importante para se ter uma ideia da própria ciência.
