\subsection{Minhas impressões iniciais sobre a ciência, por Arthur da Silveira Couto}

A ciência aparenta ser uma tentativa de formalizar a filosofia por meio do método cientifico para obter resultados que sejam mais universalmente aceitos. Os princípios de reproducibilidade científica e comunidade científica são evidencias dessa ideia, tornando necessário que os resultados obtidos por um estudo possam ser compartilhados e reproduzidos por outros pesquisadores, e princípios como empiricismo científico e universalidade científica ditam que os resultados possam ser observados ou aplicados de forma geral, tornando esse conhecimento mais compreensível e reduzindo a chance de desentendimentos. Em contraste, sem esses princípios, filósofos podem criar as próprias ideias em um vácuo intelectual, sem ferramentas que facilitem a compreensão mutua entre filósofos, nem ferramentas que impeçam esses filósofos de chegarem a conclusões mutuamente exclusivas sobre uma mesma questão filosófica. 