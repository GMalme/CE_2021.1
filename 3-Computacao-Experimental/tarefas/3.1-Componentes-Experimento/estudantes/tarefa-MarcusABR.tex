\section{Respostas de Marcus Vinicius Oliveira de Abrantes}

\subsection{A computação é uma atividade científica?}
 A princípio, não. Embora possua semelhanças, como a frequente criação de modelos e coleta de dados, a computação carece de uma abordagem rigorosa para que se possa considerá-la como uma atividade científica. Não está enraizada a nenhuma \gls{Metodologia} específica e portanto, não pode ser definida como uma.
 
 Ainda assim, como dito, a computação se estende em um amplo especto de atividades que podem ou não estar próximas da atividade científica. No campo da ciência da computação, por exemplo, existe uma comunidade que realiza experimentações a fim de descobrir soluções para dilemas computacionais, que por fim norteiam o desenvolvimento de novas tecnologias.

\subsection{A computação é uma ciência experimental?}
Sim, dado que em desde a escala de um programador que realiza testes para encontrar soluções em seu algoritmo, até  a de divulgação da versão beta de um produto feito por uma grande grande corporação, existe a \gls{Experimentacao} como método de definir um caminho a se seguir de forma a encontrar a solução mais eficiente ou evitar eventuais consequências indesejadas. Fundamentalmente, a experimentação está presente nos processos da computação de forma profunda.
    
\subsection{A computação é uma ciência empírica?}
O empirismo está sim presente. É inerente à computação lidar com a coleta de dados para posterior posterior análise. Ocorre desde a avaliação do output de um programa para medir se comportamento de um software coincide com o esperado, até  em tecnologias pesadamente baseadas na coleta, como o \gls{MachineLearning}, em que o dados coletados são usador para fazer o auto refinamento de padrões. Mesmo nas empresas que se valem da computação, tratar de dados coletados digitalmente para encontrar padrões que beneficiem a tomada de decisões é essencial para aumentar sua competitividade.