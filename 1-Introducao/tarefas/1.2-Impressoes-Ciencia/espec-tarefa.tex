\section{Tarefa: Registrar suas impressões iniciais sobre a ciência, citando pelo menos  um item no glossário}

Cada estudante, a partir do que já sabe e leu, deve criar no diretório a seguir uma seção com o seu nome, a fim de apresentar um ou dois parágrafos de sua autoria, apresentando as suas \textbf{Impressões iniciais sobre o que é a Ciência}: 

O diretório onde o arquivo deve ser criado é: 
1-Introducao/tarefas/1.2-Impressoes-Ciencia/estudantes

O nome do arquivo á ser criado deve ter a forma:
tarefa-\githubusername.tex, e deve seguir o modelo da tarefa exemplo do professor, em
1-Introducao/tarefas/1.2-Impressoes-Ciencia/estudantes/tarefa-jhcf.tex

No registro de suas impressões iniciais no texto da tarefa, você precisa:
\begin{itemize}
    \item Usar um ou mais dos itens do glossário, incluído o que você criou na tarefa \ref{tarefa:glossario};
    \item Usar uma ou mais citações a referências bibliográficas usando a tag \verb|\citet{}| ou \verb|\citep{}|. Não use a tag \verb|\url{}|. Note que todas as referências citadas devem estar registradas no arquivo RESIC.bib, gerado a partir da bibliografia no grupo RESIC em \url{https://www.zotero.org/groups/2465026/resic}, ao qual você deve ter acesso, como feito em tarefa anterior.
\end{itemize}

A sua resposta a essa atividade vale até 3\% da pontuação total da disciplina.
Veja como exemplo a resposta do professor, na primeira seção.

\subsection{Minhas impressões iniciais sobre a ciência, por Jorge Henrique Cabral Fernandes}

A ciência da computação \citep{baldwin_three-fold_1994}, uma ciência específica dentre as várias existentes, apresenta-se muito relacionada com a filosofia da ciência \citep{floridi_blackwell_2004}, ou como uma meta-ciência (A filosofia é um campo do pensamento humano que é base para a ciência). Um exemplo que comprova essa versatilidade - ou até mesmo universalidade - da Ciência da Computação é o emprego da computação, sendo como exemplo mais concreto o uso de simulação, como método de pesquisa empírica, isso é pesquisa baseada em coleta e análise de dados \cite{marcolino_a_2014,tedre_experiments_2014}. A versatilidade do computador, decorrente do software, inclusive de simulação, gera uma plataforma empírica - o computador é prodigioso na coleta, cálculo e geração de dado - para o desenho e execução de quase todo tipo de experimento científico em quaisquer dos outros campos do conhecimento. Situações mais específicas podem ser vistas no campo da \gls{EA}.


\subsection{Minhas impressões iniciais sobre a ciência, por Stefano Luppi Sposito}

Desde o início na humanidade, o ser humano possui um desejo insaciável pelo conhecimento. A busca por algo novo sempre possuiu um papel extremamente importante ao longo da evolução humana. Atualmente, definimos \gls{Ciencia} como o tipo de conhecimento que busca verdades para compreender o funcionamento das coisas \citep{gnipper_o_nodate}.

As verdades buscadas podem ser encontradas através da \gls{Experimentacao} de diversas teorias para que evidências sejam conseguidas, a fim de comprovar que teorias e pensamentos estão corretos. É possível afirmar com certeza absoluta que experimentos são uma parte fundamental na parte da ciência que tanto nos ajuda, pois através deles, a própria ciência pode ser executada, levando o homem à lua ou à menor partícula existente. \citep{rabelo_o_2011}

\subsection{Minhas impressões iniciais sobre a ciência, por Tong Zhou}

A ciência \gls{RC}

\newglossaryentry{Derek Solla Price}{
	name={Derek Solla Price},
	description={'Derek John de Solla Price (22 de janeiro de 1922 — 3 de setembro de 1983) foi um físico, historiador da ciência e cientista da informação, conhecido principalmente como o pai da cienciometria. Price é também lembrado por cunhar a expressão "grande ciência" (big science).' Fonte: \citet{wikipedia_derek_2018}
	}
}

\subsection{Minhas impressões iniciais sobre a ciência, por Gustavo Tomás de Paula}

O positivismo \citep{wikipedia_positivismo_2022} é uma corrente filosófica que defende que o conhecimento verdadeiro é derivado pela razão e lógica por meio de experiências sensoriais. No texto "Considerações Preliminares Sobre a Ciência e sua Avaliação", é possível notar essa influência, nos tópicos que se referem, principalmente, ao empiricismo e projetização da ciência. Esses métodos são amplamente utilizados na computação, no momento em que o desenvolvimento de algoritmos que simulam o comportamento de órbitas de corpos celestiais, por exemplo, é derivado da astrofísica e suas áreas correlatas, que fazem forte uso da observação e de resultados empíricos \gls{Positivismo}.


\subsection{Minhas impressões iniciais sobre a ciência, por Gabriel Rocha Fontenele}

A \gls{Ciencia} ampara-se fundamentalmente na filosofia como base para servir o propósito de descrever, explicar ou prever \gls{fenomeno}s do universo \citep{floridi_blackwell_2004}. Tal filosofia abrange a sub-área da Filosofia da Ciência, que estuda aspectos mais subjetivos da ciência ao tentar responder perguntas como "o que é o cientista?", "como os fenômenos devem ser observados?" ou "como deve se estruturar um método para produzir conhecimento científico?".

Tais perguntas concebem conceitos e definições que a ciência se apropria para estudar os fenômenos de forma objetiva -, com perguntas sobre como se manifestam, quais o fatores desencadeadores, o que provocam e explicações sobre os resultados obtidos. Para isso, a ciência faz uso do \gls{EmpirismoCientifico}: os dados coletados produzirão informações relevantes e possíveis soluções para um dado problema inserido no contexto do fenômeno.

\subsection{Minhas impressões iniciais sobre a ciência, por Alexsander Correa de Oliveira}

A ciência da computação, enquanto ciência, é uma das áreas mais subestimadas, pois computação sempre toma a frente, tanto em fundos, quanto em visibilidade. Isso tudo é apenas uma falta de conhecimento sobre como ela é estruturada, pois é sem dúvida uma das ciências mais diferentes. A aplicação de grandes \emph{break-troughs} é quase imediata, como pode ser observado em algoritmos como busca binária e as buscas em grafos criadas por Dijkstra. 
Além do mais, os problemas computacionais tendem a ser mais facilmente apresentáveis para um publico \emph{pop} \citep{wikipedia_scientific_2021}, como um exemplo podemos apresentar um dos problemas do milênio, $P = NP$, em que existem claras aplicações em jogos como xadrez. Essa facilidade de compreensão superficial, muita das vezes, acaba por gerar também o efeito já citado, o de subestimarem a real força da ciência da computação enquanto meio intelectual. 


\subsection{Minhas impressões iniciais sobre a ciência, por Enzo Nunes Leal Sampaio}


A ciência é algo difícil de definir, pois não existe um conceito único e aceito por todos sobre sua definição, segundo \citet{schwartzman_ciencia_1984}. Contudo, a sua importância para o ser humano durante toda a sua história é inegável. Ela permitiu vários avanços para a humanidade como por exemplo: aumento da expectativa de vida, aumento da produtividade de fábricas, a globalização, etc. Infelizmente, na atualidade, vários grupos anti-ciência têm duvidado dos méritos dela como mostra o artigo dos pesquisadores \citet{scherma_relatos_2020}.

Diante disso, o conhecimento sobre o método científico é crucial para que os cientistas possam tentar mostrar como a ciência é feita e que ela segue vários princípios pré-estabelecidos. Um desses princípios é o da Publicidade Científica e este princípio afirma a importância da documentação da atividade científica e é nesse ponto que o uso de ferramentas como o \gls{Bibtex} ajudam nesse processo para criar, por exemplo, os documentos via LaTex, no qual, os pesquisadores podem produzir seus registros.

\subsection{Minhas impressões iniciais sobre a ciência, por André Larrosa Chimpliganond}

Produto da curiosidade e capacidade de abstração humana, a ciência, como explicado em \cite{aranha_temas_2009}, busca entender os fenômenos de maneira racional utilizando uma \gls{Metodologia} rígida que assegure a veracidade das descobertas.
Nascida do pensamento lógico 

\subsection{Minhas impressões iniciais sobre a ciência, por Gustavo Rodrigues dos Santos}

Desde o primórdio da humanidade a ciência sempre foi uma área intrínseca ao ser humano, sempre tentamos explicar o porquê das coisas que acontecem ao nosso redor, os denominados fenômenos ( \gls{fenomeno}). Dessa forma, podemos analisar que a resposta para a pergunta \textit{O que é ciência?} vem se modificando juntamente à humanidade. Inicialmente, a ciência podia se apresentar na forma de um mito o qual retratava a cosmogonia de um povo. Porém ao longo da evolução da humanidade, a ciência tende a acompanhá-la, passando desde os filósofos gregos até a modernidade, quando ocorre um marco para a história da ciência, a Revolução Científica. Desse modo, a ciência sai de um estado nascente rumo a um estado final, sendo este o estado o qual a produção do conhecimento esteja totalmente e precisamente correta acerca dos fenômenos estudados.

Atualmente, nos encontramos no meio dessa transição, nosso conhecimento científico é repleto de erros, e estamos numa fase onde a verdade não é absoluta, como relatado por \citet{fernandes_consideracoes_2021}. Assim, como qualquer outro setor da sociedade, a ciência vivencia regularmente crises não só ligadas à \textit{``incapacidade dos paradigmas científicos vigentes para explicar a realidade''} \citep{fernandes_consideracoes_2021}. Um grande exemplo disso são os movimentos anti-ciência vistos na atual conjuntura onde é notável a crise da elitização da ciência. Esse fenômeno em nada se relaciona à não competência das atuais teorias em explicar os eventos, mas sim com o descolamento da população da comunidade acadêmica. Os cientistas são seres humanos, e portanto, naturalmente, possuem defeitos e consequentemente vieses e conceitos que muitas vezes refletem a própria sociedade a qual estão inseridos. Portanto, se almejarmos chegar a ciência em seu estado final, logo devemos nos opor à movimentos reacionários para assim dar continuidade ao processo espontâneo da ciência. Nós como sociedade devemos, primeiramente, buscar formas de superar nossos vieses e consequentemente alterar as estruturas em nosso meio que perpetuam esses vieses e quem sabe assim conseguiremos progredir de uma maneira mais fluida e assertiva para o objetivo final ou sua proximidade.


\subsection{Minhas impressões iniciais sobre a ciência, por Fernanda Macedo de Sousa}

[...] \gls{FilosofiaDaCienciaComputacional} [...]

\subsection{Minhas impressões iniciais sobre a ciência, por Fernando Ferreira Cordeiro}
A ciência é tanto um corpo de conhecimento quanto um processo. É uma maneira de descobrir e estudar metodicamente o que há no universo, suas verdades e leis naturais como dito em \gls{Ciencia}, para entender como essas coisas funcionam hoje, como funcionavam no passado e como provavelmente funcionarão no futuro. Através da aplicação de métodos metódicos rigorosos e pela analise de dados recorrentes de tais métodos, temos diariamente o que chamamos de inovação cientifica e tecnológico em diversas áreas de conhecimento.

Contudo, toda ação tem uma reação, e assim surgiu o fenômeno chamado de anti-ciência (\gls{AntiCiencia}). Definido como a a rejeição de visões e métodos científicos convencionais ou sua substituição por teorias não comprovadas ou deliberadamente enganosas, a anti-ciência pode ser vista diariamente nos tempos atuais. Muitas vezes utilizadas para ganhos políticos e nefastos, Márcio Scherma e Victor Miranda em \citep{scherma_relatos_2020} relatam como grupo neopentecostais e bolsonaristas conseguem impactar a diretrizes do país pela anti-ciência, se apoiando em notícias e falsas verdades que elevam o desejo dos cidadães acima de uma base confiável de informação.



\subsection{Minhas impressões iniciais sobre a ciência, por Mateus de Paula Rodrigues}

Em \citep{schwartzman_ciencia_1984} Simon Schwartzman não mede esforços para convencer quanto é difícil definir ciência, mas creio eu que a parte realmente dificultosa é conversar terceiros da precisão de sua definição, Aqui vou tentar defini-la de maneira mais simples.

De sua forma mais essencial, a \gls{Dado} é processo pelo qual um ou mais cientistas tem seu trabalho investigativo (seja esse uma hipótese, teoria, técnica ou que for) corrigido por seus pares.

\subsection{Minhas impressões iniciais sobre a ciência, por João Pedro Felix}

A noção do que de fato é ciência por vezes pode ser equivocada e a realidade é que muito do que se considera ciência é na realidade uma aplicação da mesma.
Como visto \hyperlink{1-Introducao/aulas/Ciencia-e-sua-Avaliacao.pdf.2}{acima}, a ciência tem por objetivo, através de um estudo metodicamente organizado, descrever, explicar e prever fenômenos (\gls{fenomeno}) a partir de dados (\gls{Dado}) obtidos através do estudo desses próprios fenômenos.

Isso por si só desconstrói a ideia que boa parte das pessoas tem do que é ciência, ao classificar áreas de estudo como a matemática não como ciência, mas sim como uma ferramenta utilizada em prol da ciência. Isso não retira uma parte sequer da importância da matemática e muito menos muda a necessidade de se investir no desenvolvimento da área, mas nos convida a refletir também sobre outras áreas que talvez funcionem muito mais como um meio pelo qual se faz ciência do que como ciência propriamente dita.

A ciência da computação por exemplo é uma dessas áreas. Boa parte das pessoas que trabalham na área não atua descrevendo, explicando e prevendo fenômenos. Até porque, podemos dizer que no fim boa parte da computação se resume a aplicação da matemática. Trata-se de uma área totalmente diferente da biologia por exemplo, não é algo que se pode tirar proveito simplesmente observando e coletando dados. A computação depende dos seres humanos para existir, bem ao contrário da vida, no caso da biologia.

A ciência da computação tem como principal objetivo intervir na realidade. Tornar situações adversas em outras mais favoráveis. Ela ainda faz uso da observação, mas de problemas externos a própria computação, ela analisa o cotidiano das pessoas, identifica aquilo que não funciona bem e procura maneiras de fazer funcionar melhor. Ou seja, a maioria dos profissionais da área de ciência da computação não atua diretamente na produção de conhecimento cientifico, mas sim em sua aplicação na vida das pessoas.

Mas será que isso não seria o suficiente para classificá-la como ciência ao invés de uma ferramenta como a matemática? A observação se faz presente, mas ao invés de acontecer sobre um vírus, uma molécula, ela acontece sobre a sociedade, assim como nas ciências humanas. Ela ainda busca descrever fenômenos, mas não a partir de palavras e sim da poderosa matemática, de modo a explicá-los não com um simples objetivo da reflexão, mas sim com uma visão atuante na intenção de intervir nesse meio para tornar as condições mais favoráveis e é justamente a intenção de intervir que traz a tona a previsão. Não há como intervir em uma realidade sem conhecê-la. É necessário ter em mente o que acontece no meio ao qual se está inserido se o objetivo for mudá-lo, e é fato que isso é fundamental no desenvolvimento de qualquer software que será utilizado por seres que vivem em uma realidade que muda constantemente.

De fato, existem contribuições dentro da área de ciência da computação que visam o desenvolvimento interno da própria área, que visam evoluir a própria computação. Assim como é fato que a ciência da computação pode muito bem ser usada como uma ferramenta em outras áreas do conhecimento para atingir objetivos próprios dentro dessas áreas. Contudo, o fato de poder ser utilizada como ferramenta não muda o fato de que existem situações que o papel do profissional é exatamente o mesmo de um cientista de qualquer outra área: descrever, explicar e prever, a maior diferença no fato de que a isso é adicionado o "intervir".

A diferença então estaria na parte de que tudo isso deveria ser feito a partir de um estudo metodicamente organizado? De fato, a maneira como se constrói software hoje em dia se difere muito da maneira como outras áreas realizam a produção cientifica. Abordagens de desenvolvimento comuns no mercado atual, como a abordagem de desenvolvimento ágil, se baseia em metas, em marcos que devem ser obtidos a cada período de tempo, sem especificar exatamente como. Se por um lado isso garante certa organização a equipe de desenvolvimento, não há como dizer que ela surge de forma metódica.

Mas e quanto as abordagens usadas em sistemas criticos? O desenvolvimento nesses casos costuma ser baseado em rigorosa documentação em que um fenômeno (problema) é analisado, descrito de maneira matemática, de forma que possa ser explicado sem ambiguidades, e assim é construído um esquema capaz de prever o comportamento do mesmo de maneira que seja possível intervir no mesmo. Feito isso, cada detalhe de como "intervir" é planejado metodicamente de modo que a primeira linha de código só seja digitada no momento que o plano estiver claro e revisado diversas vezes.

O fato de buscar a intervenção tornaria a área algo diferente de ciência? Ou o sentido de descrever, explicar e prever não deveria ser aplicado sobre uma sociedade caso o objetivo seja intervir? Fato é que se a produção cientifica não servir para alterar a realidade de nada ela servirá. Felizmente, como preferimos ver áreas A ou B pouco importa na produção do conhecimento cientifico.

\subsection{Minhas impressões iniciais sobre a ciência, por Vinícius Caixeta de Souza}

A ciência é uma atividade coletiva que estuda fenômenos do universo, não é possível realizar esses estudos sem ter em mente produções científicas feitas por outras pessoas. Devido a grande quantidade de \gls{Conhecimento-Cientifico} a ciência é dividida em várias áreas e subáreas com foco em fenômenos particulares, assim surgem comunidades científicas que realizam discussões e análises em conjunto podendo então aumentar a qualidade de produções \citep{noauthor_comunidade_nodate}. Como forma de coletividade os cientistas procuram divulgar seus estudos e produções para o mundo para que outras pessoas possam utilizar esse conhecimento em suas próprias produções.

\subsection{Minhas impressões iniciais sobre a ciência, por Lucas Gabriel Gurgel}


Ciência não possui significado fixo ao fato de que não há um consentimento entre todos os que a já tentaram introduzir \citep{schwartzman_ciencia_1984}. Ao mesmo tempo que isso se contradiz com a franca necessidade de estudiosos em busca da verdade única, essa discordância também exalta o interminável esforço para se chegar na Verdade absoluta de inúmeros estudos científicos presentes na \gls{ComunidadeCientifica}.

Todo \gls{Conhecimento-Cientifico} resultado de pesquisas validadas por meio de \gls{publicacao-cientifica} é estrutura fundamental inerente para interpretar o mundo e direcionar os objetivos a longo prazo. E mesmo sem ter certeza da definição exata de ciência, interessados possuem \gls{SensoCritico} e aplicam métodos sistemáticos muito bem definidos para fundamentarem, testarem e aplicarem suas teses. Diante disso, mesmo sem significado simples, a ciência engloba todos os resultados obtidos (com sucesso ou não) em estudos feitos, em andamento e futuros, mostrando a complexidade existente por trás da Ciência.


\subsection{Minhas impressões iniciais sobre a ciência, por Ítalo Eduardo Dias Frota}

A ciência é um escopo de \gls{Conhecimento} e um processo. A área se apoia na busca e aplicação dos conhecimentos a respeito das esferas sociais e naturais, seguindo uma \gls{Metodologia} bem definida baseada em evidências que descrevam, expliquem e possam prever um \gls{fenomeno}. Entretanto, definir a ciência não é uma tarefa fácil devido a pluralidade de aplicações e abordagens que permeiam a \gls{ComunidadeCientifica}.

Um dos aspectos mais importantes no campo científico é o da metodologia científica, que permite a formulação de hipóteses, experimentos  e verificações. Os resultados obtidos devem ser reproduzidos através de artigos e publicações que auxiliem na propagação dos pontos observados, instigando novas descobertas e indagações. Sendo assim, o processo é extremamente autocorretivo, pois está em constante evolução.

Sem a ciência e o pensamento científico, a humanidade enfrentaria dificuldades em larga escala. As descobertas e previsões documentadas pelos membros da comunidade são de extrema importância para os avanços nas mais diversas áreas, desde a saúde, até a educação e tecnologia. É imprescindível que seja dada a devida atenção às observações coletadas pela ciência, caso contrário, a humanidade estará sempre destinada ao fracasso.


\subsection{Minhas impressões iniciais sobre a ciência, por Pedro de Torres Maschio}

A ciência, esforço contínuo da \gls{ComunidadeCientifica} é a mais notável atividade humana, existindo desde os primórdios da humanidade (embora não com as configurações atuais), sendo o trabalho do \gls{Cientista} contemporâneo organizado por uma  \gls{Metodologia} que garante a condução sem vieses dos estudos e também sua reprodutibilidade. A ciência atual é marcada pelo uso das tecnologias de informação e de colaboração entre a comunidade científica, bem diferente de seu início, em que as grandes distâncias atrapalhavam a troca de informações entre cientistas de diferentes países/continentes. Sua capacidade de criar novas tecnologias e de responder grandes questões da humanidade fez com que a ciência se tornasse objeto de interesse de empresas, que também podem manter projetos de pesquisa, visando o lucro e a criação de novos produtos.

A atividade científica também sofre críticas, conforme apresenta-se em \citep{chavalarias_whats_2017}, publicação científica no qual o autor investiga as dinâmicas da ciência contemporânea. Dentre as questões levantadas, está a relevância da relação entre produção científica e avanço real na ciência; a influência das dinâmicas sociais na produção científica e os efeitos da política e “publique ou pereça” que é relevante na ciência atual. De modo geral, a ciência é aberta a críticas e isto constitui sua principal fraqueza e sua principal força, pois usa das críticas para se reinventar e gerar novos meios de existir.

\subsection{Minhas impressões iniciais sobre a ciência, por Lucas de Almeida Bandeira Macedo}

A ciência é uma das poucas atividades feitas em grupo, que contam com participantes do presente, passado e futuro. A ciência transcende tempo e espaço, sem necessariamente estudar física. A ciência representa o extremo oposto do que chamamos de egoísmo, pois não há espaço nela para quem não reconhece os esforços dos que vêm depois, e para os que não abrem espaços para que seus trabalhos sejam explorados no futuro. Nada nunca é feito sozinho, pois todo o conhecimento que alguém adquire durante a vida, foi repassado ou baseado em \gls{Conhecimento}s das pessoas que passaram pela vida dessas pessoas.

Este, entre outros vários, é o motivo da Ciência da Ciência \citep{schwartzman_ciencia_1984} ser tão importante: a ciência é complexa. As publicações, a árvore de conhecimento que levou a cada descoberta, a conexão das descobertas com o cotidiano, a diferença que cada estudo fez na humanidade... e o porquê desta descoberta ter sido possível. Entender é saber reproduzir. Não necessariamente conseguir, mas saber.

\subsection{Minhas impressões iniciais sobre a ciência, por Felipe Gomes Paradas}

A ciência é muito complexa para se definir de forma rápida e simplista, porém podemos afirmar que um de seus principais objetivos é a busca, através de observações, hipóteses (\gls{Hipotese}), experimentos, entre outros processos, pela explicação de fenômenos que ocorrem no universo. Através de estudos metódicos e sistemáticos, baseados em dados, experimentos e evidências, a ciência tem como um de seus resultados chaves a produção científica, que é um fenômeno social complexo por si só.

Utilizando a produção científica e estudos metódicos e sistemáticos, ou seja, científicos, se tem um processo bem definido para criação, divulgação e acumulação de conhecimento utilizado pela humanidade.



\subsection{Minhas impressões iniciais sobre a ciência, por Ualiton Ventura da Silva}

O texto apresentado busca descrever e definir um modelo de visão do que seria \gls{Ciencia}, apesar de que o conceito da mesma não seja consensual, assim é descrito por \citep{schwartzman_ciencia_1984}. Contudo, é fundamental que para o estudo aprofundado de uma área haja  uma definição dos pontos aos quais aquela área se atenta a descrever, assim é possível organizar o conhecimento, sendo que o texto define ciência com cunho de ciência natural.

Criando outras perspectivas de ciência, talvez o que não é visto como, será. Exemplo a ser citado é o caso da própria matemática, que observando de um ponto de vista de ciência natural não é ciência, assim é dito por \citep{bilaniuk_is_nodate}, mas o método adotado no próprio campo da matemática assemelha-se ao de outras ciências naturais, enquanto a observação empírica é a forma de validação de uma dada hipótese científica, no caso da própria matemática utilizam-se provas para então chegar a uma determinada conclusão. Assim como as ciências naturais, a matemática também possui questões em aberto que não foram concluídas até o momento por não possuírem meio de prova, exemplo a ser mencionado é a chamada “Conjectura de Goldbach” \citep{noauthor_goldbachs_2022} que estabelece que todo inteiro par maior do que 2 pode ser escrito através da soma de primos (4 = 2+2, 6 = 3+3, 8 = 3+5, …), apesar de acharmos que possivelmente seja um fato, tal conjectura não possui prova e portanto não pode ser assumida como uma verdade universal.

Portanto, apesar de alguns campos de estudos não serem categorizados como ciência, através da visão adotada existem casos onde características das quais foram apresentadas pertencem ao campo estudado. Em ciência da computação, parte dos profissionais podem exercer a atividade de cientista, empregando então dada metodologia científica, contudo, outros podem desconsiderar alguns aspectos, exemplo, para um dado problema computacional este mesmo problema poderá ser solucionado de diversas maneiras, assim, não existe uma maneira única e universal para a resolução, portanto, não apresenta uma universalidade científica.


\subsection{Minhas impressões iniciais sobre a ciência, por João Víctor Siqueira de Araujo}

Como podemos definir o que é \gls{Ciencia}? Essa pergunta a priori parece bastante simples, mas na realidade trata-se de uma questão muito mais complexa. É difícil definir um conceito para Ciência, o que se deve à inexistência de um consenso sobre uma definição exata, o que temos na verdade são noções que variam ao longo do tempo e do espaço \citep{schwartzman_ciencia_1984}. Mesmo com essa dificuldade de uma definição exata, podemos enxergar a Ciência como um \gls{fenomeno} de caráter humano e principalmente social, em virtude que a Ciência é feita por pessoas, os \gls{Cientista}s, e essas pessoas interagem socialmente através da \gls{ComunidadeCientifica}, a partir da qual se instaurou uma \gls{Cultura} que preza pelo desenvolvimento do conhecimento a partir de observações empíricas e de análise racional. Dessa forma, para se entender melhor o que é Ciência, é imprescindível a compreensão que a mesma trata-se de um fato social e, por isso, não podemos desatrelar seus aspectos sociais ao próprio conceito de Ciência, visto que esta é intrinsecamente uma atividade social.