\subsection{Minhas impressões iniciais sobre a ciência, por Enzo Nunes Leal Sampaio}

A ciência é algo difícil de definir, pois não existe um conceito único e aceito por todos sobre sua definição, segundo \citet{schwartzman_ciencia_1984}. Contudo, a sua importância para o ser humano durante toda a sua história é inegável. Ela permitiu vários avanços para a humanidade como por exemplo: aumento da expectativa de vida, aumento da produtividade de fábricas, a globalização, etc. Infelizmente, na atualidade, vários grupos anti-ciência têm duvidado dos méritos dela \citet{scherma_relatos_2020}.

Diante disso, o conhecimento sobre o método científico é crucial para que os cientistas possam tentar mostrar como a ciência é feita e que ela segue vários princípios pré-estabelecidos. Um desses princípios é o da Publicidade Científica e este princípio afirma a importância da documentação da atividade científica e é nesse ponto que o uso de ferramentas como o \gls{Bibtex} ajudam nesse processo para criar, por exemplo, os documentos via LaTex, no qual, os pesquisadores podem produzir seus registros.