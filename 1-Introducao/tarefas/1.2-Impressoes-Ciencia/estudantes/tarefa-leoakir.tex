\subsection{Minhas impressões iniciais sobre a ciência, por Léo Akira Abe Barros}

A \gls{Ciencia} é a forma que os seres humanos encontraram de entender um determinado \gls{fenomeno} com uma abordagem direta e objetiva que procura explorar todos os possíveis ramos de conhecimento até o seu fim. E embora possa ser difícil de acreditar, devido a propostas tão distintas, possui uma grande semelhança com contos narrativos.
Apesar de narrativas explorarem a subjetividade e tentarem afetar a natureza emocional humana, ambos são intrinsecamente sociais, pois ambos são mecanismos já utilizados há centenas de anos para passar \gls{Conhecimento} de geração em geração.
Mas algo que torna a ciência ainda mais eficaz através deste mecanismo é que todo o conhecimento desenvolvido por antepassados vira um avanço permanente que pode abrir múltiplas novas oportunidades a humanidade de compreender o universo à nossa volta.
É possível observar que estes avanços são tão potentes, que o crescimento não ocorre de forma linear, e uma das formas de notar é através da \gls{Cientometria}, que mede esta progressão de forma quantitativa, podendo não apenas permitir uma visualização do progresso científico, mas também nos informar para melhorar nossas decisões sobre nosso foco em diferentes ramos de conhecimento, assim como detalhado por \cite{dasilva_cientometria_2001}.