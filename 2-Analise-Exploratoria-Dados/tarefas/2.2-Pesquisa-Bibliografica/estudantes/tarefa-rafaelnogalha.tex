\chapter{Análise Bibliográfica sobre Quantum Key Distribution(QKD) Na Segurança Computacional, por Rafael Henrique Nogalha de Lima}

\section{Planejamento do estudo}

Com o crescimento do número de ataques cibernéticos a empresas governamentais e grandes corporações no mundo todo, o estudo e investimento por formas mais seguras de criptografia se tornou prioridade em diversas entidades. 

Dessa forma, com o aumento exponencial de informação produzida no mundo (\textit{BigData}), houve uma expansão no uso de tecnologias quânticas para o fim de segurança computacional, tendo em vista que a matemática computacional convencional não é capaz de ser escalável em relação a quantidade de informação produzida e assim não é mais segura também.. Assim, o investimento está sendo feito em \textit{Distribuição de chave quântica} ou QKD, pois o poder do processador quântico e a escalabilidade são praticamente infinitos comparados à computação tradicional, além de oferecer uma segurança computacional dupla, na qual é praticamente impossível de ser quebrada até mesmo por computadores quânticos. Assim, a presente análise busca evidenciar aspectos a respeito da produção científica sobre esse tema. Os pontos de pesquisa são:

\begin{itemize}
    \item Como a produção científica em relação à QKD evoluiu nos últimos anos?
    \item Quais são as principais definições e campos relacionados à QKD?
    \item Como a QKD é implementada em ambientes reais de segurança computacional
\end{itemize}