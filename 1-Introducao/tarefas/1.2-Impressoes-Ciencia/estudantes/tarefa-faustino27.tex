\subsection{Minhas impressões iniciais sobre a ciência, por Gabriel Faustino Lima da Rocha}

A ciência vem da vontade do ser humano de explicar, prever e descrever aquilo que acontece a seu redor por meio do método científico. Seguindo o princípio de que não há verdade absoluta pois as pessoas que criaram aquela hipótese \citet{noauthor_hipotese_2020} também estão sujeitos a erros, pois os dados sofrem limitações na descrição da realidade nos 5 sentidos utilizados pelo ser humano.

A ciência proporcionou um desenvolvimento tecnológico e econômico tendo iniciado desde de antes da criação do método científico, e desde então vem crescendo cada vez mais rápido com crescimento dos meios de comunicação possibilitando com que publicações, ou comunicações, científicas se propagassem de maneira mais rápida, eficiente e igualitária.