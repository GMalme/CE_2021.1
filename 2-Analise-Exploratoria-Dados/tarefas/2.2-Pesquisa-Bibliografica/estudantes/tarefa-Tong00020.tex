\chapter{Análise Bibliográfica sobre Simulação de Big Data na economia, por Tong Zhou\label{chap:bibliometria:jhcf}}


\section{Planejamento do estudo}
%O planejamento o  desenho do estudo deve descrever as motivações, questões de interesse, escopo, limitações e objetivos do trabalho.

%O planejamento do estudo deve motivar o tema escolhido e o interesse do autor.

%No caso do meu trabalho, as perguntas que o nortearam foram:
%\begin{itemize}
%    \item Qual a base de conhecimentos científicos produzida em torno do tema simulação multiagente voltada à compreensão de fenômenos sociais, com ênfase em métodos experimentais? 
%    \item Como a simulação multiagente tem sido usada para compreender fenômenos sociais, com ênfase em métodos experimentais? 
%    \item Quais os principais termos e conceitos ligados à frente de pesquisa no tema simulação multiagente de fenômenos sociais, com ênfase em métodos experimentais? 
%    \item Qual a estrutura social da comunidade, se é que existe, que pesquisa sobre o tema simulação multiagente de fenômenos sociais, com ênfase em métodos experimentais?
%\end{itemize}

\begin{itemize}
    \item 
\end{itemize}


\subsection{O que já existe de pesquisa bibliométrica sobre esse tema?}


\subsection{Uso do Bibliometrix e Biblioshiny}


\subsection{Limitações} 
Está tarefa foi feita com


\section{Coleta de dados}

Anotações feitas durante uma pesquisa em base bibliográfica:

big data and economy (experimental  or  numeric* or  statist* or  hypothes* or  empiric* or  inferen * social  or  society  or  behavi *)

A coleta de dados feita usando o Web Of Science (WoS) no dia 03 de fevereiro de 2022, acessado por meio do Portal de Periódicos da CAPES. Foram feitas buscas nas coleções Science  Citation  Index  Expanded (SCI -EXPANDED), Social Sciences  Citation  Index (SSCI), Conference Proceedings Citation Index-Science (CPI-S) e Emerging Sources Citation Index(ESCI), com foco.
Foi pesquisado artigos de 2018 a 2022.
Foram encontrados 983.

\section{Análise dos dados}

\subsection{Filtragem de registros}

Sobrou 892.

\subsection{Análise descritiva do %\textit{dataset} 
MASSA@Tong00020}



\subsection{Evolução da Produção Científica}




\subsection{Evolução das Citações}
