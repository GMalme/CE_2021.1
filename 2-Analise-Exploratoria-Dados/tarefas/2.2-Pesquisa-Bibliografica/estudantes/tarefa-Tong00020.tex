\chapter{Análise Bibliográfica sobre Simulação de Big Data na economia, por Tong Zhou\label{chap:bibliometria:Tong00020}}


\section{Planejamento do estudo}

O Big Data estuda, analisa e trata de um conjunto de dados maior e mais complexo que as processadas em sistemas tradicionais. Isso permite a resolução de problemas que não eram possíveis anteriormente.
%O planejamento o  desenho do estudo deve descrever as motivações, questões de interesse, escopo, limitações e objetivos do trabalho.

%O planejamento do estudo deve motivar o tema escolhido e o interesse do autor.

%No caso do meu trabalho, as perguntas que o nortearam foram:
%\begin{itemize}
%    \item Qual a base de conhecimentos científicos produzida em torno do tema simulação multiagente voltada à compreensão de fenômenos sociais, com ênfase em métodos experimentais? 
%    \item Como a simulação multiagente tem sido usada para compreender fenômenos sociais, com ênfase em métodos experimentais? 
%    \item Quais os principais termos e conceitos ligados à frente de pesquisa no tema simulação multiagente de fenômenos sociais, com ênfase em métodos experimentais? 
%    \item Qual a estrutura social da comunidade, se é que existe, que pesquisa sobre o tema simulação multiagente de fenômenos sociais, com ênfase em métodos experimentais?
%\end{itemize}

\begin{itemize}
    \item 
\end{itemize}


\subsection{O que já existe de pesquisa bibliométrica sobre esse tema?}


\subsection{Uso do Bibliometrix e Biblioshiny}

A pesquisa bibliométrica foi realizada com o uso do RStudio, foram usados o pacote \textit{Bibliometrix} e o aplicativo \textit{Biblioshiny} a partir do que foi apresentado em aula, como mostrado na figura ~\ref{fig:bibliometrix:workflow}. Através do acesso café do portal Periódicos CAPES, foi utilizada a \textit{Web of Science} para extrair os artigos usados na pesquisa.


\subsection{Limitações} 

Nesta tarefa foram feitas duas buscas na base de dados WoS, pois na primeira, o número de artigos não foi suficiente para gerar gráficos apresentáveis.


\section{Coleta de dados}

A coleta de dados feita usando o Web Of Science (WoS) no dia 03 de fevereiro de 2022, acessado por meio do Portal de Periódicos da CAPES. Foram feitas buscas nas coleções \textbf{Science  Citation  Index  Expanded (SCI -EXPANDED)}, \textbf{Social Sciences  Citation  Index (SSCI)}, \textbf{Conference Proceedings Citation Index-Science (CPI-S)} e \textbf{Emerging Sources Citation Index(ESCI)}. Foram colocados na pesquisa, artigos de 2010 a 2022. Foram encontrados 1265 artigos.

\subsection{Query de Busca}

Foi usada a query de busca abaixo: 

big data and economy (experimental  or  numeric* or  statist* or  hypothes* or  empiric* or  inferen * social  or  society  or  behavi *)

\subsubsection{Explicação para os termos de busca usados}


\subsection{Registros recuperados}

Os 1.265 registros obtidos como resultado da busca encontram-se em \url{https://github.com/jhcf/Comput-Experim-20212/experiments/Tong00020/PesqBibliogr/SimulacaoMultiagente/WoS-20220203/1265records.txt}

Foram utilizadas as opções \textit{Exportar registros para arquivo de texto sem formatação} e \textit{export full record / Gravar Conteúdo: Seleção personalizada, com todos os 29 campos disponíveis, inclusive referências citadas} no WoS, para que as citações também fosse usadas em análises da citações (estrutura intelectual do conhecimento). Os 1.265 registros foram recuperados em nove blocos de até 1.000 registros por vez (1-1000, 1001-1265).

\section{Análise dos dados}



\subsection{Filtragem de registros}

Sobrou 1.168 artigos ao colocar apenas "ARTICLE" como tipo de documento com o uso filtro do biblioshiny.



\subsection{Análise descritiva do %\textit{dataset} 
MASSA@Tong00020}

As informações mais gerais sobre o \textit{dataset} MASSA@Tong0020 são as seguintes:
\begin{description}
    \item[\textit{Timespan}] Os artigos da busca foram publicados a partir de 2010.
    \item [\textit{Sources (Journals, Books, etc)}] São 719 fontes publicaram os documentos no \dataset\. Em média, cada \textit{scientific journal} publicou $1.168/719=1,62$ artigos.%%%
     \item [\textit{Average years from publication}] A média do tempo de publicação dos artigos no dataset é de 3.25 anos.
     \item [\textit{Average citations per documents}] Cada artigo no dataset foi citado, em média 16.79 vezes.
     \item [\textit{Average citations per year per doc}] Após publicado, cada um dos 1.168 artigos do dataset foi citado 3.421 vezes por ano, em média.
    \item [\textit{References}] O dataset contém 61.413 referências citada.
    \item [\textit{Keywords Plus (ID)}] 2.857 distintas palavras-chave do tipo Keywords Plus (ID).
    \item [\textit{Author's Keywords (DE)}] 4.420 distintas palavras-chave indicadas pelos autores foram encontradas no \textit{dataset}.
     \item [\textit{Authors}] 10.181 nomes distintos de autores foram encontrados no \textit{dataset}.
    \item [\textit{Author Appearances}] Os 10.181 distintos (nomes de) autores foram encontrados 13.591 vezes, como autores de artigos.
    \item [\textit{Authors of single-authored documents}] Dentre os 10.181 distintos (nomes de) autores encontrados, 167 deles editaram artigos individualmente, isso é, sem co-autores.
    \item [\textit{Authors of multi-authored documents}] Dentre os 10.181 distintos (nomes de) autores encontrados, 10.014 deles editaram artigos com um ou mais co-autores"
    \item [\textit{Single-authored documents}] Dentre os 1.168 documentos presentes no \dataset\, 173 foram escritos por um único autor, e os 995 restantes foram elaborados em co-autoria.
    \item [\textit{Documents per Author}] Dentre os 10.181 distintos (nomes de) autores, cada um publicou em média 0,115 artigos.
    \item [\textit{Authors per Document}] Cada um dos 1.168 documentos presentes no \textit{dataset}  foi autorado com 8.72 autores em média ($10.181 / 1.168 = 8.72$).
    \item [\textit{Co-Authors per Documents}] As 13.591 aparições de (nomes de) autores (``Author Appearances''), sem distribuem, em média 11.6 vezes para os 1.168 documentos do \textit{dataset}.
    \item [\textit{Collaboration Index}] Os 10.181 (nomes de) autores que editaram artigos com um ou mais co-autores, colaboraram em media 10.1 vezes para editar os 1.168 artigos elaborados em co-autoria, gerando, assim, um índice de colaboração 10.1. 
    
\end{description}

     
\subsection{Evolução da Produção Científica}
A figura 
     
\subsubsection{Interpretação do Crescimento}

\subsection{Evolução das Citações}
A figura

\subsubsection{Interpretação da figura}
A figura

\subsection{\textit{Three-Field Plots (Sankey diagram)}}

\subsubsection{Interpretação da figura}
A figura

\subsection{Autores mais relevantes}
Apresentamos a seguir comentários sobre os autores mais citados nos trabalhos.

\subsection{Análises Bibliométricas: Fontes de Informação}

\subsection{Análises Bibliométricas: Autores}

\subsection{Análises Bibliométricas: Documentos}