\subsection{Minhas impressões iniciais sobre a ciência, por Guilherme Oliveira Loiola}

Com as construções baseadas no \gls{Racionalismo} e argumentações lógicas é construído o conhecimento científico que permeia diferentes evoluções necessárias na sociedade. Com resultados pautados em \gls{Metodologia} é evidencializado resultados coerentes e em parte de aceitação comum. Através desses resultados, a \gls{Ciencia} influência com seus métodos diferentes áreas como: saúde, política, segurança e outros. 

Com o \gls{Conhecimento-Cientifico} as decisões sociais não dependem apenas de opiniões pessoais, mas sim de construções aceitas pela \gls{ComunidadeCientifica}. Dessa forma, consequências diretas são observadas como no texto de \citet{frickel_new_2006} onde existe a junção da política com aspectos científicos para tentar mapear comportamentos da sociedade e assim obter respostas para os desafios propostos. Além disso, é exemplificado a eficácia da ciência também com a geração de resultados em desafios recorrentes como produção de alimentos, manutenção da saúde e melhora na qualidade de vida.  

Dentro da ciência temos práticas com controles \gls{Estatisticos} para o uso estratégico da informação na tomadas de decisões. De forma que seja antecipado parte das influências consequentes da decisão tomada. Com o estudo bibliométrico são explicitados diferentes métricas sobre a evolução de áreas emergentes e a progressão de áreas consolidadas. Assim como argumentado em \citet{santa_soriano_bibliometric_2018} a maturidade e impacto de diferentes pesquisas podem ser mensurados pelas análises das publicações científicas. Então, é estruturado respaldo científico acerca dos padrões de construção do conhecimento, logo trazendo oportunidades de participação nesse processo.

