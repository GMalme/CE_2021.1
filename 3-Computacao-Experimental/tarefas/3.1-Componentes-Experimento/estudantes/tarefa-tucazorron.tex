\section{Respostas de Artur Filgueiras Scheiba Zorron}

\label{tarefa-tucazorron-componentes-experimento}

\subsection{A computação é uma atividade científica? Justifique. }

Para responder se a computação é uma atividade científica é preciso:
\begin{itemize}
    \item Definir o que é uma atividade científica;
    \item Definir o que é computação;
    \item Analisar se a definição de computação se enquadra na definição de atividade científica. 
\end{itemize}

\subsubsection{O que é uma atividade científica?}

A atividade científica é a prática de métodos científicos utilizando comparações de resultados, análises de métricas e resultados cada vez mais conclusivos e próximos da realidade, sempre mantendo como foco a realidade sem interferências de cunho pessoal sobre as conclusões.

\subsubsection{O que é a computação?}

A computação de forma sucinta e breve se resume a criação e automação de processos de maneira digital, realizadas a partir de algoritmos onde se utiliza lógica como pretexto para validar, iterar e construir conclusões e processos.

Com base no que foi visto em \ref{chap:tucazorron:impressoes} e a partir do conceito de \gls{MetodoExperimental}, podemos perceber que o método pelo qual a computação obtém suas respostas se adequa ao que é ciência.

\subsection{A computação é uma ciência experimental? Justifique. }

Sim pois a computação testa e aprende com o processo até chegar no algoritmo final.

\subsection{A computação é uma ciência empírica? Justifique. }

Sim, pois os resultado dela fazem com que ela mesma evolua. Um exemplo claro é o Aprendizado de Máquina.