\section{Respostas de João Víctor Siqueira}

A \gls{Ciencia} pode ser definida como sendo uma investigação racional metodicamente organizada acerca de \gls{fenomeno}s quaisquer do mundo com intuito de descrever, explicar e prever o comportamento e a estrutura de tais fenômenos e é sustentada por registros produzidos sobre esses fenômenos \citep{fernandes_consideracoes_2021}. Dessa definição, é perceptível que a \gls{CienciaComp} não é uma atividade científica, pois seu objetivo é mudar a realidade de um fenômeno, gerar algo novo, e não buscar compreender a verdade por trás de um fenômeno.

Sabendo que a Ciência da Computação não é uma atividade científica, surge um questionamento a respeito da atividade profissional de um cientista da computação, o que se quer saber é se essa atividade utiliza todos os componentes de um experimento científico. Inicialmente, é válido enumerar quais são os componentes de um experimento científico: 

\begin{itemize}
    \item \textbf{Uma Classe de Fenômenos de Interesse}
    \item \textbf{Hipóteses e Modelos}
    \item \textbf{Causa e Efeito}
    \item \textbf{Procedimentos Repetíveis}
    \item \textbf{Demonstração, Manipulação e Controle de \gls{Fatores}}
    \item \textbf{Coleta de Dados}
    \item \textbf{Análise Lógica}
\end{itemize}

No que diz respeito à \textbf{\textit{classe de fenômenos de interesse}}, pode-se verificar que na Ciência da Computação a classe de interesse seria a classe dos fenômenos computacionais, como, por exemplo, o consumo de memória, a complexidade de algoritmos e até mesmo o tempo de execução de um programa.

Com relação a \textbf{\textit{hipóteses e modelos}}, as técnicas de modelagem são amplamente utilizadas pelos cientistas da computação principalmente para a modelagem de sistemas que auxilia na concepção de um \textit{software} por especificar sua estrutura e seu comportamento, servindo como um guia para a construção. Dessa forma, a ação de construir um \textit{software} baseando-se em um modelo nada mais é que uma busca por validação desse modelo.

Acerca de \textbf{\textit{causa e efeito}}, é notório que as contribuições dos cientistas da computação para o desenvolvimento de novas tecnologias trazem diversos impactos para a sociedade como um todo. Por exemplo, as pesquisas e o desenvolvimento dentro da área da inteligência artificial, principalmente no ramo de reconhecimento de padrões, ocasionaram uma mudança no \gls{Paradigma} de segurança, pois possibilitou novas técnicas de segurança como o reconhecimento facial e o reconhecimento biométrico.

Sobre \textbf{\textit{procedimentos repetíveis}}, é válido lembrar que uma das atividades dos cientistas da computação é justamente o desenvolvimento de algoritmos. A maneira em que um algoritmo é estruturado permite que um programa que o implemente possa ser, de certa forma, reproduzido por qualquer outra pessoa, desde que essa pessoa respeite a forma que o algoritmo foi definido, utilize os mesmos dados de entrada que foram usados no programa e também replique o ambiente de execução no qual o programa foi testado anteriormente. Dessa forma, pode-se perceber que, atendendo a certas condições, é possível que uma pessoa consiga obter um programa que a sua execução resulte nos mesmos valores de um programa desenvolvido por um terceiro.

No que se refere à \textbf{\textit{demonstração, manipulação e controle de fatores}}, uma das principais tarefas de um cientista da computação é justamente comparar diferentes soluções computacionais para o mesmo problema. Por exemplo, para comparar dois processadores a fim de verificar qual dos dois é mais eficiente para a execução de um dado programa, pode-se desconsiderar, por exemplo, o tempo de entrada e saída dos dados, ou ainda manipular a frequência de \textit{clock} de um dos processadores para ver se essa manipulação resultou em um melhor desempenho.

Quanto à \textbf{\textit{coleta de dados}}, é importante destacar que a Computação está intimamente relacionada com o processamento de dados de entrada e a produção de dados de saída. Nesse contexto, os dados gerados por um programa, por exemplo, são dados significativos e por isso podem ser coletados e armazenados com o intuito de se fazer um estudo sobre os mesmos. Voltando para o exemplo da comparação entre dois processadores, pode-se fazer a coleta dos dados referentes ao tempo de execução do programa em cada um dos processadores, assim como os dados referentes ao consumo de energia, por exemplo. E, posteriormente, usar esses dados para chegar a alguma conclusão sobre essa comparação.

Por fim, no que concerne à \textbf{\textit{análise lógica}}, como visto anteriormente, os processos computacionais resultam na geração de dados, que por sua vez são significativos, ou seja, fornecem algum tipo de informação. Por fornecerem informações, é então possível realizar uma análise sobre esses dados a fim de obter uma conclusão sobre o processo. Voltando ao exemplo dos processadores, haviam sido coletados dados referentes ao tempo e o consumo de energia durante a execução de um dado programa. Esses dados coletados fornecem informações das quais, por meio de uma análise lógica, é possível compreender como é o desempenho de um processador com relação ao outro. 

Com o que foi observado anteriormente, mesmo que a Ciência da Computação não seja uma atividade científica, ela pode ser compreendida como uma atividade empírica visto que a realização da atividade profissional de um cientista da computação acaba por empregar todos os componentes de um experimento científico.