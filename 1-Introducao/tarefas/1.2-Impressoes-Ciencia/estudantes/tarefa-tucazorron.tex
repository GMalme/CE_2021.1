\subsection{Minhas impressões iniciais sobre a ciência, por Artur Filgueiras Scheiba Zorron}

Durante milênios a humanidade está a procura de verdades ou de fatos. Acontecimentos e fenômenos que conseguimos prever, predizer ou até controlar. Para tal, aplicamos métodos de estudo e de análise sobre estes fenômenos. Um dos métodos mais aceitos e conclusivos que possuímos é o método científico. Este processo da busca de alguma verdade ou de algum fato é chamado por muitas pessoas de ciência. Porém segundo \citep{schwartzman_ciencia_1984}, ciência não é uma variável de controle onde podemos afirmar o que é o que não é ciência. Hoje a \gls{Comunidade Científica} entende como ciência diversas formas de procurar informações ou validar dados.