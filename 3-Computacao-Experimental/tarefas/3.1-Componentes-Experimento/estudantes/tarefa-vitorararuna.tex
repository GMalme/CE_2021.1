\section{Respostas de Vitor de Oliveira Araújo Araruna \label{tarefa-vitorararuna-componentes-eperimento}}

\subsection{A computação é uma atividade científica? Justifique. }

Para responder se a computação é uma atividade científica é preciso:
\begin{itemize}
    \item Definir o que é uma atividade científica;
    \item Definir o que é computação;
    \item Analisar se a definição de computação se enquadra na definição de atividade científica. 
\end{itemize}

\subsubsection{O que é uma atividade científica?} Atividades científicas podem ser consideradas como uma coleção de informações já conhecidas em relação aos fenômenos naturais, esses que serão correlacionados e interpretados por cientistas, a fim de gerar novos pontos de vista.

O pensamento científico convoca o raciocínio para interpretar e relacionar. Há uma disciplina da atenção dirigida:

\begin{enumerate}
    \item à objetividade, para que teorias ou sistemas resultem de avaliação imparcial, independente de tendências ou pré-julgamentos;
    \item  à exatidão, para que se consiga a ideia adequada aos fatos;
    \item  à ordenação, para análise de eventuais variações do que foi observado;
    \item aos pormenores para que não ocorra menosprezo de aspectos pouco evidentes.
\end{enumerate}

\subsubsection{O que é a computação?}
A computação pode ser estabelecida como uma investigação de soluções para um problema a partir de argumentos (inputs), a fim de obter respostas (outputs), após o processamento de outras informações, juntamente com os argumentos. Esse prcesso ocorre atrav´s de um algoritmo 

Logo, definindo características de experimento científico:

uma experiência adequada implica:

\begin{enumerate}
    \item que os antecedentes temporais sejam claros;
    \item que exista uma covariação estatisticamente significativa entre uma causa e um efeito;
    \item que não existam terceiras variáveis que possam dar uma explicação alternativa para a relação de causa e o efeito;
    \item que não haja hipóteses alternativas sobre os construtos utilizados;
\end{enumerate}

Com base nos estudos realizados até aqui, em Computação Experimental, seguindo o conceito de Metodoligia Experimental, percebe-se que o método no qual a computação obtém suas respostas, está intimamente ligada no que diz respeito ao que é ciência.

\subsection{A computação é uma ciência experimental? Justifique. }
Diante do que foi exposto anteriormente, a computação é realizada com a tentativa de encontrar respostas através de soluções com algorítimos, testes e experimentos. E essa ideia se aproxima da do método científico experimental

\subsection{A computação é uma ciência empírica? Justifique. }

O empirismo científico diz que a produção de conhecimentos científicos devem ser baseados em coletas e análises de dados obtidos sobre o objeto de estudo. Logo, a computação está diretamente relacionada com o empirismo, já que utilizamos dados e conhecimentos já analisados em grande parte dos casos que queremos buscar uma resposta.