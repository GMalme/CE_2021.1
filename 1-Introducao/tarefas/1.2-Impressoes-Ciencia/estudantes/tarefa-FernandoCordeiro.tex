\subsection{Minhas impressões iniciais sobre a ciência, por Fernando Ferreira Cordeiro}
A ciência é tanto um corpo de conhecimento quanto um processo. É uma maneira de descobrir e estudar metodicamente o que há no universo, suas verdades e leis naturais como dito em \gls{Ciencia}, para entender como essas coisas funcionam hoje, como funcionavam no passado e como provavelmente funcionarão no futuro. Através da aplicação de métodos metódicos rigorosos e pela analise de dados recorrentes de tais métodos, temos diariamente o que chamamos de inovação cientifica e tecnológico em diversas áreas de conhecimento.

Contudo, toda ação tem uma reação, e assim surgiu o fenômeno chamado de anti-ciência (\gls{AntiCiencia}). Definido como a a rejeição de visões e métodos científicos convencionais ou sua substituição por teorias não comprovadas ou deliberadamente enganosas, a anti-ciência pode ser vista diariamente nos tempos atuais. Muitas vezes utilizadas para ganhos políticos e nefastos, Márcio Scherma e Victor Miranda em \citep{scherma_relatos_2020} relatam como grupo neopentecostais e bolsonaristas conseguem impactar a diretrizes do país pela anti-ciência, se apoiando em notícias e falsas verdades que elevam o desejo dos cidadães acima de uma base confiável de informação.

