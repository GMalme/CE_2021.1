\chapter{Análise Bibliográfica sobre Graph Neural Network, por Artur Filgueiras Scheiba Zorron\label{chap:bibliometria:titofrota}}

\section{Planejamento do estudo}

\textit{Graph Neural Network}, uma junção de "\textit{graph}" e "\textit{neural network}", se trata de uma classe de rede neural para processamento de dados melhor representada por estruturas de dados de grafo. Eles foram popularizados por seu uso em aprendizado supervisionado sobre propriedades de várias moléculas.

Atualmente, \textit{graph neural networks} são muito utilizados em diversas áreas como processamento de linguagem natural, aprendizado de máquina e até em jogos como engines de xadrez.

Partindo dos pontos acima, as questões que nortearam a pesquisa foram:
\begin{itemize}
    \item Quais são as técnicas mais utilizadas com \textit{graph neural networks}?
    \item Quais são os impactos dessa tecnologia hoje em dia?
    \item Quais as principais áreas utilizando \textit{graph neural networks}?
\end{itemize}

\subsection{Uso do Bibliometrix e Biblioshiny}
Serão usadas a ferramenta e o \textit{workflow} proposto pelos autores do pacote Bibliometrix, conforme indica a figura ~\ref{fig:bibliometrix:workflow}.

\subsection{Limitações} O exercício relatado foi feito em 6 horas e meia, utilizando a base de dados Web of Science (WoS).


\section{Coleta de dados\label{MASSA:coleta}}

A coleta de dados feita usando a base Web of Science (WoS) no dia 09 de janeiro de 2022, acessado por meio do Portal de Periódicos da CAPES.

Foi usada a \textit{query} de busca ilustrada loog abaixo:

graph neural networks
graph ai
graph convolutional networks

\subsection{Explicação para os termos de busca usados\label{sec:tucazorron:query}}

A busca utilizando estes termos se deve pela lógica de tentar encontrar artigos voltados para o tema de grafos, com foco em redes neurais e inteligência artificial. Os termos podem aparecer no título, no abstract ou nas palavras chave.

O termo grafo se refere às buscas voltadas para esta estrutura de dados. Já seus adjetivos como redes neurais e inteligência artificial foram usados para entender aplicações mais específicas.

Foram obtidos mais de 15 mil registros com a \textit{query} utilizada. Na exportação, foi utilizado o formato de arquivo de texto sem formatação, com os 29 campos disponíveis.

\section{Análise dos dados}

\subsection{Filtragem de registros}

Para não poluir demais os gráficos, foi feita uma filtragem com os 2000 artigos mais relevantes do tema.

\subsection{Análise descritiva do \textit{dataset} }

A seguir, é feita uma análise bibliométrica descritiva do \textit{dataset} utilizando a função \texttt{biblioAnalysis} do Bibliometrix, que realiza diversos cálculos para levantar as taxas apresentadas.

As informações mais gerais sobre o \textit{dataset} são as seguintes:
\begin{description}
    \item [\textit{Timespan}] Os artigos filtrados foram publicados entre 1966 e 2022.
    \item [\textit{Sources (Journals, Books, etc)}] São 1024 fontes de informação que publicaram os artigos recuperados no \textit{dataset}.
    \item [\textit{Average years from publication}] A média do tempo de publicação dos artigos no \textit{dataset} é de 5.89 anos.
    \item [\textit{Average citations per documents}] Cada artigo no \textit{dataset} foi citado, em média 7.44 vezes.
    \item [\textit{Average citations per year per doc}] Após publicado, cada um dos artigos foi citado, em média, 1.35 vezes por ano.
    \item [\textit{References}] O \textit{dataset} contém 49.429 referências citadas.
    \item [\textit{Keywords Plus (ID)}] 1444 distintas palavras-chave do tipo Keywords Plus (ID) foram encontradas no \textit{dataset}.
    \item [\textit{Author's Keywords (DE)}] 4276 distintas palavras-chave indicadas pelos autores foram encontradas no \textit{dataset} .
    \item [\textit{Authors}] 6.017 distintos nomes de autores foram encontrados no \textit{dataset} .
    \item [\textit{Author Appearances}] Os autores foram encontrados 8.036 vezes, como autores de artigos.
    \item [\textit{Authors of single-authored documents}] 121 autores editaram artigos individualmente, isso é, sem co-autores.
    \item [\textit{Authors of multi-authored documents}] 5.896 autores editaram artigos com um ou mais co-autores.
    \item [\textit{Single-authored documents}] 134 documentos foram escritos por um único autor, e os restantes foram elaborados em co-autoria.
    \item [\textit{Documents per Author}] Cada autor publicou em média 0.332 artigos.
    \item [\textit{Authors per Document}] 3.01 autores por documento.
    \item [\textit{Co-Authors per Documents}] 4.02 co-autores por documento.
    \item [\textit{Collaboration Index}] 3.16 indíces de colaboração.
\end{description}

\subsection{Evolução da Produção Científica}

\begin{figure}
    \centering
    \includegraphics[width=1\textwidth]{experiments/tucazorron/PesquisaBibliometrica/GNN/3.png}
    \caption{Evolução da produção científica no \textit{dataset}}
    \label{fig:evol:anual:GNN@tucazorron}
\end{figure}

A figura \ref{fig:evol:anual:GNN@tucazorron} representa a evolução em produção científica mundial a respeito do tema, de acordo com o \textit{dataset}. Houve um crescimento a notável a partir do ano de 2016, atingindo o pico em 2022. 

O \textit{Annual Growth Rate} do \textit{dataset} é de 8.6\%, que é um valor alto comparado a média de crescimento da comunidade científica como um todo.

\subsection{Interpretação do Crescimento} a taxa de crescimento do \textit{dataset} demonstra que o tema tem chamado muita atenção nos últimos anos, provavelmente devido ao grande avanço das redes sociais pelo mundo junto da inteligência artificial.

\subsection{Evolução das Citações}

\begin{figure}
    \centering
    \includegraphics[angle=0,width=1\textwidth]{experiments/tucazorron/PesquisaBibliometrica/GNN/2.png}
    \caption{Evolução das citações ao \textit{dataset}.}
    \label{fig:evol:anual:citacoes:GNN@tucazorron}
\end{figure}

A figura \ref{fig:evol:anual:citacoes:GNN@tucazorron} apresenta a evolução da média de citações aos artigos do \textit{dataset}. Não há muita estabilidade na média anual de citações, até mesmo nos anos mais recentes.

\subsection{Interpretação das Citações}
Embora a taxa de crescimento de publicações anuais seja alta, ainda há instabilidade no que diz respeito a média de citações.

\subsection{\textit{Three-Field Plots (Sankey diagram)}}

As \textit{Three-Field Plots (Sankey diagram)} (plotagens do tipo ``Três Campos'') correlacionam três conjuntos de atributos em busca das afinidades encontradas no \textit{dataset}. Assim, são demonstrados os principais fluxos entre diferentes conjuntos.

\begin{figure}
    \centering
    \includegraphics[width=1\textwidth]{experiments/tucazorron/PesquisaBibliometrica/GNN/1.png}
    \caption{Plotagem ``Três Campos'' (Sankey plot) do \textit{dataset}: 20 Autores, Citações e 14 Palavras-Chave mais proeminentes.}
    \label{fig:GNN@tucazorron:ThreeFieldPlot}
\end{figure}

A figura \ref{fig:GNN@tucazorron:ThreeFieldPlot} apresenta a plotagem do tipo ``Três Campos'' realizada no \textit{dataset}, vinculando, ao centro, os 20 Autores mais proeminentes (AU), à esquerda, as 20 Citações mais frequentes (CR - Cited Records), e à direita, as 14 Palavras-Chave mais frequentes empregadas pelos autores.

\subsection{Interpretação da figura} \ref{fig:evol:anual:GNN@tucazorron:ThreeFieldPlot}
A maioria dos autores mais relevantes apresentados na plotagem são, aparentemente, dos Estados Unidos ou da China. Isso indica que é um assunto de extrema importância já que as duas maiores potências mundiais estão empenhadas em aprender mais sobre o tema.

\begin{figure}[htp]
    \centering
    \includegraphics[width=0.6\textwidth]{experiments/tucazorron/PesquisaBibliometrica/GNN/8.png}
    \caption{Rede de co-ocorrência de palavras aplicada ao \textit{dataset}.}
    \label{fig:GNN@tucazorron:countries}
\end{figure}

\section{Refinamento da Coleta de Dados}

Ao analisar a rede de co-ocorrência de palavras aplicada ao \textit{dataset}, foram identificadas algumas palavras no cluster que não correspondem ao assunto trabalhado. Assim, surgiu uma nova \textit{query} para levantar artigos mais apurados e relevantes.

\begin{figure}[htp]
    \centering
    \includegraphics[width=0.6\textwidth]{experiments/titofrota/PesquisaBibliometrica/Deepfakes/co-ocurrence.png}
    \caption{Rede de co-ocorrência de palavras aplicada ao \textit{dataset}.}
    \label{fig:DEEPFAKES@titofrota:redecoocorrencia}
\end{figure}

\begin{figure}[htp]
    \centering
    \includegraphics[width=0.6\textwidth]{experiments/titofrota/PesquisaBibliometrica/Deepfakes/co-ocurrence.png}
    \caption{Rede de co-ocorrência de palavras aplicada ao \textit{dataset}.}
    \label{fig:DEEPFAKES@titofrota:redecoocorrencia}
\end{figure}

A nova \textit{query} leva em conta novos termos e a busca foi feita nas publicações entre 2017 (quando o termo \textit{deepfake} surgiu) e 2022.

\lstinputlisting[numbers=left,basicstyle=\normalsize\ttfamily,caption={Query refinada de busca sobre Deepfakes.}]
{experiments/titofrota/PesquisaBibliometrica/Deepfakes/new-query.txt}


Além das justificativas para os termos usados entre as linhas 1 a 9, já descritas em \ref{MASSA:query},  justifica-se na listagem \ref{query20220203}, a inclusão da cláusula \textit{not (
 adsoption or molecular -dynamics or force -field
 or in -vitro or nanopartic* or in -vivo
 or aqueous -solution or protein or surface)}, entre as linhas 10 e 13 da \query, pois elas irão remover artigos não se enquadram no escopo da busca desejada, por usarem uma ou mais desses termos no título, resumo ou palavras-chave do artigo.
 
 Foram incluídas cláusulas como \textbf{deep-fake} e \textbf{face-swap*} com o intuito de encontrar mais conteúdos relacionados ao tema. Além disso, foram adicionadas novas cláusulas negadas para filtrar melhor os resultados, evitando que artigos médicos ou relacionados à outros temas façam parte do novo \textit{dataset}.
 
Usando a nova \textit{} de busca, foram recuperados 284 documentos. Sendo assim, aproximadamente 749 registros não se enquadravam na necessidade de busca.
Uma nova análise dos dados recuperados é apresentada a seguir.

\section{Nova Análise dos Dados}

\subsection{Nova filtragem de registros}

São aplicados dois filtros aos 284 documentos recuperados:
\begin{itemize}
    \item Remoção dos registros de documentos que não são artigos \textit{full paper}, isso é, artigos completos publicados em revistas;
    \item Remoção dos registros de artigos científicos que não fazem parte do \textit{core} da bibliografa, segundo a Lei de Bradford.
\end{itemize}

Após a filtragem, foram obtidos apenas 79 registros, que correspondem ao novo \textit{dataset}.
08
\subsection{Análise descritiva do \textit{dataset} refinado}

\begin{table}[]
    \centering
\csvautotabular[separator=semicolon
%,filter not strcmp={\csvcolii}{}
]{experiments/titofrota/PesquisaBibliometrica/Deepfakes/main-info.csv}
    \caption{Principais dados descritivos do dataset refinado.}
    \label{tab:DEEPFAKE:Main}
\end{table}

Logo, de acordo com a tabela \ref{tab:DEEPFAKE:Main}, é possível notar que entre 2017 e 2022 houveram 240 artigos publicados em 94 revistas diferentes.