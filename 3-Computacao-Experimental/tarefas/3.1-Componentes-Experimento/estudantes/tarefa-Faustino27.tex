\section{Respostas de Gabriel Faustino Lima da Rocha}

\subsection{A computação é uma atividade científica?}

Sendo a primeira das perguntas que vamos responder, usaremos o conceito de \gls{Computacao} e dos conceitos de ciência vistos anteriormente na matéria, podemos definir a computação como uma ciência, é um estudo organizado que utiliza de entradas (input), que podem ser qualquer fenômeno do universo, para gerar uma saída (output), que descreve ou prevê esse fenômeno. Por exemplo podemos utilizar de algum algorítimo para prever a trajetória de um objeto no espaço, por meio de cálculos físicos feitos rapidamente pelo computador. Além disso diversos estudos na computação feitos pela \gls{ComunidadeCientifica}, tentam aperfeiçoar os algoritmos utilizados em diversas áreas do conhecimento. A computação foi inclusive responsável por facilitar a \gls{ComunicacaoCientifica} a longas distâncias por meio da internet, possibilitando consultas e comunicação entre cientistas quase instantâneas ou o acesso a pesquisas já feitas várias área. 
%m
\subsection{A computação é uma ciência experimental?}
A computação é sim uma ciência experimental, pois além de ser utilizada por outras áreas da ciência para criar modelos e simular possibilidades, várias áreas da computação surgiram como experimentos como por exemplo a internet e armazenamento de dados, ambos poderiam ter sido feitos de diversas formas, mas através da experimentação foram aperfeiçoados para modelos melhores e mais otimizados. Dessa forma o modelo atual de computação surgiu após vários outros modelos computacionais que foram testados e modificados para melhor suprirem as necessidades do usuário.

\subsection{A computação é uma ciência empírica?}
A computação também pode ser considerada uma ciência empírica, pois a observação do comportamento de programas e como ele se comporta com diferentes entras nas quais ele foi feito para tratar, corrigindo problemas ou casos específicos nos quais o programa não sabe com lidar, pode ser considerado uma ciência empírica, pois muitas vezes só é possível corrigir e efetivamente perceber a ocorrência desses problemas nos algoritmos após a observação de seu comportamento perante as entradas esperadas na hora de sua construção.