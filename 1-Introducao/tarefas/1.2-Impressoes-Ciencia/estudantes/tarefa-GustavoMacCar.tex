\subsection{Minhas impressões iniciais sobre a ciência, por Gustavo Macedo de Carvalho}

O termo Ciência está relacionado a reconhecer a existência de algo, ter ciência de algo. Quase tudo que conhecemos hoje, como medicamentos, aparelhos eletrônicos, alimentos industrializados, meios de transporte, a escrita, etc, foi, ao longo da existência da nossa espécie, sendo descoberto ou inventado. Nos dias atuais, tudo que temos parece normal, pois já foi incorporado na nossa rotina, porém, desde os primórdios da espécie humana, lentamente foram feitas descobertas que possibilitaram os surgimento de novas descobertas ou invenções. Foi-se tomando ciência de novos fenômenos e, assim, o ser humano começava a fazer Ciência, com "C" maiúsculo.

Hoje, o termo Ciência pode ser entendido como \citep{wikipedia_scientific_2022} o uso do conhecimento aceito para facilitar o entendimento de novos aspectos do mundo. A Ciência influencia as pessoas e muda a forma como elas vivem. É possível perceber a influencia da Ciência nos grupos humanos ao observar como fatores como cultura, expectativa de vida, mortalidade infantil, acesso a saneamento básico e, até mesmo a \gls{Territorialidade} mudaram ao longo da História. Ainda que, a obervação de tais fatores também deve levar em consideração aspectos políticos, econômicos e sociais.