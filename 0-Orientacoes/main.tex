\chapter{Orientações Iniciais}

Leia atentamente as orientações a seguir, tendo em vista que lhe auxiliarão no melhor desempenho neste curso-disciplina. 

\section{Importância deste documento}

Este documento contém o registro de todas as evidências de aprendizagem suas, e dos demais estudantes que fazem parte do curso Computação Experimental.
Quando for editar este documento tome cuidado para sempre deixá-lo em plena condição compilável, e sem erros. Busque também resolver os \textit{warnings} causados pelas suas edições.

Todos os pacotes de trabalho que você depositar deverão ser acessíveis, de forma direta ou indireta, a partir do arquivo ``main.tex'', no diretório raiz.

\section{Uso sincronizado de Repositório no Github e Overleaf}

Todo o código deste documento está na plataforma Overleaf.com, mantido sincronizado como o repositório CE-20212 na plataforma Github.com, disponível na url \url{https://github.com/jhcf/Comput-Experim-20212}, que é o repositório \textbf{origin}.


O acesso de gravação no Repositório ser-lhe-á concedido pelo professor do curso ao final da primeira semana de aula.
Dessa forma, para poder realizar a disciplina todos os estudantes devem ter uma conta pessoal em github.com, bem como uma conta pessoal em overleaf.com. Inicie registrando o seu nome completo e \textbf{github username} no arquivo autores.tex, onde você também deve escolher uma cor para uso no registro de alterações, conforme disponível no pacote ``changes''.

\section{Ingresso no grupo Zotero ``RESIC'', para  compartilhar referências bibliográficas com a turma}

Você deve cadastrar-se como usuário do Zotero, registrar-se no grupo Zotero de compartilhamento de referências bibliográficas RESIC.
A inscrição no grupo é feita no endereço abaixo:
\url{https://www.zotero.org/groups/2465026/resic}

\begin{quote}
Opcionalmente, você deve instalar a versão \textit{standalone} do Zotero em uma máquina de seu uso, a fim de ter acesso facilitado à base de referências bibliográficas a ser usada durante a execução da disciplina. Caso não instale, poderá acessar a base diretamente pelo navegador.
\end{quote}

Após ingressar no grupo, visite a referência ao livro ``Complex Adaptive Systems: An introduction to computational models of social life'', que está dentro da coleção ``Complexity -> Complex Adaptive Systems (CAS)''.

Busque acessar o livro completo ou o seu sumário, como disponível em \url{https://www.google.com.br/books/edition/Complex_Adaptive_Systems/XQUHZC8wcdMC}. Se possível adquira o livro em formato digital ou impresso. Ele será usado como referência para discussões avançadas sobre simulação, mais adiante na disciplina.

Veja o que contém o sumário do livro, parte do prefácio e introdução, e use o Zotero para adicionar uma pequena nota com sua impressão inicial sobre o livro, vinculada a essa referência. A Nota deve conter o seu nome completo e qualquer observação que você ache relevante, sobre o livro referenciado.

Depois de registrar uma nota vinculada à referência do livro citado, use o Zotero para exportar essa referência no formato / estilo ABNT [use a opção ``Create Bibliography'' no Zotero online, ou ``criar bibliografia a partir do item selecionado''  no Zotero standalone], e deposite o texto da referência formatada no texto online da tarefa T1 (ver Moodle/Aprender da disciplina).

\section{Depósito dos pacotes de trabalho referentes à execução das tarefas}

A execução de toda e qualquer tarefa pontuável neste curso é feita por meio do depósito de um pacote de trabalho no repositório, contendo textos, programas de computador e outros dados coletados e (ou) analisados. Todos os textos e códigos no repositório devem estar em formato não compactado. Apenas os arquivos de dados muito grandes devem ser compactados em formato zip. Não usar rar, gz etc. 

Todo pacote de trabalho a ser avaliado precisa estar integralmente armazenado no Repositório CE-20212 / origin (e consequentemente sincronizável com o Overleaf).

\subsection{Onde depositar os pacotes de trabalho?}

Toda e qualquer inserção de texto, programa de computador, dados, enfim, qualquer documento, feito por estudante, deve ocorrer em um dos seguintes pontos:
\begin{enumerate}
    \item Dentro de um subdiretório com o código da tarefa, no diretório de ``experiments'' do estudante, onde o nome do diretório de \texttt{experiments} de um estudante é o seu github username (veja, por exemplo, o professor, que tem como github username: jhcf);
    \item Dentro dos diretórios ``tarefas'', nos diretórios dos temas de estudo;
    \item No arquivo ``autores.tex'';
    \item No arquivo ``packages-estudantes.tex'', onde eventualmente podem ser inseridos novos pacotes para apoiar o uso de algum recurso específico; e
    \item Na substituição do arquivo RESIC.bib por outro mais recente, obtido pela exportação completa da biblioteca RESIC que se encontra na plataforma Zotero, na url \url{https://www.zotero.org/groups/2465026/resic}.
\end{enumerate}

Não serão avaliados os pacotes de trabalhos entregues em local distinto do especificado, ou não acessíveis por meio do output em PDF, resultante da compilação de ``main.tex''.

\section{Entrega das tarefas}

Toda pontuação concedida ao/à estudante será feita mediante:
\begin{enumerate}
    \item Sincronização plena entre o documento no Overleaf.com e o repositório no Github.com;
    \item O registro de execução da correspondente tarefa no ambiente Moodle do curso, feita pelo estudante até o limite de prazo informado;
\end{enumerate}

Se a tarefa é feita em grupo, cada um dos membros do grupo deve obrigatoriamente fazer o registro da execução da tarefa, com as mesmas informações. Os membros do grupo devem evitar qualquer duplicação de código e textos no Overleaf. Ou seja, depositar os pacotes de trabalho referentes a cada tarefa em apenas um dos diretórios de experimentos de um dos membros do grupo. Até o final do curso, todos os membros de um grupo devem aparecer como produtores de pelo menos um pacote de trabalho feito pelo grupo.

\section{Escrevendo o texto}

Cuidados ao escrever texto:
\begin{description}
\item [Reconhecimento de autorias] Faça citações a textos e ideias que não são de sua autoria, usando referências registradas no grupo Zotero RESIC. Sempre informe os nomes completos dos autores dos relatórios, logo após o título do relatório;
\item [Grafia] A boa ortografia e gramática são essenciais à valorização de um trabalho. Descuido com essa questão revela, de forma geral, descuido e (ou) desinteresse pelo próprio trabalho, influenciando a pontuação obtida;
\item [Referências a códigos e dados] Toda e qualquer de escrita de relatório deverá fazer referência explícita ao diretório no Repositório, onde se encontram os códigos e dados usados para produção do relatório. No ambiente Moodle não será aceita a entrega de arquivos compactados contendo os resultados de realização das tarefas;
\end{description}

\section{Organização do ambiente experimental}

A organização do ambiente experimental de cada estudante é essencial para a realização adequada das tarefas.
Alguns dos problemas típicos que refletem descuido com a organização do trabalho de laboratório, são:

\begin{enumerate}
\item Problemas com merge durante operação no repositório git;
\item Uso inadequado de caracteres com acentuação nos nomes dos arquivos;
\item Não informar os ponteiros adequados, seja por meio das tags \LaTeX~ input e (ou) includepdf, de modo que a tarefa fica inacessível e invisível na tabela de conteúdos gerada a partir da compilação do documento main.tex.
\end{enumerate}

\subsection{Problemas com merge durante operações de push no Git}

Cada estudante é inteiramente responsável por fazer os merges manuais para entrega de sua tarefa no Repositório, caso o branch correspondente ao seu pacote de trabalho não consiga ser feito de forma automática. 

Estudantes que deixarem merges em conflito, prejudicando o processo de envio dos trabalhos dos demais, poderão perder pontos pela não entrega do trabalho no prazo, bem como pelo conflito causado para os demais estudantes do curso-disciplina. 

Busque informações sobre como resolver merges no livro Pro Git \cite{chacon_pro_2014}, ou em  urls como \url{https://www.zotero.org/groups/2465026/resic/collections/C7BG9S2W/items/HCY5X8PT/note/9M2BFISL/collection}

\subsection{Uso inadequado de nomes para arquivos}

Devido ao fato de que estaremos trabalhando em um ambiente laboratorial compartilhado, com uso de muitas linguagens combinadas, como \LaTeX, Python e R, entre outras, é fundamental adotar um padrão de nomeação para arquivos e diretórios. Atualmente esse padrão é composto pelas seguintes regras:
\begin{enumerate}
    \item Não use caracteres de acentuação em nomes de arquivos;
    \item Não use espaço em branco em nomes de arquivos;
    \item Não use underline em nomes de arquivos. Usar hífen em vez de Underline;
    \item Para nomes de arquivos longos use a \url{https://pt.wikipedia.org/wiki/CamelCase}, e eventualmente misture com o uso de hifens, para melhor legibilidade.
\end{enumerate}

\subsection{Não colocar os ponteiros adequados para que a tarefa seja visível}

É necessário que todo trabalho a ser avaliado esteja visível a partir da tabela de conteúdos gerada pela compilação do documento main.tex, sejam em um capítulo, seção ou subseção em \LaTeX.

