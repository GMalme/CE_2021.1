\section{Respostas de Gustavo Macedo de Carvalho}

Antes de argumentar sobre a atividade profissional de um cientista da computação utilizar todos os componentes de um experimento científico, temos que definir qual é essa atividade profissional. Portanto, para efeitos desse texto, cientista da computação é alguém que produz ciência utilizando a computação como um objeto de estudo.

Parando agora para refletir sobre a computação ser uma atividade científica, temos que levar em consideração o contexto em que essa "computação" é realizada. Podemos considerar um caso em que a computação é utilizada somente como ferramenta na produção da ciência, nesse caso, a computação, muito provavelmente, não está envolvida na criação da hipótese e, portanto, não constitui uma atividade cientifica por si só. Porém, podemos considerar um caso em que a computação serve como o objeto fim de uma verdade que se quer descobrir, quando ela é o próprio objeto de estudo. Nesse caso, temos que a computação está envolvida na criação da hipótese, no planejamento da experimentação, do experimento em si e da análise dos dados, constituindo assim uma atividade científica.

Quando pensamos na profissão de um cientista da computação conforme definido aqui, podemos afirmar que sim, a atividade profissional se encaixa no cenário em que a computação constitui uma atividade científica, ainda que no mundo real, os cientistas da computação possam auxiliar na produção de ciência de outras áreas do conhecimento, de modo que a computação assume um papel mais parecido com o de uma ferramenta, o que não constitui atividade científica, muito embora possa ser parte de uma atividade científica.