\section{Respostas de Mateus de Paula Rodrigues\label{tarefa-MoustacheGolem-componentes-eperimento}}

\section{A atividade profissional de um cientista da computação
utiliza todos os componentes de um experimento científico?}

Não, vamos pensar mais um pouco,  vamos considerar o que são os componentes de um experimento cientifico? E em seguida, o que um cientista da computação utilizaria?

\subsection{Componentes de um experimento cientifico.}

para nossa discussão, vamos considerar os Componentes de um experimento cientifico como sendo:

\begin{itemize}
    \item \gls{Hipotese}
    \item Teoria
    \item Experimentação e 
    \item Analise
\end{itemize}


Esses componentes são inspirados no ciclo do método cientifico onde, primeiro e pensado uma hipótese e a partir dessa uma teoria é formada, em seguida é executado experimentação para obter dados, uma analise sobre esses dados confirmara ou não a teoria original, se confirmada a teoria é fortalecida, se não a teoria é quebrada e se torna necessário a criação de nova hipótese com base nos resultados anteriores para criação de uma nova teoria Mais forte.

\subsection{O que um cientista da computação utilizaria:}
Um cientista da computação normalmente executa atividades envolvendo manutenção e criação de software, a partir daqui jé é possível observar que apesar do nome, cientistas da computação, estão geralmente mais ocupados desenvolvendo do que engajando cientificamente com sua área, apesar disso é possível traçar alguns aspectos do processo cientifico, principalmente em analise, o processo computacional envolve Criação de \gls{Dado}s de maneira intrínseca, assim analise desses Dados é parte frequente da computação.

Experimentação também se encaixa, essa faz parte de diversos processos, até artísticos e musicais, então não é difícil encontra-la na criação de software, já os passos restantes, criação de hipótese e teoria, não são necessariamente componentes utilizados por cientista da computação, afinal esses normalmente não estão com âmbitos de descobrir algo sobre um aspecto do mundo, onde normalmente a criação de Hipóteses acontece, e sem hipótese e difícil ter uma teoria.

Assim, é fácil concluir que não, cientistas da computação, apesar do nome, não necessariamente engajam cientificamente com seu fenômeno de interesse.