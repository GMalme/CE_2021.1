\subsection{Minhas impressões iniciais sobre a ciência, por Gustavo Rodrigues dos Santos}

Desde o primórdio da humanidade a ciência sempre foi uma área intrínseca ao ser humano, sempre tentamos explicar o porquê das coisas que acontecem ao nosso redor, os denominados fenômenos ( \gls{fenomeno}). Dessa forma, podemos analisar que a resposta para a pergunta \textit{O que é ciência?} vem se modificando juntamente à humanidade. Inicialmente, a ciência podia se apresentar na forma de um mito o qual retratava a cosmogonia de um povo. Porém ao longo da evolução da humanidade, a ciência acompanha a evolução, passando desde os filósofos gregos até a modernidade, quando ocorre um marco para a história da ciência, a Revolução Científica. Desse modo, a ciência sai de um estado nascente rumo a um estado final, sendo este o estado o qual a produção do conhecimento esteja totalmente e precisamente correta acerca dos fenômenos estudados.

Atualmente, nos encontramos no meio dessa transição, nosso conhecimento ciêntifico é repleto de erros, e estamos numa fase onde a verdade não é absoluta, como relatado por \citet{fernandes_consideracoes_2021}. Assim, como qualquer outro setor da sociedade, a ciência vivencia regularmente crises não só ligadas à \textit{incapacidade dos paradigmas cientı́ficos vigentes para explicar a realidade} \citep{fernandes_consideracoes_2021}. Um grande exemplo disso são os movimentos anti-ciência vistos na atual conjuntura onde é notável a crise da elitização da ciência. Esse fenômeno em nada se relaciona à não competência das atuais teorias em explicar os eventos, mas sim com o descolamento da população da comunidade acadêmica. Os cientistas são seres humanos, e portanto, naturalmente, possuem defeitos e consequentemente dotados de vieses e conceitos que muitas vezes refletem a própria sociedade a qual estão inseridos. Logo, para dar continuidade ao processo espontâneo da ciência, nós como sociedade devemos buscar formas de superar nossos vieses 


% A ciência da computação \citep{baldwin_three-fold_1994}, uma ciência específica dentre as várias existentes, apresenta-se muito relacionada com a filosofia da ciência \citep{floridi_blackwell_2004}, ou como uma meta-ciência (A filosofia é um campo do pensamento humano que é base para a ciência). Um exemplo que comprova essa versatilidade - ou até mesmo universalidade - da Ciência da Computação é o emprego da computação, sendo como exemplo mais concreto o uso de simulação, como método de pesquisa empírica, isso é pesquisa baseada em coleta e análise de dados \citep{tedre_experiments_2014}. A versatilidade do computador, decorrente do software, inclusive de simulação, gera uma plataforma empírica - o computador é prodigioso na coleta, cálculo e geração de dado - para o desenho e execução de quase todo tipo de experimento científico em quaisquer dos outros campos do conhecimento. Situações mais específicas podem ser vistas no campo da \gls{EA}.
