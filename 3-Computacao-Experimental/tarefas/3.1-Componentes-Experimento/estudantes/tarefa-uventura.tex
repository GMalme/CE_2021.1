\section{Respostas de Ualiton Ventura da Silva\label{tarefa-ualiton-ventura-componentes-eperimento}}

Vários aspectos devem ser considerados para descrever uma atividade científica, como pode ser observado em “Principais Componentes de um Experimento”, dos aspectos que compõem uma atividade científica estão as características de observar um \gls{fenomeno}, hipotetizar, criar modelos, observar causas e efeitos, ser capaz de reproduzir o experimento novamente, demonstrar e manipular o experimento observado, coletar dados e os analisar logicamente. Tratando-se da computação, é necessário que seja considerado o fato de que a mesma é um campo extremamente vasto e composto por várias ramificações com objetivos diferentes, portanto, para analisar a computação de maneira adequada é necessário entender sua definição.

Observando-se de uma maneira geral, tem-se que computação pode ser definida como “a busca de solução para um problema a partir de entradas (inputs), de forma a obter resultados (outputs) depois de processada a informação através de um algoritmo”[Wikipédia]. Desta forma, a ação de “computar” poderá não ser definida como uma atividade científica, mas sim como um meio que possibilita chegar a uma atividade científica, isto é, caso considere apenas o ato de a partir de entradas criar saídas.

Apesar da mesma não ser em si uma atividade científica, observa-se que através da formulação de um \gls{Algoritmo}, alguns dos aspectos de uma atividade científica podem ser considerados. Um dos fatores que deve ser colocado em sua definição é a de que um algoritmo visa a resolução de um problema, seja ele abstrato ou concreto.

Com a elaboração de algoritmos, é objetivado a criação de um determinado fenômeno desejado, para a formulação deste resultado é necessário que elabore-se \gls{hipotese}s sobre o que criaria tal comportamento, assim, vários modelos e exemplos podem ser obtidos, por fim, produzem-se saídas através destes modelos que são observadas e verifica-se e analisa-se as saídas obtidas, que podem estar certas ou erradas. Para a validação de um algoritmo pode-se utilizar-se da matemática como meio de prova e demonstração, contudo, este procedimento nem sempre é o usual.
 
Analisando as definições de um algoritmo, o mesmo possui diversos aspectos de uma atividade científica, contudo em relação a capacidade de ser possível reproduzir um experimento, isto é um fator que poderá ocorrer mas que depende do que deseja-se reproduzir do algoritmo, seu comportamento poderá ser replicado, mas suas respostas podem não ter coerência com o fenômeno desejado, assim, para a reprodutibilidade depende exclusivamente do que está sendo analisado. Portanto, a concepção de algoritmos computacionais poderá ser observada como uma atividade experimental, assim, a computação possui característica experimental.

Como a computação trata-se da produção de saídas, isto depende de observação de resultados, tornando assim uma atividade empírica, através da elaboração de seus algoritmos.

Observando os fatos analisados, um cientista da computação pode utilizar diversos dos aspectos de um experimento científico, contudo, apesar de ser possível empregar tal metodologia talvez parte destes aspectos poderá não ser levado em consideração, sendo que a observação de fenômeno é uma das características principais que pode acabar não sendo empregada, considerando que o único fenômeno é o obtido através dos comportamentos desejados, como os que surgem através de algoritmos. Outro fator que pode não ser utilizado trata-se de formalismos e metodologia ordenada.


