\chapter{Análise Bibliográfica sobre Computação Quântica, por Raylan da Silva Sales\label{chap:bibliometria:jhcf}}

\section{Planejamento do estudo}

O planejamento o  desenho do estudo deve descrever as motivações, questões de interesse, escopo, limitações e objetivos do trabalho.

O planejamento do estudo deve motivar o tema escolhido e o interesse do autor.

No caso do meu trabalho, as perguntas que o nortearam foram:
\begin{itemize}
    \item Qual a base de conhecimentos científicos produzida em torno do tema simulação multiagente voltada à compreensão de fenômenos sociais, com ênfase em métodos experimentais? 
    \item Como a simulação multiagente tem sido usada para compreender fenômenos sociais, com ênfase em métodos experimentais? 
    \item Quais os principais termos e conceitos ligados à frente de pesquisa no tema simulação multiagente de fenômenos sociais, com ênfase em métodos experimentais? 
    \item Qual a estrutura social da comunidade, se é que existe, que pesquisa sobre o tema simulação multiagente de fenômenos sociais, com ênfase em métodos experimentais?
\end{itemize}

Um termo que vem ganhando forma ao longo dos últimos anos é a "Computação Quântica". Quando usado para referenciar os avanços tecnológicos, esse termo se refere a ciência que elabora e estuda o desenvolvimento de produções tecnológicas relacionadas a softwares e algoritmos. 

A computação quântica transcende os limites do desenvolvimento tecnológico que haviam em tempos passados e possibilitam o avanço de um futuro longínquo de tecnologias avançadas, por exemplo no avanço da criação de super máquinas capazes de resolver cálculos e tarefas que demandam de um grau de complexidade elevadíssimo.

Para abordar o tema de uma forma mais crua, é importante se atentar a diversos conceitos e questões básicas relacionadas a Computação Quântica, como:

\begin{itemize}
    \item Como funciona a Computação Quântica?
    \item O que é um computador quântico?
    \item Relações entre a computação e a mecânica quântica.
    \item Os avanços e impactos da computação quântica na atualidade.
\end{itemize}

\subsection{O que já existe de pesquisa bibliométrica sobre esse tema?}

Durante os últimos anos foram feitas várias pesquisas a cerca da Computação Quântica. Este tema está em constante evolução, e com isso vários autores estão interessados em escrever sobre, não só pra expressar suas ideias, como também para expor resultados obtidos em experimentos e análises feitas através da Computação Quântica.

\subsection{Uso do Bibliometrix e Biblioshiny}
Para realização da observação e análise de vários resultados, foram utilizados o Biliometrix e Biblioshiny, que foram necessários para adquirir informações um pouco precisas as análises descritivas e pesquisas feitas acerca do tema.

\subsection{Limitações}
A produção descritiva do trabalho foi feita durante uma semana. Esse demora se deu por conta da dificuldade em utilizar as ferramentas necessárias para a conclusão das análises. Porém, assim que o autor se tornou capaz de utilizá-las com maestria, a conclusão do trabalho se fez real.

\section{Coleta de Dados}

A coleta de dados feita usando o WoS no dia 06 de fevereiro de 2022, acessado por meio do Portal de Periódicos da CAPES.

Foram feitas pesquisas utilizando o termo \textbf{Neuroscience and Artificial Intelligence} para buscar resultados que condissessem com o tema desta análise.


