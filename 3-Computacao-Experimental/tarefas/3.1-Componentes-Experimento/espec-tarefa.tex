\section{Especificação da Tarefa: Reflexão sobre os Fundamentos da Computação Experimental}

\subsection{Motivação}

Veja os slides disponíveis em \ref{fundamentos:ce} e a(s) correspondente(s) aula(s) gravada(s), feita(s) pelo professor, em fevereiro de 2022. Elas abordam os componentes de um experimento. A atividade de um cientista da computação utiliza todos esses componentes? Como e por que sim? Como e por que não? 

\subsection{Pergunta a ser respondida por você}

Com base no apresentado e discutido, crie em \textbf{3-Computacao-Experimental / tarefas / 3.1-Componentes-Experimento / estudantes} o texto de uma seção contendo o seu nome, e escreva nela um texto com pelo menos 250 palavras, que ofereça uma resposta ou reflexão sobre a seguinte questão central:
\begin{quote}
A atividade profissional de um cientista da computação utiliza todos os componentes de um experimento científico?
\end{quote}


\subsection{Três perguntas basilares, para iniciar}

Algumas perguntas basilares que permitem o desenvolvimento pleno de uma resposta são a seguir apresentadas:
\begin{itemize}
    \item A computação é uma atividade científica? 
    \item A computação é uma ciência experimental? 
    \item A computação é uma ciência empírica? 
    \item Por que sim e por que não? 
Justifique. 
\end{itemize}

\subsection{O que precisa ser feito na tarefa}

Na sua resposta à questão central, você precisa necessariamente referenciar pelo menos três termos do glossário, sendo:
\begin{itemize}
    \item Dois termos de glossário já existentes;
    \item Um novo termo que você vai criar em \textbf{ 1-Introducao / tarefas / 1.1-Glossario / estudantes / tarefa-\githubusername}, contendo uma nova definição e exemplo de algum termo usado nos slides disponíveis em \ref{fundamentos:ce}.
\end{itemize}

\subsection{Exemplo Inicial}

Veja como exemplo inicial, o texto de provocação feito pelo professor em \ref{tarefa-jhcf-componentes-eperimento}, em resposta às três perguntas basilares, e
desenvolva seus próprios argumentos.
