\section{Respostas de Leonardo Alves Riether\label{tarefa-LeoRiether-componentes-eperimento}}

Nesta seção, será discutida a resposta para a pergunta "A atividade profissional de um
cientista da computação utiliza todos os componentes de um experimento científico?". Antes
de discutirmos a resposta em si, precisamos primeiro entender quais as possíveis
atividades que um cientista da computação pode exercer, para somente então verificar se
elas de fato utilizam os componentes de um experimento científico ou não.

\subsection{O que faz um cientista da computação}

Sendo profissionais da área de \gls{Computacao}, cientistas da computação de modo geral
trabalham na resolução de problemas para determinadas entradas, por meio de algoritmos que
constroem uma resposta. A presença de algoritmos concretos, de natureza construtiva , e
que levam em conta a utilização de memória e tempo de execução, é um dos fatores que
distinguem a \gls{CienciaComp} da Matemática, além de deixar a área mais "voltada ao mundo
real".

Essa definição, no entanto, é bem ampla. Um cientista da computação pode trabalhar tanto
em áreas puramente teóricas; como em Teoria da Computação ou elaboração de Estruturas de
Dados; quanto em áreas práticas, por exemplo desenvolvendo software ou analisando a
segurança de sistemas computacionais. Notamos que as áreas mais práticas possuem bastante
interação com o mundo externo aos sistemas computacionais. É preciso lidar com dados
inseridos por usuários, mensagens recebidas de outros computadores, falhas no hardware,
etc. Mesmo assim, as áreas teóricas produzem conhecimento que deve ser aplicável à
prática, e portanto também precisam lidar bem com esse tipo de situação.

Tendo isso, podemos comparar as atividades exercidas pelos cientistas da computação com os
experimentos científicos.

\subsection{Componentes de um Experimento Científico}
Um \gls{experimento} científico possui vários componentes que o definem:

\begin{enumerate}
    \item 
\end{enumerate}