\chapter{Tarefa 4: Análise Bibliométrica com apoio de R/R Studio e Bibliometrix}

A tarefa vale 20 pontos, e consiste em produzir individualmente uma análise bibliométrica inicial, abordando um tema de computação que  interessa ao autor/à autora. 

A análise bibliométrica, o produto da tarefa, deve estar apresentada num capítulo da parte \ref{part:estudos:exploratorios}, no relatório da turma no Overleaf, no diretório:

{\footnotesize
\verb|2-Analise-Exploratoria-Dados/tarefas/2.2-Pesquisa-Bibliografica/estudantes/|
}

, em um arquivo de nome ``tarefa-\githubusername.tex''. 

Ver um exemplo com a complexidade referencial da tarefa, no arquivo ``tarefa-jhcf.tex'', no diretório:

{\footnotesize
\verb|2-Analise-Exploratoria-Dados / tarefas / 2.2-Pesquisa-Bibliografica / estudantes/|, 
}

na Seção \ref{MASSA@jhcf:questoes} até a Seção  \ref{MASSA@jhcf:analise}, do capítulo \ref{chap:bibliometria:jhcf}. 

A análise deve conter texto, figuras, tabelas  etc. Todas as figuras, tabelas, dados etc, incluídas no texto, devem estar montadas no diretório:

{\footnotesize
\verb|/experiments / <githubusername> / AnaliseBibliometrica / <tema-pesquisa>|. 
}

\verb|<tema-pesquisa>| é um nome curto, em formato \textit{CamelCase}, que você dará ao diretório onde os dados, figuras, etc, do seu estudo, estarão montados. Não use caracteres acentuados em nomes de arquivos no Overleaf.

Veja um exemplo de dados e figuras em:

{\footnotesize
\verb|/experiments / jhcf / PesqBiblogr / SimulacaoMultiagente|.
}

Para a compilação da tarefa é necessário fazer o \textit{input} do arquivo de texto em:

{\footnotesize
\verb|2-Analise-Exploratoria-Dados / tarefas / 2.1-Pesquisa-Bibliografica / estudantes / main.tex|.
}

O \textit{dataset} de análise bibliométrica deve conter, minimamente, 250 (500) registros bibliográficos.

A análise precisa ser realizada e descrita em cinco etapas, e deve seguir as orientações feitas em \ref{metodo:analise:bibliografica}, e no detalhamento proposto por \citet{aria_bibliometrix_2017}:
\begin{enumerate}
    \item \textit{Study design} (Planejamento do estudo);

    \item  \textit{Data collection} (Coleta de dados);

    \item \textit{Data analysis} (Análise dos dados);

    \item \textit{Data visualization} (Visualização dos dados representados de forma gráfica, em vários formatos, vários tipos de diagrama);

    \item  \textit{Interpretation} (Interpretação, traçar conclusões, reflexões, sugestões de aprofundamento).
\end{enumerate}

Na edição do \LaTeX~ deve-se atentar aos seguintes aspectos:
\begin{enumerate}
    \item Todos os dados, inclusive as imagens, usadas na produção da análise, deve estar inseridos no diretório de experimentos individuais do estudante;
    \item Todas as figuras e gráficos inseridos na análise devem ser individualmente rotulados com \textit{label}, sem conflitar com os \textit{labels} já existentes, devem ter um título (\textit{caption}) descritivo do que apresenta a figura e o nome do \textit{dataset} usado, e também a figura/gráfico deve ser explicitamente descritas e citadas, usando referencia (ref);
    \item As figuras deve ser automaticamente dimensionadas, e eventualmente rotacionadas,  para se adaptar à largura e (ou) altura do texto na página. 
\end{enumerate}

\section{Critérios opcionais}

Os critérios de avaliação a seguir não foram explicitados aos estudantes quando da apresentação da tarefa original:
\begin{enumerate}
    \item O tema deve ser explicado, com pelo menos 200 palavras;
    \item Na apresentação do tema de estudo devem ser citadas pelo menos duas referências bibliográficas já previamente conhecidas pelo autor. Não se pode usar a Wikipedia como referência;
    \item Deve haver pelo menos um gráfico ou tabela para explorar cada uma das três estruturas do conhecimento: conceitual, social e intelectual;
    \item Todas as perguntas de pesquisa formuladas no início do estudo devem ser avaliadas quanto ao atingimento de respostas, nas conclusões;  
    \item O capítulo precisa citar e apresentar brevemente, pelo menos cinco referências bibliográficas relevantes, e presentes no \dataset\ explorado.
\end{enumerate}