\chapter{Análise Bibliográfica sobre Computação Quântica, por Raylan da Silva Sales\label{chap:bibliometria:jhcf}}

\section{Planejamento do estudo}

O planejamento o  desenho do estudo deve descrever as motivações, questões de interesse, escopo, limitações e objetivos do trabalho.

O planejamento do estudo deve motivar o tema escolhido e o interesse do autor.

No caso do meu trabalho, as perguntas que o nortearam foram:
\begin{itemize}
    \item Qual a base de conhecimentos científicos produzida em torno do tema simulação multiagente voltada à compreensão de fenômenos sociais, com ênfase em métodos experimentais? 
    \item Como a simulação multiagente tem sido usada para compreender fenômenos sociais, com ênfase em métodos experimentais? 
    \item Quais os principais termos e conceitos ligados à frente de pesquisa no tema simulação multiagente de fenômenos sociais, com ênfase em métodos experimentais? 
    \item Qual a estrutura social da comunidade, se é que existe, que pesquisa sobre o tema simulação multiagente de fenômenos sociais, com ênfase em métodos experimentais?
\end{itemize}

Um termo que vem ganhando forma ao longo dos últimos anos é a "Computação Quântica". Quando usado para referenciar os avanços tecnológicos, esse termo se refere a ciência que elabora e estuda o desenvolvimento de produções tecnológicas relacionadas a softwares e algoritmos. 

A computação quântica transcende os limites do desenvolvimento tecnológico que haviam em tempos passados e possibilitam o avanço de um futuro longínquo de tecnologias avançadas, por exemplo no avanço da criação de super máquinas capazes de resolver cálculos e tarefas que demandam de um grau de complexidade elevadíssimo.

Para abordar o tema de uma forma mais crua, é importante se atentar a diversos conceitos e questões básicas relacionadas a Computação Quântica, como:

\begin{itemize}
    \item Como funciona a Computação Quântica?
    \item O que é um computador quântico?
    \item Relações entre a computação e a mecânica quântica.
    \item Os avanços e impactos da computação quântica na atualidade.
\end{itemize}

\subsection{O que já existe de pesquisa bibliométrica sobre esse tema?}

Durante os últimos anos foram feitas várias pesquisas a cerca da Computação Quântica. Este tema está em constante evolução, e com isso vários autores estão interessados em escrever sobre, não só pra expressar suas ideias, como também para expor resultados obtidos em experimentos e análises feitas através da Computação Quântica.

\subsection{Uso do Bibliometrix e Biblioshiny}
Para realização da observação e análise de vários resultados, foram utilizados o Biliometrix e Biblioshiny, que foram necessários para adquirir informações um pouco precisas as análises descritivas e pesquisas feitas acerca do tema.

\subsection{Limitações}
A produção descritiva do trabalho foi feita durante uma semana. Esse demora se deu por conta da dificuldade em utilizar as ferramentas necessárias para a conclusão das análises. Porém, assim que o autor se tornou capaz de utilizá-las com maestria, a conclusão do trabalho se fez real.

\section{Coleta de Dados}

Os dados utilizados durante todo o processo de análise, foram coletados utilizando o WoS, cujo acesso foi feito por meio do Portal de Periódicos da CAPES.

Para realização das pesquisas feitas durante toda a produção, foi usado o termo \textbf{Quantum Computing}, foi feito o uso do termo em inglês, pois foi notado que através do uso do temo nessa forma, mais produções em torno do tema eram encontradas.

\section{Análise dos Dados}

\subsection{Filtragem de registros}

Através dos dados gerados pelo bibliometrix, usando o biblioshiny, foram obtidos 574 resultados, dentre esses resultados, estão livros, documentos e etc. Foram Aplicados os filtros para ter uma visão mais abrangente dos dados analisados, fazendo assim o \dataset exibiu artigos na integra. No total foram obtidos 1000 documentos como resultado pelo \dataset.

\subsection{Análise descritiva do \dataset }

A análise bibliométrica descritiva faz uma descrição inicial do \dataset\  . Para explicação detalhada de como são calculadas as diversas taxas geradas pelo Bibliometrix veja a documentação do \textit{package} a partir da página \url{https://cran.r-project.org/web/packages/bibliometrix/index.html}. A análise bibliométrica descritiva é gerada pela função \texttt{biblioAnalysis}.

As informações mais gerais sobre o \dataset\ são as seguintes:

\begin{description}
    \item [\textit{Timespan}] Os artigos que atenderam aos critérios de busca e filtragem foram publicados a partir de 1991, até 2022. Ou seja, não foram encontrados registros entre 1945 e 1987.
    \item [\textit{Sources (Journals, Books, etc)}] São 574 fontes de informação que publicaram os documentos recuperados no \dataset\.
    \item [\textit{Average years from publication}] A média do tempo de publicação dos artigos no \dataset\ é de 9,11 anos.
    \item [\textit{Average citations per documents}] Cada artigo no \dataset\ foi citado, em média 14,37 vezes
    \item [\textit{Average citations per year per doc}] Após publicado, cada um dos 574 artigos do \dataset\   foram citados 1.652 vezes por ano, em média.
    \item [\textit{References}] O \dataset\ contém 20.835 referências citadas (tags CR).
    \item [\textit{Keywords Plus (ID)}] 678 distintas palavras-chave do tipo Keywords Plus (ID)\footnote{\textit{KeyWords Plus} são ``termos de índice gerados automaticamente a partir dos títulos de artigos citados. Os termos do KeyWords Plus devem aparecer mais de uma vez na bibliografia e são ordenados de frases com várias palavras a termos únicos. O KeyWords Plus aumenta o número de resultados tradicional de palavras-chave ou títulos.'' Fonte: \url{https://images.webofknowledge.com/WOKRS410B4/help/pt_BR/WOS/hp_full_record.html}} foram encontradas no \dataset\   . 
    \item [\textit{Author's Keywords (DE)}] 1.665 distintas palavras-chave indicadas pelos autores foram encontradas no \dataset\  .
    \item [\textit{Authors}] 2.298 distintos nomes de autores foram encontrados no \dataset\  \footnote{Um mesmo autor pode ter uma ou mais diferentes grafias no \dataset\  , e serão reconhecidos dois ou mais autores diferentes, embora de fato sejam apenas um. Isso significa que a quantidade de \textbf{nomes de autores} equivale à quantidade de \textbf{autores}. Adicionalmente, é possível que distintos autores sejam reconhecidos com o mesmo nome, isso é, que sejam homônimos. Ou seja, o \dataset\   em geral conterá erros de contagem na quantidade de autores reais.}.
    \item [\textit{Author Appearances}] 2.298 distintos (nomes de) autores foram encontrados 2.926 vezes, como autores de artigos.
    \item [\textit{Authors of single-authored documents}] Dentre os 1.116 distintos (nomes de) autores encontrados, 117 deles editaram artigos individualmente, isso é, sem co-autores.
    \item [\textit{Authors of multi-authored documents}] Dentre os 1.116 distintos (nomes de) autores encontrados, 999 deles editaram artigos com um ou mais co-autores"
    \item [\textit{Single-authored documents}] Dentre os 349 documentos presentes no \dataset\  , 123 foram escritos por um único autor, e os 226 restantes foram elaborados em co-autoria.
    \item [\textit{Documents per Author}] Dentre os 1.116 distintos (nomes de) autores, cada um publicou em média 0,313 artigos.
    \item [\textit{Authors per Document}] Cada um dos 349 documentos presentes no \dataset\  foram autorados com 3,2 autores em média ($1.116 / 349 = 3,19$).
    \item [\textit{Co-Authors per Documents}] As 1.187 aparições de (nomes de) autores (``Author Appearances''), se distribuem, em média 3.4 vezes para os 349 documentos do \dataset\ .
    \item [\textit{Collaboration Index}] Os 1.116 (nomes de) autores que editaram artigos com um ou mais co-autores, colaboraram em media 4.42 vezes para editar os 349 artigos elaborados em co-autoria.
\end{description}



