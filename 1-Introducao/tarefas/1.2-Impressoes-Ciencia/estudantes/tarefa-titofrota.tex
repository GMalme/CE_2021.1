\subsection{Minhas impressões iniciais sobre a ciência, por Ítalo Eduardo Dias Frota}

A ciência é um escopo de \gls{Conhecimento} e um processo. A área se apoia na busca e aplicação dos conhecimentos a respeito das esferas sociais e naturais, seguindo uma \gls{Metodologia} bem definida baseada em evidências que descrevam, expliquem e possam prever um \gls{fenomeno}. Entretanto, definir a ciência não é uma tarefa fácil devido a pluralidade de aplicações e abordagens que permeiam a \gls{ComunidadeCientifica}.

Um dos aspectos mais importantes no campo científico é o da metodologia científica, que permite a formulação de hipóteses, experimentos  e verificações. Os resultados obtidos devem ser reproduzidos através de artigos e publicações que auxiliem na propagação dos pontos observados, instigando novas descobertas e indagações. Sendo assim, o processo é extremamente autocorretivo, pois está em constante evolução.

Sem a ciência e o pensamento científico, a humanidade enfrentaria dificuldades em larga escala. As descobertas e previsões documentadas pelos membros da comunidade são de extrema importância para os avanços nas mais diversas áreas, desde a saúde, até a educação e tecnologia. É imprescindível que seja dada a devida atenção às observações coletadas pela ciência, caso contrário, a humanidade estará sempre destinada ao fracasso.
