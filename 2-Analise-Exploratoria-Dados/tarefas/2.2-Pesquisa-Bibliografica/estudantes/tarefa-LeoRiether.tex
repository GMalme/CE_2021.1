\chapter{Análise Bibliográfica sobre Cálculo Lambda, por Leonardo Alves Riether\label{chap:bibliometria:LeoRiether}}

\section{Planejamento do estudo}
O lambda cálculo, ou cálculo-$\lambda$, é um modelo computacional tão poderoso quanto a Máquina de Turing, mas com certas propriedades que o deixam mais elegante e fácil de analisar. Algumas dessas propriedades, como descrito por \citet{lynn_lambda_nodate}, são a simplicidade, praticidade e correção. Por conta disso, várias linguagens funcionais, como Haskell e Kind\citep{kindelia_kind_nodate}, são baseadas nesse modelo de computação.

Com a atual tendência de linguagens de programação implementarem cada vez mais recursos funcionais, é possível que avanços na teoria do cálculo-$\lambda$ tragam benefícios práticos para essas linguagens, o que pode incentivar o estudo dessa área do conhecimento pelos pesquisadores.

É interessante, portanto, analisar o que já foi publicado sobre esse cálculo, de forma a extrair métricas sobre a história e evolução da área ao longo do tempo.

Com base nisso, este trabalho objetivou responder as seguintes perguntas:
\begin{itemize}
    \item Como é o histórico do nível de intensidade de publicações sobre cálculo-$\lambda$ da comunidade científica? Os pesquisadores estão mais interessados nele agora que no passado, ou menos? 
    \item Quais são os países e instituições que mais produzem conteúdo científico sobre o cálculo lambda? 
    \item Quais são os termos mais importantes relacionados a esse sistema computacional?
\end{itemize}

\subsection{O que já existe de pesquisa bibliométrica sobre esse tema?}
Pesquisas na Web of Science, SCOPUS e Google Scholar não encontraram resultados relevantes, portanto não foi possível encontrar nenhuma pesquisa bibliométrica sobre o cálculo-$\lambda$ feita anteriormente.

\subsection{Uso do Bibliometrix e Biblioshiny}
O estudo será feito com auxílio das ferramentas Bibliometrix e Biblioshiny, executados por meio do RStudio, conforme o \textit{worflow} recomendado pelos autores do pacote, indicado na figura ~\ref{fig:bibliometrix:workflow}.

\subsection{Limitações}
A pesquisa foi realizada no período de uma semana, e portanto não foi possível analisar muito a fundo. Além disso, a única base de dados utilizada foi a Web of Science, mantida pela Clarivate. Em um estudo posterior, seria interessante ter mais tempo para fazer a pesquisa e utilizar mais de uma base de dados bibliográfica.


\section{Coleta de dados\label{leoriether:lammetrics:coleta}}
Os dados obtidos foram coletados utilizando a ferramenta Web of Science, da Clarivate, no dia 08 de fevereiro de 2022, acessada por meio do Portal de Periódicos da CAPES. A coleção utilizada foi a Science Core Collection, edição Science Citation Index Expanded (SCI-EXPANDED), que possui artigos indexados desde 1945 sobre as áreas de ciências exatas e naturais.

Como o objetivo era analisar todas as publicações relacionadas ao estudo do lambda cálculo, a query utilizada na Web of Science utiliza as palavras chave "lambda" e "calculus", ou "calculi"; a pesquisa, porém, foi limitada a categorias de Matemática, Lógica e Computação, a fim de omitir resultados relacionados à Física. A query produzida é mostrada na listagem \ref{leoriether-lammetrics-query}.

\lstinputlisting[numbers=left,basicstyle=\normalsize\ttfamily,caption={\query\  de busca no WoS sobre lambda cálculo},label=leoriether-lammetrics-query]{experiments/LeoRiether/AnaliseBibliometrica/LambdaCalculus/WoS-20220208/query.txt}

Com isso, foram obtidos 2612 registros bibliográficos sobre o cálculo-$\lambda$, incluindo algumas de suas variantes, como o cálculo mu-lambda, o cálculo-$\lambda$ não tipado e o com tipagem simples. Após exportação dos registros em blocos de até mil itens, com a opção \textit{Exportar registros para arquivo de texto sem formatação}, os arquivos foram concatenados, formando um arquivo de 7.99MB com todos os dados, que pode ser encontrado no GitHub da disciplina, em \url{https://github.com/jhcf/Comput-Experim-20212/blob/main/experiments/LeoRiether/AnaliseBibliometrica/LambdaCalculus/WoS-20220208/recs.txt}.

\section{Análise dos dados}

\subsection{Filtragem de registros}
Foi aplicado um filtro ao \dataset\  inicial, que possuia 2612 registros, para que a análise fosse feita apenas em registros de artigos publicados em revistar científicas, assumindo que o conhecimento científico publicado nessas revistas é mais rigoroso. Após a filtragem, obtivemos 1787 registros de artigos sobre o Lambda Cálculo, que serão chamados de LC@LeoRiether.

\subsection{Análise descritiva do \dataset\   LC@LeoRiether}

A primeira análise a ser feita é a análise descritiva do \dataset\  LC@LeoRiether. Para isso, abrimos no Biblioshiny a seção \textit{Dataset > Main Information}, que calcula as informações necessárias. A partir diss, conseguimos as seguintes informações gerais sobre o dataset:

\begin{description}
    \item [\textit{Timespan}] Os artigos encontrados pela query no Web of Science datam desde 1970 até 2022.
    \item [\textit{Sources (Journals, Books, etc)}] Foram encontradas 311 fontes de informação que publicaram artigos presentes no \dataset\ LC@LeoRiether. Assim, cada fonte publicou em média $\frac{1787}{311} =5.75$ artigos.
    \item [\textit{Average years from publication}] A média do tempo de publicação dos artigos no dataset foi de 14.5 anos.
    \item [\textit{Average citations per documents}] Cada documento foi citado, em média, 12.33 vezes.
    \item [\textit{Average citations per year per doc}] Cada documento foi citado, em média, 0.9031 vezes por ano.
    \item [\textit{References}] O \dataset\  LC@LeoRiether possui 33701 referências citadas (tags CR)
    \item [\textit{Keywords Plus (ID)}] Foram encontradas 1538 palavras-chave distintas do tipo Keywords Plus (ID).
    \item [\textit{Author's Keywords (DE)}] Foram encontradas 3341 palavras-chave distintas indicadas pelas autores.
    \item [\textit{Authors}] 2259 nomes de autores distintos estão presentes no \dataset\  LC@LeoRiether.
    \item [\textit{Author Appearances}] Os 2259 distintos (nomes de) autores foram encontrados 3560 vezes, como autores de artigos.
        
    \item [\textit{Authors of single-authored documents}] Há 513 autores, dentre os 2259 autores do \dataset\  , que editaram publicações individualmente
    \item [\textit{Authors of multi-authored documents}] Há 1746 autores, dentre os 2259 autores do \dataset\  , que editaram publicações com um ou mais co-autores
    \item [\textit{Single-authored documents}] Dentre os 1787 documentos do \dataset\  LC@LeoRiether, 672 deles foram escritos por um autor individualmente, enquanto os outros 1115 foram escritos colaborativamente.
    \item [\textit{Documents per Author}] Em média, cada autor publicou 0.791 documentos.
    \item [\textit{Authors per Document}] Em média, cada documento foi produzido por 1.26 autores.
    \item [\textit{Co-Authors per Documents}] No \dataset\ LC@LeoRiether, há uma média de 1.99 co-autores por artigo publicado.
    \item [\textit{Collaboration Index}] O índice de colaboração encontrado no \dataset\  LC@LeoRiether é de 1.57.
\end{description}

\subsection{Evolução da Produção Científica}
Com o Bibliometrix, podemos analisar como a produção científica de artigos sobre o cálculo-$\lambda$ evoluiu ao longo dos anos, segundo o \dataset\  LC@LeoRiether. Ao realizar essa análise, encontramos a imagem \ref{fig:evol:anual:LC@LeoRiether}. 

\begin{figure}
    \centering
    \includegraphics[width=1\textwidth]{experiments/LeoRiether/AnaliseBibliometrica/LambdaCalculus/WoS-20220208/Images/AnnualScientificProduction.png}
    \caption{Evolução da produção científica no \dataset\   LC@LeoRiether.}
    \label{fig:evol:anual:LC@LeoRiether}
\end{figure}

O \textit{Annual Growth Rate} encontrado no \dataset\  LC@LeoRiether é de 5.8\%, levemente maior que a taxa de crescimento anual da publicação científica mundial, de cerca de 3.3\%.

\subsection{Interpretação do Crescimento}
Há dois pontos interessantes que podemos observar no gráfico \ref{fig:evol:anual:LC@LeoRiether}. O primeiro deles é como a produção científica nessa área de estudo subiu abruptamente a partir de 1991, possivelmente indicando uma descoberta importante que levou a um grande aumento de interesse da comunidade. O segundo ponto é que, diferentemente de gráficos de evolução de produção científica de algumas outras áreas, não há um aumento exponencial no número de publicações por ano -- esse aumento é mais lento e contínuo.

\subsection{Evolução das Citações}
Outra análise que pode ser feita sobre o \dataset\  LC@LeoRiether é a de evolução de citações. A figura \ref{fig:evol:anual:citacoes:LC@LeoRiether} mostra os resultados obtidos a partir dessa análise.

\begin{figure}
    \centering
    \includegraphics[width=1\textwidth]{experiments/LeoRiether/AnaliseBibliometrica/LambdaCalculus/WoS-20220208/Images/AverageCitationsPerYear.png}
    \caption{Evolução das citações ao \dataset\   LC@LeoRiether.}
    \label{fig:evol:anual:citacoes:LC@LeoRiether}
\end{figure}

Com base na figura, observamos que há um crescimento aproximadamente constante, porém pequeno, de citações por ano. Note que o eixo de média de citações por ano não tem uma grande extensão, indo de 0 citações até um máximo de somente 2.2, em 2015. Além disso, vemos, como no gráfico de produção científica anual da figura \ref{fig:evol:anual:LC@LeoRiether} da seção anterior, um pico de média de citações entre 1991 e 1993. Dessa vez, no entanto, esse pico é seguido de uma queda no número de citações até 1997.

