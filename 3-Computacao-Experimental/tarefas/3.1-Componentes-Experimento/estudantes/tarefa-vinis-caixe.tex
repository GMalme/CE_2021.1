\section{Respostas de Vinícius Caixeta de Souza}

\subsection{A computação é uma atividade científica?}
Para responder esta pergunta e as seguintes primeiro é preciso investigar a definição de \gls{Computacao}, na qual busca a solução de um problema a partir de dados de entrada e saída que são processados por um \gls{Algoritmo}.

Quando se quer solucionar um problema é criada uma hipótese que deve ser comprovada ou refutada a partir de experimentos, os quais investigam causa e efeito a partir da manipulação de determinados fatores e dos dados do resultado final. Na área de computação algoritmos são criados para solucionar diversos problemas de ordenar listas até realizar buscas nelas, como a quantidade de dados pode atingir altos valores e os recursos físicos de computadores são limitados é preciso que os algoritmos sejam eficientes, ou seja, precisam ser corretos e terem tempo de execução mais otimizado possível. Para isso, cientistas da computação precisam criar hipóteses de algoritmos que sejam melhores que os já existentes, logo é considerada uma atividade científica.

\subsection{A computação é uma ciência experimental?}
Como mencionado anteriormente os cientistas da computação criam hipóteses na área sobre algoritmos, para poder comprovar ou refutá-los são realizados experimentos, um dos possíveis é a correção de um algoritmo de ordenação, no qual é preciso provar a partir de lemas e propriedades matemáticas se ao dar uma lista nos dados de entrada o dado de saída é uma mesma lista porém ordenada e permutada. Outra experimentação pode ser a de testes unitários em determinado algoritmo para verificar se a partir de determinados dados de entrada ele resulta em um dado de saída previsto, no qual deve valer para qualquer hardware por causa da \gls{ReprodutibilidadeCientifica}, comprovando então que a computação é uma ciência experimental.

\subsection{A computação é uma ciência empírica?}
Na subseção anterior pode-se observar caso os experimentos tenham resultados esperados, na primeira parte existem assistentes de provas para determinar se um algoritmo e correto e na segunda os testes unitários são feitos pelos próprios cientistas.