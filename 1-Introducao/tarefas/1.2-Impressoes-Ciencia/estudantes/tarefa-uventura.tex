\subsection{Minhas impressões iniciais sobre a ciência, por Ualiton Ventura da Silva}

O texto apresentado busca descrever e definir um modelo de visão do que seria \gls{Ciencia}, apesar de que o conceito da mesma não seja consensual, assim é descrito por \citep{schwartzman_ciencia_1984}. Contudo, é fundamental que para o estudo aprofundado de uma área haja  uma definição dos pontos aos quais aquela área se atenta a descrever, assim é possível organizar o conhecimento, sendo que o texto define ciência com cunho de ciência natural.

Criando outras perspectivas de ciência, talvez o que não é visto como, será. Exemplo a ser citado é o caso da própria matemática, que observando de um ponto de vista de ciência natural não é ciência, assim é dito por \citep{bilaniuk_is_nodate}, mas o método adotado no próprio campo da matemática assemelha-se ao de outras ciências naturais, enquanto a observação empírica é a forma de validação de uma dada hipótese científica, no caso da própria matemática utilizam-se provas para então chegar a uma determinada conclusão. Assim como as ciências naturais, a matemática também possui questões em aberto que não foram concluídas até o momento por não possuírem meio de prova, exemplo a ser mencionado é a chamada “Conjectura de Goldbach” \citep{noauthor_goldbachs_2022} que estabelece que todo inteiro par maior do que 2 pode ser escrito através da soma de primos (4 = 2+2, 6 = 3+3, 8 = 3+5, …), apesar de acharmos que possivelmente seja um fato, tal conjectura não possui prova e portanto não pode ser assumida como uma verdade universal.

Portanto, apesar de alguns campos de estudos não serem categorizados como ciência, através da visão adotada existem casos onde características das quais foram apresentadas pertencem ao campo estudado. Em ciência da computação, parte dos profissionais podem exercer a atividade de cientista, empregando então dada metodologia científica, contudo, outros podem desconsiderar alguns aspectos, exemplo, para um dado problema computacional este mesmo problema poderá ser solucionado de diversas maneiras, assim, não existe uma maneira única e universal para a resolução, portanto, não apresenta uma universalidade científica.
