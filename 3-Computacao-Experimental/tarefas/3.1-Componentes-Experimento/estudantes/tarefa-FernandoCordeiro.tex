\section{Respostas de Fernando Ferreira Cordeiro}

% computação é atividade cientifica?
% computação é ciência experimental?
% computação é ciência empírica?

\subsection{A computação é uma atividade científica? Justifique.
}
Para responde está pergunta é preciso entender o que é Atividade Científica e o que é a Computação.
A \gls{Ciencia} é dita como sendo qualquer sistema de conhecimento que se preocupa com o mundo físico e seus fenômenos e que envolve 
observações imparciais e experimentação sistemática. Sendo Atividade Científica a atividade que compõe o processo de produção de conhecimento cientifico

A \gls{Computacao} por sua vez, como definido no glossário, é "a busca da solução para um problema a partir de entradas de forma a
obter resultados através de um algoritmo." A computação, contudo, não é um exatamente um objeto de produção cientifica, visto todas as áreas que a compõem, os computadores são um ferramenta capaz de ser utilizada pela ciência para a produção desses conhecimentos.

Dito isso, tem se tornado indispensável no próprio avanço da ciência, visto todo o poder processamento de dados (evidencias) que uma produção cientifica necessita, até mesmo para evolução da própria computação.

Ao analisarmos dessa foma, é possível dizer a computação se aproxima bastante da atividade cientifica.

\subsection{A computação é uma ciência experimental?}

Com base no que foi dito anteriormente, a computação utiliza de um processo de experimentação para desenvolver seu conhecimento científico, sendo assim, é um ciência experimental.

\subsection{A computação é uma ciência empírica?}

Como discutido anteriormente, a computação quando gera suas evidências (\gls{ColetaDados}) para comprovar suas hipóteses, aproxima-se de uma atividade empírica.