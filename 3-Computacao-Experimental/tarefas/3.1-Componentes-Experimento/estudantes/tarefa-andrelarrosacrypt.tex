\section{Resposta de André Larrosa Chimpliganond}

% computação é atividade cientifica?
% computação é ciência experimental?
% computação é ciência empírica?


%def comp: A busca da solução para um problema a partir de entradas de forma a obter resultados através de um algoritmo

\subsection{A computação é uma atividade científica?}

Partindo da definição de \gls{Computacao}, vamos decompô-la para entender todas as etapas e compará-las com o processo da atividade científica.

Primeiro, se buscamos uma solução para um problema, significa que temos interesse em certa área de aplicação. Segundo, ao utilizarmos um \gls{Algoritmo}, estamos construindo uma possível solução utilizando um processo lógico que pode ser alterado e aprimorado para melhor resolver o problema. Terceiro, queremos obter um resultado com esse processo, esse resultado por sua vez será utilizado dentro da área de conhecimento a qual o problema inicial faz parte.

Agora, vamos comparar nossa decomposição da definição de computação com o processo da atividade científica. Qualquer atividade científica tem uma área de interesse, ponto abordado no primeiro passo do nosso estudo. Também temos uma hipótese que leva em consideração relações de \gls{Causalidade}. Esses fatores estão relacionados com os o segundo ponto, desenvolvimento do algoritmo. A programação de um algoritmo segue um processo lógico e metódico que visa moldar o problema de acordo com algum \gls{Paradigma} de programação. Pela própria natureza, segue uma relação de causa e efeito. Uma versão inicial do algoritmo, segue a hipótese do desenvolvedor de como o problema se comporta e como pode ser resolvido. Além disso, um algoritmo é um processo repetível, em que podemos controlar variáveis e realizar experimentações. Esse último ponto é parte essencial da \gls{Ciencia}. Por último temos que analisar os dados obtidos e fazer inferências sobre os resultados. Em computação, realizamos essa etapa em dois momentos. Um primeiro momento é quando estudamos os dados em relação ao problema de maneira isolada, queremos verificar se o algoritmo foi capaz de modelar e resolver o problema de maneira satisfatória. Caso a conclusão seja que os resultados não são suficientes, desenvolvemos uma outra versão do algoritmos (muito relacionado com o processo de experimentação). Por outro lado, se a solução é suficiente, passamos a analisar os resultados em relação à área de interesse como um todo. Aqui, fazemos o estudo das consequências dos dados descobertos de um ponto de vista mais amplo.

Essas análises mostram que a computação é uma atividade científica.

\subsection{A computação é uma ciência experimental?}

Como discutido anteriormente, a computação utiliza de um processo de experimentação para desenvolver seus estudos, logo é um ciência experimental.

\subsection{A computação é uma ciência empírica?}

Como argumentado, a computação comprova suas hipóteses ao gerar evidências (dados de saída), portanto é uma ciência empírica.