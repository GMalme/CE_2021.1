\newglossaryentry{ComunidadeCientifica}{
name = {Comunidade Científica},
description  = {A comunidade científica consiste no corpo de cientistas, suas relações e interações, e nos meios necessários à manutenção destas. Ela é normalmente dividida em "sub-comunidades", cada uma trabalhando em um campo particular dentro da ciência. Contudo, assim como a ciência é única, também o é a comunidade científica.
Fonte: \citep{wikipedia_ciencia_2022}.
}
}

\newglossaryentry{Teoria}{
name = {Teoria},
description  = {Uma definição científica de teoria é a de que ela é uma síntese aceita de um vasto campo de conhecimento, consistindo em hipóteses necessariamente falseáveis - mas não por isto erradas, dúbias ou tão pouco duvidosas - que foram e são devidamente e permanentemente confrontadas entre si e com os fatos no conjunto de evidências científicas, que, juntamente com as hipóteses, alicerçam o conceito. As hipóteses, em casos específicos, devido à simplicidade e ampla abrangência, podem ser elevadas ao status de leis.
Fonte: \citep{fiani_teoria_2009}.
}
}

