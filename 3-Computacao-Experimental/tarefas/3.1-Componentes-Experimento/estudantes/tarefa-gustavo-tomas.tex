\section{Respostas de Gustavo Tomás}

Para responder a pergunta sobre se a atividade profissional de um cientista da \gls{Computacao} utiliza todos os componentes de um experimento científico, é preciso listar e verificar quais são esses componentes. Em suma, os componentes de um experimento são os seguintes: 

\begin{itemize}
    \item Uma classe de fenômenos de interesse
    \item Hipóteses e Modelos
    \item Causa e Efeito
    \item Procedimentos repetíveis
    \item Demonstração, manipulação e controle de fatores
    \item Coleta de dados
    \item Análise lógica
\end{itemize}

Primeiramente, é preciso verificar se a ciência da computação possui uma classe de \gls{fenomeno} de interesse. Observando aspectos sociais (como as relações entre grupos e também competições entre os mesmos) é possível perceber que há uma forte tendência à criação de grupos e relações entre os profissionais na área de trabalho, de modo que há possibilidade da ocorrência de fenômenos de interesse no campo da computação.

Em seguida, surge a questão de hipóteses e modelos na computação. De fato, observando a história do campo em estudo, é possível notar a formulação de hipóteses (algumas interessantes são a do teste de Turing e a definição de inteligência artificial). Modelos também estão presentes, dado que a computação herda da matemática esses atributos.

Sobre causa e efeito: o desenvolvimento dos processadores ao passar dos anos auxilia na resposta a essa pergunta. O desenvolvimento dos transistores permitiu que o processo de fabricação fosse mais rápido e eficiente, de modo que houve redução no preço de processadores. Essa redução, por sua vez, permitiu que pessoas com rendas medianas pudessem adquirir um computador pessoal, aumentando o acesso à informação e, por sua vez, a qualidade de vida do indivíduo.

Procedimentos repetíveis. A priori, não é possível perceber que tipos de procedimentos são realizáveis em computação. Entretanto, após uma observação mais detalhada, nota-se que o processo de criação de um algoritmo segue uma série de etapas, para desenvolvimento e aprimoramento desses algoritmos (como se nota pela versão de um sistema ou aplicativo).

Sobre a demonstração, manipulação e controle de fatores. É evidente que experimentos realizados (como no teste de modelos) possuem diversas variáveis que às vezes são difíceis de controlar. Em um ambiente virtual, entretanto, as variáveis são formadas por bits, de forma que é possível (na maior parte dos casos), ajustar valores para observar os diferentes resultados. Além disso, sabe-se que esses resultados podem ser repetidos com alta confiabilidade, devido ao modo como as linguagens de programação são projetadas.

Em seguida, tem-se a coleta de \gls{Dado}. Como dito na seção anterior, é possível realizar o controle das variáveis utilizadas e os resultados obtidos podem ser repetidos com confiança. Dito isso, esses resultados podem ser armazenados de diversas formas, desde um simples arquivo de texto até uma planilha detalhada com diferentes valores.

Por fim, tem-se a análise lógica. Na simulação do comportamento de algoritmos, é preciso analisar os resultados e verificar se estes estão condizentes com os resultados esperados. É possível realizar uma análise acerca desses resultados e extrair uma conclusão satisfatória.

Feito a análise de como esses sete requisitos estão presentes na computação, é possível afirmar, afinal, que a computação utiliza todos os componentes de um experimento científico e é, por isso, uma ciência.