\section{Respostas de Raylan da Silva Sales}

\subsection{A computação é uma atividade científica?}

A computação está sujeita a análise em relação ao questionamento, se ela é ou não uma atividade científica, porém como foi afirmado em momentos anteriores, a computação embora não seja uma atividade científica, está por alguns momentos de sua formulação, inserida em algumas partes do ciclo de produção do conhecimento científico. Levando em conta o ciclo de produção do conhecimento científico, a computação olhando de forma abstrata, possui a fase de hipotetização, fase essa que é usada bastante como um dos componentes de uma \gls{Experimentacao}, já que um experimento tem entre tantos outros processos envolvidos, o processo de apoiar ou refutar uma hipótese criada.

A fase de análise de dados na \gls{Computacao}, também está presente na formulação do fato de que a computação seja uma atividade científica, pois no método científico, um experimento é um procedimento empírico que valida modelos ou hipóteses concorrentes, e de a cordo com o que foi estudado na seção, esse fato se mantém bem forte quando nos referimos a procedimentos empíricos.

E também há uma fase de planejamento que envolve todo o apanhado de \gls{Heurísticas}, que colabora ainda mais para a justificativa de que a computação é uma atividade científica, tendo em vista que um experimento geralmente testa uma hipótese, envolvendo um ciclo de algumas etapas para o comprovação ou refutação da hipótese em questão. Esse teste seria sob uma certa perspectiva, uma expectativa sobre como um determinado processo ou processos funcionam.

No entanto, a computação mesmo seguindo abstratamente e não objetivamente o ciclo de o que molda uma atividade científica, não é considerada uma atividade científica, já que não está em busca da verdade, nem faz parte do que chamamos de produção do conhecimento já que como falado recentemente, não está em busca da verdade.


Levando em conta de que na computação o conceito de simulação seja bastante utilizado, tanto na forma de implementar um modelo muitas vezes bastante complexo, quanto quando queremos saber como um objeto se comportará ou a procedência de sua \gls{eficácia}, fazendo isso através de modelos \gls{Estatíscos}. A computação é sim uma ciência experimental, pois são através de vários testes, simulação, experimentos e etc que algumas hipóteses são provadas.

Em relação a ser uma ciência empírica, se torna fácil provar que isso ocorre na computação, pois de acordo com o conceito do que é empírico, a computação possui certas partes que corroboram com o conceito de algo empírico, ou seja se baseia na experiência e na observação, metódicas ou não.
