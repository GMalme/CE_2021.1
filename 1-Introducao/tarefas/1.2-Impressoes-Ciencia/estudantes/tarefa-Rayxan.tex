\subsection{Minhas impressões iniciais sobre a ciência, por Raylan da Silva Sales}

De grande importância para o avanço contínuo da humanidade, a \gls{Ciencia} se mantêm em posição unânime no que se refere ao grau de abrangência que ela traz, desde a evolução de certas coisas até destruição das mesmas. Revoluções cientificas acontecem a todo momento e algumas delas são registradas no momento que são feitas, porém só ganham importância em um futuro longínquo, isso aconteceu por anos, o que me faz pensar que a ciência nunca irá acabar, seu progresso contínuo e resgate de conceitos de muito tempo atrás, me faz pensar que todo e qualquer movimento relacionado a ciência é importante. Através de meios \gls{Estatísticos} é possível analisar com dados reais o grande e contínuo alcance de muitas coisas na ciência, como o avanço de pandemias, mudanças climáticas recorrentes, medição de periculosidade em áreas de risco explosivo e etc. Apesar de seu conceito não ser único, a maioria dos "pontos de vista" em relação a ciência que envolvem evolução, criação e destruição devem ser captados como um método aproximado de descrever o que pode ser a ciência \citep{schwartzman_ciencia_1984}. 