\subsection{Minhas impressões iniciais sobre a ciência, por Gabriel dos Santos Martins}

A humanidade vem sempre evoluindo através dos tempos e isso deve-se ao fato de estarmos constantemente aprendendo, estudando e procurando melhorar as coisas e querendo ou não a \gls{Ciência} está envolvida em todo esse processo. Um exemplo disso é ver o quanto o ser humano entende a natureza e consegue prever \gls{fenomeno}s da natureza para evitar grandes tragédias mas que ainda sim muitas dessas acabam acontecendo e dizimando a vida de centenas e até milhares de pessoas, o que nos levam a estudar mais a ciência por trás desses \gls{fenomeno}s e mais uma vez a \gls{Ciência} faz parte da vida do ser humano. 

Para o estudo e evolução, são utilizado métodos científicos que segundo \citep{} é, basicamente, um conjunto de regras para se realizar uma experiência, com o objetivo de produzir um novo conhecimento, além de corrigir conhecimentos pré-existentes.
Em cada situação ou área de estudo, é utilizado um método científico mais adequado para tal caso, pois por exemplo, para conseguir avançar na cura de doenças mais perigosas hoje em dia são utilizados métodos científicos que levam em consideração tais doenças e estudos feitos anteriormente mas para avançar na tecnologia de carros e computadores é outro assunto diferente e logo é utilizando métodos científicos para essa área de estudo. 