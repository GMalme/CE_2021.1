\chapter{Análise Bibliográfica sobre o uso da Inteligência Artificial em carros autônomos, por Conrado Nunes\label{chap:bibliometria:Conras21}}

\section{Planejamento do estudo}

Ultimamente a necessidade de soluções tecnológicas tem crescido bastante. A tecnologia da informação é uma área em constante crescimento, e dentre as diversas ramificações, a inteligência artificial vem tendo sua participação significativa nas soluções tecnológicas. A IA está presente no nosso dia a dia mesmo que não seja percebida, seja em algoritmos de sugestão, até em carros que dirigem de forma totalmente autônoma.

Algumas empresas como a tesla, uber, e etc vem investimento na autonomia dos carros, um carro que dirige de forma totalmente autônoma sem interferência humana. Em alguns estados dos EUA já está sendo testado o transporte de passageiros com esses carros.

\subsection{O que já existe de pesquisa bibliométrica sobre esse tema?}

Citar MCA gontijo, A inteligência artificial pode ser usada em várias áreas diferentes, por esse motivo existem várias pesquisas para suas respectivas áreas.

A pesquisa é base para um posterior aprofundamento no campo da Cientometria, como fez \cite{chavalarias_whats_2017}.

\subsection{Uso do Bibliometrix e Biblioshiny}
Serão usadas a ferramenta e o \textit{workflow} proposto pelos autores do pacote Bibliometrix, conforme indica a figura ~\ref{fig:bibliometrix:workflow}.

\subsection{Limitações} O exercício relatado foi feito em apenas uma semana, envolvendo entre 5 a 10 horas de trabalho.


\begin{itemize}
\item O objetivo é exercitar inicialmente, e relatar, o uso da técnica de análise bibliométrica, para fins didáticos.
\end{itemize}


\section{Coleta de dados}

A coleta de dados feita usando o WoS no dia 02 de fevereiro de 2021, acessado por meio do Portal de Periódicos da CAPES.

artificial intelligence and (autonomous) and (experimental  or  numeric* or  statist* or  hypothes* or  empiric* or  inferen *)

Foram feitas buscas nas coleções Science  Citation  Index  Expanded (SCI -EXPANDED) e Social  Sciences  Citation  Index (SSCI), que contém registros relativos a vários campos do conhecimento, no qual o SCI-EXPANDED foca mais na área das ciências exatas e naturais, enquanto que o SSCI indexa artigos da área das ciências sociais. Observe que os artigos nessas duas coleções são indexados desde 1945. 

Foi usada a \textit{query} de busca ilustrada nas linhas 1 a 9 da listagem \ref{query20210803-2}.

\lstinputlisting[numbers=left,basicstyle=\normalsize\ttfamily,caption={Query de busca sobre simulação multiagente de fenômenos socials, com ênfase em métodos experimentais.},label=query20210803-2]
{experiments/jhcf/PesqBibliogr/SimulacaoMultiagente/WoS-20210803/classico-mais-citacoes/query.txt}