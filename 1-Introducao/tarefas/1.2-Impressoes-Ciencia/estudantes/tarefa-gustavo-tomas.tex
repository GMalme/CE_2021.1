\subsection{Minhas impressões iniciais sobre a ciência, por Gustavo Tomás de Paula}

O positivismo \citep{wikipedia_positivismo_2022} é uma corrente filosófica que defende que o conhecimento verdadeiro é derivado pela razão e lógica por meio de experiências sensoriais. No texto "Considerações Preliminares Sobre a Ciência e sua Avaliação", é possível notar essa influência, nos tópicos que se referem, principalmente, ao empiricismo e projetização da ciência. Esses métodos são amplamente utilizados na computação, no momento em que o desenvolvimento de algoritmos que simulam o comportamento de órbitas de corpos celestiais, por exemplo, é derivado da astrofísica e suas áreas correlatas, que fazem forte uso da observação e de resultados empíricos. Uma definição mais precisa pode ser encontrada em \gls{Positivismo}.
