\section{Respostas de Pedro de Torres Maschio\label{tarefa-pedro-maschio-componentes-eperimento}}



\subsection{A computação é uma atividade científica? Justifique. }

Uma das formas de definir \gls{Ciencia} é expô-la como um sistema de obtenção de conhecimento por meio de um método científico  bem definido, o exercício da ciência teve como consequência a formação de uma \gls{ComunidadeCientifica} para suportá-la. Por outro lado, a computação refere-se a uma busca para solução de um problema a partir de entradas, de modo a encontrar as saídas referentes a essas entradas; a computação deu origem a comunidade de cientistas da computação.

A atividade científica, por sua vez, é o exercício da ciência, esta que é admitida como um conjunto de conhecimentos a respeito de fenômenos naturais, então ordenados, correlacionados e devidamente interpretados
\cite{decourt_atividade_nodate}.

O processo de computação segue passos parecidos com a atividade científica, como o \gls{planejamento} do experimento (ou, neste caso, do modelo computacional), o teste desse modelo e a observação e análise dos resultados. Com base nisso, pode-se dizer que a computação é uma atividade científica.

\subsection{A computação é uma ciência experimental? Justifique. }

Uma ciência experimental é aquela que faz uso de experimentos para comprovar hipóteses teóricas. A condução dos experimentos segue passos pré-determinados e com uma metodologia muito bem definida; os experimentos são feitos com intuito de provar alguma hipótese, fazendo uso da causalidade. A computação também faz uso da causalidade para prover seus os resultados. A computação, a seu modo, também possui experimentos computacionais, enquadrando-se assim como uma ciência experimental.

\subsection{A computação é uma ciência empírica? Justifique. }

Uma ciência empírica se apoia em observações do mundo real, na experiência vivida, não em teorias ou métodos científicos. A computação se faz do mesmo modo, tendo em vista sua natureza de ciência exatas e da terra.