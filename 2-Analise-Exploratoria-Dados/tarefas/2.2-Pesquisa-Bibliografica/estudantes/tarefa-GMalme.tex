\chapter{Análise Bibliográfica sobre O auxílio da indústria dos jogos na evolução do Hardware, por Gabriel Martins de Almeida}

\section{Planejamento do estudo}

Este é um projeto com o propósito de treinar o uso de técnicas de análises bibliométricas para fins de aprendizado. Em resumo, vivemos em um mundo altamente integrado com a tecnologia e embora muitos profissionais, por exemplo, um design gráfico necessita de um processador potente para ter uma boa eficiência no seu trabalho, um usuário comum não necessita do mesmo para o seu ambiente profissional, mas os jogos acabam por levar está necessidade para os clientes usuais. Logo, o lazer relacionado aos jogos aumenta a busca por hardware , então o objetivo é responder as seguintes perguntas.

\begin{itemize}
    \item Quais são os principais usos para o hardware comprado por consumidores comuns?
    \item A evolução do hardware e o aumento dos requisitos de sistemas para jogos estão relacionados?
    \item Com a popularização dos jogos o preço do hardware subiu ou desceu? Quais acontecimentos podem está ligado a esta possível variação?
\end{itemize}

\subsection{Limitações} As limitações estão relacionadas ao tempo, pois é uma atividade introdutória então não foi possível ler os artigos resultantes das buscas e a falta de experiência com a ferramenta pode levar a alguns resultados inesperados.

\section{Coleta de dados} 
Os dados foram coletados na base de dados Web of Science no dia 9 de fevereiro de 2022, acessado por meio do Portal de Periódicos da CAPES

\subsection{Query de Busca}
\begin{verbatim}
A query utilizada para a busca foi:
(game* or  specification or requirements ) 
and 
(advances or computer) 
and 
(hardware or purchase)
\end{verbatim}

\subsection{Explicação para os termos de busca usados}

O primeiro termo utilizado busca informações sobre requisitos e especificações para determinado jogo sendo utilizado o termo game* para buscar dados sobre um jogo e também podendo retornar uma busca para as especificações e requisitos relacionados aos jogadores. O segundo termo busca relacionar o termo a avanços tecnológicos na computação que também está relacionado ao tema da pesquisa, o terceiro tem o proposito de complementar os anteriores tentando buscar informações sobre compras de componentes.

\subsection{Registros recuperados}

A busca retornou 6858 resultados, sendo utilizado uma exportação personalizada passando também como parâmetro as referências citadas, contagem de referências citadas, total de usos e artigos interessantes. Logo, foi feita uma exportação de 5 arquivos de texto sem formatação com um tamanho de 1000 registros e após foi realizada uma concatenação em apenas um arquivo para ser possível passar os 6858 resultados da pesquisa para o biblioshiny.

\section{Análise dos dados}

\subsection{Filtragem de registros}
Após uma filtragem que deu enfoque para os artigos científicos retirando, por exemplo, notas,  