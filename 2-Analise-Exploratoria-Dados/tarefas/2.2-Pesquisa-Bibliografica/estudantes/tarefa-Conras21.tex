\chapter{Análise Bibliográfica sobre o uso da Inteligência Artificial, por Conrado Nunes\label{chap:bibliometria:Conras21}}

\section{Planejamento do estudo}
O planejamento o  desenho do estudo deve descrever as motivações, questões de interesse, escopo, limitações e objetivos do trabalho.

O planejamento do estudo deve motivar o tema escolhido e o interesse do autor.

No caso do meu trabalho, as perguntas que o nortearam foram:
\begin{itemize}
    \item Qual a base de conhecimentos científicos produzida em torno do tema Inteligência Artificial? 
    \item Como a inteligência artificial tem sido usada no dia a dia? 
    \item Qual a estrutura social da comunidade, que contribui para o avanço da Inteligência Artificial?
\end{itemize}

\subsection{O que já existe de pesquisa bibliométrica sobre esse tema?}

Citar MCA gontijo, A inteligência artificial pode ser usada em várias áreas diferentes, por esse motivo existem várias pesquisas para suas respectivas áreas.

A pesquisa é base para um posterior aprofundamento no campo da Cientometria, como fez \cite{chavalarias_whats_2017}.

\subsection{Uso do Bibliometrix e Biblioshiny}
Serão usadas a ferramenta e o \textit{workflow} proposto pelos autores do pacote Bibliometrix, conforme indica a figura ~\ref{fig:bibliometrix:workflow}.

\subsection{Limitações} O exercício relatado foi feito em apenas uma semana, envolvendo entre 5 a 10 horas de trabalho de cada autor.


\begin{itemize}
\item O objetivo é exercitar inicialmente, e relatar, o uso da técnica de análise bibliométrica, para fins didáticos.
\end{itemize}