\section{Respostas de Alexsander Correa de Oliveira}

Em ciências, toda atividade feita pelos profissionais tende a ter um caminho bem formado: há um interesse em algum fenômeno, e portanto, em pesquisar e se aprofundar nos mecanismos de tal aspecto físico. Cientistas da computação se distinguem categoricamente da imagem formada de um cientista profissional, mas por que isso?

Bem, para respondermos essa pergunta, primeiro temos de entender como a estrutura de uma ciência se dá para conseguirmos entender o que diferencia computação de outras atividades científicas. Inicialmente, será explicado o conceito passo a passo do que é experimentação, este que será agregado mais tarde a explicação de atividade científica.

\subsection{Explicando experimentação}

    Experimentos, em termos simples, são ferramentas para demonstrar a veracidade de uma determinada hipótese.
    
    Eles são constituídos por:
\begin{itemize}
    \item Uma Classe de Fenômenos de Interesse - Os objetos de interesse do cientista;
    \item Hipóteses e \gls{ModeloCientifico} - São dúvidas/pensamentos sobre a classe de fenômenos;
    \item Causa e Efeito - É a construção de \gls{Causalidade} dentro da hipótese/modelo;
    \item Procedimento Repetíveis - Procedimentos que podem ser feitos por outras pessoas e acabam no mesmo resultado;
    \item Demonstração, Manipulação e Controle de Fatores - Servem para analisar fatores irregulares/imprevisíveis com objetivo de diminuir o erro;
    \item Coleta de Dados;
    \item Análise Lógica - Passo final de um experimento. Utilizamos lógica para construir conclusões a partir dos resultados dos passos anteriores;
\end{itemize}

\subsection{Considerações sobre a atividade científica}
    Noções de como atividades científicas são conduzidas estão bem encapsuladas no conceito de experimentação. Conseguimos muito bem imaginar físicos com seus grandes equipamentos testando novas hipóteses. Mas podemos dizer o mesmo para profissionais de computação? O que realmente pode ser dito como científico no dia a dia de um programador, quando comparamos tais atividades com a rotina de um físico?

\subsection{O que cientistas da computação fazem?}
    Cientistas da computação podem ter vários trabalhos diferentes, mas todos acabam se voltando para desenvolvimento de sistemas. Tal atividade se assemelha bastante a experimentação, principalmente quando há uma metodologia de para a criação do projeto:
    \begin{itemize}
        \item Ideia - É o resultado de querer algo dentro de uma área de interesse. Pode ser visto como equivalente dos primeiros dois passos da experimentação discutidos anteriormente;
        \item Estudo de mercado - É a analise tanto lógica quanto estatística do objetivo e usuários do projeto. Ex: "Meu projeto é uma rede social de fotos. Quais serão meus usuários?". Isso pode ser atingido vendo tendências mercadológicas ou necessidades que podem surgir no futuro. Este passo tem um escopo que se assemelha a coleta de dados e causalidade do processo de experimentação;
        \item Desenvolvimento do sistema - É realmente onde o projeto é feito. Pode ser visto como equivalente de procedimentos repetíveis, mas os furos nessa comparação são notáveis;
        \item Testes - São equivalentes a demonstração, manipulação e controle de fatores, pois englobam tanto testes estruturais do software, quanto testes de mercado, onde analisamos a noção publica de um protótipo do projeto. E a partir deles podem ser feitas mudanças de parâmetros em quaisquer área do sistema;
        \item Lançamento - O projeto foi finalizado, e agora pode seguir dois caminhos. Aceitar o produto como finalizado ou continuar desenvolvendo, voltando assim para o terceiro passo. E é aqui que vemos o principal problema das comparações de computação com atividades científicas. O objetivo nunca foi sanar uma dúvida ou provar uma hipótese. Então quando chegamos no fim do projeto, que na experimentação seria análise lógica, conseguimos ver que os paradigmas são completamente diferentes. Enquanto um cientista publica seu artigo e o tem analisado, um profissional de computação lança seu produto. O sucesso de um sistema é bastante subjetivo, enquanto que o de um artigo é objetivo, a partir do momento que ele pode ser desmentido por qualquer pessoa com acesso a ele;
    \end{itemize}

\subsection{Conclusões}
    Os argumentos para as semelhanças entre a atividade científica e as de computação podem ser convincentes, porém ao colocarmos a prova suas diferenças, fica aparente que o objetivo das duas é categoricamente distinto. Contudo, devemos reconhecer que isso não é negativo, dado que ferramentas e metodologias diferentes são utilizados para atingir objetivos diferentes e mesmo que tenhamos chegado na conclusão de que profissionais de computação não realizam atividade científica em sua completude, vemos sim uma forte inspiração que a \gls{Ciencia} traz a todo meio.