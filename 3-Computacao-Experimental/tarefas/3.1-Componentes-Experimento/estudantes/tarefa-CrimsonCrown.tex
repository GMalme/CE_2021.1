\section{Respostas de Arthur da Silveira Couto}

Para responder à pergunta, primeiro precisamos definir os componentes de um experimento científico. Como definido na seção \ref{fundamentos:ce}, os componentes são:
\begin{itemize}
    \item Uma classe de fenômenos de interesse.
    \item Hipóteses e modelos.
    \item Causa e Efeito.
    \item Procedimentos repetíveis.
    \item Demonstração, manipulação e controle de fatores.
    \item Coleta de dados.
    \item Análise lógica.
\end{itemize}

Todo cientista da computação tem seus tópicos de interesse. Os fenômenos computacionais são muito estudados, mas além disso, cientistas da computação podem utilizar técnicas computacionais para estudar fenômenos mais associados com outras áreas. A ciência da computação é extremamente versátil e abrangente, porém pode se dizer que o primeiro componente de um experimento está sendo suprido, já que sempre há um fenômeno de interesse para cada cientista.

Hipóteses são sempre utilizadas no planejamento de um experimento ou projeto na ciência da computação. Um programa pode ser entendido como um modelo matemático e computacional de um determinado fenômeno, atrelando a computação à modelagem científica. Esse componente está bem presente na atividade de um cientista da computação.

