\subsection{Minhas impressões iniciais sobre a ciência, por Jônatas Gomes Barbosa da Silva}

A \gls{Ciencia} nasceu com as tentativas de descobrir respostas para os demais questionamentos humanos, as origens e suas fontes. Com a dedicação a isso, nasceram \gls{metodologia} e práticas que sistematizaram as descobertas e as práticas humanas na Terra. Dessa forma, a ciência pode ser definida como o estudo metodicamente organizado de quaisquer fenômenos que ocorrem no universo, com finalidade de descrever, explicar e prever o comportamento e a estrutura de tais fenômenos, amparado pelo registros (dados) produzidos acerca de tais fenômenos \citep{fernandes_ciencia_2018}. 

Analisar a ciência como um campo que consegue estudar o todo e a si mesma é intrigante, por ser um conceito que se entrelaça com vários conceitos antropológicos, sobre a formação da nossa sociedade. Em tempos obscuros como os atuais, estudar o que é ciência e sua importância, se faz fundamental.