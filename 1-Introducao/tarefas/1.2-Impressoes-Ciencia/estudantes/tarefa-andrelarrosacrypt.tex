\subsection{Minhas impressões iniciais sobre a ciência, por André Larrosa Chimpliganond}

Produto da curiosidade e capacidade de abstração humana, a Ciência, como explicado em \cite{aranha_temas_2009}, busca entender os fenômenos de maneira racional utilizando uma \gls{Metodologia} rígida que assegure a veracidade das descobertas. Embora a ciência, no sentido mais abrangente, possa ser datada junto com as primeiras civilizações humanas, a Ciência contemporânea surgiu com a Revolução Científica.

A Revolução Científica iniciou-se na Europa no século XVI e foi uma série de eventos envolvendo as mais diversas áreas do conhecimento que mudou a perspectiva humana sobre o universo. Nesse período, foi criado o método científico, que eleva um processo científico, estabelecendo relações causais e não apenas correlacionais.