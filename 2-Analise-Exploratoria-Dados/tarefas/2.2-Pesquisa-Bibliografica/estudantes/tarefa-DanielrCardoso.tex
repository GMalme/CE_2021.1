\chapter{Análise Bibliográfica sobre o uso de transformers no processamento de linguagem natural, por Daniel Rodrigues Cardoso}

\section{Planejamento do estudo}
O processamento de linguagem natural é uma das áreas que mais tem movimentado pesquisas no ramo da inteligencia artificial e com o passar dos anos novas técnicas estão surgindo. Uma técnica que tem ganhado bastante notoriedade se da por meio do uso de transformers e que sera o elemento de estudo desta pesquisa.

Esta pesquisa foi concebida com o intuito de sanar as seguintes questões:
\begin{itemize}
    \item Como o uso de transformers para processamento de linguagem natural evoluiu durante o tempo?
    
    \item De que forma o uso de transformers impactou os estudos em processamento de linguagem natural?
    
    \item Quais os autores mais relevantes?
\end{itemize}



\subsection{Limitações}Durante a produção deste trabalho a maior limitação foi o tempo disponível para o desenvolvimento do mesmo. foram empregadas cerca de 9 horas, tempo este que foi divido entre aprender a utilizar as ferramentas necessárias (R Studio e bibliometrix) e na coleta e análise dos dados.

\section{Coleta de dados}
A coleta de dados foi realizada utilizando a ferramenta de busca Web Of Science no dia 9 de fevereiro de 2022, acessado por meio do Portal de Periódicos da Capes.

\subsection{Query de Busca}
Para a busca, foi usada a \query\ abaixo:
\begin{verbatim}
(transformer*) 
and (Natural Language Processing or nlp)
\end{verbatim}
\subsubsection{Explicação para os termos de busca usados}
O foco deste trabalho era obter resultados que tratassem do uso de transformers no processamento de linguagem natural, desta forma a palavra chave transformer* foi adotada, para refinar a busca também foi incluído o termo Natural Language Processing e a sua sigla (NLP) .

\subsection{Registros recuperados}
Os 1018 registros recuperados pela query citada anteriormente foram exportados utilizando a opção de exportação para aquivo de texto sem formatação. Foram necessarios 

\section{Análise dos dados}

\subsection{Filtragem de registros}

\subsection{Análise descritiva do \dataset\   }

\subsection{Evolução da Produção Científica}

\subsection{Evolução das Citações}

