\section{Respostas de Bruno Sanguinetti R. de Barros}

\subsection{A computação é uma atividade científica?}

Sendo atividade cientifica uma forma de adquirir conhecimento por meio do método cientifico, e este método consistindo de etapas que ao final desenvolvem um novo conhecimento, uma correção (evolução) ou um aumento na área de incidência de conhecimentos anteriormente existentes, e \gls{Computacao} definida como a busca de solução para um problema a partir de entradas de forma a obter resultados com iterações e procedimentos preestabelecidos. A atribuição do "título" atividade científica à  \gls{Computacao} em sua definição de termo é incompatível.

Porem, esta conclusão só se demonstra verdadeira quando comparamos as definições dos termos em si. Relativizando o termo  \gls{Computacao} e considerando o que o antevêem e o que o sucede. É possível interpretar computação como atividade científica caso o processo computacional e as etapas que o precedem e sucedem sigam determinados padrões e protocolos científicos, como a elaboração de uma hipótese, emprego de um experimento controlado com ou sem  \gls{Modelos}, observação da \gls{Experiência Científica} , análise dos resultados chegando a uma conclusão.

Apesar da possível caracterização da \gls{Computacao} como atividade científica, existem termos capazes de associar \gls{Computacao} e atividade científica, como é o caso do termo \gls{CienciaComp}. Que através de \gls{MetodoCientifico} e emprego da  \gls{Computacao} desenvolve novos conhecimentos científicos, este conhecimento pode ser usado em novas atividades científicas ou apenas em \gls{Computacao} aplicada. Cabe ao cientista da computação empregar seus conhecimentos de \gls{Ciência} para sintetizar a computação em uma atividade ou conhecimento cientifico. Sendo essa a maior distinção entre \gls{Computacao} e \gls{CienciaComp}.

Em síntese, é imprudente caracterizar computação em si como uma atividade científica quando existem outros termos específicos que descrevem essa associação. Apesar de  \gls{Computacao} o compartilhar de várias etapas e métodos semelhantes ou ate iguais para seu desenvolvimento e aplicação em soluções.