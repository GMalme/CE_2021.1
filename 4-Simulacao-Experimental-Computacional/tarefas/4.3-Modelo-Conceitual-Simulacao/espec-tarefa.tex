Uso de ambiente de modelagem conceitual CMapTools-CMapCloud e modelos conceituais do projeto CE20211

deve ser feita em trio de alunos, tem por objetivo introduzir o emprego de modelagem conceitual no desenvolvimento do modelo multiagente CE20211, usando o ambiente cmaptools / cmapcloud.

1. Registre-se como usuário em https://cmapcloud.ihmc.us/, (ambiente de "Mapas Conceituais em Nuvem") e informe o seu usuário e email no texto da tarefa, para posteriormente obter acesso ao ambiente de modelagem conceitual online, que lhe será concedido pelo professor.

Baixe e instale em seu computador a ferramenta CMapTools, e configure em Preferências os seus dados de usuário (login e senha) para ter acesso aos "Mapas Conceituais em Nuvem".

Crie, no seu computador local, e posteriormente no ambiente compartilhado, usando sua instalação local do CMapTools, um Mapa Conceitual inicial, que tem por objetivo explicar os conceitos subjacentes à pergunta de pesquisa que você deve solucionar para melhor calibrar os parâmetros da simulação multiagente em desenvolvimento, conforme vai orientar o professor em aula. Insira, nas propriedades desse Mapa Conceitual em desenvolvimento (Editar/Propriedades), os seguintes metadados:
1. Questão Focal?
2. Palavras-chave
3. Cabeçalho com a questão focal
4. Rodapé com o seu nome completo e email.

Exporte o seu mapa como um PDF e o deposite no diretório estudantes, da tarefa. 
Depois, no mesmo diretório estudantes, crie um arquivo tex, com o seu nome, que inclui o gráfico do modelo e o apresenta de forma textual.
Por fim, insira no arquivo main, o seu documento latex.

+ 2 pontos
Exporte o seu mapa para o ambiente CmapCloud, conforme instruções em email do professor, no fórum https://aprender3.unb.br/mod/forum/discuss.php?d=161175#p443119

