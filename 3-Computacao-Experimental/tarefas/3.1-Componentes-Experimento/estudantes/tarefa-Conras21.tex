\section{Respostas de Conrado Nunes}

A \gls{Ciência} é um modo de conhecer fundamentado em um método, o método científico. O método é o modo de funcionamento das ciências, é fundamentado na observação, na experimentação e na produção de teorias e leis.

As teorias são constantemente testadas, visando sua comprovação ou substituição por outra teoria que resista à checagem.

O objetivo da ciência é explicar, descrever e prever os fenômenos a partir do desenvolvimento de procedimentos metodológicos que possam ser constantemente verificados e reproduzidos.

\subsection{A computação é uma atividade científica?}
A essência da atividade científica é, descobrir por descobrir, desvelar por desvelar. Essa práxis da ciência, essa nova determinação do ser dos entes tem, no entanto, uma maneira peculiar de desvelar o ente em sua totalidade e transformá-lo em natureza.
Portanto pode-se considerar que a computação é sim uma atividade científica, já que ela possibilidade novas descobertas.

\subsection{A computação é uma ciência experimental?}
As \gls{ciências experimentais} são todas aquelas ciências que se utilizam de experimentação para comprovar os seus postulados teóricos.

Tais ciências são muito importantes para o aprendizado dos estudantes, pois aproximam estes do fazer cotidiano das ciências modernas e os capacita com teoria e prática científica.

As ciências experimentais, têm nesse contexto uma contribuição pedagógica muito grande para oferecer e deve, desse modo ser explorada em sala de aula.

Então podemos sim dizer que a computação é uma ciência experimental, tendo em vista que para comprovar a teoria, se utiliza o experimento.

\subsection{A computação é uma ciência empírica?}

Para a ciência, empírico é um tipo de evidência inicial para comprovar alguns métodos científicos, o primeiro passo é a observação, para então fazer uma pesquisa através do método científico. Nas ciências, muitas pesquisas são realizadas inicialmente através da observação e da experiência.

A computação utiliza suas \gls{hipótese}s gerando dados de saída com isso ela pode ser considerada uma ciência empírica.

\subsection{Conclusões}

A ciência da computação utiliza sim todos os métodos de um experimento científico, no entanto de uma forma mais peculiar, ela está mais para uma ferramenta para auxiliar na atividade científica de diversas áreas profissionais, o que acaba fazendo com que ela se torne uma ciência. 