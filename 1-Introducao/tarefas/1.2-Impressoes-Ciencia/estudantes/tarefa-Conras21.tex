\subsection{Minhas impressões iniciais sobre a ciência, por Conrado Nunes Barbosa Neto}

A ciência\citep{gnipper_o_nodate} é algo difícil para se definir, no entanto podemos vê-la como aquele tipo de conhecimento que busca compreender verdades ou leis naturais para explicar o funcionamento das coisas e do universo em geral.  É por isso que cientistas fazem observações, verificações, medições, análises e classificações, procurando entender os fatos e traduzi-los para uma linguagem estatística. E é aí que entra o método científico. O método científico é, basicamente, um conjunto de regras para se realizar uma experiência, com o objetivo de produzir um novo conhecimento, além de corrigir conhecimentos pré-existentes. Essas regras são necessárias justamente para coibir a subjetividade, direcionando a pesquisa para a produção de conhecimentos válidos — em suma, científicos. 
