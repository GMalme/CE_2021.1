\newglossaryentry{Observabilidade}{
name={Observabilidade},
    description={O conceito de observabilidade pode ser definido como “uma medida que descreve quão bem podem os estados de um sistema serem inferidos a partir do conhecimento de suas saídas externas”. No contexto da área de computação, a observabilidade indica até que ponto um sistema, incluindo sua infraestrutura, aplicativos e as interações entre os mesmos, pode ser monitorado. Com isso, os responsáveis por esse sistema são capazes de detectar comportamentos indesejáveis, permitindo inclusive a prevenção de falhas futuras. Fonte: \citet[p. 14]{arantes_observabilidade_2021}}}

\newglossaryentry{Experiencia}{
	name={Experiência},
	description={Experiência refere-se a eventos conscientes em geral, mais especificamente a percepções, ou ao conhecimento prático e á familiaridade produzidos por estes processos conscientes.
	Fonte: \citet{wikipedia_experiencia_2022}
	}
}

