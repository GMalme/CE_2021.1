\subsection{Minhas impressões iniciais sobre a ciência, por Leonardo Alves Riether}

A ciência é uma poderosa ferramenta que temos para gerar conhecimento, por meio do estudo metódico de fenômenos observáveis que ocorrem à nossa volta. Por causa de limitações, tanto dos humanos quanto dos instrumentos utilizados para fazer medições e obter dados, o conhecimento científico é sempre imperfeito. Mesmo assim, ele pode ser muito útil. Fazendo uma analogia com a frase de \citet{box_all_2018}, que diz que todos os modelos estão errados, mas alguns são úteis, podemos argumentar que o conhecimento científico funciona da mesma forma: os modelos científicos são imperfeitos, mas geram avanços societárias significativos.

O estudo do conhecimento científico é feito pela área da filosofia chamada de \gls{epistemologia}, também conhecida como filosofia da ciência. Para estudar a ciência, é preciso considerar todas as etapas que levam à criação do conhecimento, desde a geração de hipóteses até a lógica utilizada para formar conclusões.
