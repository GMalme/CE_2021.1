\newglossaryentry{fenomeno}{
	name={fenômeno},
	description={consite em um acontecimento observável, particularmente algo especial (literalmente "algo que pode ser visto", derivado da palavra grega phainomenon = "observável"). Fonte: \cite{noauthor_o_nodate} Exemplo: podemos observar diarimente o acontecimento advindo da natureza, a gravidade, portanto chamamos a gravidade de fenômeno da natureza, entretanto podemos ter outros tipos de fenômenos, por exemplo, o exôdo rural, sendo este um fenômeno social de acordo com as ciências humanas. Fonte: \cite{noauthor_exodo_nodate}}
}

\newglossaryentry{MetodoCientifico}{
	name={Método Científico},
	description={O método científico consiste em um conjunto de regras que conduzem uma pesquisa científica. Essa pesquisa, é resultado de indagações feitas a partir uma motivação pessoal do cientista. Para tentar respondê-las, o cientista iniciará seu processo de busca por respostas, utilizando, então este método. Dessa forma, ao se definir um conjunto de passo a serem seguidos o método científico promove a reprodutibilidade de seus resultados, garantindo que o próprio ou outros cientistas, caso necessário, consigam reproduzir os passos e compararem os resultados obtidos. Existem quatro etapas no método científico: Observação do método, Formulação de hipóteses, Realização do experimento e Aceitação ou rejeição da hipótese formulada. Após finalizar todas as etapas, as afirmações extraídas dos resultados do estudo são denominadas teorias.}
}