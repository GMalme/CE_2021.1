\section{Respostas de Jônatas Gomes Barbosa da Silva}
A fim de investigar sobre a utilização dos componentes de um experimento científico na atividade profissional de um cientista da computação, serão dissertadas duas perguntas: Como é a atividade profissional de um cientista da computação? e Quais são e como são os componentes de um experimento científico?

\subsection{A atividade profissional de um cientista da computação}
Atualmente, o mercado de trabalho para graduados em ciência da computação tem sido focado para geração de produtos de consumo. Geralmente, o recém graduado que resolve não seguir carreira acadêmica, acaba sendo levado para carreiras como engenheiro de software, desenvolvimento web, ciência de dados, entre outros. Dessa forma, pela falta de demanda ou maior demanda com salários mais atraentes, suas funções são desviadas de sua formação. 

\subsection{Os componentes de um experimento Científico}
Um \gls{experimento} é um procedimento para apoiar ou refutar uma hipótese. Esse procedimento deve seguir determinadas etapas para aferir sua eficiência. Os componentes de um experimento científico são:
\begin{itemize}
    \item Uma Classe de Fenômenos de Interesse;
    \item Hipóteses e Modelos;
    \item Causa e Efeito;
    \item Procedimento Repetíveis;
    \item Demonstração, Manipulação e Controle de Fatores;
    \item Coleta de Dados; e
    \item Análise Lógica;
\end{itemize}

Estes componentes são etapas a serem realizadas que mantém a veracidade e a conceituação de um experimento científico. Parte desses componentes estão presentes em muitos processos da gerência de projetos de \gls{Computacao}.

\subsection{Comparações}
Um exemplo dessa utilização é visto no modelo Scrum.

Dentro da área de gerenciamento de projetos, o \textit{Scrum} é um framework de desenvolvimento, entrega e manutenção de produtos em um ambiente complexo, focado em desenvolvimento de software. Muito comum em diversas organizações e empresas, esse modelo de trabalho é dividido em alguns eventos, que de acordo com a organização, podem ser alterados ou criados. São eles:

\begin{itemize}
    \item Planejamento de Sprint - No início do desenvolvimento das atividades, há a definição dos objetivos, a seleção das metas e se acorda sobre o que deve ser feito para que as metas e os objetivos sejam alcançados.
    \item Reuniões diárias (Daily Scrum) - Evento diário em que os integrantes da organização ou equipe se reúnem para aferirem o andamento da sprint e dos resultados dos seus trabalhos para alcançar as metas
    \item Revisão da Sprint - Evento do final do período de trabalho, os trabalhos completos são apresentados à  equipe e ao gerente de projetos, e se discute o impacto das metas que não foram alcançadas, guiando o desenvolvimento de um plano de continuação do que está em desenvolvimento.
    \item Retrospectiva da Sprint - Momento em que há uma análise dos eventos passados durante a sprint a fim de encontrar possíveis melhorias e destacar pontos negativos do desenvolvimento para que o prosseguimento dos trabalhos evolua.
\end{itemize}

É fácil notar a coincidência de várias etapas do Scrum com as etapas do \gls{MetodoCientifico}, seja nas etapas ou em seus objetivos. Ambos os modelos se distanciam pelos seus objetivos: O Scrum é voltado para uma lógica mercadológica, que se volta para agilizar a produção e aumentar o entrosamento da equipe, enquanto o Método Científico se volta para explicar fenômenos e constituições. Os métodos, no entanto, ainda perpassam pela computação e suas características inerentes, como instrumento da ciência, de análise e de observação.

\subsection{Conclusão}
A atividade profissional de um cientista da computação utiliza dos componentes de um experimento ci