\subsection{Minhas impressões iniciais sobre a ciência, por Felipe Gomes Paradas}

A ciência é muito complexa para se definir de forma rápida e simplista, porém podemos afirmar que um de seus principais objetivos é a busca, através de observações, hipóteses (\gls{Hipotese}), experimentos, entre outros processos, pela explicação de fenômenos que ocorrem no universo. Através de estudos metódicos e sistemáticos, baseados em dados, experimentos e evidências, a ciência tem como um de seus resultados chaves a produção científica, que é um fenômeno social complexo por si só.

Utilizando a produção científica e estudos metódicos e sistemáticos, ou seja, científicos, se tem um processo bem definido para criação, divulgação e acumulação de conhecimento utilizado pela humanidade.

