\section{Respostas de Ítalo Eduardo Dias Frota\label{tarefa-titofrota-componentes-eperimento}}

\subsection{A computação é uma atividade científica?}

A resposta para esta pergunta necessita da definição de elementos-chave sobre as atividades científicas. Primeiramente, é preciso definir quais são as pontos que caracterizam essas produções como tais. 

Primeiramente, uma atividade científica obedece um ciclo de produção, explicitado por \cite[p.2]{barton_graphical_1999}, existem quatro fases no ciclo de produção científico:

\begin{enumerate}
    \item Ideação ou hipotetização; 
    \item Planejamento de experimentação;
    \item Experimentação;
    \item Análise dos dados empíricos.
\end{enumerate}

Cada uma dessas fases está relacionada à componentes de experimento. Na fase de ideação, é necessário haver uma classe de \gls{fenomeno} de interesse, obtida através da observação de aspectos sociais, físicos, químicos, biológicos, computacionais e etc. 

Na fase de planejamento de experimentação, a utilização de modelos e hipóteses tem o objetivo de tornar uma parte ou característica específica do mundo mais fácil de entender e definir, o que também possibilita o trabalho quando é impossível criar condições experimentais onde os resultados possam ser medidos de forma direta.

Quando se trata de experimentação, todos os componentes já citados fazem parte desse processo, mas ainda há a presença de elementos-chave para a produção científica. Um deles é o de causa e efeito, onde são levantadas abstrações que definem as influências pelas quais um evento, processo, estado ou objeto contribuem para a ocorrência de algum efeito. 

Os experimentos podem variar bastante quanto à forma e complexidade, dessa maneira, são definidos procedimentos repetíveis para que seja possível observar e replicar fenômenos visando a posterior \gls{Analise} dos resultados obtidos, contribuindo para o bom entendimento e exploração do objeto estudado. Neste momento, é importante a demonstração, manipulação e controle de fatores, projetadas para minimizar os efeitos de variáveis diferentes da variável independente única, com foco no aumento da confiabilidade dos resultados. Logo, idealmente, todas as variáveis de um \gls{experimento} devem ser controladas para que o experimento funcione da forma que foi proposto. Por fim, deve ser realizada a coleta de dados.

A etapa de análise dos dados empíricos ocorre por uso da lógica, dedutiva, indutiva, matemática, estatística e algorítmica. Aqui, são construídos argumentos que referenciam teorias e estão diretamente relacionados com os dados obtidos através da experimentação. Dessa forma, é possível levantar conclusões a respeito do objeto de estudo.

Se tratando de \gls{Computacao} sob a lente científica, é possível defini-la como o processamento de dados de entrada que geram dados de saída, logo, é uma ação empírica que produz insumos para diversos fins.

Passada a etapa de concepção de hipóteses, a etapa de codificação de \textit{softwares/hardwares} faz parte do planejamento e execução de experimentos, que produzem dados empíricos, que posteriormente são analisados e interpretados pelos usuários finais. Tal ciclo é observado em diversas aplicações da \gls{Computacao}, onde há a necessidade de alta confiabilidade dos resultados obtidos. Por fim, conclui-se que a computação é muito semelhante a atividade científica. 

\subsection{A computação é uma ciência experimental?}

É notável que os processos de computação etão em congruência com os componentes da experimentação científica, logo, a produção de \gls{Conhecimento} configura a área como uma ciência experimental.

\subsection{A computação é uma ciência empírica?}

Sim, pois ao seguir as etapas supracitadas e gerar dados que comprovam as ideias e hipóteses, se caracteriza como uma atividade empírica.