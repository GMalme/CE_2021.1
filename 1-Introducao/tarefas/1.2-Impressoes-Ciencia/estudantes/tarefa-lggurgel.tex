\subsection{Minhas impressões iniciais sobre a ciência, por Lucas Gabriel Gurgel}


Ciência não possui significado fixo ao fato de que não há um consentimento entre todos os que a já tentaram introduzir \citep{schwartzman_ciencia_1984}. Ao mesmo tempo que isso se contradiz com a franca necessidade de estudiosos em busca da verdade única, essa discordância também exalta o interminável esforço para se chegar na Verdade absoluta de inúmeros estudos científicos presentes na \gls{ComunidadeCientifica}.

Todo \gls{Conhecimento-Cientifico} resultado de pesquisas validadas por meio de \gls{publicacao-cientifica} é estrutura fundamental inerente para interpretar o mundo e direcionar os objetivos a longo prazo. E mesmo sem ter certeza da definição exata de ciência, interessados possuem \gls{SensoCritico} e aplicam métodos sistemáticos muito bem definidos para fundamentarem, testarem e aplicarem suas teses. Diante disso, mesmo sem significado simples, a ciência engloba todos os resultados obtidos (com sucesso ou não) em estudos feitos, em andamento e futuros, mostrando a complexidade existente por trás da Ciência.
