\section{Tarefa 4: Análise Bibliométrica com apoio de R/R Studio e Bibliometrix}

A tarefa vale 20 pontos, e consiste em produzir individualmente uma análise bibliométrica inicial, abordando um tema de computação que  interessa ao autor. 

A análise bibliométrica, o produto da tarefa, deve estar apresentada num capítulo da parte \ref{part:estudos:exploratorios}, no relatório da turma no Overleaf, no diretório ``2-Analise-Exploratoria-Dados/tarefas/2.2-Pesquisa-Bibliografica/estudantes/'', em um arquivo de nome "tarefa-\githubusername.tex. Ver um exemplo com a complexidade referencial, no arquivo ``tarefa-jhcf.tex'', no diretório ``2-Analise-Exploratoria-Dados/tarefas/2.2-Pesquisa-Bibliografica/estudantes/'', que pode ser lido no capítulo \ref{chap:bibliometria:jhcf}. 

A análise deve conter texto, figuras, tabelas  etc. Todas as figuras, tabelas, dados etc, incluídas no texto, devem estar montadas no diretório /experiments/<githubusername>/AnaliseBibliometrica/\textbf{<tema-pesquisa>}. 

\textbf{<tema-pesquisa>} é um nome curto, em formato CamelCase, que você dará ao diretório onde os dados, figuras, etc, do seu estudo, estarão montados. Não use caracteres acentuados em nomes de arquivos no Overleaf.

Veja um exemplo de dados e figuras em 
/experiments / jhcf / PesqBiblogr / SimulacaoMultiagente.


Para a compilação da tarefa é necessário fazer o input do arquivo de texto em "2-Analise-Exploratoria-Dados / tarefas / 2.1-Pesquisa-Bibliografica / estudantes / main.tex".

O \textit{dataset} de análise bibliométrica deve conter, minimamente, 250 registros bibliográficos.


A análise precisa ser realizada e descrita em cinco etapas, e deve seguir as orientações feitas em \ref{metodo:analise:bibliografica}, e no detalhamento proposto por \citet{aria_bibliometrix_2017}:
\begin{enumerate}
    \item \textit{Study design} (Planejamento do estudo);

    \item  \textit{Data collection} (Coleta de dados);

    \item \textit{Data analysis} (Análise dos dados);

    \item \textit{Data visualization} (Visualização dos dados representados de forma gráfica, em vários formatos, vários tipos de diagrama);

    \item  \textit{Interpretation} (Interpretação, traçar conclusões, reflexões, sugestões de aprofundamento).
\end{enumerate}

Na edição do \LaTeX~ deve-se atentar aos seguintes aspectos:
\begin{enumerate}
    \item Todos os dados, inclusive as imagens, usadas na produção da análise, deve estar inseridos no diretório de experimentos individuais do estudante;
    \item Todas as figuras e gráficos inseridos na análise devem ser individualmente rotulados com label, sem conflitar com os labels já existentes, devem ter um título (caption) descritivo do que apresenta a figura e o nome do dataset usado, e também a figura/gráfico deve ser explicitamente descritas e citadas, usando referencia (ref);
    \item As figuras deve ser automaticamente dimensionadas, e eventualmente rotacionadas,  para caber na largura e (ou) altura do texto. 
\end{enumerate}