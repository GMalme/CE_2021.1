\subsection{Minhas impressões iniciais sobre a ciência, por Fernanda Macedo de Sousa}

Em síntese ao que fora discutido em aula sobre a \gls{Ciencia}, a ciência pode ser descrita como uma atividade humana que emprega um método científico para estudar fenômenos que ocorrem no Universo, buscando descrever, explicar e prever o comportamento e a estrutura desses fenômenos. O observador, ou \gls{Cientista}, emprega um método científico munido de instrumentos de observação para observar esses fenômenos e produzir registros sobre o fenômeno (dado). Esses registros sobre o \gls{fenomeno} formam a base empírica que contém o \gls{Conhecimento-Cientifico} produzido nesse processo, e que serão também eventualmente analisados empregando-se um método científico. 

A ciência é fundamentada em princípios da produção científica. Ao aplicar o empirismo científico ao desenvolvimento de software observa-se que alguns dos princípios da organização metódica da produção científica podem ser aplicados. Entretanto, o desenvolvimento de software não é uma atividade científica. Os princípios do método cientifico são apenas parcialmente aplicáveis nas atividades realizadas na computação. 

Ao comparar a ciência propriamente com a computação, nota-se que a ciência busca explicar as coisas como são, sem criar algo novo além do conhecimento científico que explica os fenômenos, enquanto a computação visa mudar a realidade das pessoas e cria novos fenômenos por meio da \gls{tecnologia}.

Há muitas formas e oportunidades para estudo da ciência, esse  estudo científico da ciência é chamado de ciência da ciência \citep{clauset_data-driven_2017}. A ciência ampara-se na filosofia e dentre os ramos da filosofia do conhecimento encontra-se a filosofia da ciência que divide-se em diversas áreas de estudo, dentre eles, a computação. Em vista disso, para compreender mais profundamente a estrutura do conhecimento científico na computação a  \gls{FilosofiaDaCienciaComputacional} pode ser explorada.
