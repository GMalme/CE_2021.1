\subsection{Minhas impressões iniciais sobre a ciência, por Tatiana Franco Pereira}

Conforme apresentado nesta disciplina, \gls{Ciencia} pode ser definida como o estudo metodicamente organizado dos fenômenos que ocorrem no universo, com o intuito de descrever, explicar e prever o comportamento e a estrutura dos mesmos.  A realização da ciência é uma atividade essencialmente social, realizada por humanos através da produção científica e a sua organização metódica é baseada em alguns princípios. Entre eles, podemos citar o princípio da \gls{Observabilidade} e o princípio da \gls{Comunidade_Cientifica}, o último podendo ser definido como sendo um agrupamento de cientistas que conduzem atividades de investigação e produção do \gls{Conhecimento} \citep{wikipedia_ciencia_2022}. 

Por consequência, os resultados obtidos a partir da produção científica é capaz de auxiliar a tomada de decisão de profissionais das áreas de conhecimento relacionadas ao \gls{fenomeno} que fora utilizado como objeto de estudo da pesquisa em questão. Sabe-se que considerável parte do avanço tecnológico e econômico experimentado pela humanidade é decorrente de avanços científicos. No entanto, é importante ressaltar que um dos principais fundamentos da ciência é que não há verdade ou conhecimento absoluto. Devido a erros inerentes a humanos, limitações na descrição da realidade e instrumentos de coleta de dados, a verdade científica será sempre parcial e incompleta, e a medida que ela evolui, conhecimentos que antes eram tidos como verdades podem  se mostrar serem inverdades.  Apesar disso, a importância e efeitos positivos que a ciência tem perante a nossa sociedade são inegáveis.