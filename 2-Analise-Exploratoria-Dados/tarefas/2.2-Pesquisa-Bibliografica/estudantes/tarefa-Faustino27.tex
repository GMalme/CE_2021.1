\chapter{Análise Bibliográfica sobre Prova de Conhecimento Zero na Computação, por Gabriel Faustino Lima da Rocha}

\section{Planejamento do estudo}
O pensamento de que para provar a veracidade de uma afirmação é necessário explicar todo o processo que faz a afirmação se tornar verdadeira é comum entre as pessoas, porém uma área da matemática demonstra que isso nem sempre é verdadeiro, através da prova de conhecimento zero.

Por meio desse conhecimento surgiram perguntas sobre os possíveis usos desse campo na computação.

\begin{itemize}
    \item Como a prova de conhecimento zero está sendo utilizada na computação? 
    \item Qual área de pesquisa sofreu maior influencia da prova de conhecimento zero? 
    \item Houve o surgimento de novas áreas devido ao estudo da prova de conhecimento zero?
\end{itemize}

\section{Ferramentas utilizadas}
O pacote Bibliometrix da linguagem R, mais especificamente o biblioshiny, foi utlizado para a criação de dados quantitativos e bibliométricos. O base de dados utlizada foi o "Web of Science", disponível no portal de periódicos da CAPES.

\section{Limitações}
O projeto foi feito ao longo de