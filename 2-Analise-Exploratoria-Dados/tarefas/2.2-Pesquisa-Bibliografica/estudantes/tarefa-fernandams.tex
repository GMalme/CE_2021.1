\chapter{Análise Bibliográfica sobre \textit{Feedbacks} Automáticos no Ensino de Programação em Cursos de Graduação, por Fernanda Macedo de Sousa \label{chap:bibliometria:fernandams}}

\section{Planejamento do estudo}

% O planejamento o  desenho do estudo deve descrever as motivações, questões de interesse, escopo, limitações e objetivos do trabalho.

O ensino de programação nos anos iniciais das graduações tem sido um grande desafio didático e metodológico. Os ambientes Virtuais de aprendizagem (AVA’s) e demais plataformas de ensino emergem como possíveis soluções para sanar as dificuldades encontradas neste processo, porém a maioria destes ambientes não dispõe da dinamicidade necessária, que é intrínseca ao processo de ensino aprendizagem e, principalmente, ao ensino da primeira linguagem de programação. Neste contexto, o presente trabalho tem como intento responder à seguinte questão de pesquisa (RQ): O que nos diz a literatura sobre o uso de feedbacks automáticos no ensino de programação em cursos de graduação nas universidades de todo o mundo? A fim de responder ao RQ foram criadas 10 questões de investigação:

\begin{itemize}
    \item RQ1). Qual é a quantidade de artigos publicados por ano relacionados a este tema? 
    \item RQ2). Quais foram os países que mais produziram estes trabalhos? 
    \item RQ3). Quais fontes de informação publicaram mais artigos sobre esse tópico?
    \item RQ4). Quais são as principais plataformas de apoio ao ensino de programação utilizados atualmente?
    \item RQ5). Quais as principais linguagens de programação (Python, C, C++, Java, etc.) mais utilizadas no ensino da primeira linguagem de programação nos cursos de graduação?
    \item RQ6). Dentre os artigos analisados, qual a porcentagem dos que apresentaram resultados positivos em relação a implementação dos sistemas de feedbacks?
    \item RQ7). Estes feedbacks foram adaptados ao perfil do aluno ou são respostas padronizadas?
    \item RQ8). Os artigos analisados apresentaram alguma forma de classificação dos discentes por meio de suas respostas?
    \item RQ9). Os artigos apresentam alguma avaliação da percepção de quem faz uso dos feedbacks automáticos?
    \item RQ10). Qual o tipo de resposta apresentada nos feedbacks?
\end{itemize}

% \subsection{O que já existe de pesquisa bibliométrica sobre esse tema?}

\subsection{Uso do Bibliometrix e Biblioshiny}
Serão usadas a ferramenta e o \textit{workflow} proposto pelos autores do pacote Bibliometrix, conforme indica a figura ~\ref{fig:bibliometrix:workflow}.

\subsection{Limitações} 

O exercício relatado foi feito em X horas, utilizando a base de dados Scopus.

\section{Coleta de dados \label{FeedAuto:coleta}}

A coleta de dados foi realizada usando a base de dados Scopus no dia 09 de fevereiro de 2022, acessada por meio do Portal de Periódicos da CAPES.

Para a busca foram considerados os artigos publicados nos últimos cinco anos (2017 a 2021).

\subsection{Query de Busca}

Foi usada a \query\  de busca ilustrada nas linhas 1 a 14 da listagem \ref{query20220209-1}.

\lstinputlisting[numbers=left,basicstyle=\normalsize\ttfamily,caption={\query\  de busca sobre o uso de \textit{feedbacks} automáticos no ensino de programação em cursos de graduação},label=query20220209-1]
{experiments/fernandams/AnaliseBibliometrica/FeedbacksAutomaticosEnsProgGrad/Scopus-20220209/query.txt}

\subsubsection{Explicação para os termos de busca usados\label{FeedAuto:query}}

A busca consistiu de 

Os termos

\subsection{Registros recuperados}

\section{Análise dos dados}

\subsection{Filtragem de registros}

\subsection{Análise descritiva do \dataset\ FeedAuto@fernandams}

\subsection{Evolução da Produção Científica}

\subsection{Interpretação do Crescimento}

Percebe-se que esta é uma área de pesquisa em crescimento e que várias frentes de estudo estão sendo desenvolvidas em todo o mundo.

\subsection{Evolução das Citações}

\subsection{Interpretação das Citações}

\subsection{\textit{Three-Field Plots (Sankey diagram)} \label{FeedAuto:Sankey}}

\subsection{Interpretação da figura \ref{fig:FeedAuto@fernandams:ThreeFieldPlot}}

\subsubsection{Autores mais relevantes\label{FeedAuto:Sankey:AutoresRelevantes}}
