%% Keywords usadas: (graphic processing unit or GPU) and (lighting or light or shadow*)

\chapter{Análise Bibliográfica sobre Graphics Processing Unit, por Gustavo Tomás}

\section{Planejamento do estudo}

O objetivo do trabalho é analisar o impacto e popularização das tecnologias utilizadas pelas GPUs (Graphic Processing Units) no processamento e simulação da luz. Para isso, foram utilizadas as ferramentas Bibliometrix e Biblioshiny. Algumas perguntas para basear a pesquisa são:

\begin{itemize}
    \item Qual tecnologia obteve maior crescimento ao passar dos anos?
    \item Quais autores possuem o maior impacto no desenvolvimento dessas tecnologias?
\end{itemize}

\subsection{Limitações} O exercício relatado foi feito em cerca de uma semana, entre os dias 02 e 10 de fevereiro de 2022 e a base de dados utilizada foi Web Of Science (WoS).

\section{Coleta de dados}

A coleta de dados foi feita usando o WoS no dia 04/02/2022, por meio do Portal de Periódicos da CAPES. Foram feitas buscas nas coleções Science Citation Index Expanded (SCI-EXPANDED) e Social Sciences Citation Index (SSCI), mas com o foco em registros relativos a área de ciências naturais e exatas. A busca utilizada foi a seguinte:

\begin{verbatim}
(graphic processing unit or GPU) 
and
(lighting or light or shadow*)
\end{verbatim}

Essa busca consiste em dois termos, sendo que o primeiro é composto pela GPU (por extenso ou pela sigla) e o segundo pelas palavras luz ou iluminação ou sombra(s). Dessa forma, foram encontrados 1311 registros, sendo que nesse trabalho foram utilizados os primeiros 1000 registros.

\section{Análise dos dados}

\subsection{Filtragem de registros}
Antes da análise, foram aplicados filtros aos registros, de forma que apenas registros do tipo \textit{article}, de qualquer ano e com qualquer número de citações, fossem analisados. O resultado consiste em 850/1000 registros originais.

\subsection{Análise descritiva do dataset}

As informações mais gerais sobre o \textit{dataset} são as seguintes:
\begin{description}
    \item  [\textit{Timespan}] Os artigos analisados foram publicados entre os anos de 1992 e 2022.
    \item  [\textit{Sources}] O dataset é composto por 360 fontes diferentes (dentre artigos, livros e outros).
    \item 
    [\textit{Average years from publication}] A média do tempo de publicação é de 6 anos.
    \item 
    [\textit{Average citations per document}] Cada artigo no dataset foi citado em média 18.55 vezes.
    \item 
    [\textit{References}] O dataset contém 27463 referências citadas.
\end{description}

\subsection{Evolução da Produção Científica}

O gráfico em \ref{fig:gpu-prod-cient} apresenta a evolução da produção científica, mostrando uma forte tendência de crescimento a partir de 2007.

\begin{figure}[ht]
    \centering
    \includegraphics[width=12cm]{experiments/gustavo-tomas/AnaliseBibliometrica/GPUs/Graficos/gpu-prod-cient.png}
    \caption{Evolução da produção científica}
    \label{fig:gpu-prod-cient}
\end{figure}

O gráfico mostra um enorme crescimento na área, com destaque ao ano de 2021 com 120 documentos produzidos, um crescimento de 25\% em relação ao ano de 2020.

Nos anos de 2019-2021, em particular, foram introduzidas novas placas de vídeo no mercado, com poder computacional bem maior que as anteriores (é possível comparar as placas 1080 com as gerações 2080 e 3080 e perceber um aumento substancial de qualidade).

\subsection{Evolução das Citações}

A figura em \ref{fig:gpu-citation-year} mostra o número de citações médias feitas por ano. Nota-se uma grande quantidade de citações no ano de 2007, provavelmente devido a um artigo muito citado por outros autores. Nota-se também o que em uma faixa de 20 anos (1992-2021) o número de citações por ano aumentou de 0.2 para 1.0, um aumento de 5 vezes.

\begin{figure}[ht]
    \centering
    \includegraphics[width=12cm]{experiments/gustavo-tomas/AnaliseBibliometrica/GPUs/Graficos/gpu-citation-year.png}
    \caption{Evolução das citações por ano}
    \label{fig:gpu-citation-year}
\end{figure}

\subsection{\textit{Gráfico de três campos}}

O gráfico em \ref{fig:gpu-three-field} mostra um gráfico de três campos feitos com os dados acerca das referências, autores e palavras-chaves mais relevantes.

\begin{figure}[ht]
    \centering
    \includegraphics[width=12cm]{experiments/gustavo-tomas/AnaliseBibliometrica/GPUs/Graficos/gpu-three-field.png}
    \caption{Gráfico de três campos analisando palavras-chave}
    \label{fig:gpu-three-field}
\end{figure}

Observando as palavras-chaves, é possível perceber que termos como iluminação, processamento em paralelo e \textit{ray tracing} são expressões relevantes no contexto de GPUs.

Observando os autores é possível perceber que a maior parte é de origem asiática e que esses autores utilizam principalmente \textit{masuda n 2006 opt express} como referência.

\subsection{Refinamento da coleta de dados}

O gráfico em \ref{fig:gpu-co-occur} mostra a rede de co-ocorrências de palavras-chave. Dentro os termos é possível perceber que os assuntos tratados são relevantes ao tópico de GPUs, com destaque ao campo em verde que separa assuntos relativos a dispersão de luz em meios turbulentos (\textit{turbid media}).

\begin{figure}[ht]
    \includegraphics[width=12cm]{experiments/gustavo-tomas/AnaliseBibliometrica/GPUs/Graficos/gpu-co-ocurr.png}
    \caption{Rede de co-ocorrências de palavras-chave}
    \label{fig:gpu-co-occur}
\end{figure}
