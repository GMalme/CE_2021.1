\section{Respostas de Artur Filgueiras Scheiba Zorron \label{tarefa-CaioMassucato-componentes-eperimento}}

\subsection{A computação é uma atividade científica? Justifique. }

Para responder se a computação é uma atividade científica é preciso:
\begin{itemize}
    \item Definir o que é uma atividade científica;
    \item Definir o que é computação;
    \item Analisar se a definição de computação se enquadra na definição de atividade científica. 
\end{itemize}

\subsubsection{O que é uma atividade científica?} Atividade Científica pode ser admitida como um conjunto de conhecimentos dos fenômenos naturais então ordenados, correlacionados e interpretados. 

O pensamento científico convoca o raciocínio para interpretar e relacionar. Há uma disciplina da atenção dirigida:

\begin{enumerate}
    \item à objetividade, para que teorias ou sistemas resultem de avaliação imparcial, independente de tendências ou pré-julgamentos;
    \item  à exatidão, para que se consiga a ideia adequada aos fatos;
    \item  à ordenação, para análise de eventuais variações do que foi observado;
    \item aos pormenores para que não ocorra menosprezo de aspectos pouco evidentes.
\end{enumerate}

\subsubsection{O que é a computação?}

A computação pode ser definida como a busca de solução para um problema a partir de entradas (inputs), de forma a obter resultados (outputs) depois de processada a informação através de um \gls{Algoritmo}.

Logo, definindo características de experimento científico:

uma experiência adequada implica:

\begin{enumerate}
    \item que os antecedentes temporais sejam claros;
    \item que exista uma covariação estatisticamente significativa entre uma causa e um efeito;
    \item que não existam terceiras variáveis que possam dar uma explicação alternativa para a relação de causa e o efeito;
    \item que não haja hipóteses alternativas sobre os construtos utilizados;
\end{enumerate}

Com base no que foi visto em \ref{chap:CaioMassucato:impressoes} e a partir do conceito de \gls{MetodoExperimental}, podemos perceber que o método pelo qual a computação obtém suas respostas se adequa ao que é ciência.

\subsection{A computação é uma ciência experimental? Justifique. }

Com base no que foi exposto anteriormente, a maneira como a Computação é utilizada para criar soluções, algoritmos e testes se aproxima da produção do conhecimento pelo método científico.

\subsection{A computação é uma ciência empírica? Justifique. }

Com base no que foi exposto anteriormente, a maneira como a Computação é utilizada para criar soluções, algoritmos e testes, utilizando \gls{Dados} de entrada e saída para gerar soluções se aproxima da produção do conhecimento científico com base no consumo e geração de dados, logo, pode ser classificada como uma atividade empírica.