\section{Respostas de Rafael Henrique Nogalha de Lima\label{tarefa-rafaelnogalha-componentes-eperimento}}


\subsection{A computação é uma atividade científica? Justifique}

Vamos primeiro entender o que é \gls{Computacao}. A partir dessa definição, temos os seguintes pontos em relação à ela: por meio dela há o estudo de algoritmos e resolução de problemas do mundo real e ela da \gls{Suporte} a análise de outras áreas. Dessa forma, como a atividade científica é definida como um conjunto de conhecimentos dos fenômenos naturais então ordenados, correlacionados e interpretados, diferente do conhecimento vulgar, que decorre apenas de observação rotineira de fatos, afastada de buscas de interpretação racional e de conexões possíveis; logo, a computação é uma atividade científica sim.

\subsection{A computação é uma ciência experimental? Justifique}

Primeiro vamos conceituar a ciência experimental, são todas aquelas ciências que se utilizam de experimentação para comprovar os seus postulados teóricos. Logo, como discutido acima, a computação é sim uma ciência experimental, visto que utiliza de provas formais para comprovar um \gls{Algoritmo} e um Teorema, por exemplo. Dessa forma, a \gls{Computacao} é uma ciência experimental.

\subsection{A computação é uma ciência empírica? Justifique}

Como a computação utiliza comprova suas hipóteses gerando dados de saída, então ela é uma ciência empírica.