\section{Respostas de João Pedro Felix}

Partindo do princípio que a ciência tem por objetivo descrever, explicar e prever fenômenos (\gls{fenomeno}) a partir de dados (\gls{Dado}), tudo isso através de um estudo metodicamente organizado, fica claro que a posição da área de ciência da computação enquanto ciência fica em cheque. Ainda assim, é interessante analisarmos se a atividade profissional de um cientista da computação utiliza todos os componentes de um experimento científico.

Um experimento científico nasce dos seguintes componentes:

Uma classe de fenômenos de interesse. No caso da computação poderíamos citar qualquer problema que possa ser resolvido computacionalmente, como traçar a melhor rota para fugir do trânsito.

Hipóteses e modelos. O processo de definir um modelo computacional, isto é, as bases para que o programa que será desenvolvido de fato solucione problema que caracteriza o \gls{fenomeno} de interesse, pode se encaixar neste componente.

Causa e efeito. A \gls{Causalidade} diz respeito a influência que um evento tem de produzir outro evento. É evidente os impactos gerados pela computação no mundo inteiro, sem ela, este texto por exemplo não teria sido elaborado.

Procedimentos Repetíveis. A própria natureza da computação é baseada no fato de que toda vez que um programa for executado sob as mesmas circunstâncias, este deve apresentar os mesmos resultados. Um exemplo desta abordagem seria a da comparação de eficiência entre dois algoritmos de ordenação. Ambos algoritmos podem ser disponibilizados e uma análise pode ser refeita por qualquer pessoa que detenha o conhecimento necessário. Os resultados serão os mesmos, desde que feitos corretamente.

Controle de fatores. Ainda no exemplo anterior, seria possível desconsiderar fatores na análise como a variação na frequência do processador utilizado durante a execução do algoritmo. Seria possível até realizar a análise ignorando completamente as interferências dos componentes físicos do sistema.

Coleta de dados. A computação tem como principio básico o processamento de dados de entrada e a geração de dados de saída. Isso a caracteriza como uma atividade claramente empírica. Naturalmente, isso se dá através da coleta de dados.

Análise Lógica. A comprovação de um sistema computacional pode se dar de diferentes maneiras, desde uma argumentação formal matemática, até uma análise empírica por parte dos usuários durante o uso do sistema.

Com tudo isso em mente, podemos concluir que, embora a computação não se encaixe muito bem na definição de ciência, a computação trata-se de uma atividade empírica que se aproxima bastante do processo de produção do conhecimento científico via método experimental.