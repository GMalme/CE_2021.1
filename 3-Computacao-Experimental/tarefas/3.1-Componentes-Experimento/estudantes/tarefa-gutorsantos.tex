\section{Respostas de Gustavo Rodrigues dos Santos\label{tarefa-gutorsantos-tarefa-5-1}}

\subsection{A computação é uma atividade científica?}

Com base nas discussões \ref{chap:gutorsantos:impressoes-ciencia} e apoiado nas aulas e documentos (\ref{fundamentos:ce}) fornecidos pelo professor, neste breve ensaio, tentaremos responder à pergunta ``A computação é uma atividade científica?''. Portanto devemos definir alguns outros conceitos e guiar e enriquecer a análise da formulação da resposta.

\begin{quote}
Definir o que é uma atividade científica;

Definir o que é computação;

Analisar se a definição de computação se enquadra na definição de atividade científica. 
\end{quote}

\subsubsection{O que é uma atividade científica?} 

Como visto em \cite{fernandes_consideracoes_2021}, vemos que a produção do conhecimento científico deve ser pautada em alguns princípios para que os resultados obtidos possam garantir uma confiabilidade e veracidade acerca do estudo. Portanto, ao longo da história da humanidade e da ciência, foi desenvolvido o \gls{MetodoCientifico} para garantir a execução de alguns desses princípios, sendo o mais evidente, o princípio da Reprodutibilidade.

Nesse sentido, a atividade científica é a manifestação da produção do conhecimento. Ou seja, a interação, dentro da rede complexa da academia, entre alunos, professores, pesquisadores, instituições e organizações de publicação configuram essencialmente a atividade científica a medida que promovem a manutenção do ciclo da produção do conhecimento. Cada entidade nessa complexa rede possui suas devidas atribuições para a continuidade do ciclo. 

Existem os alunos, que renovam o pensamento dentro do meio acadêmico e consomem os estudos de outros cientistas. Estes são, por sua vez, orientados pelos professores e pesquisadores que além disso, fornecem estudos a serem utilizados como referência teórica pelos recém-chegados bem como por outros pesquisadores. Logo, esses pesquisadores devem estar localizados dentro de um contexto que permita o ``fazer ciência'', diante disso, são as instituições que fornecem este ambiente próprio para a condução das pesquisas, sendo por possuírem bons ambientes físicos de trabalho quanto pelo financiamento de capital para a execução dos estudos. Por fim, o conhecimento deve ser propagado --- seguindo o princípio da publicidade ---  e então as organizações de publicação ao disponibilizar o conhecimento, em meios de comunicação, para que este seja consumido por diversos outros cientistas ao redor do mundo, cumprem este papel e, consequentemente, evita-se que o conhecimento fique restrito apenas à bolha de um certa instituição. Incontestavelmente, existem outros agentes que influenciam nesse ciclo, como os governos e empresas privadas, no entanto, este exemplo ilustra bem o cerne da atividade científica.


\subsubsection{O que é a computação?}
Tomando o verbo, computar, de mesmo radical da palavra computação, o dicionário dicionário define como calcular. Essa definição possui sentido, quando se analisa a história da computação, assim, chegando na abstração máxima da computação, esta se configura como uma maneira de manipular números, baseada principalmente na matemática e na lógica, em busca de um resultado. 

\subsubsection{Jeitos de se fazer computação}
Como desenvolvimento acerca da computação, obteve-se diversos avanços e com isso, naturalmente, surgiram novos usos da computação. A função de fornecer uma entrada contendo um conjunto de dados, processá-los e obter uma saída com um resultado ainda está presente no dia a dia da computação, nunca deixará de estar pois este é a essência dessa atividade. Pode-se hoje, invés de,  por exemplo, utilizar o computador para obter números de uma trajetória balística, como feito no ENIAC, criar janelas gráficas, para que então seja possível, dentro delas, exibir-se outros elementos gráficos, ou seja, há um processamento porém não obtêm-se um resultado puramente numérico. 

Nesse sentido, houve a possibilidade da criação de softwares que escaparam da esfera técnica permitindo que usuários sem conhecimento em computação utilizassem os computadores, dando origem, assim, aos computadores pessoais. Consequentemente, formou-se um mercado propício para que empresas explorassem o potencial acerca do desenvolvimento de softwares tanto para uso pessoal quanto empresarial. Portanto, dentro das organizações foram sendo criados métodos para um desenvolvimento coeso e eficaz. Há diversos métodos de desenvolvimento porém quase todos possuem pelo menos algumas fases principais em comum

\begin{enumerate}
    \item Planejamento -- fase que consiste na formulação do problema a ser resolvido, geralmente é debatido com o cliente;
    \item Projeto -- consiste na formulação de um solução parcial ou total para o problema;
    \item Testagem -- consiste na execução de testes para verificar se a solução encontrada corresponde ao esperado;
    \item Revisão -- geralmente, apresenta-se a solução construída para o cliente e este deve verificar se a solução é compatível ao que foi pedido e ou se faltam alguns elementos, caso esteja tudo certo, o projeto é finalizado, caso contrário, alguns passos são revistos;
    
\end{enumerate}
\subsection{Conclusão }

Com base no exposto, a resposta para o questionamento inicial não é única, depende da onde está ocorrendo essa atividade. Considerando a exerção da computação dentro do âmbito empresarial, é possível traçar um paralelo e perceber que os métodos de desenvolvimento utilizado pelas equipes dentro das empresas, de certo modo, se aproximam do método científico. Entretanto, apesar da semelhança a computação exercida nas corporações não podem se enquadrar na definição de atividade científica. 

Por outro lado, olhando para área da Ciência da Computação, como o próprio nome já evidência, a computação exercida nesse campo configura uma atividade científica.

Nesse sentido, é visível a diferença dentro dos cursos de graduação, Análise e Desenvolvimento de Sistemas (ADS) e Ciência da Computação (CIC), em ADS possui foco na aplicação prática da computação voltado ao desenvolvimento de sistemas já em CIC nota-se que o foco é voltado para a teorização de fenômenos (\gls{fenomeno}) computacionais. Sendo assim, é perceptível a intima ligação dessas duas áreas --- a atividade científica produz conhecimentos que serão utilizados na prática pelos desenvolvedores.

Em síntese, a computação tem um vasta possibilidade de áreas do conhecimento e de aplicabilidade. Dessa forma, ela pode ou não configurar um tipo de atividade científica. Além disso, essa ampla possibilidade contribui positivamente para o desenvolvimento da computação em si, uma vez que permite a incorporação de diversos perfis de profissionais. 




