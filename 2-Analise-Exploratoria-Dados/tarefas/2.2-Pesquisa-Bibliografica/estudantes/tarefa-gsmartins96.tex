\chapter{Análise Bibliográfica sobre Malware, por Gabriel dos Santos Martins\label{chap:bibliometria:gsmartins96}}

\section{Planejamento do estudo}

... texto sobre o assunto ...

Partindo dos pontos acima, as questões que nortearam a pesquisa foram:
\begin{itemize}
    \item Questão 1?
    \item Questão 2?
    \item Questão 3? 
    \item Questão 4?
\end{itemize}

\subsection{Uso do Bibliometrix e Biblioshiny}
Serão usadas a ferramenta e o \textit{workflow} proposto pelos autores do pacote Bibliometrix, conforme indica a figura ~\ref{fig:bibliometrix:workflow}.

\subsection{Limitações} O exercício relatado foi feito em 8 horas, utilizando a base de dados Web of Science (WoS)

\section{Coleta de dados\label{MASSA:coleta}}

A coleta de dados feita usando a base Web of Science (WoS) no dia 08 de janeiro de 2022, acessado por meio do Portal de Periódicos da CAPES.

\subsection{Explicação para os termos de busca usados\label{sec:titofrota:query}}

\section{Análise dos dados}

\subsection{Filtragem de registros}

\subsection{Análise descritiva do \textit{dataset} }

\subsection{Evolução da Produção Científica}

\subsection{Interpretação do Crescimento}

\subsection{Evolução das Citações}

\subsection{Interpretação das Citações}

\subsection{\textit{Three-Field Plots (Sankey diagram)}}

\section{Refinamento da Coleta de Dados}

\section{Nova Análise dos Dados}

\subsection{Nova filtragem de registros}

\subsection{Análise descritiva do \textit{dataset} refinado}
