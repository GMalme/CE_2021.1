\subsection{Minhas impressões iniciais sobre a ciência, por Pedro de Torres Maschio}

A ciência, esforço contínuo da \gls{ComunidadeCientifica} é a mais notável atividade humana, existindo desde os primórdios da humanidade (embora não com as configurações atuais), sendo o trabalho do \gls{Cientista} contemporâneo organizado por uma  \gls{Metodologia} que garante a condução sem vieses dos estudos e também sua reprodutibilidade. A ciência atual é marcada pelo uso das tecnologias de informação e de colaboração entre a comunidade científica, bem diferente de seu início, em que as grandes distâncias atrapalhavam a troca de informações entre cientistas de diferentes países/continentes. Sua capacidade de criar novas tecnologias e de responder grandes questões da humanidade fez com que a ciência se tornasse objeto de interesse de empresas, que também podem manter projetos de pesquisa, visando o lucro e a criação de novos produtos.

A atividade científica também sofre críticas, conforme apresenta-se em \citep{chavalarias_whats_2017}, publicação científica no qual o autor investiga as dinâmicas da ciência contemporânea. Dentre as questões levantadas, está a relevância da relação entre produção científica e avanço real na ciência; a influência das dinâmicas sociais na produção científica e os efeitos da política e “publique ou pereça” que é relevante na ciência atual. De modo geral, a ciência é aberta a críticas e isto constitui sua principal fraqueza e sua principal força, pois usa das críticas para se reinventar e gerar novos meios de existir.