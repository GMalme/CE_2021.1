\section{Respostas de Stefano Luppi Spósito}

\subsection{A computação é uma atividade científica?}
Podemos definir a computação como a busca de soluções a partir de entradas, de forma a se obter saídas que são processadas por algoritmos. A computação não visa a produção de conhecimentos científicos, uma vez que a computação nada mais é do que uma ferramenta que pode ou não ser utilizada na produção destes conhecimentos, uma vez que ela pode reunir dados ou ajudar na confirmação de hipóteses.
Portanto, podemos afirmar que a computação não pode ser considerada como uma atividade científica.
\subsection{A computação é uma ciência experimental?}
Sabemos que a \gls{Ciencia} Experimental busca comprovar suas teorias através da \gls{Experimentacao}. Podemos considerar a computação como uma \gls{Ciencia} que utiliza de experimentos para comprovar suas teorias, visto que os chamados 'casos de teste' são criados para testar algoritmos e teses, com o intuito de prová-los, além de permitirem a criação de \gls{ModelosCientificos} que podem ser utilizados para facilitar o entendimento do experimento e do que se quer provar.
\subsection{A computação é uma ciência empírica?}
A computação pode ser considerada uma ciência empírica, uma vez que através da experiência que se tem na computação, novos resultados podem ser alcançados através de uma incessante série de tentativas e erros, com o objetivo de alcançar um determinado resultado, seja ele um experimento para comprovar um fato científico ou não.