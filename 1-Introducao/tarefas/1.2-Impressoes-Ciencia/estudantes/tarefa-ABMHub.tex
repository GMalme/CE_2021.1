\subsection{Minhas impressões iniciais sobre a ciência, por Lucas de Almeida Bandeira Macedo}

A ciência é uma das poucas atividades feitas em grupo, que contam com participantes do presente, passado e futuro. A ciência transcende tempo e espaço, sem necessariamente estudar física. A ciência representa o extremo oposto do que chamamos de egoísmo, pois não há espaço nela para quem não reconhece os esforços dos que vêm depois, e para os que não abrem espaços para que seus trabalhos sejam explorados no futuro. Nada nunca é feito sozinho, pois todo o conhecimento que alguém adquire durante a vida, foi repassado ou baseado em \gls{Conhecimento}s das pessoas que passaram pela vida dessas pessoas.

Este, entre outros vários, é o motivo da Ciência da Ciência \citep{schwartzman_ciencia_1984} ser tão importante: a ciência é complexa. As publicações, a árvore de conhecimento que levou a cada descoberta, a conexão das descobertas com o cotidiano, a diferença que cada estudo fez na humanidade... e o porquê desta descoberta ter sido possível. Entender é saber reproduzir. Não necessariamente conseguir, mas saber.