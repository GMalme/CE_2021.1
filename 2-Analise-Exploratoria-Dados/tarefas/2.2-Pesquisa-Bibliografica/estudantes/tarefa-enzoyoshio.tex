\chapter{Análise Bibliográfica sobre Programação Competitiva, por Enzo Yoshio \label{chap:bibliometria:}}

\section{Planejamento do estudo\label{}}

    O planejamento de estudos é importante para os estudos científicos e o método científico pois sem um planejamento e questionamento do que a pesquisa deve procurar, o objetivo e o foco da pesquisa pode se perder sem esse planejamento e a organização dos pesquisadores feita previamente. Dito isso, durante o planejamento, é necessário levantar alguns questionamentos para nortear a
 

\begin{itemize}
    \item Qual a relevância e a atenção dada a pesquisa do uso de tecnologia na educação nos últimos anos?
    \item A gameficação tem sido considerada como uma técnica pertinente na avaliação de métodos de ensino
    \item Nos últimos anos com o evento da pandemia, este assunto tem alçado maior importância, dados os desafios apresentados?
\end{itemize}

\subsection{O que já existe de pesquisa bibliométrica sobre esse tema?}

\cite{} 
\cite{}.


\subsection{Limitações} O trabalho foi desenvolvido ao decorrer de uma semana, empregando por volta de 5 a 10 horas.

\section{Coleta de dados\label{MASSA:coleta}}

A coleta de dados feita usando o mecanismo Web of Science no dia 09 de fevereiro de 2022, acessado por meio do Portal de Periódicos da CAPES.


\subsection{Query de Busca}

Foi usada a \query\  de busca ilustrada nas linhas 1 a 9 da listagem \ref{query20210803-2}.

\lstinputlisting[numbers=left,basicstyle=\normalsize\ttfamily,caption={\query\  de busca sobre simulação multiagente de fenômenos socials, com ênfase em métodos experimentais.},label=query20210803-2]
{experiments/jhcf/PesqBibliogr/SimulacaoMultiagente/WoS-20210803/classico-mais-citacoes/query.txt}

\subsubsection{Explicação para os termos de busca usados\label{}}


%Os termos \texttt{experimental}, \texttt{numeric*}, \texttt{statist*}, \texttt{hypothes*}, 
%\texttt{empiric*}
%e \texttt{inferen} (linhas 1 e 2 da query) foram usados na primeira cláusula da \query\  para recuperar artigos que tenham em seu título, palavras-chave e resumo, termos relacionados a métodos experimentais,
%métodos numéricos,
%métodos estatísticos,
%teste de hipóteses,
%métodos empíricos e métodos inferenciais.

Tendo como tópico central a educação o termo educ*  foi utilizado. Com enfoque no meio digital, usa-se o termo digita* e para aumentar a abragência para  a técnica de gameficação, o termo or gamefic*. Como este trabalho tem enfoque nas melhoras trazidas pelo meio digital, e não nas consequências e desafios de seu uso, os termos improv*. Ainda, para reduzir a generalidade e tratar mais das vantagens no contexto de aprendisado, também foi o usado o termo learn*





A $4^{a}$ cláusula, linha 9,  s.

\subsection{Registros recuperados}

Os  registros obtidos como resultado da busca encontram-se em \url{}. 

Os 7795 registros obtidos na busca podem ser encontrados no link ...
Foi ultilziada o opção de exportação para aquivo de texto sem formatação, contemplando todos as 29 opções de campo. Os 7795 registros foram recuperado em grupos de 1000 para posteriormente serem concatenados.

Foram utilizadas as opções \textit{Exportar registros para arquivo de texto sem formatação} e \textit{export full record / Gravar Conteúdo: Seleção personalizada, com todos os 29 campos disponíveis, inclusive referências citadas} no WoS, para que as citações também fosse usadas em análises da citações (estrutura intelectual do conhecimento). Os 8115 registros foram recuperados em nove blocos de até 1.000 registros por vez (1-1000, 1001-2000, 2001-3000, ..., 8001-8115).


\section{Análise dos dados}

\subsection{Filtragem de registros}
Antes da análise, é possível aplicar filtros sobre os registros obtidos.

Foi aplicado um filtro ao \dataset\   inicial, com 8.115 registros, que continham pŕevias de artigos, artigos de conferência, capítulos de livro etc. Foram mantidos apenas os registros de artigos publicados em revistas científicas\footnote{A suposição é que que o conhecimento de maior qualidade sobre o tema está nas publicações em revistas.}. Após a aplicação desse filtro, 5.787 registros foram mantidos no \dataset, que será doravante chamado

Os documentos coletados foram filtrados para apenas aqueles caracterizados como artigos publicados em revistas científicas. Além disso, foram tratados sob a lei de Bradford. Dos 7795 arquivos obtídos, foram filtrados  1124 como pertinentes a análise.

\subsection{Análise descritiva do \dataset\   }

Constam as informações gerais sobre os conjuntos de dados:
As informações mais gerais sobre o \dataset\   são as seguintes:
\begin{description}
    \item [\textit{Timespan}] Os artgiso obtidos da busca após a filtragem foram publicados entre o período de 1991 até 2022
    \item [\textit{Sources (Journals, Books, etc)}] Os artigos possuem ao todo 29 fontes diferentes, sendo assim, em média, 39 artigos por revista.
    \item [\textit{Average years from publication}] A média de tempo de publicação dos artigos é de 4.26 anos
    \item [\textit{Average citations per documents}] A média de citações por documentos é 18,35.
    \item [\textit{Average citations per year per doc}] Após o ano de sua publicação, das um dos dos artigos foi citado em média 2713 vezes por ano.
    \item [\textit{References}] Ao todo, foram feitas 46553  referências por entre os artigos coletados
    \item [\textit{Keywords Plus (ID)}] 1828 palavras chaves do tipo ID (Keyword Plus)
    \item [\textit{Author's Keywords (DE)}]  3211 palavras chaves escolhidas pelos autores dos artigos

    \item [\textit{Authors}]  
    \item [\textit{Author Appearances}] 
    \item [\textit{Authors of single-authored documents}] Dos 3874 autoes, 81 escreveram artigos individualmente
    \item [\textit{Authors of multi-authored documents}] Dos 3874 autores, 3793 são autores de artigos escritos em colaboração com outros.
    \item [\textit{Single-authored documents}] Dos 1124 documentos, 85 foram escritos indidualmente, os 1039 restantes foram produzidos em co-autoria.
    \item [\textit{Documents per Author}] A média de documentos por autor é de 0.29
    \item [\textit{Authors per Document}] A média de autores por dcumento é de 3.45.
    \item [\textit{Co-Authors per Documents}] As 4543 aparições de autores se distribuem em 4.04 por documento
    \item [\textit{Collaboration Index}] O indíce de colaboração (total de autores em artigos escritos em co-autoria / total de artigos escritos em co-autoria) é de 3,65.
\end{description}
