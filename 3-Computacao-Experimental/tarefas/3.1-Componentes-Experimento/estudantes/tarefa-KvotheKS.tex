\section{Respostas de Alexsander Correa de Oliveira}

Em ciências, toda atividade feita pelos profissionais tende a ter um caminho bem formado: há um interesse em algum fenômeno, e portanto, em pesquisar e se aprofundar nos mecanismos de tal aspecto físico. Cientistas da computação se distinguem categoricamente da imagem formada de um cientista profissional, mas por que isso?

Bem, para respondermos essa pergunta, primeiro temos de entender como a estrutura de uma ciência se dá para conseguirmos entender o que diferencia computação de outras atividades científicas. Inicialmente, será explicado o conceito passo a passo do que é experimentação, este que será agregado mais tarde a explicação de atividade científica.

\subsection{Explicando experimentação}

    Experimentos, em termos simples, são ferramentas para demonstrar a veracidade de uma determinada hipótese.
    
    Eles são constituídos por:
\begin{itemize}
    \item Uma Classe de Fenômenos de Interesse - Os objetos de interesse do cientista;
    \item Hipóteses e Modelos - São dúvidas/pensamentos sobre a classe de fenômenos;
    \item Causa e Efeito - É a construção de \gls{Causalidade} dentro da hipótese/modelo;
    \item Procedimento Repetíveis - ;
    \item Demonstração, Manipulação e Controle de Fatores
    \item Coleta de Dados
    \item Análise Lógica
\end{itemize}


\subsection{Explicando atividade científica}


\subsection{O que cientistas da computação fazem?}


\subsection{Conclusões}