\section{Respostas de Gabriel Martins de Almeida}

\subsection{A computação é uma atividade científica?}
A computação é uma atividade científica, o motivo é por existirem métodos que auxiliam na construção de computadores e de aplicações para os mesmos, por exemplo, a engenharia de software que contém vários métodos para se implementar um software ou o estudo da organização e arquitetura de computadores que também contém várias maneiras de se implementar um hardware, então a computação se distingue de um conhecimento não científico por existirem esses estudos da \gls{Computacao} que são ordenados, correlacionados e interpretados, mas ela pode assumir um caráter mais prático e menos teórico como em projetos de desenvolvedores individuais e em usos casuais por usuários comuns onde o conhecimento decorre de aplicações rotineiras e o objetivo não é a busca por interpretação racional e apenas recorrer à mesma.

\subsection{A computação é uma ciência experimental?}
A computação é uma  \gls{Ciencia} experimental em vários aspectos, por exemplo, ao estudar o consumo de um hardware quando se aplica determinada modificação e como é possível aumentar a eficiência energética do mesmo, também ao realizar vários testes em um programa para comprovar que o mesmo está funcionando adequadamente e também ao provarmos a lógica de determinada aplicação utilizando programas de prova como o Coq, mas ela pode assumir uma faceta mais prática quando se trata de aplicações privadas onde também se realizam testes em um escopo reduzido cujo o propósito é verificar a funcionalidade da implementação em vez de estudar os efeitos das mudanças na mesma.

\subsection{A computação é uma ciência empírica?}
A computação é uma ciência empírica, isso se dá ao fato de ser necessário comprovar a funcionalidade de suas implementações, por exemplo, com testes, provas algébricas e com relatórios sobre os resultados, então a partir desses resultados é necessário realizar uma \gls{Analise} para garantir a eficiência de determinado e realmente verificar que os resultados são positivos. 