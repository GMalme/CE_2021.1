\section{Respostas de Leonardo Alves Riether\label{tarefa-LeoRiether-componentes-experimento}}

Nesta seção, será discutida a resposta para a pergunta "A atividade profissional de um
cientista da computação utiliza todos os componentes de um experimento científico?". Antes
de discutirmos a resposta em si, precisamos primeiro entender quais as possíveis
atividades que um cientista da computação pode exercer, para somente então verificar se
elas de fato utilizam os componentes de um experimento científico ou não.

\subsection{O que faz um cientista da computação}

Sendo profissionais da área de \gls{Computacao}, cientistas da computação de modo geral
trabalham na resolução de problemas para determinadas entradas, por meio de algoritmos que
constroem uma resposta. A presença de algoritmos concretos, de natureza construtiva , e
que levam em conta a utilização de memória e tempo de execução, é um dos fatores que
distinguem a \gls{CienciaComp} da Matemática, além de deixar a área mais "voltada ao mundo
real".

Essa definição, no entanto, é bem ampla. Um cientista da computação pode trabalhar tanto
em áreas puramente teóricas; como em Teoria da Computação ou elaboração de Estruturas de
Dados; quanto em áreas práticas, por exemplo desenvolvendo software ou analisando a
segurança de sistemas computacionais. Notamos que as áreas mais práticas possuem bastante
interação com o mundo externo aos sistemas computacionais. É preciso lidar com dados
inseridos por usuários, mensagens recebidas de outros computadores, falhas no hardware,
etc. Mesmo assim, as áreas teóricas produzem conhecimento que deve ser aplicável à
prática, e portanto também precisam lidar bem com esse tipo de situação.

Tendo isso, podemos comparar as atividades exercidas pelos cientistas da computação com os
experimentos científicos.

\subsection{Componentes de um Experimento Científico}
Um \gls{experimento} científico possui vários componentes que o definem. Vamos compará-los
com os sistemas que um cientista da computação cria:

\begin{enumerate}
    \item \textbf{Uma classe de fenômenos (\gls{fenomeno}) de interesse} -- Tanto a
    Ciência quanto a Computação possuem fenômenos de interesse, portanto elas são
    semelhantes nesse ponto.

    \item \textbf{Hipóteses e Modelos} -- Apesar da computação de certo modo formular
    hipóteses, os modelos computacionais são bem diferentes dos científicos. Enquanto um
    modelo científico visa aproximar ou explicar um fenômeno do mundo real, um modelo
    computacional pode implementar qualquer função computável, que pode ser desde um
    cálculo matemático até um servidor de arquivos. Assim, os modelos computacionais
    tendem a criar algo novo para o mundo, os científicos apenas explicam o que já existe
    nele.

    \item \textbf{Causa (\gls{causalidade}) e Efeito} -- A natureza matemática de funções computacionais
    determina que existe uma relação de causa e efeito muito grande na computação. É fácil
    ver essa relação em linguagens funcionais puras, porque para uma mesma entrada, uma
    função pura sempre dará a mesma saída. Podemos, assim, dizer que aquela entrada
    "causa" uma saída específica. 

    \item \textbf{Procedimentos Repetíveis} -- Os procedimentos em sistemas computacionais
    não só são repetíveis, como são muito mais repetíveis que os científicos.

    \item \textbf{Demonstração, Manipulação e Controle de Fatores} -- Tanto os
    experimentos científicos quanto os computacionais apresentam essas características.

    \item \textbf{Coleta de Dados} -- A maioria dos sistemas computacionais coleta dados,
    mas isso não é necessário para que um sistema seja considerado computacional. Um
    exemplo clássico disso é o programa que todo programador escreve quando começa a
    aprender uma nova linguagem: o \textit{Hello World}. Outro exemplo seria um programa
    para calcular os primeiros $N$ dígitos de $\pi$. Mesmo que $N$ possa ser dado como
    entrada pelo usuário, isso não é necessário, já que é possível deixar $N$
    "\textit{hardcoded}" no código.

    \item \textbf{Análise Lógica} -- Sempre é necessário analisar logicamente o resultado
    de um trabalho em computação, seja ele prático ou teórico, para evitar a presença de
    bugs ou teorias mal formuladas.

\end{enumerate}

\subsection{Conclusão}

Apesar de existirem certas similaridades, a atividade profissional de um cientista da
computação não é necessariamente um experimento científico. Dito isso, é possível realizar
experimentos científicos com auxílio computacional, sendo um exemplo disso é a criação de
simulações, que modelam o mundo real por meio de ferramentas computacionais.

