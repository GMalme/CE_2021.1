\newglossaryentry{epistemologia}{
    name={Epistemologia},
    description={É o ramo da filosofia que estuda o conhecimento científico, os princípios, as hipóteses e os resultados das diversas ciências. O termo foi cunhado por James Frederick Ferrier, que originalmente se referia apenas ao conhecimento científico, mas com o tempo o termo começou a ser usado de forma mais abrangente, podendo se referir ao conhecimento humano em geral. Fonte: \citet{noauthor_epistemologia_nodate}}.
}

\newglossaryentry{causalidade}{
    name={Causalidade},
    description={
    Condição segundo a qual uma causa produz um efeito. Princípio de causalidade, relação necessária entre a causa e o efeito. (Enuncia-se: "Todo fato tem uma causa, e as mesmas causas produzem, nas mesmas condições, os mesmos efeitos.").
    Fonte: \citet{dicio_causalidade_2022}.
    }
}
