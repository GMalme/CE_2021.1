\section{Respostas de Fernanda Macedo de Sousa\label{tarefa-fernandams-componentes-eperimento}}

Um \gls{experimento} é um procedimento utilizado para apoiar ou refutar uma hipótese, um experimento científico possui os seguintes componentes: uma classe de fenômenos de interesse; hipóteses e modelos; causa e efeito; procedimentos repetíveis; demonstração, manipulação e controle de fatores; coleta de dados e análise lógica. Ao pensar na computação, é possível encontrar esses componentes em algumas atividades. Por exemplo, um teste de software apresenta uma manipulação controlada de fatores e registro de dados, ou seja, partes do procedimento da experimentação. Entretanto, nem toda coleta de dados significa que o que está sendo feito é um experimento visto que experimentos sempre contam com procedimentos repetíveis e análise lógica dos resultados.

Em vista disso, para analisar se a atividade profissional de um cientista da computação utiliza todos os componentes de um experimento científico é importante refletir a respeito de algumas questões como a computação ser ou não ser uma \gls{AtividadeCientifica}, uma ciência experimental, e uma ciência empírica.

A \gls{Computacao} aproxima-se de uma atividade científica pois envolve a construção de modelos, e por vezes a construção de teoria, simulação e \gls{Experimentacao}. Por exemplo, ao comparar a ideação de um modelo computacional útil e empírico com a ideação ou hipotetização científica, ou mesmo o desenvolvimento de software ou hardware com parte do planejamento de uma experimentação, ou como dito anteriormente sobre teste de software, pensar no teste sistemático de um sistema computacional com parte do planejamento e execução de experimentos, ou por fim, relacionar o uso de um sistema de computação com a análise dos dados (empíricos), é possível compreender essa aproximação.

Portanto, a computação aproxima-se da produção do conhecimento científico pelo método experimental quando usada no sentido de ideação, codificação, teste e uso de computadores.  Nesse caso, a computação aproxima-se da produção do conhecimento científico baseado no consumo e geração de dados, sendo assim, uma atividade empírica uma vez que empírico significa estar baseado na \gls{ColetaDados}, observação ou registro, metódicas ou não.